\documentclass[a4paper]{article}
\usepackage{parskip,color,soul,tabto}
\usepackage[pdfstartview=FitPage, colorlinks=true, linkcolor=black, citecolor=blue, urlcolor=blue, linktoc=all]{hyperref}
\parskip=0pt
\parindent -0.7cm
\rightskip -0.7cm
\begin{document}

\vspace*{-2cm}
Nb \tabto{0cm}1/327 (article\_id: 328)\par
TI \tabto{0cm}\hl{ABSORPTI}VE-\hl{CAPACIT}Y - A NEW PERSPECTIVE ON LEARNING AND INNOVATION\par
AU \tabto{0cm}COHEN, LEVINTHAL\par
PY \tabto{0cm}1990, SO ADMINISTRATIVE SCIENCE QUARTERLY\par
DT \tabto{0cm}Article\par
PG \tabto{0cm}25, NR 62, TC 7703\par
DE \tabto{0cm}null\par
ID \tabto{0cm}null\par
AB \tabto{0cm}null\par
\clearpage

\vspace*{-2cm}
Nb \tabto{0cm}2/327 (article\_id: 329)\par
TI \tabto{0cm}\hl{Absorpti}ve \hl{capacit}y: On the creation and acquisition of technology in development\par
AU \tabto{0cm}Keller\par
PY \tabto{0cm}1996, SO JOURNAL OF DEVELOPMENT ECONOMICS\par
DT \tabto{0cm}Article\par
PG \tabto{0cm}29, NR 35, TC 109\par
DE \tabto{0cm}HUMAN CAPITAL, LONG-RUN GROWTH, OPENNESS, TECHNOLOGICAL CHANGE, TRANSITIONAL DYNAMICS\par
ID \tabto{0cm}DEVELOPING-COUNTRIES, ENDOGENOUS GROWTH, GAPS, INDUSTRIAL\par
AB \tabto{0cm}Recent studies on technological transformation in developing countries emphasize that successful industrialization requires both technological information and a good understanding of its implementation. In an open economy context it is shown that the sustained gains in growth derived from a move towards an open trade regime evaporate as soon as one gives up the arguably inadequate notion of 'technology' as being all-inclusive. In this paper, technology is only implementable as the labor force has built up the corresponding skills. Trade liberalization disseminates new technologies and goods to the domestic economy, and has beneficial effects on the level of final output and consumption. But sustained growth gains are only forthcoming if in addition to the arrival of new technologies also skills are accumulated at a higher rate than before the regime change. I argue that this explains part of the variation in the long-run growth effects of trade liberalization in newly industrializing countries.\par
\clearpage

\vspace*{-2cm}
Nb \tabto{0cm}3/327 (article\_id: 330)\par
TI \tabto{0cm}Relative \hl{absorpti}ve \hl{capacit}y and interorganizational learning\par
AU \tabto{0cm}Lane, Lubatkin\par
PY \tabto{0cm}1998, SO STRATEGIC MANAGEMENT JOURNAL\par
DT \tabto{0cm}Article\par
PG \tabto{0cm}17, NR 78, TC 1300\par
DE \tabto{0cm}\hl{ABSORPTI}VE \hl{CAPACIT}Y, ALLIANCES, KNOWLEDGE, ORGANIZATIONAL LEARNING, R \& D\par
ID \tabto{0cm}COMPENSATION STRATEGY, COMPETITIVE ADVANTAGE, DECISIONS, DIVERSIFICATION, EMPIRICAL-ANALYSIS, FINANCIAL RESEARCH, FIRM PERFORMANCE, INNOVATION, PUBLICATION RECORDS, R-AND-D\par
AB \tabto{0cm}Much of the prior research on interorganizational learning has focused on the role of \hl{absorpti}ve \hl{capacit}y, a firm's ability to value, assimilate, and utilize new external knowledge. However, this definition of the construct suggests that a firm has an equal \hl{capacit}y to learn from all other organizations. We reconceptualize the Jinn-level construct \hl{absorpti}ve \hl{capacit}y as a learning dyad-level construct, relative \hl{absorpti}ve \hl{capacit}y. One firm's ability to learn from another firm is argued to depend on the similarity of both firms' (1) knowledge bases, (2) organizational structures and compensation policies, and (3) dominant logics. We then test the model using a sample of pharmaceutical-biotechnology RED alliances. As predicted, the similarity of the partners' basic knowledge, lower management formalization, research centralization, compensation practices, and research communities were positively related to interorganizational learning. The relative \hl{absorpti}ve \hl{capacit}y measures are also shown to have greater explanatory power than the established measure of \hl{absorpti}ve \hl{capacit}y, R\&D spending. (C) 1998 John Wiley \& Sons, Ltd.\par
\clearpage

\vspace*{-2cm}
Nb \tabto{0cm}4/327 (article\_id: 331)\par
TI \tabto{0cm}\hl{Absorpti}ve \hl{capacit}y, coauthoring behavior, and the organization of research in drug discovery\par
AU \tabto{0cm}Cockburn, Henderson\par
PY \tabto{0cm}1998, SO JOURNAL OF INDUSTRIAL ECONOMICS\par
DT \tabto{0cm}Article\par
PG \tabto{0cm}26, NR 30, TC 354\par
DE \tabto{0cm}null\par
ID \tabto{0cm}FIRM, INNOVATION, PRODUCTIVITY, R-AND-D, SCIENCE\par
AB \tabto{0cm}We examine the interface between for-profit and publicly funded research in pharmaceuticals. Firms access upstream basic research through investments in \hl{absorpti}ve \hl{capacit}y in the form of in-house basic research and 'pro-publication' internal incentives. Some firms also maintain extensive connections to the wider scientific community, which we measure using data on coauthorship of scientific papers between pharmaceutical company scientists and publicly funded researchers. 'Connectedness' is significantly correlated with firms' internal organization, as well as their performance in drug discovery. The estimated impact of 'connectedness' on private research productivity implies a substantial return to public investments in basic research.\par
\clearpage

\vspace*{-2cm}
Nb \tabto{0cm}5/327 (article\_id: 332)\par
TI \tabto{0cm}Coevolution of firm \hl{absorpti}ve \hl{capacit}y and knowledge environment: Organizational forms and combinative capabilities\par
AU \tabto{0cm}Van den Bosch, Volberda, de Boer\par
PY \tabto{0cm}1999, SO ORGANIZATION SCIENCE\par
DT \tabto{0cm}Article; Proceedings Paper\par
PG \tabto{0cm}18, NR 59, TC 331\par
DE \tabto{0cm}\hl{ABSORPTI}VE \hl{CAPACIT}Y, COMBINATIVE CAPABILITIES, KNOWLEDGE ENVIRONMENT, MICRO- AND MACROCOEVOLUTION, MULTIMEDIA INDUSTRIAL COMPLEX, ORGANIZATION FORMS\par
ID \tabto{0cm}ALLIANCES, CULTURE, FORTUNE FAVORS, INNOVATION, MANAGEMENT, PREPARED FIRM\par
AB \tabto{0cm}This paper advances the understanding of \hl{absorpti}ve \hl{capacit}y for assimilating new knowledge as a mediating variable of organization adaptation. Many scholars suggest a firm's \hl{absorpti}ve \hl{capacit}y plays a key role in the process of coevolution (Lewin et al., this issue). So far, most publications, in following Cohen and Levinthal (1990), have considered the level of prior related knowledge as the determinant of \hl{absorpti}ve \hl{capacit}y. We suggest, however, that two specific organizational determinants of \hl{absorpti}ve \hl{capacit}y should also be considered: organization forms and combinative capabilities. We will show how these organizational determinants influence the level of \hl{absorpti}ve \hl{capacit}y, ceteris paribus the level of prior related knowledge. Subsequently, we will develop st framework in which \hl{absorpti}ve \hl{capacit}y is related to both micro- and macro-coevolutionary effects. This framework offers an explanation of how knowledge environments coevolve with the emergence of organization forms and combinative capabilities that are suitable for absorbing knowledge. We will illustrate the framework by discussing two longitudinal case studies of traditional publishing firms moving into the turbulent knowledge environment of an emerging multimedia industrial complex.\par
\clearpage

\vspace*{-2cm}
Nb \tabto{0cm}6/327 (article\_id: 333)\par
TI \tabto{0cm}Meet me halfway: research joint ventures and \hl{absorpti}ve \hl{capacit}y\par
AU \tabto{0cm}Kamien, Zang\par
PY \tabto{0cm}2000, SO INTERNATIONAL JOURNAL OF INDUSTRIAL ORGANIZATION\par
DT \tabto{0cm}Article\par
PG \tabto{0cm}18, NR 29, TC 101\par
DE \tabto{0cm}\hl{ABSORPTI}VE \hl{CAPACIT}Y, JOINT VENTURES, SUBGAME PERFECT NASH EQUILIBRIUM\par
ID \tabto{0cm}DUOPOLY, ENDOGENOUS SPILLOVERS, POLICY, RESEARCH-AND-DEVELOPMENT, TECHNOLOGY\par
AB \tabto{0cm}We propose a representation of a firm's 'effective' R\&D effort level that reflects how both its R\&D approach and R\&D budget influences its ability to realize spillovers from other firms' R\&D activity, i.e. its '\hl{absorpti}ve \hl{capacit}y', and generalizes the commonly employed representation. The ability to choose an R\&D approach is accommodated by positing a three-stage game in which the choice of an R\&D approach is made in its first stage. The firms' R\&D budgets and output levels are chosen in the game's second and third stages, respectively. It is found that when firms cooperate in the setting of their R\&D budgets, i.e. form a research joint venture, they choose identical broad R\&D approaches. On the other hand, if they do not form a research joint venture, then they choose firm-specific R\&D approaches unless there is no danger of exogenous spillovers. The analysis suggests that the commonly employed representation of firms' effective R\&D investment levels implicitly presupposes that the firms have chosen to cooperate in setting their R\&D budgets. (C) 2000 Elsevier Science B.V. All rights reserved.\par
\clearpage

\vspace*{-2cm}
Nb \tabto{0cm}7/327 (article\_id: 334)\par
TI \tabto{0cm}R\&D competition, \hl{absorpti}ve \hl{capacit}y, and market shares\par
AU \tabto{0cm}Campisi, Mancuso, Nastasi\par
PY \tabto{0cm}2001, SO JOURNAL OF ECONOMICS-ZEITSCHRIFT FUR NATIONALOKONOMIE\par
DT \tabto{0cm}Article\par
PG \tabto{0cm}24, NR 23, TC 9\par
DE \tabto{0cm}COST-REDUCING R\&D, DYNAMIC NONCOOPERATIVE FEEDBACK GAME, EXTRA-INDUSTRY R\&D, STOCK OF TECHNOLOGICAL KNOWLEDGE\par
ID \tabto{0cm}COST REDUCTION, INDUSTRY, INNOVATION\par
AB \tabto{0cm}This paper deals with an oligopolistic industry where firms are engaged in cost-reducing R\&D activity to maximize their market shares. The existence and uniqueness of a feedback-Nash-optimal R\&D strategy for each firm are discussed. Our simulations highlight that variations in spillovers hardly influence the firms R\&D investment, if their \hl{absorpti}ve \hl{capacit}ies to exploit extramural knowledge depend on their R\&D efforts. Moreover, extramural knowledge cannot completely replace in-house R\&D. However, a high level of public R\&D favors the firm with the most restrictive R\&D expenditure constraint and/or with the lowest initial R\&D stock provided it invests in R\&D.\par
\clearpage

\vspace*{-2cm}
Nb \tabto{0cm}8/327 (article\_id: 335)\par
TI \tabto{0cm}The role of R\&D intensity, technical development and \hl{absorpti}ve \hl{capacit}y in creating entrepreneurial wealth in high technology start-ups\par
AU \tabto{0cm}Deeds\par
PY \tabto{0cm}2001, SO JOURNAL OF ENGINEERING AND TECHNOLOGY MANAGEMENT\par
DT \tabto{0cm}Article\par
PG \tabto{0cm}19, NR 81, TC 65\par
DE \tabto{0cm}ENTREPRENEURSHIP, HIGH TECHNOLOGY VENTURES, SCIENTIFIC CAPABILITIES, TECHNICAL CAPABILITIES\par
ID \tabto{0cm}CAPABILITIES, COMPETITIVE ADVANTAGE, EMPIRICAL-ANALYSIS, FIRM RESOURCES, INITIAL PUBLIC OFFERINGS, INNOVATION, KNOWLEDGE, SCIENCE, SHAREHOLDER VALUE, US PHARMACEUTICAL-INDUSTRY\par
AB \tabto{0cm}This study uses 80 newly public pharmaceutical biotechnology companies to explore the relationship between a high technology venture's R\&D intensity, technical capabilities and \hl{absorpti}ve \hl{capacit}y and the amount of entrepreneurial wealth created by the venture. A novel measure of \hl{absorpti}ve \hl{capacit}y based on co-citation analysis of a firm's scientific publications is developed and several indicators of technical capabilities are used to develop early and late stage measures of a firm's technical capabilities. The results provide strong evidence of a positive relationship between a high technology venture's R\&D intensity, late stage technical capabilities and \hl{absorpti}ve \hl{capacit}y and the amount of entrepreneurial wealth coated by a high technology venture. (C) 2001 Elsevier Science B.V. All rights reserved.\par
\clearpage

\vspace*{-2cm}
Nb \tabto{0cm}9/327 (article\_id: 336)\par
TI \tabto{0cm}Knowledge transfer in intraorganizational networks: Effects of network position and \hl{absorpti}ve \hl{capacit}y on business unit innovation and performance\par
AU \tabto{0cm}Tsai\par
PY \tabto{0cm}2001, SO ACADEMY OF MANAGEMENT JOURNAL\par
DT \tabto{0cm}Article\par
PG \tabto{0cm}9, NR 30, TC 933\par
DE \tabto{0cm}null\par
ID \tabto{0cm}CENTRALITY, FIRM, ORGANIZATIONS, POWER\par
AB \tabto{0cm}Drawing on a network perspective on organizational learning, I argue that organizational units can produce more innovations and enjoy better performance if they occupy central network positions that provide access to new knowledge developed by other units. This effect, however, depends on units' \hl{absorpti}ve \hl{capacit}y, or ability to successfully replicate new knowledge. Data from 24 business units in a petrochemical company and 36 business units in a food-manufacturing company show that the interaction between \hl{absorpti}ve \hl{capacit}y and network position has significant, positive effects on business unit innovation and performance.\par
\clearpage

\vspace*{-2cm}
Nb \tabto{0cm}10/327 (article\_id: 337)\par
TI \tabto{0cm}\hl{Absorpti}ve \hl{capacit}y, learning, and performance in international joint ventures\par
AU \tabto{0cm}Lane, Salk, Lyles\par
PY \tabto{0cm}2001, SO STRATEGIC MANAGEMENT JOURNAL\par
DT \tabto{0cm}Article\par
PG \tabto{0cm}23, NR 74, TC 540\par
DE \tabto{0cm}\hl{ABSORPTI}VE \hl{CAPACIT}Y, INTERNATIONAL JOINT VENTURES, KNOWLEDGE, LEARNING, PERFORMANCE\par
ID \tabto{0cm}BARGAINING POWER, BEHAVIOR, CHOICE, CONTEXT, DIVERSITY, INNOVATION, KNOWLEDGE TRANSFER, STRATEGIC ALLIANCES, TRUST, UNITED-STATES\par
AB \tabto{0cm}This paper proposes and tests a model of IJV learning and performance that segments \hl{absorpti}ve \hl{capacit}y into the three components originally proposed by Cohen and Levinthal (1990). First, trust between an IJV's parents and the IJV's relative \hl{absorpti}ve \hl{capacit}y, with its foreign parent are suggested to influence its ability to understand new knowledge held by foreign parents. Second, an IJV's learning structures and processes are proposed to influence its ability to assimilate new knowledge from those parents. Third, the IJV's strategy and training competence are suggested to shape its ability to apply the assimilated knowledge. Revisiting the Him.-arian IJVs studied by Lyles and Salk, (1996) 3 years later, we find support for the knowledge understanding and application predictions, and partial support for the knowledge assimilation prediction. Unexpectedly, our results suggest that trust and management support from foreign parents are associated with IJV performance but not learning. Our model and results offer a new perspective on IJV learning and performance as well as initial insights into how those relationships change over time. Copyright (C) 2001 John Wiley \& Sons, Ltd.\par
\clearpage

\vspace*{-2cm}
Nb \tabto{0cm}11/327 (article\_id: 338)\par
TI \tabto{0cm}The critical factors for technology \hl{absorpti}ve \hl{capacit}y\par
AU \tabto{0cm}Lin, Tan, Chang\par
PY \tabto{0cm}2002, SO INDUSTRIAL MANAGEMENT \& DATA SYSTEMS\par
DT \tabto{0cm}Article\par
PG \tabto{0cm}9, NR 61, TC 30\par
DE \tabto{0cm}CORPORATE CULTURE, INTERACTION, R\&D, RESOURCES, TECHNOLOGY TRANSFER\par
ID \tabto{0cm}DIFFUSION, FIRM, INNOVATION, JOINT VENTURES, KNOWLEDGE MANAGEMENT, ORGANIZATIONS, PERSPECTIVE, PRODUCTIVITY, RESEARCH-AND-DEVELOPMENT, STRATEGIES\par
AB \tabto{0cm}There are numerous studies concerning the key success factors for technology transfer performance, but little empirical research has been conducted on technology \hl{absorpti}ve \hl{capacit}y. In the real world, firms cannot successfully assimilate and apply external knowledge without greater \hl{absorpti}ve \hl{capacit}y. It is worthwhile exploring the critical factors of \hl{absorpti}ve \hl{capacit}y through its impact on transfer performance. Results reveal significant associations between technology \hl{absorpti}ve \hl{capacit}y and the critical factors technology diffusion channels, interaction mechanisms, and R\&D resources. Organizational cultures impact on interaction mechanisms, R\&D resources, \hl{absorpti}ve \hl{capacit}y and transfer performance. Different organizations will experience different technology transfer performance. Focuses explicitly on technology \hl{absorpti}ve \hl{capacit}y within the field of empirical technology transfer research. The findings are important for management practice, especially for firms carrying out technology transfer in developing countries.\par
\clearpage

\vspace*{-2cm}
Nb \tabto{0cm}12/327 (article\_id: 339)\par
TI \tabto{0cm}\hl{Absorpti}ve \hl{capacit}y: A review, reconceptualization, and extension\par
AU \tabto{0cm}Zahra, George\par
PY \tabto{0cm}2002, SO ACADEMY OF MANAGEMENT REVIEW\par
DT \tabto{0cm}Article; Proceedings Paper\par
PG \tabto{0cm}19, NR 103, TC 1831\par
DE \tabto{0cm}null\par
ID \tabto{0cm}COMPETITIVE ADVANTAGE, DYNAMIC CAPABILITIES, FIRM, INDUSTRY, INNOVATION, KNOWLEDGE, MANAGEMENT, ORGANIZATIONS, STRATEGIC ALLIANCES, TECHNOLOGY-TRANSFER\par
AB \tabto{0cm}Researchers have used the \hl{absorpti}ve \hl{capacit}y construct to explain various organizational phenomena. In this article we review the literature to identify key dimensions of \hl{absorpti}ve \hl{capacit}y and offer a reconceptualization of this construct. Building upon the dynamic capabilities view of the firm, we distinguish between a firm's potential and realized \hl{capacit}y. We then advance a model outlining the conditions when the firm's potential and realized \hl{capacit}ies can differentially influence the creation and sustenance of its competitive advantage.\par
\clearpage

\vspace*{-2cm}
Nb \tabto{0cm}13/327 (article\_id: 340)\par
TI \tabto{0cm}R\&D with spillovers and endogenous \hl{absorpti}ve \hl{capacit}y\par
AU \tabto{0cm}Kaiser\par
PY \tabto{0cm}2002, SO JOURNAL OF INSTITUTIONAL AND THEORETICAL ECONOMICS-ZEITSCHRIFT FUR DIE\par
DT \tabto{0cm}Article\par
PG \tabto{0cm}18, NR 29, TC 12\par
DE \tabto{0cm}null\par
ID \tabto{0cm}ANTITRUST, COMPETITION, COOPERATIVE RESEARCH, INNOVATION, RESEARCH JOINT VENTURES\par
AB \tabto{0cm}This paper derives a three-stage Cournot duopoly game for research cooperation, research expenditures, and product-market competition. The amount of knowledge firms can absorb is made dependent on their own research efforts, e.g., firms' \hl{absorpti}ve \hl{capacit}y is treated as an endogenous variable. It is shown that cooperating firms invest more in R\&D than noncooperating firms if spillovers are sufficiently large. The degree of market competition is a key determinant of the effects of research cooperation on research efforts, implying that existing models, which usually assume perfect competition, might be too restrictive.\par
\clearpage

\vspace*{-2cm}
Nb \tabto{0cm}14/327 (article\_id: 341)\par
TI \tabto{0cm}R\&D and \hl{absorpti}ve \hl{capacit}y: Theory and empirical evidence\par
AU \tabto{0cm}Griffith, Redding, Van Reenen\par
PY \tabto{0cm}2003, SO SCANDINAVIAN JOURNAL OF ECONOMICS\par
DT \tabto{0cm}Article\par
PG \tabto{0cm}20, NR 49, TC 98\par
DE \tabto{0cm}\hl{ABSORPTI}VE \hl{CAPACIT}Y, ENDOGENOUS GROWTH, R\&D, TOTAL FACTOR PRODUCTIVITY (TFP)\par
ID \tabto{0cm}CATCH, CONVERGENCE, COUNTRIES, INDUSTRIES, INNOVATION, MODEL, PATTERNS, PRODUCTIVITY GROWTH, SPILLOVERS, TECHNOLOGY FLOWS\par
AB \tabto{0cm}This paper presents a single unified framework that integrates the theoretical literature on Schumpeterian endogenous growth and major strands of the empirical literature on R\&D, productivity growth and productivity convergence. Starting from a structural model of endogenous growth following Aghion and Howitt (1992, 1998), we provide microeconomic foundations for the reduced-form equations for total factor productivity (TFP) growth frequently estimated empirically using industry-level data. R\&D affects both innovation and the assimilation of others' discoveries ("\hl{absorpti}ve \hl{capacit}y"). Long-run cross-country differences in productivity emerge endogenously, and the analysis implies that many existing studies underestimate R\&D's social rate of return by neglecting \hl{absorpti}ve \hl{capacit}y.\par
\clearpage

\vspace*{-2cm}
Nb \tabto{0cm}15/327 (article\_id: 342)\par
TI \tabto{0cm}\hl{Absorpti}ve \hl{capacit}y and the efficiency of research partnerships\par
AU \tabto{0cm}Scott\par
PY \tabto{0cm}2003, SO TECHNOLOGY ANALYSIS \& STRATEGIC MANAGEMENT\par
DT \tabto{0cm}Article; Proceedings Paper\par
PG \tabto{0cm}7, NR 21, TC 18\par
DE \tabto{0cm}null\par
ID \tabto{0cm}INNOVATION\par
AB \tabto{0cm}The paper advances the hypothesis that research partnerships expand a firm's \hl{absorpti}ve \hl{capacit}y. The paper juxtaposes the multimarket contact of firms with their patent cross-citations to provide a test of the \hl{absorpti}ve \hl{capacit}y hypothesis in general and an indirect test of the hypothesis that research partnerships expand \hl{absorpti}ve \hl{capacit}y. The findings support the hypotheses. In the context of those findings, the paper discusses the role for public policy toward research partnerships.\par
\clearpage

\vspace*{-2cm}
Nb \tabto{0cm}16/327 (article\_id: 343)\par
TI \tabto{0cm}Organizational \hl{absorpti}ve \hl{capacit}y and responsiveness: An empirical investigation of growth-oriented SMEs\par
AU \tabto{0cm}Liao, Welsch, Stoica\par
PY \tabto{0cm}2003, SO ENTREPRENEURSHIP-THEORY AND PRACTICE\par
DT \tabto{0cm}Article; Proceedings Paper\par
PG \tabto{0cm}23, NR 79, TC 91\par
DE \tabto{0cm}null\par
ID \tabto{0cm}CONSTRUCTION, ENVIRONMENT, FIRMS, INFORMATION, INNOVATION, KNOWLEDGE, MARKET ORIENTATION, PERFORMANCE, STRATEGIC CHOICE, VENTURES\par
AB \tabto{0cm}This study examines the relationship between firm \hl{absorpti}ve \hl{capacit}y and organizational responsiveness in the context of growth-oriented small and medium-sized enterprises (SMEs). By testing the different dimensions of \hl{absorpti}ve \hl{capacit}y, external knowledge acquisition and intrafirm knowledge dissemination were found to be positively related to organizational responsiveness. In addition, the relationships between \hl{absorpti}ve \hl{capacit}y and organizational responsiveness were moderated by environmental dynamism and the SMEs' strategic orientation. Results demonstrate that the responsiveness of growth-oriented SMEs is expected to increase if (1) they have well-developed capabilities in external knowledge acquisition and intrafirm knowledge dissemination; (2) they have a well-developed external knowledge acquisition capability and adopt a more proactive strategy, such as being a prospector; (3) they face a turbulent environment and have a well developed internal knowledge dissemination capability. Implications and future research directions are provided.\par
\clearpage

\vspace*{-2cm}
Nb \tabto{0cm}17/327 (article\_id: 344)\par
TI \tabto{0cm}Meet me halfway but don't rush: \hl{absorpti}ve \hl{capacit}y and strategic R\&D investment revisited\par
AU \tabto{0cm}Grunfeld\par
PY \tabto{0cm}2003, SO INTERNATIONAL JOURNAL OF INDUSTRIAL ORGANIZATION\par
DT \tabto{0cm}Article\par
PG \tabto{0cm}19, NR 21, TC 34\par
DE \tabto{0cm}\hl{ABSORPTI}VE \hl{CAPACIT}Y, R\&D INVESTMENT, R\&D SPILLOVERS, RJV\par
ID \tabto{0cm}COOPERATION, IMPERFECTLY APPROPRIABLE RESEARCH, RESEARCH JOINT VENTURES, SPILLOVERS\par
AB \tabto{0cm}In this paper, we analyse how R\&D investment decisions are affected by R\&D spillovers between firms, taking into consideration that more R\&D investment improves the ability to learn from competing firms-the so-called \hl{absorpti}ve \hl{capacit}y effect of R\&D. Contrary to earlier studies, we show that \hl{absorpti}ve \hl{capacit}y effects of own R\&D do not necessarily drive up the incentive to invest in R\&D. This only happens when the market size is small or the \hl{absorpti}ve \hl{capacit}y effect is weak. Otherwise, firms will actually choose to cut down on R\&D. Furthermore, \hl{absorpti}ve \hl{capacit}y effects also increase the critical rate of spillovers that determines whether a research joint venture generates more R\&D investment than a non-cooperative setting. Finally, we show that strong learning effects of own R\&D are not necessarily good for welfare. Moreover, if the market size is large, welfare will be at its highest when the learning effect is small. (C) 2003 Elsevier B.V. All rights reserved.\par
\clearpage

\vspace*{-2cm}
Nb \tabto{0cm}18/327 (article\_id: 345)\par
TI \tabto{0cm}MNC knowledge transfer, subsidiary \hl{absorpti}ve \hl{capacit}y, and HRM\par
AU \tabto{0cm}Minbaeva, Pedersen, Bjorkman, Fey, Park\par
PY \tabto{0cm}2003, SO JOURNAL OF INTERNATIONAL BUSINESS STUDIES\par
DT \tabto{0cm}Article\par
PG \tabto{0cm}14, NR 48, TC 304\par
DE \tabto{0cm}\hl{ABSORPTI}VE \hl{CAPACIT}Y, HRM, KNOWLEDGE TRANSFER\par
ID \tabto{0cm}HUMAN-RESOURCE MANAGEMENT, IMPACT, INTERNATIONAL-JOINT-VENTURES, MANUFACTURING PERFORMANCE, MOTIVATION, MULTINATIONAL-CORPORATIONS, ORGANIZATIONAL PERFORMANCE, PERSPECTIVE, PRODUCTIVITY, SYSTEMS\par
AB \tabto{0cm}Based on a sample of 169 subsidiaries of multinational corporations (MNCs) operating in the USA, Russia, and Finland, this paper investigates the relationship between MNC subsidiary human resource management (HRM) practices, \hl{absorpti}ve \hl{capacit}y, and knowledge transfer. First, we examine the relationship between the application of specific HRM practices and the level of the \hl{absorpti}ve \hl{capacit}y. Second, we suggest that \hl{absorpti}ve \hl{capacit}y should be conceptualized as being comprised of both employees' ability and motivation. Further, results indicate that both ability and motivation (\hl{absorpti}ve \hl{capacit}y) are needed to facilitate the transfer of knowledge from other parts of the MNC.\par
\clearpage

\vspace*{-2cm}
Nb \tabto{0cm}19/327 (article\_id: 346)\par
TI \tabto{0cm}\hl{Absorpti}ve \hl{capacit}y and the effects of foreign direct investment and equity foreign portfolio investment on economic growth\par
AU \tabto{0cm}Durham\par
PY \tabto{0cm}2004, SO EUROPEAN ECONOMIC REVIEW\par
DT \tabto{0cm}Article\par
PG \tabto{0cm}22, NR 42, TC 74\par
DE \tabto{0cm}ECONOMIC GROWTH, EQUITY FOREIGN PORTFOLIO INVESTMENT, FINANCIAL DEVELOPMENT, FOREIGN DIRECT INVESTMENT\par
ID \tabto{0cm}ANOMALIES, ASIAN FINANCIAL CRISIS, CORPORATE GOVERNANCE, COUNTRIES, LIBERALIZATION, OUTLIERS, REGRESSIONS, STOCK MARKETS\par
AB \tabto{0cm}This study examines the effects of foreign direct investment (FDI) and equity foreign portfolio investment (EFPI) on economic growth using data on 80 countries from 1979 through 1998. The results largely suggest that lagged FDI and EFPI do not have direct, unmitigated positive effects on growth, but some data are consistent with the view that the effects of FDI and EFPI are contingent on the '\hl{absorpti}ve \hl{capacit}y' of host countries, with particular respect to financial or institutional development. Moreover, extreme bound analysis (EBA) of significant results indicates that the estimates are robust compared to other empirical studies on growth. Published by Elsevier B.V.\par
\clearpage

\vspace*{-2cm}
Nb \tabto{0cm}20/327 (article\_id: 347)\par
TI \tabto{0cm}Prospects for developing \hl{absorpti}ve \hl{capacit}y through internal information provision\par
AU \tabto{0cm}Lenox, King\par
PY \tabto{0cm}2004, SO STRATEGIC MANAGEMENT JOURNAL\par
DT \tabto{0cm}Article\par
PG \tabto{0cm}15, NR 39, TC 122\par
DE \tabto{0cm}\hl{ABSORPTI}VE \hl{CAPACIT}Y, DIFFUSION, INFORMATION PROVISION\par
ID \tabto{0cm}BEHAVIOR, DRUG DISCOVERY, ENVIRONMENT, FIRM, INDUSTRY, INNOVATION, MANAGEMENT, ORGANIZATIONS, PERFORMANCE, PRODUCTIVITY\par
AB \tabto{0cm}Theories of \hl{absorpti}ve \hl{capacit}y propose that knowledge gained from prior experience facilitates the identification, selection, and implementation of related profitable practices. Researchers have investigated how managers may develop \hl{absorpti}ve \hl{capacit}y by building internal knowledge stocks, but few have focused on the distribution of this knowledge within the firm and the role managers play in administering information to organizational subunits. In this paper, we explore the degree to which managers can develop \hl{absorpti}ve \hl{capacit}y by directly providing information to agents in the organization that might potentially adopt a new practice. We find that the effectiveness of managerial information provision depends on the degree to which potential adopters have information from other sources. We find that information from previous adopters and past events reduces the effect of information provision, while experience with related practices amplifies it. Our research helps clarify when \hl{absorpti}ve \hl{capacit}y may provide a sustained competitive advantage. Copyright (C) 2004 John Wiley Sons, Ltd.\par
\clearpage

\vspace*{-2cm}
Nb \tabto{0cm}21/327 (article\_id: 348)\par
TI \tabto{0cm}The effects of knowledge attribute, alliance characteristics, and \hl{absorpti}ve \hl{capacit}y on knowledge transfer performance\par
AU \tabto{0cm}Chen\par
PY \tabto{0cm}2004, SO R \& D MANAGEMENT\par
DT \tabto{0cm}Article\par
PG \tabto{0cm}11, NR 59, TC 81\par
DE \tabto{0cm}null\par
ID \tabto{0cm}COOPERATION, EVOLUTION, FIRM, INNOVATION, JOINT VENTURES, MANAGEMENT, ORGANIZATIONAL CAPABILITIES, PERSPECTIVE, STRATEGIC ALLIANCES, TECHNOLOGY\par
AB \tabto{0cm}The main purpose of this study is to examine the effects of knowledge attribute, alliance characteristics, and firm's \hl{absorpti}ve \hl{capacit}y on the performance of knowledge transfer. Regression analysis was used to test the hypotheses in a sample of 137 alliance cases. The findings suggest that knowledge transfer performance is positively affected by the explicitness of knowledge and firm's \hl{absorpti}ve \hl{capacit}y; that equity-based alliance will transfer tacit knowledge more effectively while contract-base alliance is more effective for the transfer of explicit knowledge; and that trust and adjustment have positive effects while conflict possesses a curvilinear effect on knowledge transfer performance.\par
\clearpage

\vspace*{-2cm}
Nb \tabto{0cm}22/327 (article\_id: 349)\par
TI \tabto{0cm}Systemic \hl{absorpti}ve \hl{capacit}y: creating early-to-market returns through R\&D alliances\par
AU \tabto{0cm}Newey, Shulman\par
PY \tabto{0cm}2004, SO R \& D MANAGEMENT\par
DT \tabto{0cm}Article\par
PG \tabto{0cm}10, NR 36, TC 15\par
DE \tabto{0cm}null\par
ID \tabto{0cm}ADVANTAGE, DOMINANT DESIGNS, ENVIRONMENT, EXPLOITATION, FIRM, INNOVATION, KNOWLEDGE, MANAGEMENT, MODEL, RESOURCE-BASED VIEW\par
AB \tabto{0cm}For most complex emergent technologies, product-market success depends on efficient linkages between changing lead innovators within the R\&D process. In this paper, our unit of analysis is a complex high technology product and the system of alliance linkages formed to progress a product through R\&D milestones. We present a model and evidence for advancing our understanding of how achieving early-to-market returns depends on systemic \hl{absorpti}ve \hl{capacit}y. This systemic \hl{absorpti}ve \hl{capacit}y is the cumulative efficiency in the use of \hl{absorpti}ve \hl{capacit}y to link changing lead innovators across successive milestones in R\&D product development. We advance propositions of how systemic \hl{absorpti}ve \hl{capacit}y can explain performance differences between rival product development systems competing for early-to-market returns with similar products through accelerating speed to market, cost and quality advantages. These explanations are contrasted with the conclusions of previous studies that have focused on \hl{absorpti}ve \hl{capacit}y of single firms or single alliances in RD.\par
\clearpage

\vspace*{-2cm}
Nb \tabto{0cm}23/327 (article\_id: 350)\par
TI \tabto{0cm}Venture team human capital and \hl{absorpti}ve \hl{capacit}y in high technology new ventures\par
AU \tabto{0cm}Hayton, Zahra\par
PY \tabto{0cm}2005, SO INTERNATIONAL JOURNAL OF TECHNOLOGY MANAGEMENT\par
DT \tabto{0cm}Article; Proceedings Paper\par
PG \tabto{0cm}19, NR 60, TC 17\par
DE \tabto{0cm}\hl{ABSORPTI}VE \hl{CAPACIT}Y, ACQUISITIONS., ALLIANCES, HUMAN CAPITAL, TOP MANAGEMENT TEAM, VENTURING\par
ID \tabto{0cm}CONTEXTUAL FACTORS, CORPORATE ENTREPRENEURSHIP, FINANCIAL PERFORMANCE, IMPACT, INDUSTRY, INNOVATION, INTERNATIONAL-JOINT-VENTURES, KNOWLEDGE, STRATEGIC ALLIANCES, TOP MANAGEMENT\par
AB \tabto{0cm}In high technology environments, rapid technological change means that the value of a firm's existing knowledge is quickly eroded. In order to acquire needed capabilities, high technology new ventures engage in venturing strategies such as acquisitions and joint ventures. The top management team is an important source of inherited knowledge for these ventures and therefore influences their \hl{absorpti}ve \hl{capacit}y. We hypothesise that high technology new ventures' ability to learn through venturing activities is influenced by their top management teams' human capital. This study analyses the human capital characteristics of the top management teams of 340 high technology new ventures from the USA. It is found that human capital diversity of the top management teams moderates the relationship between venturing activities and innovation and financial performance in these ventures. In contrast, the level of human capital has no effect. The implications for management practice and future research are discussed.\par
\clearpage

\vspace*{-2cm}
Nb \tabto{0cm}24/327 (article\_id: 351)\par
TI \tabto{0cm}Knowledge creation: \hl{absorpti}ve \hl{capacit}y, organizational mechanisms, and knowledge storage/retrieval capabilities\par
AU \tabto{0cm}Chou\par
PY \tabto{0cm}2005, SO JOURNAL OF INFORMATION SCIENCE\par
DT \tabto{0cm}Article\par
PG \tabto{0cm}13, NR 55, TC 25\par
DE \tabto{0cm}\hl{ABSORPTI}VE \hl{CAPACIT}Y, KNOWLEDGE CREATION, ORGANIZATIONAL MECHANISMS, ORGANIZATIONAL MEMORY\par
ID \tabto{0cm}ADAPTATION, FIRM, INFORMATION-TECHNOLOGY, INNOVATION, MANAGEMENT, PERSPECTIVE, SYSTEMS\par
AB \tabto{0cm}Drawing on a knowledge-based perspective of the firm (KBV), this study develops a framework that delineates the interrelationships among 'the roles of individuals as well as organizations,' 'the IT capabilities of knowledge storage/retrieval,' and 'knowledge creation.' In order to test the feasibility of this framework, we conducted an empirical study. This study employed a survey instrument, which collected data from 1000 respondents from organizations in manufacturing, trade, transportation, service industries, and academic institutions. A total of 271 useable responses were analyzed. The major contributions of this research are to: (a) develop a knowledge management framework based on individual and organizational perspectives; (b) identify the impact of individuals' \hl{absorpti}ve \hl{capacit}y, and organizational mechanisms, on knowledge creation; and (c) specify the moderating effect of organizational memory. The implications of the study are provided, and further research directions are proposed.\par
\clearpage

\vspace*{-2cm}
Nb \tabto{0cm}25/327 (article\_id: 352)\par
TI \tabto{0cm}Frontier technology, \hl{absorpti}ve \hl{capacit}y and distance\par
AU \tabto{0cm}Kneller\par
PY \tabto{0cm}2005, SO OXFORD BULLETIN OF ECONOMICS AND STATISTICS\par
DT \tabto{0cm}Article\par
PG \tabto{0cm}23, NR 45, TC 30\par
DE \tabto{0cm}null\par
ID \tabto{0cm}DIFFUSION, ECONOMIC-GEOGRAPHY, GEOGRAPHIC LOCALIZATION, GROWTH, KNOWLEDGE, OUTPUT, R-AND-D, SPILLOVERS, TOTAL FACTOR PRODUCTIVITY, TRADE\par
AB \tabto{0cm}Recent literature on international technology diffusion has demonstrated the positive effect in foreign country productivity on the domestic economy. Using a sample of Organization for Economic Co-operation and Development (OECD) manufacturing industries we search for evidence that the effect of this foreign technology varies according to the level of \hl{absorpti}ve \hl{capacit}y and physical distance. We find evidence that both help to explain differences in the level of productivity across countries, but that \hl{absorpti}ve \hl{capacit}y is quantitatively more important. Physical distance had a greater effect at the start of the time period and in industries in which trade is local and technology is high-tech.\par
\clearpage

\vspace*{-2cm}
Nb \tabto{0cm}26/327 (article\_id: 353)\par
TI \tabto{0cm}\hl{Absorpti}ve \hl{capacit}y configurations in supply chains: Gearing for partner-enabled market knowledge creation\par
AU \tabto{0cm}Malhotra, Gosain, El Sawy\par
PY \tabto{0cm}2005, SO MIS QUARTERLY\par
DT \tabto{0cm}Review\par
PG \tabto{0cm}43, NR 115, TC 222\par
DE \tabto{0cm}\hl{ABSORPTI}VE \hl{CAPACIT}Y, CONFIGURATION APPROACHES, INTERORGANIZATIONAL INFORMATION SYSTEMS, KNOWLEDGE MANAGEMENT, PROCESS MODULARITY, RICH INFORMATION, SUPPLY CHAIN\par
ID \tabto{0cm}CLUSTER-ANALYSIS, COLLABORATION, COOPERATION, FIRM, INFORMATION-TECHNOLOGY, INNOVATION, ORGANIZATIONAL MEMORY, PRODUCT DEVELOPMENT, STRATEGIC MANAGEMENT RESEARCH, SYSTEMS\par
AB \tabto{0cm}The need for continual value innovation is driving supply chains to evolve from a pure transactional focus to leveraging interorganizational partner ships for sharing information and, ultimately, market knowledge creation. Supply chain partners are (1) engaging in interlinked processes that enable rich (broad-ranging, high quality, and privileged) information sharing, and (2) building information technology infrastructures that allow them to process information obtained from their partners to create new knowledge. This study uncovers and examines the variety of supply chain partnership configurations that exist based on differences in capability platforms, reflecting varying processes and information systems. We use the \hl{absorpti}ve \hl{capacit}y lens to build a conceptual framework that links these configurations with partner-enabled market knowledge creation. \hl{Absorpti}ve \hl{capacit}y refers to the set of organizational routines and processes by which organizations acquire, assimilate, transform, and exploit knowledge to produce dynamic organizational capabilities.
Through an exploratory field study conducted in the context of the RosettaNet consortium effort in the IT industry supply chain, we use cluster analysis to uncover and characterize five supply chain partnership configurations (collectors, connectors, crunchers, coercers, and collaborators). We compare their partner-enabled knowledge creation and operational efficiency, as well as the shortcomings in their capability platforms and the nature of information exchange. Through the characterization of each of the configurations, we are able to derive research propositions focused on enterprise \hl{absorpti}ve \hl{capacit}y elements. These propositions provide insight into how partner-enabled market knowledge creation and operational efficiency can be affected, and highlight the interconnected roles of coordination information and rich information. The paper concludes by drawing implications for research and practice from the uncovering of these configurations and the resultant research propositions. It also highlights fertile opportunities for advances in research on knowledge management through the study of supply chain contexts and other interorganizational partnering arrangements.\par
\clearpage

\vspace*{-2cm}
Nb \tabto{0cm}27/327 (article\_id: 354)\par
TI \tabto{0cm}Technology transfer from acquisition FDI and the \hl{absorpti}ve \hl{capacit}y of domestic firms: An empirical investigation\par
AU \tabto{0cm}Girma\par
PY \tabto{0cm}2005, SO OPEN ECONOMIES REVIEW\par
DT \tabto{0cm}Article\par
PG \tabto{0cm}13, NR 21, TC 18\par
DE \tabto{0cm}ACQUISITIONS, FOREIGN DIRECT INVESTMENT, PRODUCTIVITY\par
ID \tabto{0cm}FOREIGN-INVESTMENT, MODEL, PERFORMANCE, PRODUCTIVITY\par
AB \tabto{0cm}This paper seeks to establish the role of \hl{absorpti}ve \hl{capacit}y in technology transfer from acquisition FDI in U.K. manufacturing. It finds that the rate of productivity change following a foreign takeover is higher the higher the pre-acquisition productivity level of the acquired firm, indicating the importance of \hl{absorpti}ve \hl{capacit}y. However, beyond some critical level of initial productivity, the rate of technology transfer due to foreign acquisition starts to decline. It seems that U.K. acquisition targets that had been operating nearer the domestic technology frontier have less to gain from their association with foreign multinationals.\par
\clearpage

\vspace*{-2cm}
Nb \tabto{0cm}28/327 (article\_id: 355)\par
TI \tabto{0cm}\hl{Absorpti}ve \hl{capacit}y and connectedness: Why competing firms also adopt identical R\&D approaches\par
AU \tabto{0cm}Wiethaus\par
PY \tabto{0cm}2005, SO INTERNATIONAL JOURNAL OF INDUSTRIAL ORGANIZATION\par
DT \tabto{0cm}Article\par
PG \tabto{0cm}15, NR 20, TC 13\par
DE \tabto{0cm}\hl{ABSORPTI}VE \hl{CAPACIT}Y, APPROPRIABILITY, INNOVATION, R\&D, SPILLOVERS\par
ID \tabto{0cm}COOPERATION, DUOPOLY, INNOVATION, ME HALFWAY, RESEARCH JOINT VENTURES, SPILLOVERS\par
AB \tabto{0cm}This paper explores the endogenous determination of R\&D appropriability through the firms' choice of R\&D approaches. Whereas identical broad R\&D approaches 'connect' firms with their R\&D environment and maximize \hl{absorpti}ve \hl{capacit}ies, the opposite holds for idiosyncratic R\&D approaches. Our model shows that competing firms choose identical R\&D approaches in order to maximize knowledge flows between each other. In essence, this frees firms from the 'prisoner's dilemma' of aggressive investment in R\&D. Our analysis contrasts with Kamien and Zang's (2000) [Kamien, M., Zang, I., 2000. Meet me halfway: research joint ventures and \hl{absorpti}ve \hl{capacit}y. International Journal of Industrial Organization 18 995-1012] finding that competing firms chose idiosyncratic R\&D approaches. We demonstrate that their model also yields a Nash equilibrium for broad identical R\&D approaches. (c) 2005 Elsevier B.V. All rights reserved.\par
\clearpage

\vspace*{-2cm}
Nb \tabto{0cm}29/327 (article\_id: 356)\par
TI \tabto{0cm}\hl{Absorpti}ve \hl{capacit}y and productivity spillovers from FDI: A threshold regression analysis\par
AU \tabto{0cm}Girma\par
PY \tabto{0cm}2005, SO OXFORD BULLETIN OF ECONOMICS AND STATISTICS\par
DT \tabto{0cm}Article\par
PG \tabto{0cm}26, NR 45, TC 95\par
DE \tabto{0cm}null\par
ID \tabto{0cm}BENEFIT, DOMESTIC FIRMS, FOREIGN DIRECT-INVESTMENT, GEOGRAPHIC LOCALIZATION, HYPOTHESIS, INNOVATION, MODEL, PERFORMANCE, R-AND-D, TECHNOLOGY\par
AB \tabto{0cm}This paper explores whether the effect of foreign direct investment (FDI) on productivity growth is dependent on \hl{absorpti}ve \hl{capacit}y using recently developed threshold regression techniques. In manufacturing sectors where technology-exploiting multinationals are prevalent, the results point to the presence of nonlinear threshold effects: the productivity benefit from FDI increases with \hl{absorpti}ve \hl{capacit}y until some threshold level beyond which it becomes less pronounced. But there is also a minimum \hl{absorpti}ve \hl{capacit}y threshold level below which productivity spillovers from FDI are negligible or even negative. On the contrary, no evidence of productivity spillovers is found in sectors where FDI appears to be motivated by technology-sourcing considerations.\par
\clearpage

\vspace*{-2cm}
Nb \tabto{0cm}30/327 (article\_id: 357)\par
TI \tabto{0cm}Cluster \hl{absorpti}ve \hl{capacit}y - Why do some clusters forge ahead and others lag behind?\par
AU \tabto{0cm}Giuliani\par
PY \tabto{0cm}2005, SO EUROPEAN URBAN AND REGIONAL STUDIES\par
DT \tabto{0cm}Review\par
PG \tabto{0cm}20, NR 128, TC 97\par
DE \tabto{0cm}\hl{ABSORPTI}VE \hl{CAPACIT}Y, FIRM KNOWLEDGE BASE, INDUSTRIAL CLUSTERS\par
ID \tabto{0cm}COLLECTIVE EFFICIENCY, CONVERGENCE, GLOBAL VALUE CHAINS, INDUSTRIAL CLUSTERS, INNOVATION, KNOWLEDGE SPILLOVERS, LOCAL FIRMS, POLICY, REGIONAL-DEVELOPMENT, TECHNOLOGY\par
AB \tabto{0cm}This article provides a firm-centred interpretation of why some industrial clusters forge ahead and others lag behind. It argues that the dynamic growth of a cluster depends on its \hl{absorpti}ve \hl{capacit}y and therefore on the \hl{capacit}y of firms to absorb external knowledge and diffuse it into the intra-cluster knowledge system. This article speculates on the relationship existing between the heterogeneity of firms' knowledge bases with both intra- and extra-cluster knowledge systems. It concludes by illustrating that a conceptual link exists between firm-level knowledge bases, the cluster \hl{absorpti}ve \hl{capacit}y and its potential for growth.\par
\clearpage

\vspace*{-2cm}
Nb \tabto{0cm}31/327 (article\_id: 358)\par
TI \tabto{0cm}Strategic flexibility, rigidity and barriers to the development of \hl{absorpti}ve \hl{capacit}y in business markets: Themes and research perspectives\par
AU \tabto{0cm}Matthyssens, Pauwels, Vandenbempt\par
PY \tabto{0cm}2005, SO INDUSTRIAL MARKETING MANAGEMENT\par
DT \tabto{0cm}Article\par
PG \tabto{0cm}8, NR 19, TC 29\par
DE \tabto{0cm}\hl{ABSORPTI}VE \hl{CAPACIT}Y, MARKETING STRATEGY, RIGIDITY, STRATEGIC FLEXIBILITY\par
ID \tabto{0cm}CAPABILITIES, ORIENTATION\par
AB \tabto{0cm}The marketing and strategy literature hail strategic flexibility as a key success factor in creating continuously customer value and generating competitive advantage. However, empirical evidence indicates that rigidity in market strategies and actions is more the rule than the exception in organizations. The focus of this special issue is on better understanding rigidity and flexibility in business markets. This lead article seeks to elaborate on why companies face rigidity and how they can create flexibility. To do this, we relate rigidity in organizations to the concepts of dominant logic, industry recipe and persistence. The case illustrations highlight barriers to the development of \hl{absorpti}ve \hl{capacit}y in business organizations. Identifying such barriers is a first step in better understanding how companies can remain agile and flexible in demanding and fast changing markets. The paper then proceeds with a brief introduction to the other contributions of this special issue and concludes with a research agenda. (c) 2005 Elsevier Inc. All rights reserved.\par
\clearpage

\vspace*{-2cm}
Nb \tabto{0cm}32/327 (article\_id: 359)\par
TI \tabto{0cm}\hl{Absorpti}ve \hl{capacit}y in the software industry: Identifying dimensions that affect knowledge and knowledge creation activities\par
AU \tabto{0cm}Matusik, Heeley\par
PY \tabto{0cm}2005, SO JOURNAL OF MANAGEMENT\par
DT \tabto{0cm}Article\par
PG \tabto{0cm}24, NR 59, TC 42\par
DE \tabto{0cm}\hl{ABSORPTI}VE \hl{CAPACIT}Y, INNOVATION, KNOWLEDGE, LEARNING, SOFTWARE INDUSTRY\par
ID \tabto{0cm}CAPABILITIES, COLLABORATION, DETERMINANTS, FIRM, INNOVATION, METAANALYSIS, NETWORKS, ORGANIZATIONS, PERFORMANCE, STRATEGIC ALLIANCES\par
AB \tabto{0cm}The ability of the firm to effectively use external knowledge (its \hl{absorpti}ve \hl{capacit}y) is important to firm competitiveness and innovativeness. However the wide array of approaches to studying \hl{absorpti}ve \hl{capacit}y has obscured our understanding of what drives the effective use of external knowledge. The authors show that \hl{absorpti}ve \hl{capacit}y is composed of mutliple dimensions: (a) the firm's relationship to its external environment; (b) the structure, routines, and knowledge base of the main value creation group(s); and (c) individuals' \hl{absorpti}ve abilities. Their data illustrate that each of these dimensions contributes to increased knowledge or knowledge creation activities.\par
\clearpage

\vspace*{-2cm}
Nb \tabto{0cm}33/327 (article\_id: 360)\par
TI \tabto{0cm}Increased aid vs \hl{absorpti}ve \hl{capacit}y: Challenges and opportunities towards 2015\par
AU \tabto{0cm}de Renzio\par
PY \tabto{0cm}2005, SO IDS BULLETIN-INSTITUTE OF DEVELOPMENT STUDIES\par
DT \tabto{0cm}Article\par
PG \tabto{0cm}10, NR 13, TC 3\par
DE \tabto{0cm}null\par
ID \tabto{0cm}null\par
AB \tabto{0cm}This article examines several constraints on aid effectiveness associated with the key problem of \hl{absorpti}ve \hl{capacit}y, and then proposes an array of actions to address these problems. The constraints take diverse forms: macroeconomic; institutional and policy-oriented; technical and managerial; and those associated with donor behaviour. The constraints can be eased if donors do their homework; harmonise and align their strategies; design interventions with cared pay attention to macroeconomic management; renew their focus on infrastructure; and adopt innovative delivery mechanisms. The article concludes with an agenda for research which can facilitate such efforts. It is important to analyse successful examples of macroeconomic management in periods when aid flows increased; political economy factors driving countries' development strategy choices, including the incentives created by domestic politics and by aid relationships; as well as particularly effective sectoral strategies, the use of natural resource rents, and of reform efforts that allowed for a significant step-change in public sector performance.\par
\clearpage

\vspace*{-2cm}
Nb \tabto{0cm}34/327 (article\_id: 361)\par
TI \tabto{0cm}\hl{Absorpti}ve \hl{capacit}y, technological opportunity, knowledge spillovers, and innovative effort\par
AU \tabto{0cm}Nieto, Quevedo\par
PY \tabto{0cm}2005, SO TECHNOVATION\par
DT \tabto{0cm}Article\par
PG \tabto{0cm}17, NR 66, TC 96\par
DE \tabto{0cm}\hl{ABSORPTI}VE \hl{CAPACIT}Y, INNOVATION, KNOWLEDGE SPILLOVERS, R\&D, SPAIN, TECHNOLOGICAL OPPORTUNITY\par
ID \tabto{0cm}DEMAND, DETERMINANTS, DRUG DISCOVERY, FIRM SIZE, HIGH-TECH INDUSTRIES, MANUFACTURING-INDUSTRIES, MARKET-STRUCTURE, PRODUCTIVITY, RESEARCH-AND-DEVELOPMENT, STRATEGIC ALLIANCES\par
AB \tabto{0cm}This paper analyses the influence of two variables related with industrial structure (technological opportunity and knowledge spillovers) and one management variable (\hl{absorpti}ve \hl{capacit}y) on the innovative efforts developed by firms. These relationships are investigated in a total of 406 Spanish manufacturing companies with an established degree of innovative activity. In addition the nature of the variable,\hl{absorpti}ve \hl{capacit}y' is considered and an index is suggested that would render this concept operational through analysis of the factors defining it and by which the process of building it up is influenced. As a result of this study it is demonstrated that the \hl{absorpti}ve \hl{capacit}y variable determines innovative effort to a greater extent than the two structural variables. It is also shown that \hl{absorpti}ve \hl{capacit}y has a moderating effect on the relationship between technological opportunity and innovative effort being this one of the most remarkable results obtained from the work. (c) 2004 Elsevier Ltd. All rights reserved.\par
\clearpage

\vspace*{-2cm}
Nb \tabto{0cm}35/327 (article\_id: 362)\par
TI \tabto{0cm}Managing potential and realized \hl{absorpti}ve \hl{capacit}y: How do organizational antecedent's matter?\par
AU \tabto{0cm}Jansen, Van den Bosch, Volberda\par
PY \tabto{0cm}2005, SO ACADEMY OF MANAGEMENT JOURNAL\par
DT \tabto{0cm}Article\par
PG \tabto{0cm}17, NR 90, TC 371\par
DE \tabto{0cm}null\par
ID \tabto{0cm}CENTRIPETAL FORCES, COMBINATIVE CAPABILITIES, COMPETITIVE ADVANTAGE, DYNAMIC CAPABILITIES, FIRM, INTERRATER AGREEMENT, KNOWLEDGE STRUCTURES, PRODUCT DEVELOPMENT, SOCIALIZATION TACTICS, TECHNOLOGICAL INNOVATION\par
AB \tabto{0cm}Exploring how organizational antecedents affect potential and realized \hl{absorpti}ve \hl{capacit}y, this study identifies differing effects for both components of \hl{absorpti}ve \hl{capacit}y. Results indicate that organizational mechanisms associated with coordination capabilities (cross-functional interfaces, participation in decision making, and job rotation) primarily enhance a unit's potential \hl{absorpti}ve \hl{capacit}y. Organizational mechanisms associated with socialization capabilities (connectedness and socialization tactics) primarily increase a unit's realized \hl{absorpti}ve \hl{capacit}y. Our findings reveal why units may have difficulty managing levels of potential and realized \hl{absorpti}ve \hl{capacit}y and vary in their ability to create value from their \hl{absorpti}ve \hl{capacit}y.\par
\clearpage

\vspace*{-2cm}
Nb \tabto{0cm}36/327 (article\_id: 363)\par
TI \tabto{0cm}Technology spillovers, \hl{absorpti}ve \hl{capacit}y and economic growth\par
AU \tabto{0cm}Lai, Peng, Bao\par
PY \tabto{0cm}2006, SO CHINA ECONOMIC REVIEW\par
DT \tabto{0cm}Article; Proceedings Paper\par
PG \tabto{0cm}21, NR 61, TC 33\par
DE \tabto{0cm}\hl{ABSORPTI}VE \hl{CAPACIT}Y, ECONOMIC GROWTH, HUMAN CAPITAL, R\&D, TECHNOLOGY SPILLOVERS\par
ID \tabto{0cm}CHINA, DEVELOPING-COUNTRIES, DISTORTIONS, FOREIGN DIRECT-INVESTMENT, INCREASING RETURNS, INTEGRATION, INTERNATIONAL-TRADE, LONG-RUN GROWTH, MONOPOLISTIC COMPETITION, PRODUCTIVITY GROWTH\par
AB \tabto{0cm}By establishing an endogenous growth model with knowledge-driven R\&D, this paper aims to investigate the relationship between international technology spillovers, the host country's \hl{absorpti}ve capability and endogenous economic growth. The solution to the competitive equilibrium problem shows that long-run growth arises from improvements in \hl{absorpti}ve capability and higher human capital stocks, while the relationships between openness, the technology gap and the steady-state growth rate are uncertain. Econometric estimates of China's economic growth are obtained using province level data covering the period 1996-2002. The estimates indicate that technology spillovers depend on the host country's human capital investment and degree of openness, and that FDI is a more significant spillover channel than imports. (c) 2006 Elsevier Inc. All rights reserved.\par
\clearpage

\vspace*{-2cm}
Nb \tabto{0cm}37/327 (article\_id: 364)\par
TI \tabto{0cm}Spillovers and \hl{absorpti}ve \hl{capacit}y in a patent race\par
AU \tabto{0cm}Halmenschlager\par
PY \tabto{0cm}2006, SO MANCHESTER SCHOOL\par
DT \tabto{0cm}Article\par
PG \tabto{0cm}18, NR 25, TC 3\par
DE \tabto{0cm}null\par
ID \tabto{0cm}DEVELOPMENT COMPETITION, INDUSTRY, INNOVATION, MARKET-STRUCTURE, RESEARCH JOINT VENTURES, RESEARCH-AND-DEVELOPMENT, UNCERTAINTY\par
AB \tabto{0cm}This paper explores the impacts of \hl{absorpti}ve \hl{capacit}y on the behavior of innovating firms competing in a dynamic patent race. We introduce spillovers and \hl{absorpti}ve \hl{capacit}ies in the Fudenberg, Gilbert, Stiglitz and Tirole patent race with memory. First, we show that incorporating \hl{absorpti}ve \hl{capacit}y has different effects according as the firm leads or follows in the race. While the leader is getting soft, the follower revives, allowing him to possibly catch up with the leader. Second, we find that spillovers with \hl{absorpti}ve \hl{capacit}ies tend to increase the pace of innovation, a somewhat paradoxical outcome.\par
\clearpage

\vspace*{-2cm}
Nb \tabto{0cm}38/327 (article\_id: 365)\par
TI \tabto{0cm}Frontier technology and \hl{absorpti}ve \hl{capacit}y: Evidence from OECD manufacturing industries\par
AU \tabto{0cm}Kneller, Stevens\par
PY \tabto{0cm}2006, SO OXFORD BULLETIN OF ECONOMICS AND STATISTICS\par
DT \tabto{0cm}Article\par
PG \tabto{0cm}21, NR 44, TC 47\par
DE \tabto{0cm}null\par
ID \tabto{0cm}AGGREGATE PRODUCTION FUNCTION, DIFFUSION, ECONOMIC-GROWTH, EFFICIENCY, INEFFICIENCY, OUTPUT GROWTH, PANEL, RESEARCH-AND-DEVELOPMENT, SPECIFICATION, TOTAL FACTOR PRODUCTIVITY\par
AB \tabto{0cm}In this paper, we examine the three facets of technology: its creation, dispersion and \hl{absorpti}on. We investigate whether differences in \hl{absorpti}ve \hl{capacit}y help to explain cross-country differences in the level of productivity. We utilize stochastic frontier analysis to investigate two potential sources of this inefficiency - differences in human capital and R\&D - for nine industries in 12 Organization for Economic Co-operation and Development (OECD) countries over the period 1973-91. We find that inefficiency in production does indeed exist and it depends upon the level of human capital of the country's workforce. Evidence that the amount of R\&D an industry undertakes is also important is less robust.\par
\clearpage

\vspace*{-2cm}
Nb \tabto{0cm}39/327 (article\_id: 366)\par
TI \tabto{0cm}Faculty support for the objectives of university-industry relations versus degree of R\&D cooperation: The importance of regional \hl{absorpti}ve \hl{capacit}y\par
AU \tabto{0cm}Azagra-Caro, Archontakis, Gutierrez-Gracia, Fernandez-de-Lucio I\par
PY \tabto{0cm}2006, SO RESEARCH POLICY\par
DT \tabto{0cm}Article\par
PG \tabto{0cm}19, NR 51, TC 37\par
DE \tabto{0cm}\hl{ABSORPTI}VE \hl{CAPACIT}Y, UNIVERSITY-INDUSTRY RELATIONS\par
ID \tabto{0cm}ECONOMICS, GROWTH, INNOVATION, INTELLECTUAL PROPERTY, SCIENCE, SYSTEMS, TECHNOLOGY-TRANSFER\par
AB \tabto{0cm}The growing importance of regions in the analysis of innovation and the pressure on European universities to interact with their environment justify this article. It argues that faculty support for the objectives of university-industry relations (UIR) does not vary across disciplines and does not respond to university encouragement in a region with low \hl{absorpti}ve \hl{capacit}y. These results are in contrast with those obtained in studies of technology leading countries like the USA. Furthermore, incentives for UIR may generate unpredicted dynamics while instruments to cooperate are not significant. Finally, support for the objectives of UIR should not be confused with the degree of R\&D cooperation. The former is sensitive to university age while the latter is sensitive to gender, discipline, commitment to R\&D and university encouragement. Empirical evidence is obtained from a sample of faculty from the Valencian Community (Spain) and analysed through a set of models for discrete choice. (c) 2005 Elsevier B.V. All rights reserved.\par
\clearpage

\vspace*{-2cm}
Nb \tabto{0cm}40/327 (article\_id: 367)\par
TI \tabto{0cm}Innovation and regional \hl{absorpti}ve \hl{capacit}y: the labour market dimension\par
AU \tabto{0cm}Roper, Love\par
PY \tabto{0cm}2006, SO ANNALS OF REGIONAL SCIENCE\par
DT \tabto{0cm}Article; Proceedings Paper\par
PG \tabto{0cm}11, NR 27, TC 13\par
DE \tabto{0cm}null\par
ID \tabto{0cm}KNOWLEDGE, PERSPECTIVE, SPILLOVERS, TECHNOLOGY, UK\par
AB \tabto{0cm}In 2003, Eurostat published an 'experimental' dataset on regional innovation levels derived from the Second Community Innovation Survey. This dataset, part of the European Innovation Scoreboard, also contains a range of regional labour market indicators. In this paper, we report an exploratory analysis of this data, focussing on how the labour market characteristics of regions shape regions' \hl{absorpti}ve \hl{capacit}y (RACAP) and their ability to assimilate knowledge from public and externally conducted R\&D. In particular, we aim to establish whether labour market aspects of RACAP are more important for innovation in prosperous or lagging regions of the European Union (EU).\par
\clearpage

\vspace*{-2cm}
Nb \tabto{0cm}41/327 (article\_id: 368)\par
TI \tabto{0cm}\hl{Absorpti}ve \hl{capacit}y: Enhancing the assimilation of time-based manufacturing practices\par
AU \tabto{0cm}Tu, Vonderembse, Ragu-Nathan, Sharkey\par
PY \tabto{0cm}2006, SO JOURNAL OF OPERATIONS MANAGEMENT\par
DT \tabto{0cm}Article\par
PG \tabto{0cm}19, NR 79, TC 65\par
DE \tabto{0cm}\hl{ABSORPTI}VE \hl{CAPACIT}Y, EMPIRICAL RESEARCH, OPERATIONS STRATEGY, TIME-BASED MANUFACTURING\par
ID \tabto{0cm}COMPETITIVE ADVANTAGE, CONFIRMATORY FACTOR-ANALYSIS, FORTUNE FAVORS, INFORMATION-TECHNOLOGY, MANAGEMENT, ORGANIZATIONS, PERFORMANCE, PREPARED FIRM, SYSTEMS, UNITED-STATES\par
AB \tabto{0cm}Increasingly, manufacturers are making radical changes in management practices and investing heavily in advanced technologies as they attempt to achieve sustainable competitive advantage. Organizations seeking to implement these changes should have an internal environment that emphasizes knowledge assimilation and sharing and creates continuous learning capability, i.e., \hl{absorpti}ve \hl{capacit}y. This study reviews the construct of \hl{absorpti}ve \hl{capacit}y, develops a valid and reliable instrument to measure it, and examines its impact on the organization's ability to assimilate innovative manufacturing technology and management practice. To illustrate the links, this study tests the relationships among \hl{absorpti}ve \hl{capacit}y, time-based manufacturing practices, and value to customer. Structural equation modeling, applied to a relatively large sample (n = 303), indicates strong, positive, and direct relationships between \hl{absorpti}ve \hl{capacit}y and time-based manufacturing practices, and between time-based manufacturing practices and value to customer. The managerial implications of these empirical findings are also discussed. (c) 2005 Elsevier B.V. All rights reserved.\par
\clearpage

\vspace*{-2cm}
Nb \tabto{0cm}42/327 (article\_id: 369)\par
TI \tabto{0cm}Developing \hl{absorpti}ve \hl{capacit}y in mature organizations - The change agent's role\par
AU \tabto{0cm}Jones\par
PY \tabto{0cm}2006, SO MANAGEMENT LEARNING\par
DT \tabto{0cm}Article\par
PG \tabto{0cm}22, NR 81, TC 28\par
DE \tabto{0cm}BOUNDARY SPANNERS, CHANGE AGENTS, GATEKEEPERS, MIDDLE MANAGERS\par
ID \tabto{0cm}BOUNDARY, COMMUNICATION, DYNAMIC CAPABILITIES, ENVIRONMENT, FIRM, INNOVATION, KNOWLEDGE, PERFORMANCE, PRODUCT DEVELOPMENT, STRATEGIC MANAGEMENT\par
AB \tabto{0cm}A considerable amount of research into how organizations absorb new knowledge was prompted by the work of Cohen and Levinthal. In a recent literature review Zahra and George identify two distinct elements of \hl{absorpti}ve \hl{capacit}y (potential and realized). This article contributes to the study of managerial agency in the \hl{absorpti}on of new knowledge and skills. Zahra and George's model is extended to incorporate key roles associated with knowledge transfer, including gatekeepers, boundary spanners and change agents. Empirical data are drawn from a longitudinal study of a mature manufacturing firm based in North Wales. Change was initiated by the owner in response to the loss of the company's major customer-the Ministry of Defence. The main change agent was a recently recruited middle manager who used his mass production experience to improve managerial communications and introduce more efficient working practices to the shopfloor.\par
\clearpage

\vspace*{-2cm}
Nb \tabto{0cm}43/327 (article\_id: 370)\par
TI \tabto{0cm}\hl{Absorpti}ve \hl{capacit}y in high-technology markets: The competitive advantage of the haves\par
AU \tabto{0cm}Narasimhan, Rajiv, Dutta\par
PY \tabto{0cm}2006, SO MARKETING SCIENCE\par
DT \tabto{0cm}Article\par
PG \tabto{0cm}15, NR 57, TC 47\par
DE \tabto{0cm}\hl{ABSORPTI}VE \hl{CAPACIT}Y, DYNAMIC CAPABILITIES, HIGH-TECHNOLOGY MARKETS, KNOWLEDGE ACQUISITION, MARKETING CAPABILITY : ORGANIZATIONAL LEARNING, RESOURCE-BASED VIEW\par
ID \tabto{0cm}DYNAMIC CAPABILITIES, FIRM, GENERALIZED-METHOD, INFORMATION, INNOVATION, ORIENTATION, PANEL DATA, PATENT CITATIONS, RESEARCH-AND-DEVELOPMENT, RESOURCE-BASED VIEW\par
AB \tabto{0cm}he rapid rate of knowledge obsolescence in many high-technology markets makes it imperative for firms to renew their technological bases constantly. Given its critical importance, excellence in renewal of technological base would serve as a dynamic capability. Drawing on past literature, we identify this dynamic capability associated with acquiring and utilizing external technological know-how with the notion of \hl{absorpti}ve \hl{capacit}y (AC).
We ask the following questions: (a) What would cause some firms to have a higher AC than others? and, (b) What is the impact of AC on a firm's profitability?
We build a conceptual framework suggesting that marketing, R\&D, and operations capabilities have a significant positive impact on a firm's AC. We test our framework on a data set of firms in high-technology markets. Using an econometric technique called stochastic frontier estimation, we infer the AC of firms from an observation of the know-how they actually absorb. We find that firm-specific capabilities significantly impact AC. Also, we find that AC has a significant impact on profitability and that this impact is moderated by the pace of technological change: the greater the pace of change, the greater the impact.\par
\clearpage

\vspace*{-2cm}
Nb \tabto{0cm}44/327 (article\_id: 371)\par
TI \tabto{0cm}A framework for ex ante project risk assessment based on \hl{absorpti}ve \hl{capacit}y\par
AU \tabto{0cm}Cuellar, Gallivan\par
PY \tabto{0cm}2006, SO EUROPEAN JOURNAL OF OPERATIONAL RESEARCH\par
DT \tabto{0cm}Article\par
PG \tabto{0cm}16, NR 39, TC 8\par
DE \tabto{0cm}\hl{ABSORPTI}VE \hl{CAPACIT}Y, IS IMPLEMENTATION, IS ORGANIZATIONAL RISK, IS RISK ASSESSMENT\par
ID \tabto{0cm}COMBINATIVE CAPABILITIES, FIRM, INFORMATION TECHNOLOGY, INNOVATION, KNOWLEDGE, PERSPECTIVE, SYSTEMS\par
AB \tabto{0cm}This paper explores the applicability of the concepts of \hl{absorpti}ve \hl{capacit}y and "ba" to ex ante project risk. We develop a hybrid framework to explain knowledge transfer based on these concepts-one that proposes a hybrid transference process. We then apply this framework to develop a methodology and metric for assessing ex ante software project risk, the risk that a new technology introduced into an organization may not be used as designed or may not achieve the anticipated benefits. As a preliminary validation of these concepts, we describe three case studies, employing the framework and metric to show how examining \hl{absorpti}ve \hl{capacit}y can help to assess the risk level of software projects. (c) 2005 Elsevier B.V. All rights reserved.\par
\clearpage

\vspace*{-2cm}
Nb \tabto{0cm}45/327 (article\_id: 372)\par
TI \tabto{0cm}The reification of \hl{absorpti}ve \hl{capacit}y: A critical review and rejuvenation of the construct\par
AU \tabto{0cm}Lane, Koka, Pathak\par
PY \tabto{0cm}2006, SO ACADEMY OF MANAGEMENT REVIEW\par
DT \tabto{0cm}Review\par
PG \tabto{0cm}31, NR 120, TC 455\par
DE \tabto{0cm}null\par
ID \tabto{0cm}BIOTECHNOLOGY FIRMS, COMPETITIVE ADVANTAGE, FOREIGN SUBSIDIARIES, HIGH-TECHNOLOGY, INTERNATIONAL-JOINT-VENTURES, KNOWLEDGE TRANSFER, ORGANIZATIONAL KNOWLEDGE, RESEARCH-AND-DEVELOPMENT, RESOURCE-BASED VIEW, STRATEGIC ALLIANCES\par
AB \tabto{0cm}We conduct a detailed analysis of 289 \hl{absorpti}ve \hl{capacit}y papers from 14 journals to assess how the construct has been utilized, examine the key papers in the field, and identify the substantive contributions to the broader literature using a thematic analysis. We argue that research in this area is fundamentally driven by five critical assumptions that we conclude have led to its reification and that this reification has led to stifling of research in this area. To address this, we propose a model of \hl{absorpti}ve \hl{capacit}y processes, antecedents, and outcomes.\par
\clearpage

\vspace*{-2cm}
Nb \tabto{0cm}46/327 (article\_id: 373)\par
TI \tabto{0cm}\hl{Absorpti}ve \hl{capacit}y in European manufacturing: a Delphi study\par
AU \tabto{0cm}Jung-Erceg, Pandza, Armbruster, Dreher\par
PY \tabto{0cm}2007, SO INDUSTRIAL MANAGEMENT \& DATA SYSTEMS\par
DT \tabto{0cm}Article\par
PG \tabto{0cm}15, NR 55, TC 8\par
DE \tabto{0cm}DELPHI METHOD, EUROPE, KNOWLEDGE CAPTURE, MANUFACTURING INDUSTRIES\par
ID \tabto{0cm}DYNAMIC CAPABILITIES, ENVIRONMENTS, FIRM PERFORMANCE, INDUSTRY, INNOVATION, KNOWLEDGE MANAGEMENT, POLICY, PRODUCT DEVELOPMENT, RESOURCE-BASED VIEW, TECHNOLOGY\par
AB \tabto{0cm}Purpose - This paper sets out to discuss the results of a specific part of a Europe-wide Delphi study that considers issues of \hl{absorpti}ve \hl{capacit}y in European manufacturing. Owing to the importance to competitiveness of increasing innovative capabilities in manufacturing it is highly relevant to explore how a wide community of manufacturing experts experience the phenomenon of \hl{absorpti}ve \hl{capacit}y and sense future developments.
Design/methodology/approach - A two round Delphi method was designed in which more than 3,000 experts from 22 European countries assessed 101 statements. This paper discusses eight statements focused on the issue of \hl{absorpti}ve \hl{capacit}y.
Findings - The results show a general consensus about the influence of different inter-firm relationships in acquiring external knowledge and a diversified knowledge structure for assimilating the acquired knowledge. The study also indicates some potential challenges and contradictions in managing inter-firm relationships and knowledge diversity as well as perceived barriers for future developments of \hl{absorpti}ve \hl{capacit}y.
Research limitations/implications - Delphi survey is an empirical method subject to the limitation of testing or inductively building theoretical concepts.
Practical implications - The results of the Delphi study are predominantly centred on policy implication and on informing strategic decision-making at manufacturing firms.
Originality/value - This paper discusses one of the biggest Delphi surveys ever conducted in Europe. Its comprehensiveness increases the value of the results.\par
\clearpage

\vspace*{-2cm}
Nb \tabto{0cm}47/327 (article\_id: 374)\par
TI \tabto{0cm}Knowledge sharing, \hl{absorpti}ve \hl{capacit}y, and innovation capability: an empirical study of Taiwan's knowledge-intensive industries\par
AU \tabto{0cm}Liao, Fei, Chen\par
PY \tabto{0cm}2007, SO JOURNAL OF INFORMATION SCIENCE\par
DT \tabto{0cm}Article\par
PG \tabto{0cm}20, NR 50, TC 61\par
DE \tabto{0cm}\hl{ABSORPTI}VE \hl{CAPACIT}Y, INNOVATION CAPABILITY, KNOWLEDGE MANACIEMENT, KNOWLEDGE SHARING, LISREL\par
ID \tabto{0cm}FIRMS, IMPACT, INFORMATION, MANAGEMENT, ORGANIZATIONAL RESEARCH, PERFORMANCE, PROSPECTS\par
AB \tabto{0cm}This research investigates the relationships between knowledge sharing, \hl{absorpti}ve \hl{capacit}y, and innovation capability in Taiwan's knowledge-intensive industries. We propose statistical hypotheses and a LISREL model to study these based on the data sampled from 170 Taiwanese firms, including electronic, financial insurance and medical industries, yielding 355 valid returned research samples. By testing three hypotheses, this study finds that \hl{absorpti}ve \hl{capacit}y is the intervening factor between knowledge sharing and innovation capability. It also shows that knowledge sharing has a positive effect on \hl{absorpti}ve \hl{capacit}y, and that a completely mediating model exhibits both model generalization and extension characteristics through multiple model comparison in different industry population samples. Finally, managerial implications are discussed and a brief conclusion is presented.\par
\clearpage

\vspace*{-2cm}
Nb \tabto{0cm}48/327 (article\_id: 375)\par
TI \tabto{0cm}Effects of technology \hl{absorpti}ve \hl{capacit}y and technology proactivity on organizational learning, innovation and performance: An empirical examination\par
AU \tabto{0cm}Garcia-Morales, Ruiz-Moreno, Llorens-Montes\par
PY \tabto{0cm}2007, SO TECHNOLOGY ANALYSIS \& STRATEGIC MANAGEMENT\par
DT \tabto{0cm}Review\par
PG \tabto{0cm}32, NR 158, TC 42\par
DE \tabto{0cm}null\par
ID \tabto{0cm}COMPETITIVE ADVANTAGE, CONCEPTUAL-FRAMEWORK, FIRM PERFORMANCE, JOINT VENTURES, KNOWLEDGE MANAGEMENT, PRODUCT DEVELOPMENT, PSYCHOLOGICAL SAFETY, RESEARCH-AND-DEVELOPMENT, RESOURCE-BASED VIEW, STRATEGIC MANAGEMENT\par
AB \tabto{0cm}Technology is crucial for organizations in the knowledge society, but little empirical research has been conducted on technology \hl{absorpti}ve \hl{capacit}y and technology proactivity. Based on existing theoretical studies, this article formulates a global model to analyse how technology \hl{absorpti}ve \hl{capacit}y and technology proactivity influence organizational learning and organizational innovation, and how these dynamics capabilities affect organizational performance. The model also shows how organizational learning affects organizational innovation. The hypotheses are tested using data from 246 Spanish technological firms. The paper discusses the findings and provides several implications for future research. The findings are important for management practice, especially for firms where technology is the main strategic element.\par
\clearpage

\vspace*{-2cm}
Nb \tabto{0cm}49/327 (article\_id: 376)\par
TI \tabto{0cm}Perceived \hl{absorpti}ve \hl{capacit}y of individual users in performance of Enterprise Resource Planning (ERP) usage: The case for Korean firms\par
AU \tabto{0cm}Park, Suh, Yang\par
PY \tabto{0cm}2007, SO INFORMATION \& MANAGEMENT\par
DT \tabto{0cm}Article\par
PG \tabto{0cm}13, NR 67, TC 55\par
DE \tabto{0cm}ENTERPRISE RESOURCE PLANNING (ERP), KNOWLEDGE TRANSFER, ORGANIZATIONAL SUPPORT, USERS' \hl{ABSORPTI}VE \hl{CAPACIT}Y, USERS' PERFORMANCE OF ERP USAGE\par
ID \tabto{0cm}ASSIMILATION, CRITICAL SUCCESS FACTORS, IMPLEMENTATION, INFORMATION-TECHNOLOGY, INNOVATION, KNOWLEDGE TRANSFER, ORGANIZATIONS, PERSPECTIVE, SELF-EFFICACY, SYSTEMS\par
AB \tabto{0cm}We examined the effect of \hl{absorpti}ve \hl{capacit}y of users on their use of ERP in a Korean context. The three components considered were understanding, assimilating, and applying ERP knowledge. We found that the \hl{capacit}ies of users to assimilate and apply the knowledge had both direct and indirect effects on its value. The users' ability to understand ERP knowledge was found to influence its performance by their assimilating and applying the knowledge. We also found that organizational support moderated the relationship between their \hl{absorpti}ve \hl{capacit}y and performance. (C) 2007 Elsevier B.V. All rights reserved.\par
\clearpage

\vspace*{-2cm}
Nb \tabto{0cm}50/327 (article\_id: 377)\par
TI \tabto{0cm}\hl{Absorpti}ve \hl{capacit}y: Valuing a reconceptualization\par
AU \tabto{0cm}Todorova, Durisin\par
PY \tabto{0cm}2007, SO ACADEMY OF MANAGEMENT REVIEW\par
DT \tabto{0cm}Article\par
PG \tabto{0cm}13, NR 86, TC 286\par
DE \tabto{0cm}null\par
ID \tabto{0cm}COMPETITIVE ADVANTAGE, DYNAMIC CAPABILITIES, FIRM CAPABILITIES, INNOVATION, KNOWLEDGE, PERSPECTIVE, RELATIONAL VIEW, SOCIAL-STRUCTURE, STRATEGY, TECHNOLOGICAL DISCONTINUITIES\par
AB \tabto{0cm}Zahra and George (2002) suggested a reconceptualization of the \hl{absorpti}ve \hl{capacit}y construct in order to reduce ambiguity in empirical studies. A rereading of the seminal Cohen and Levinthal (1990) article in light of current research on learning and innovation directs our attention to serious ambiguities and omissions in Zahra and George's reconceptualization. We suggest a reintroduction of "recognizing the value," an alternative understanding of "transformation." a clarification of "potential \hl{absorpti}ve \hl{capacit}y," an elaboration of the impact of socialization mechanisms, an investigation of the role of "power relationships," and an inclusion of feedback loops in a dynamic model of \hl{absorpti}ve \hl{capacit}y.\par
\clearpage

\vspace*{-2cm}
Nb \tabto{0cm}51/327 (article\_id: 378)\par
TI \tabto{0cm}Cultural differences and capability transfer in cross-border acquisitions: the mediating roles of capability complementarity, \hl{absorpti}ve \hl{capacit}y, and social integration\par
AU \tabto{0cm}Bjorkman, Stahl, Vaara\par
PY \tabto{0cm}2007, SO JOURNAL OF INTERNATIONAL BUSINESS STUDIES\par
DT \tabto{0cm}Review\par
PG \tabto{0cm}15, NR 129, TC 101\par
DE \tabto{0cm}CAPABILITY TRANSFER, CROSS-BORDER ACQUISITIONS, CULTURAL DIFFERENCES\par
ID \tabto{0cm}DECISION-MAKING, GLOBAL STRATEGY, HORIZONTAL ACQUISITIONS, KNOWLEDGE TRANSFER, LEARNING PERSPECTIVE, MULTINATIONAL-CORPORATIONS, NATIONAL CULTURE, POSTACQUISITION INTEGRATION, STRATEGIC ALLIANCES, VALUE CREATION\par
AB \tabto{0cm}This paper presents an integrative model of the impact of cultural differences on capability transfer in cross-border acquisitions. We propose that cultural differences affect the post- acquisition capability transfer through their impact on social integration, potential \hl{absorpti}ve \hl{capacit}y, and capability complementarity. Two dynamic variables - the use of social integration mechanisms, and the degree of operational integration of the acquired unit - are proposed to moderate the effects of cultural differences on social integration and potential \hl{absorpti}ve \hl{capacit}y. The implications for acquisition research and practice are discussed.\par
\clearpage

\vspace*{-2cm}
Nb \tabto{0cm}52/327 (article\_id: 379)\par
TI \tabto{0cm}The impact of \hl{absorpti}ve \hl{capacit}y on technological acquisitions engineering consulting companies\par
AU \tabto{0cm}Haro-Dominguez, Arias-Aranda, Llorens-Montes, Moreno\par
PY \tabto{0cm}2007, SO TECHNOVATION\par
DT \tabto{0cm}Article\par
PG \tabto{0cm}9, NR 51, TC 12\par
DE \tabto{0cm}\hl{ABSORPTI}VE \hl{CAPACIT}Y, INNOVATION, TECHNOLOGICAL ACQUISITIONS\par
ID \tabto{0cm}COMPETITIVE ADVANTAGE, FIRM, INNOVATION, JOINT VENTURES, KNOWLEDGE, PARTNERSHIPS, PERFORMANCE, PERSPECTIVE, RESEARCH-AND-DEVELOPMENT, STRATEGIC ALLIANCES\par
AB \tabto{0cm}In this study, the influence of \hl{absorpti}ve \hl{capacit}y on the decisions about technological acquisitions is analyzed on the basis of the impact on the firm performance. Such relationships are studied over a sample of 250 Spanish engineering consulting firms. The results obtained show that the degree of \hl{absorpti}ve \hl{capacit}y influences positively both external and internal acquisition types of technology. Both technological decisions types influence organizational performance significantly and positive even though evidence shows a significant and negative relation between these decisions. (c) 2007 Elsevier Ltd. All rights reserved.\par
\clearpage

\vspace*{-2cm}
Nb \tabto{0cm}53/327 (article\_id: 380)\par
TI \tabto{0cm}Optimal cognitive distance and \hl{absorpti}ve \hl{capacit}y\par
AU \tabto{0cm}Nooteboom, Van Haverbeke, Duysters, Gilsing, van den Oord\par
PY \tabto{0cm}2007, SO RESEARCH POLICY\par
DT \tabto{0cm}Article\par
PG \tabto{0cm}19, NR 84, TC 237\par
DE \tabto{0cm}\hl{ABSORPTI}VE \hl{CAPACIT}Y, COGNITIVE DISTANCE, EXPLORATION AND EXPLOITATION, INNOVATION, INTER-FINN ALLIANCES\par
ID \tabto{0cm}COUNT DATA, INNOVATION, INTERFIRM NETWORKS, JOINT VENTURES, LOCAL SEARCH, MARKET-STRUCTURE, RESEARCH-AND-DEVELOPMENT, RESOURCE-BASED VIEW, SOCIAL-STRUCTURE, STRATEGIC ALLIANCES\par
AB \tabto{0cm}In this paper we test the relation between cognitive distance and innovation performance of firms engaged in technology-based alliances. The key finding is that the hypothesis of an inverted U-shaped effect of cognitive distance on innovation performance of firms is confirmed. Moreover, as expected, we found that the positive effect for firms is much higher when engaging in more radical, exploratory alliances than in more exploitative alliances. The effect of cumulative R\&D turns out to be mixed. It appears to increase \hl{absorpti}ve \hl{capacit}y, as expected, but there is clear evidence that it also reduces the effect of cognitive distance on novelty value, making it increasingly difficult to find additional novelty. (C) 2007 Elsevier B.V. All rights reserved.\par
\clearpage

\vspace*{-2cm}
Nb \tabto{0cm}54/327 (article\_id: 381)\par
TI \tabto{0cm}\hl{Absorpti}ve \hl{capacit}y, R\&D spillovers, and public policy\par
AU \tabto{0cm}Leahy, Neary\par
PY \tabto{0cm}2007, SO INTERNATIONAL JOURNAL OF INDUSTRIAL ORGANIZATION\par
DT \tabto{0cm}Article\par
PG \tabto{0cm}20, NR 35, TC 31\par
DE \tabto{0cm}\hl{ABSORPTI}VE \hl{CAPACIT}Y OF R\&D, COMPETITION POLICY, INDUSTRIAL POLICY, R\&D SPILLOVERS, RESEARCH JOINT VENTURES\par
ID \tabto{0cm}2 FACES, INDUSTRIES, INNOVATION, ME HALFWAY, RESEARCH JOINT VENTURES, TECHNOLOGY\par
AB \tabto{0cm}Empirical evidence strongly suggests that R\&D increases a firm's "\hl{absorpti}ve \hl{capacit}y" (its ability to absorb spillovers from other firms) as well as contributing directly to profitability. We explore the theoretical implications of this. We specify a general model of the \hl{absorpti}ve \hl{capacit}y process and show that costly \hl{absorpti}on both raises the effectiveness of own R\&D and lowers the effective spillover coefficient. This weakens the case for encouraging research joint ventures, even if there is complete information sharing between members It also implies an additional strategic pay-off to policies that raise the level of extra-industry knowledge. (C) 2007 Elsevier B.V. All rights reserved.\par
\clearpage

\vspace*{-2cm}
Nb \tabto{0cm}55/327 (article\_id: 382)\par
TI \tabto{0cm}Relative backwardness, \hl{absorpti}ve \hl{capacit}y and knowledge spillovers\par
AU \tabto{0cm}Falvey, Foster, Greenaway\par
PY \tabto{0cm}2007, SO ECONOMICS LETTERS\par
DT \tabto{0cm}Article\par
PG \tabto{0cm}5, NR 9, TC 9\par
DE \tabto{0cm}ECONOMIC GROWTH, KNOWLEDGE SPILLOVERS\par
ID \tabto{0cm}null\par
AB \tabto{0cm}Panel data are used to investigate North-South trade-related knowledge spillovers. We find that \hl{absorpti}ve \hl{capacit}y increases the benefits of knowledge spillovers, and that spillovers have least impact in countries closest to and farthest from the technological frontier. (c) 2007 Elsevier B.V. All rights reserved.\par
\clearpage

\vspace*{-2cm}
Nb \tabto{0cm}56/327 (article\_id: 383)\par
TI \tabto{0cm}Innovation activities, use of appropriation instruments and \hl{absorpti}ve \hl{capacit}y: Evidence from Spanish firms\par
AU \tabto{0cm}Arbussa, Coenders\par
PY \tabto{0cm}2007, SO RESEARCH POLICY\par
DT \tabto{0cm}Article\par
PG \tabto{0cm}14, NR 31, TC 44\par
DE \tabto{0cm}\hl{ABSORPTI}VE \hl{CAPACIT}Y, APPROPRIATION INSTRUMENTS, INNOVATION, MULTILEVEL LOGIT MODELS\par
ID \tabto{0cm}COOPERATION, EXPENDITURES, KNOWLEDGE, MODELS, R-AND-D, SERVICES, SPILLOVERS, TECHNOLOGICAL-INNOVATION\par
AB \tabto{0cm}This article empirically investigates the relationship between innovation activities of firms, their use of appropriation instruments and their \hl{absorpti}ve \hl{capacit}y. We study a wide range of manufacturing and service industries, not just high-tech, and a wide range of innovation activities, not just R\&D. We use multilevel logit models for complex samples to disentangle industry from firm-specific effects. We find that within an industry, firms that invest in appropriation instruments to reduce outgoing spillovers tend to conduct more R\&D and downstream activities than firms that do not. Acquisition of technology is not related to the use of appropriation instruments. The effects of incoming spillovers (measured through \hl{absorpti}ve \hl{capacit}y) on innovation activities of firms are industry specific and stronger for firms that invest in appropriation instruments. For this type of firm, both the capability to scan the external environment for technology and the capability to integrate new technology are related to the innovation activities. For firms that do not invest in appropriation instruments, only scanning capabilities are related. (C) 2007 Elsevier B.V. All rights reserved.\par
\clearpage

\vspace*{-2cm}
Nb \tabto{0cm}57/327 (article\_id: 384)\par
TI \tabto{0cm}Multinational Enterprises, Technology Diffusion, and Host Country \hl{Absorpti}ve \hl{Capacit}y: A Note\par
AU \tabto{0cm}Elmawazini, Manga, Saadi\par
PY \tabto{0cm}2008, SO GLOBAL ECONOMIC REVIEW\par
DT \tabto{0cm}Article\par
PG \tabto{0cm}8, NR 27, TC 3\par
DE \tabto{0cm}\hl{ABSORPTI}VE \hl{CAPACIT}Y, HUMAN CAPITAL, MULTINATIONAL ENTERPRISES, PRODUCTIVITY GROWTH, TECHNOLOGY DIFFUSION\par
ID \tabto{0cm}BENEFIT, DOMESTIC FIRMS, ECONOMETRICS, FOREIGN DIRECT-INVESTMENT, GROWTH, MODEL, PANEL-DATA, PRODUCTIVITY, SPECIFICATION, SPILLOVERS\par
AB \tabto{0cm}Previous empirical studies show mixed support for the hypothesis that the impact of technology diffusion from multinational enterprises (MNEs) on host country productivity growth depends on host country \hl{absorpti}ve \hl{capacit}y. One explanation is that the results of these empirical studies are sensitive to the measures of \hl{absorpti}ve \hl{capacit}y used. This paper contributes to the empirical literature by investigating average years of schooling and total factor productivity gap as measures of host country \hl{absorpti}ve \hl{capacit}y in 38 developed and developing countries. Panel data regression equations are estimated using a cross-sectionally heteroskedastic and timewise autoregressive (CHTA) model. The paper has two main results. The first result does not support the hypothesis that the technology diffusion from MNEs has a positive impact on the productivity growth in developing countries. The second result is that the total factor productivity gap is more appropriate than average years of schooling to measure host country \hl{absorpti}ve \hl{capacit}y. This may suggest that the results of previous studies that used average years of schooling should be interpreted with caution.\par
\clearpage

\vspace*{-2cm}
Nb \tabto{0cm}58/327 (article\_id: 385)\par
TI \tabto{0cm}Utilizing and adaptation of the \hl{absorpti}ve \hl{capacit}y concept in a virtual enterprise context\par
AU \tabto{0cm}Beckett\par
PY \tabto{0cm}2008, SO INTERNATIONAL JOURNAL OF PRODUCTION RESEARCH\par
DT \tabto{0cm}Article; Proceedings Paper\par
PG \tabto{0cm}10, NR 18, TC 4\par
DE \tabto{0cm}\hl{ABSORPTI}VE \hl{CAPACIT}Y, ICT TOOLS, VIRTUAL ENTERPRISE\par
ID \tabto{0cm}INNOVATION, KNOWLEDGE TRANSFER\par
AB \tabto{0cm}The author is involved in a multi-year programme to establish a number of large-scale manufacturing SME collaboration projects. It was anticipated that some web-based tools used previously by larger firms could be adapted, but limits in the capabilities of the participant firms and the emergence of a different business model led to a change in approach. Participants had quite diverse internal ICT capabilities that might be used for bidding collaboratively for manufacturing contracts, or for operating as an extended enterprise. Some way of mapping the functional project needs and comparing these with participant capabilities was needed. In this paper an adaptation of the notion of \hl{absorpti}ve \hl{capacit}y where both a firm's resource base and its knowledge base are considered is used as a framework to better understand participant requirements. This was subsequently used in building a web-based virtual enterprise support system that utilized simple, lowest cost system components.\par
\clearpage

\vspace*{-2cm}
Nb \tabto{0cm}59/327 (article\_id: 386)\par
TI \tabto{0cm}Resources, \hl{absorpti}ve \hl{capacit}y, and technology sourcing\par
AU \tabto{0cm}Ouyang\par
PY \tabto{0cm}2008, SO INTERNATIONAL JOURNAL OF TECHNOLOGY MANAGEMENT\par
DT \tabto{0cm}Article\par
PG \tabto{0cm}20, NR 59, TC 5\par
DE \tabto{0cm}\hl{ABSORPTI}VE \hl{CAPACIT}Y, CHINA, KNOWLEDGE TRANSFER, LEARNING, RESOURCES, TECHNOLOGY SOURCING\par
ID \tabto{0cm}CAPABILITIES, COMPETENCE, COMPETITIVE ADVANTAGE, FIRM, JOINT-VENTURES, KNOWLEDGE, MODEL, PERFORMANCE, SCOPE, STRATEGIC ALLIANCES\par
AB \tabto{0cm}This paper explores the determinants of firms' external technology sourcing mode. Instead of focusing on opportunistic behaviour in knowledge transactions and assuming minimising transaction costs as a firms' objective, I focus on firms' \hl{absorpti}ve \hl{capacit}y and emphasise gains from knowledge transfer in technology sourcing. I analyse bow a firm matches a sourcing mode with its \hl{absorpti}ve \hl{capacit}y to enhance knowledge transfer in technology sourcing, and I find positive relations between a firms' \hl{absorpti}ve \hl{capacit}y and interactive technology sourcing modes.\par
\clearpage

\vspace*{-2cm}
Nb \tabto{0cm}60/327 (article\_id: 387)\par
TI \tabto{0cm}How does FDI inflow affect productivity of domestic firms? The role of horizontal and vertical spillovers, \hl{absorpti}ve \hl{capacit}y and competition\par
AU \tabto{0cm}Marcin\par
PY \tabto{0cm}2008, SO JOURNAL OF INTERNATIONAL TRADE \& ECONOMIC DEVELOPMENT\par
DT \tabto{0cm}Article\par
PG \tabto{0cm}19, NR 47, TC 10\par
DE \tabto{0cm}\hl{ABSORPTI}VE \hl{CAPACIT}Y, COMPETITION, FDI, PANEL DATA, PRODUCTIVITY, SPILLOVERS\par
ID \tabto{0cm}ENTERPRISES, FOREIGN DIRECT-INVESTMENT, GAP, MATTER, MODEL, OWNERSHIP, PANEL-DATA, TECHNOLOGY-TRANSFER, TRADE\par
AB \tabto{0cm}This paper examines the existence of externalities associated with foreign direct investment (FDI) in a host country by exploiting firm-level panel data covering the Polish corporate sector. We distinguish between horizontal spillovers (from foreign to domestic firms operating in the same industry) and two types of vertical spillovers: backward (from FDI in downstream industries) and forward spillovers (from FDI in upstream industries). The main findings are as follows. Local firms benefit from foreign presence in the same industry and in downstream industries. The \hl{absorpti}ve \hl{capacit}y of domestic firms is highly relevant to the size of spillovers: vertical spillovers are larger for RD-intensive firms, while firms investing in other (external) types of intangibles benefit more from horizontal spillovers. Competitive pressure facilitates backward spillovers, while market power increases the extent of forward spillovers. Horizontal spillovers are particularly strong in services, while the remaining results, including backward spillovers and the role of \hl{absorpti}ve \hl{capacit}y and competition, are mainly driven by manufacturing. Host country equity participation in foreign firms is consistent with higher unconditional productivity spillovers to domestic firms. A number of robustness checks yield results qualitatively similar to those obtained in the baseline specification.\par
\clearpage

\vspace*{-2cm}
Nb \tabto{0cm}61/327 (article\_id: 388)\par
TI \tabto{0cm}A note on export-platform Foreign Direct Investment, training and \hl{absorpti}ve \hl{capacit}y\par
AU \tabto{0cm}Chatterji, Montagna\par
PY \tabto{0cm}2008, SO JOURNAL OF INTERNATIONAL TRADE \& ECONOMIC DEVELOPMENT\par
DT \tabto{0cm}Article\par
PG \tabto{0cm}10, NR 18, TC 1\par
DE \tabto{0cm}\hl{ABSORPTI}VE \hl{CAPACIT}Y, FOREIGN DIRECT INVESTMENT, SOPHISTICATED TECHNOLOGY, TRAINING\par
ID \tabto{0cm}COUNTRIES\par
AB \tabto{0cm}The empirical literature on FDI suggests that investment in training is the major source of human resource development activities undertaken by MNEs, particularly those with sophisticated technologies, and host countries' \hl{absorpti}ve \hl{capacit}y plays an important role in attracting FDI. We develop a model of export-platform FDI that provides theoretical rationalisation of the role played by a host country's \hl{absorpti}ve \hl{capacit}y in determining MNEs' location decisions as well as their level of investment and training-and, through this, the extent to which they contribute to human capital formation in the host country.\par
\clearpage

\vspace*{-2cm}
Nb \tabto{0cm}62/327 (article\_id: 389)\par
TI \tabto{0cm}Gatekeepers of knowledge versus platforms of knowledge: From potential to realized \hl{absorpti}ve \hl{capacit}y\par
AU \tabto{0cm}Lazaric, Longhi, Thomas\par
PY \tabto{0cm}2008, SO REGIONAL STUDIES\par
DT \tabto{0cm}Article\par
PG \tabto{0cm}16, NR 54, TC 33\par
DE \tabto{0cm}GATEKEEPER, HIGH-TECHNOLOGY CLUSTERS, KNOWLEDGE, PLATFORM OF KNOWLEDGE, SOPHIA ANTIPOLIS\par
ID \tabto{0cm}CLUSTER, FIRM, INDUSTRIAL DISTRICTS, INNOVATION, KNOW-HOW, NETWORK, PROXIMITY, RULES, SPILLOVERS, TECHNOLOGY\par
AB \tabto{0cm}The development of clusters rests on geographical proximity, cognitive interactions as well as entrepreneurial initiatives. Sophia Antipolis, a multi-technology cluster in Valbonne, France, is a good illustration of the type of challenges local systems of innovation face in creating positive knowledge externalities. This paper shows that if the existence of 'gatekeepers of knowledge' can generate the potential implementation of '\hl{absorpti}ve \hl{capacit}y', its effective realization requires some additional effort regarding the transfer of knowledge into the cluster. The concept of 'platform of knowledge' defined shows how a project of knowledge codification could generate externalities by creating new opportunities for effectively combining and absorbing knowledge.\par
\clearpage

\vspace*{-2cm}
Nb \tabto{0cm}63/327 (article\_id: 390)\par
TI \tabto{0cm}Spatial mobility of knowledge transfer and \hl{absorpti}ve \hl{capacit}y: analysis and measurement of the impact within the geoeconomic space\par
AU \tabto{0cm}Coccia\par
PY \tabto{0cm}2008, SO JOURNAL OF TECHNOLOGY TRANSFER\par
DT \tabto{0cm}Article\par
PG \tabto{0cm}18, NR 60, TC 10\par
DE \tabto{0cm}\hl{ABSORPTI}VE \hl{CAPACIT}Y, ECONOMIC IMPACT, KNOWLEDGE SPILLOVER, LEARNING PROCESS, PATTERNS OF TECHNOLOGICAL INNOVATION, RESEARCH INSTITUTES, TECHNOLOGY TRANSFER\par
ID \tabto{0cm}ACADEMIC RESEARCH, DEVELOPMENT SPILLOVERS, DIFFUSION, GEOGRAPHY, INNOVATION, LOCALIZATION, ORGANIZATION, PROXIMITY, RESEARCH-AND-DEVELOPMENT, TECHNOLOGY-TRANSFER\par
AB \tabto{0cm}This paper analyses the spatial mobility of knowledge and technology transfer and measures the economic impact on the geo-economic space. The data of laboratories operating in different research and technological fields are used. The results show that, when the distance from the source of knowledge (research institute) to users increases, the impact of knowledge and technology transfer decreases with damped pulsations. The magnitude of knowledge and technology transfer shows a high intensity within the industrial district because small businesses are able to acquire externally scientific knowledge, without conducting in-house research, but by interactions with public scientific bodies and adopting both collective rules that act as collective knowledge devices, making collective learning possible, and skilled labor.\par
\clearpage

\vspace*{-2cm}
Nb \tabto{0cm}64/327 (article\_id: 391)\par
TI \tabto{0cm}Exploring the \hl{absorpti}ve \hl{capacit}y to innovation/productivity link for individual engineers engaged in IT enabled work\par
AU \tabto{0cm}Deng, Doll, Cao\par
PY \tabto{0cm}2008, SO INFORMATION \& MANAGEMENT\par
DT \tabto{0cm}Article\par
PG \tabto{0cm}13, NR 62, TC 20\par
DE \tabto{0cm}\hl{ABSORPTI}VE \hl{CAPACIT}Y, ENGINEERING WORK, INTUITIVE PROBLEM SOLVING, IT ENABLED WORK, SYSTEMATIC PROBLEM SOLVING, TASK INNOVATION, TASK PRODUCTIVITY\par
ID \tabto{0cm}FIRM, IMPACT, INDUSTRY, INFORMATION-TECHNOLOGY, INNOVATION, KNOWLEDGE TRANSFER, ORGANIZATIONS, PERFORMANCE, PERSPECTIVE, RESEARCH-AND-DEVELOPMENT\par
AB \tabto{0cm}The hypothesis that \hl{absorpti}ve \hl{capacit}y leads to greater innovation/productivity has been supported at the country, inter-organizational, organizational, and group levels. We adapted the \hl{absorpti}ve \hl{capacit}y concept to individuals engaged in IT enabled engineering work, which is a situated and emergent phenomenon that requires individuals to posses or develop ability to acquire new task and computer knowledge; use or develop analytical and intuitive problem solving skills to assimilate and integrate these two types of knowledge; and apply them to their work.
A model was developed linking the \hl{absorpti}ve \hl{capacit}y of individuals, through enhanced IT utilization for problem solving/decision support, to task innovation and productivity. It was tested using a sample of 208 engineers using computers in their work. The results suggested that using IT innovatively and productively in such a work environment requires a mix of task knowledge, computer knowledge, and problem solving modalities. (C) 2007 Elsevier B.V. All rights reserved.\par
\clearpage

\vspace*{-2cm}
Nb \tabto{0cm}65/327 (article\_id: 392)\par
TI \tabto{0cm}The effect of international venturing on firm performance: The moderating influence of \hl{absorpti}ve \hl{capacit}y\par
AU \tabto{0cm}Zahra, Hayton\par
PY \tabto{0cm}2008, SO JOURNAL OF BUSINESS VENTURING\par
DT \tabto{0cm}Article\par
PG \tabto{0cm}26, NR 91, TC 68\par
DE \tabto{0cm}\hl{ABSORPTI}VE \hl{CAPACIT}Y, ACQUISITIONS, ALLIANCES, CORPORATE VENTURE CAPITAL, INTERNATIONAL VENTURING\par
ID \tabto{0cm}ACQUISITIONS, COMPETENCE, CORPORATE ENTREPRENEURSHIP, ECONOMY, EXPANSION, INNOVATION, JOINT-VENTURES, KNOWLEDGE, PERSPECTIVE, STRATEGIES\par
AB \tabto{0cm}Companies have vigorously pursued opportunities for profitability and growth through international venturing. Yet, research evidence on the performance benefits of international venturing activities has been contradictory. Applying an organizational learning framework, we propose that the expected effects of international venturing activities on financial performance depend on companies' \hl{absorpti}ve \hl{capacit}y. Data from 217 global manufacturing companies show that \hl{absorpti}ve \hl{capacit}y moderates the relationship between international venturing and firms' profitability and revenue growth. These results urge executives to build internal R\&D and innovative capabilities in order to successfully exploit the new knowledge acquired from foreign markets. (c) 2007 Elsevier Inc. All rights reserved.\par
\clearpage

\vspace*{-2cm}
Nb \tabto{0cm}66/327 (article\_id: 393)\par
TI \tabto{0cm}Exploring the antecedents of potential \hl{absorpti}ve \hl{capacit}y and its impact on innovation performance\par
AU \tabto{0cm}Fosfuri, Tribo\par
PY \tabto{0cm}2008, SO OMEGA-INTERNATIONAL JOURNAL OF MANAGEMENT SCIENCE\par
DT \tabto{0cm}Article\par
PG \tabto{0cm}15, NR 44, TC 111\par
DE \tabto{0cm}\hl{ABSORPTI}VE \hl{CAPACIT}Y, INNOVATION, KNOWLEDGE MANAGEMENT\par
ID \tabto{0cm}BIOTECHNOLOGY, DYNAMIC CAPABILITIES, EMPIRICAL-ANALYSIS, EXPLORATION, FIRM, INDUSTRY, KNOWLEDGE, RESEARCH-AND-DEVELOPMENT, SEARCH, SPILLOVERS\par
AB \tabto{0cm}This paper builds upon the theoretical framework developed by Zahra and George [\hl{Absorpti}ve \hl{capacit}y: a review, reconceptualization, and extension. Academy of Management Review 2002;27:185-203] to empirically explore the antecedents of potential \hl{absorpti}ve \hl{capacit}y (PAC), i.e. the ability to identify and assimilate external knowledge flows. Based on a sample of 2464 innovative Spanish firms, we find evidence that R\&D cooperation, external knowledge acquisition and experience with knowledge search are key antecedents of a firm's PAC. Also, during periods of important internal reshaping, when there are significant changes in strategy, design of the organization and marketing, firms exert more effort to accumulate PAC. Finally, we find that PAC is a source of competitive advantage in innovation, especially in the presence of efficient internal knowledge flows that help reduce the distance between potential and realized \hl{capacit}y. (c) 2006 Elsevier Ltd. All rights reserved.\par
\clearpage

\vspace*{-2cm}
Nb \tabto{0cm}67/327 (article\_id: 394)\par
TI \tabto{0cm}Learning how to restructure: \hl{Absorpti}ve \hl{capacit}y and improvisational views of restructuring actions and performance\par
AU \tabto{0cm}Bergh, Lim\par
PY \tabto{0cm}2008, SO STRATEGIC MANAGEMENT JOURNAL\par
DT \tabto{0cm}Review\par
PG \tabto{0cm}24, NR 133, TC 21\par
DE \tabto{0cm}\hl{ABSORPTI}VE \hl{CAPACIT}Y, CORPORATE RESTRUCTURING, DIVESTITURE, ORGANIZATIONAL IMPROVISATION, ORGANIZATIONAL LEARNING, SPIN-OFF\par
ID \tabto{0cm}ACQUISITION EXPERIENCE, ASSET SALES, COMMERCIAL BANKING INDUSTRY, ECONOMIC-PERFORMANCE, FIRM PERFORMANCE, FOREIGN DIRECT-INVESTMENT, LONGITUDINAL ANALYSIS, ORGANIZATIONAL IMPROVISATION, SHAREHOLDER WEALTH, SPIN-OFF\par
AB \tabto{0cm}This paper examines the role of teaming in corporate restructuring. Drawing from two viewpoints of organizational learning, \hl{absorpti}ve \hl{capacit}y and organizational improvisation, we examine whether experience with corporate restructuring modes (sell-offs, spin-ofts) influences subsequent restructuring and financial performance. Consistent with an \hl{absorpti}ve \hl{capacit}y view, cumulative and repetitive experience with sell-offs was related to the adoption of an ensuing sell-off and to higher performance. Conversely, and consistent with an organizational improvisation view, short-term and contemporaneous experience with spin-ofts was related to the subsequent use of spin-offs and to increases in financial performance. The findings contribute to a dynamic explanation of corporate restructuring and its influence on financial performance, illustrate differences between learning in a repetitive situation and teaming when repetition is rare, and indicate when \hl{absorpti}ve \hl{capacit}y and organizational improvisational views are most profitable. Overall, these findings show that different kinds of restructuring experiences were associated with different modes of restructuring and performance records. Considered collectively, the organizational learning perspective offers insights into why some corporate restructuring strategies appear as intentional and deliberate actions while others resemble more spontaneous and simultaneous responses. Copyright (C) 2008 John Wiley \& Sons, Ltd.\par
\clearpage

\vspace*{-2cm}
Nb \tabto{0cm}68/327 (article\_id: 395)\par
TI \tabto{0cm}Managerial ties, \hl{absorpti}ve \hl{capacit}y, and innovation\par
AU \tabto{0cm}Gao, Xu, Yang\par
PY \tabto{0cm}2008, SO ASIA PACIFIC JOURNAL OF MANAGEMENT\par
DT \tabto{0cm}Article\par
PG \tabto{0cm}18, NR 75, TC 47\par
DE \tabto{0cm}\hl{ABSORPTI}VE \hl{CAPACIT}Y, COMMUNITY, INNOVATION, MANAGERIAL TIES\par
ID \tabto{0cm}COMPETITION, CONSEQUENCES, EMERGING ECONOMIES, FIRM PERFORMANCE, NETWORKS, ORGANIZATIONS, PERSPECTIVE, PRODUCT DEVELOPMENT, RESEARCH-AND-DEVELOPMENT, SPILLOVERS\par
AB \tabto{0cm}Managerial ties-the boundary-spanning ties and interpersonal connections of top managers-contribute to a corporation's innovativeness in emerging economies because of the absence of market supporting institutions, transparent laws, and clear regulations. Moreover, managerial ties are apt to interact with \hl{absorpti}ve \hl{capacit}y to facilitate knowledge sharing and innovation. This paper examines the joint influence of managerial ties and \hl{absorpti}ve \hl{capacit}y in two communities in China, one characterized by a high level of foreign direct investment (FDI) and the other consisting mainly of local corporations. We find that \hl{absorpti}ve \hl{capacit}y moderates the effect of managerial ties on a corporation's innovativeness. Furthermore, when examining the two communities separately, we find that business ties and university ties have opposite effects.\par
\clearpage

\vspace*{-2cm}
Nb \tabto{0cm}69/327 (article\_id: 396)\par
TI \tabto{0cm}\hl{Absorpti}ve \hl{capacit}y and source-recipient complementarity in designing new products: An empirically derived framework\par
AU \tabto{0cm}Abecassis-Moedas, Ben Mahmoud-Jouini\par
PY \tabto{0cm}2008, SO JOURNAL OF PRODUCT INNOVATION MANAGEMENT\par
DT \tabto{0cm}Article; Proceedings Paper\par
PG \tabto{0cm}18, NR 49, TC 25\par
DE \tabto{0cm}null\par
ID \tabto{0cm}ALLIANCES, CAPABILITIES, CHAIN, COORDINATION, CORE NECESSITIES, FIRMS, INDUSTRIAL, INNOVATION, PERFORMANCE, PERSPECTIVE\par
AB \tabto{0cm}Numerous publications are dedicated to \hl{absorpti}ve \hl{capacit}y and new product development (NPD). Most are centered on the recipient team, and very few consider the effects of the source team knowledge characteristics on the knowledge \hl{absorpti}on and the NPD performance. This paper analyzes the type of the external knowledge sourced from outside the organization and the process through which it is used by the recipient firm and the effect on NPD performance. This is done through a specific type of source team knowledge, the design, and through the NPD process in industries (clothing and construction) where it plays a key role. NPD cases were analyzed and clustered in three categories of design \hl{absorpti}on processes. From these categories, a conceptual framework of the source-recipient knowledge complementarity and its impact on the NPD performance is proposed. The main result is that the complementarity between the recipient and the source knowledge is a critical aspect of the \hl{absorpti}on process and therefore of the NPD performance. From a managerial perspective, this research highlights the role of design in the NPD process and how the combination of design knowledge with prior knowledge (marketing or technological) is related to NPD performance.\par
\clearpage

\vspace*{-2cm}
Nb \tabto{0cm}70/327 (article\_id: 397)\par
TI \tabto{0cm}Analyzing the determinants of firm's \hl{absorpti}ve \hl{capacit}y: beyond R\&D\par
AU \tabto{0cm}Vega-Jurado, Gutierrez-Gracia, Fernandez-De-Lucio\par
PY \tabto{0cm}2008, SO R \& D MANAGEMENT\par
DT \tabto{0cm}Article; Proceedings Paper\par
PG \tabto{0cm}14, NR 31, TC 34\par
DE \tabto{0cm}null\par
ID \tabto{0cm}ANTECEDENTS, CAPABILITIES, INNOVATION PERFORMANCE, KNOWLEDGE TRANSFER, SEARCH, VENTURES\par
AB \tabto{0cm}This article proposes a new model for analyzing the determinants of \hl{absorpti}ve \hl{capacit}y in companies. We suggest that \hl{absorpti}ve \hl{capacit}y is determined not only by research and development activities, but also by a set of internal factors, which we group into three basic categories: organizational knowledge, formalization, and social integration mechanisms. In addition, we suggest that these factors may influence all components of the firm's \hl{absorpti}ve \hl{capacit}y, and that the influence can be positive or negative depending on the applicability of the knowledge being absorbed. This paper thus advances the understanding of \hl{absorpti}ve \hl{capacit}y by exploring a largely ignored aspect in the literature: the role of knowledge attributes. We show how the model can be operationalized and empirically tested and provide preliminary evidence supporting most of the propositions in the analytical model.\par
\clearpage

\vspace*{-2cm}
Nb \tabto{0cm}71/327 (article\_id: 398)\par
TI \tabto{0cm}The role of intermediation and \hl{absorpti}ve \hl{capacit}y in facilitating university-industry linkages - An empirical study of TAMA in Japan\par
AU \tabto{0cm}Kodama\par
PY \tabto{0cm}2008, SO RESEARCH POLICY\par
DT \tabto{0cm}Article; Proceedings Paper\par
PG \tabto{0cm}17, NR 14, TC 32\par
DE \tabto{0cm}\hl{ABSORPTI}VE \hl{CAPACIT}Y, INDUSTRIAL CLUSTER, INTERMEDIARY, SME, UNIVERSITY-INDUSTRY LINKAGE\par
ID \tabto{0cm}INNOVATION, KNOWLEDGE\par
AB \tabto{0cm}This paper analyzes two elements necessary for building an efficient regional technology-transfer system between universities and firms, namely, an intermediary organization and regional firms that have a developed '\hl{absorpti}ve \hl{capacit}y', touching in particular upon the tacit knowledge aspects. Based on an empirical study of the TAMA cluster project (in the western part of the Tokyo Metropolitan Area), which is a model project of the 'Industrial Cluster Plan' in Japan, we examine the intermediation effect of the TAMA Association and the '\hl{absorpti}ve \hl{capacit}y' of various product-developing SMEs. These two elements are interrelated because the participation of the product-developing SMEs is a prerequisite for the effective functioning of an intermediary such as the TAMA Association. Our analysis also shows that university-industry linkages and inter-firm linkages lead to different outcomes. (C) 2008 Elsevier B.V. All rights reserved.\par
\clearpage

\vspace*{-2cm}
Nb \tabto{0cm}72/327 (article\_id: 399)\par
TI \tabto{0cm}Combining mapping and citation network analysis for a better understanding of the scientific development: The case of the \hl{absorpti}ve \hl{capacit}y field\par
AU \tabto{0cm}Calero-Medina, Noyons\par
PY \tabto{0cm}2008, SO JOURNAL OF INFORMETRICS\par
DT \tabto{0cm}Article\par
PG \tabto{0cm}8, NR 19, TC 19\par
DE \tabto{0cm}BIBLIOMETRIC MAPPING, CITATION NETWORK ANALYSIS, HUBS AND AUTHORITIES, MAIN PATH ANALYSIS, MAIN RESEARCH STREAM\par
ID \tabto{0cm}INNOVATION, KNOWLEDGE, TRAJECTORIES\par
AB \tabto{0cm}The general aim of this paper is to show the results of a study in which we combined bibliometric mapping and citation network analysis to investigate the process of creation and transfer of knowledge through scientific publications. The novelty of this approach is the combination of both methods. In this case we analyzed the citations to a very influential paper published in 1990 that contains, for the first time, the term \hl{Absorpti}ve \hl{Capacit}y. A bibliometric map identified the terms and the theories associated with the term while two techniques from the citation network analysis recognized the main papers during 15 years. As a result we identified the articles that influenced the research for some time and linked them into a research tradition that can be considered the backbone of the "\hl{Absorpti}ve \hl{Capacit}y Field". (C) 2008 Elsevier Ltd. All rights reserved.\par
\clearpage

\vspace*{-2cm}
Nb \tabto{0cm}73/327 (article\_id: 400)\par
TI \tabto{0cm}\hl{Absorpti}ve \hl{Capacit}y: A Process Perspective\par
AU \tabto{0cm}Easterby-Smith, Graca, Antonacopoulou, Ferdinand\par
PY \tabto{0cm}2008, SO MANAGEMENT LEARNING\par
DT \tabto{0cm}Article\par
PG \tabto{0cm}19, NR 27, TC 46\par
DE \tabto{0cm}\hl{ABSORPTI}VE \hl{CAPACIT}Y, BOUNDARIES, DYNAMIC CAPABILITIES, ORGANIZATIONAL LEARNING, POLITICS\par
ID \tabto{0cm}BOUNDARIES, DYNAMIC CAPABILITIES, FRAMEWORK, INDUSTRY, INNOVATION, KNOWLEDGE MANAGEMENT, NETWORKS, POLITICS\par
AB \tabto{0cm}\hl{Absorpti}ve \hl{capacit}y is regarded as an important factor in both corporate innovation and general competitive advantage. The concept was initially developed largely from reviews of the literature and has subsequently been extended by empirical studies, although some people suggest that progress since 1990 has been disappointing. This article argues that this limited development results from the dominance of quantitative studies which have failed to develop insights into the processes of \hl{absorpti}ve \hl{capacit}y, and builds on recent qualitative studies which have successfully opened up new perspectives. Using case studies drawn from three different sectors, the article argues that a process perspective on \hl{absorpti}ve \hl{capacit}y should include the role of power in the way knowledge is absorbed by organizations, and provide better understanding of the nature of boundaries within and around organizations.\par
\clearpage

\vspace*{-2cm}
Nb \tabto{0cm}74/327 (article\_id: 401)\par
TI \tabto{0cm}From capability to connectivity-\hl{Absorpti}ve \hl{capacit}y and exploratory alliances in biopharmaceutical firms: A US-Europe comparison\par
AU \tabto{0cm}Xia, Roper\par
PY \tabto{0cm}2008, SO TECHNOVATION\par
DT \tabto{0cm}Review\par
PG \tabto{0cm}10, NR 107, TC 29\par
DE \tabto{0cm}\hl{ABSORPTI}VE \hl{CAPACIT}Y, ALLIANCES, BIOTECHNOLOGY, EUROPE, US\par
ID \tabto{0cm}BIOTECHNOLOGY START-UPS, COMPLEMENTARY ASSETS, INCUMBENTS ADVANTAGE, INNOVATION STRATEGIES, INTERFIRM COOPERATION, KNOWLEDGE, PHARMACEUTICAL-INDUSTRY, PRODUCT DEVELOPMENT, RESEARCH-AND-DEVELOPMENT, STRATEGIC ALLIANCES\par
AB \tabto{0cm}In this paper, we explore the relationship between aspects of firms' potential \hl{absorpti}ve \hl{capacit}y (PACAP) on their involvement in exploratory alliances. Our study is based on survey data from firms in the US and European (the UK, Germany, France and Ireland) biopharmaceutical sectors. We use zero inflated negative binomial (ZINB) models to capture the number of exploratory alliances in which firms are engaged, and find that the assimilation dimension of PACAP is significantly more important than the acquisition dimension. More specifically, we find that skill levels and continuous R\&D play an important role in determining biopharmaceutical firms' exploratory alliance activity, while R\&D intensity proves relatively unimportant. Our results also highlight differences between the determinants of alliance behaviour in the US and Europe: in the US firms' skill levels prove more significant, while in Europe continuity of R\&D proves more significant. Commonalities are also observed, however, with firms' strategic focus and an inverted 'U' shaped relationship between firm size and alliance engagement evident in both areas. (C) 2008 Published by Elsevier Ltd.\par
\clearpage

\vspace*{-2cm}
Nb \tabto{0cm}75/327 (article\_id: 402)\par
TI \tabto{0cm}IS integration and business performance: The mediation effect of organizational \hl{absorpti}ve \hl{capacit}y in SMEs\par
AU \tabto{0cm}Francalanci, Morabito\par
PY \tabto{0cm}2008, SO JOURNAL OF INFORMATION TECHNOLOGY\par
DT \tabto{0cm}Review\par
PG \tabto{0cm}16, NR 103, TC 24\par
DE \tabto{0cm}BUSINESS PERFORMANCE, CFA, IS INTEGRATION, ORGANIZATIONAL \hl{ABSORPTI}VE \hl{CAPACIT}Y, SEM, SMES\par
ID \tabto{0cm}COMPETITIVE ADVANTAGE, CORPORATE-STRATEGY, ERP, FRAMEWORK, IMPACT, INFORMATION-TECHNOLOGY, INNOVATION, MANUFACTURING FLEXIBILITY, PRODUCTIVITY, SYSTEMS\par
AB \tabto{0cm}A fundamental result of the information technology (IT) and business performance literature is that IT is not a driver of performance per se. However, it can be associated with higher performance if accompanied by organizational change. The identification of the variables describing organizational change is still on-going work. This paper focuses on organizational \hl{absorpti}ve \hl{capacit}y and analyses its effects on the relationship between IT and business performance in small and medium enterprises (SMEs). Organizational \hl{absorpti}ve \hl{capacit}y measures the ability of an organization to complete a learning process. A significant learning effort is typically associated with IT, as it represents a complex technology. To cope with IT's complexity, implementation is typically incremental and is accompanied by a continuous integration effort of data and applications. The degree of integration of a company's information system (IS), called IS integration, is a proxy of IT maturity and quality. In this study, we explore the effect of IS integration on business performance through \hl{absorpti}ve \hl{capacit}y, that is, we hypothesize that \hl{absorpti}ve \hl{capacit}y has a mediation role between IS integration and business performance. The proposed research model is tested with data surveyed from 466 SMEs sited in Italy, for which exports constitute more than half of their revenues. Results indicate that organizational \hl{absorpti}ve \hl{capacit}y has a mediation effect. Alternative models attributing to \hl{absorpti}ve \hl{capacit}y a role different from mediation are found to be non-significant.\par
\clearpage

\vspace*{-2cm}
Nb \tabto{0cm}76/327 (article\_id: 403)\par
TI \tabto{0cm}International spillovers and \hl{absorpti}ve \hl{capacit}y: A cross-country cross-sector analysis based on patents and citations\par
AU \tabto{0cm}Mancusi\par
PY \tabto{0cm}2008, SO JOURNAL OF INTERNATIONAL ECONOMICS\par
DT \tabto{0cm}Article\par
PG \tabto{0cm}11, NR 29, TC 24\par
DE \tabto{0cm}\hl{ABSORPTI}VE \hl{CAPACIT}Y, CITATIONS, KNOWLEDGE PRODUCTION FUNCTION, PATENTS, R\&D SPILLOVERS\par
ID \tabto{0cm}2 FACES, DYNAMICS, INNOVATION, KNOWLEDGE SPILLOVERS, MANUFACTURING FIRMS, MARKET VALUE, MODELS, PANEL, PRODUCTIVITY, RESEARCH-AND-DEVELOPMENT\par
AB \tabto{0cm}This paper brings together the issues of knowledge spillovers and \hl{absorpti}ve \hl{capacit}y, by assessing the role of prior R\&D experience in enhancing a country's ability to understand and improve upon external knowledge. International spillovers innovative productivity in laggard countries, while technological leaders are a source rather than a destination of knowledge flows. Quantitative estimates of the effect of \hl{absorpti}ve \hl{capacit}y on innovative performance, through knowledge spillovers, show that \hl{absorpti}ve \hl{capacit}y increases the elasticity of a laggard country's innovation to international spillovers, while its marginal effect is negligible for countries at the technological frontier. (C) 2008 Elsevier B.V. All rights reserved.\par
\clearpage

\vspace*{-2cm}
Nb \tabto{0cm}77/327 (article\_id: 404)\par
TI \tabto{0cm}R\&D/Returns Causality: \hl{Absorpti}ve \hl{Capacit}y or Organizational IQ\par
AU \tabto{0cm}Knott\par
PY \tabto{0cm}2008, SO MANAGEMENT SCIENCE\par
DT \tabto{0cm}Article\par
PG \tabto{0cm}14, NR 23, TC 15\par
DE \tabto{0cm}\hl{ABSORPTI}VE \hl{CAPACIT}Y, ORGANIZATIONAL IQ, R\&D, SPILLOVERS\par
ID \tabto{0cm}DIFFUSION, GROWTH, INDUSTRY MATTER, INNOVATION, KNOWLEDGE, SPILLOVERS\par
AB \tabto{0cm}\hl{Absorpti}ve \hl{capacit}y is the principle that assimilating new knowledge requires prior knowledge. The attendant prescription is to invest more in R\&D to derive greater benefit from the R\&D of others (spillovers). Empirical tests of R\&D productivity typically find \hl{absorpti}ve \hl{capacit}y (R\&D* rival R\&D) to be significant. This result poses a puzzle, however: What can a firm conducting 50\% of industry R\&D learn from a set of firms each conducting 5\%? Aren't the laggard firms merely playing catch-up? Yet, if this is so, why is the interaction term significant?
One possible resolution to this puzzle is that the correlation between R\&D spending and returns is really about innate ability (IQ) rather than investment behavior (\hl{absorpti}ve \hl{capacit}y). In this view the causality between capability and behavior is reversed. It is not that firms obtain higher returns by investing more in R \& D; it is that some firms have higher returns to R\&D, thus they invest more. I conduct an empirical test of the competing views and find that (1) firms differ in the output elasticities of their own R\&D (IQ) as well as the elasticities of spillovers from rivals, (2) \hl{absorpti}ve \hl{capacit}y becomes insignificant when accounting for that heterogeneity, (3) R\&D investment increases with IQ, but (4) R\&D investment has no impact on firm's ability to benefit from spillovers.\par
\clearpage

\vspace*{-2cm}
Nb \tabto{0cm}78/327 (article\_id: 405)\par
TI \tabto{0cm}\hl{Absorpti}ve \hl{capacit}y, knowledge circulation and coal cleaning innovation: The Netherlands in the 1930s\par
AU \tabto{0cm}Davids, Tai\par
PY \tabto{0cm}2009, SO BUSINESS HISTORY\par
DT \tabto{0cm}Article\par
PG \tabto{0cm}23, NR 29, TC 0\par
DE \tabto{0cm}\hl{ABSORPTI}VE \hl{CAPACIT}Y, BARVOYS SYSTEM, COAL CLEANING, DUTCH STATE MINES, EXPECTATIONS, HYDROCYCLONE, INNOVATION, KNOWLEDGE, LOESS WASHING SYSTEM, TROMP SYSTEM\par
ID \tabto{0cm}COMPETITION\par
AB \tabto{0cm}Before World War II, Dutch State Mines, the national, state owned coal corporation, was confronted with major challenges, specifically that foreign coal was sold at dumping prices in the home market. At the same time, coal cleaning needed to be improved in order to offer higher quality coal against lower coal processing costs. In this paper we illustrate how State Mines relied on its innovative \hl{capacit}y in order to overcome the economic, technological and market changes. The coal cleaning innovations at State Mines show how \hl{absorpti}ve \hl{capacit}y was of prime importance for the firm's innovative \hl{capacit}y. External knowledge acquisition as well as internal knowledge building proved to be relevant, although the balance changed over time. While initially acquisition and assimilation of external knowledge (potential \hl{absorpti}ve \hl{capacit}y) were essential to improve the existing coal cleaning processes, internal knowledge building was needed to come to real improvements in coal cleaning. The establishment of a licensing company was essential to exploit this knowledge. An important feature was that State Mines was always well aware of its lack of capabilities and knowledge and open to search for and learn from knowledge outside its firm boundaries. Moreover, expectations determined the search for external knowledge.\par
\clearpage

\vspace*{-2cm}
Nb \tabto{0cm}79/327 (article\_id: 406)\par
TI \tabto{0cm}\hl{Absorpti}ve \hl{capacit}y, network embeddedness and local firm's knowledge acquisition in the Global Manufacturing Network\par
AU \tabto{0cm}Wu, Liu\par
PY \tabto{0cm}2009, SO INTERNATIONAL JOURNAL OF TECHNOLOGY MANAGEMENT\par
DT \tabto{0cm}Article\par
PG \tabto{0cm}18, NR 36, TC 10\par
DE \tabto{0cm}\hl{ABSORPTI}VE \hl{CAPACIT}Y, GLOBAL MANUFACTURING NETWORK, GMN, KNOWLEDGE ACQUISITION, NETWORK EMBEDDEDNES\par
ID \tabto{0cm}COMPETITIVE CAPABILITIES, EXTERNAL NETWORKS, INNOVATION, INTERNATIONAL-JOINT-VENTURES, MANAGEMENT, PERFORMANCE, SOCIAL-STRUCTURE, TECHNOLOGY\par
AB \tabto{0cm}Global Manufacturing Network (GMN) is an innovative manufacturing system that provides great opportunities for local firms in developing countries to acquire knowledge and upgrade through collaborations in GMN. Using data of manufacturing firms, we confirm the conceptual model that local firm's \hl{absorpti}ve \hl{capacit}y can contribute to its knowledge acquisition and performance in the context of GMN. Furthermore, our finding also suggests that firm's abilities to understand external knowledge will be more positively associated with its knowledge acquisition when the information embeddedness is high, and firm's abilities to apply external knowledge and knowledge acquisition will be more positively associated with the firm performance when the work embeddedness is high.\par
\clearpage

\vspace*{-2cm}
Nb \tabto{0cm}80/327 (article\_id: 407)\par
TI \tabto{0cm}Exporting, RD, and \hl{absorpti}ve \hl{capacit}y in UK establishments\par
AU \tabto{0cm}Harris, Li\par
PY \tabto{0cm}2009, SO OXFORD ECONOMIC PAPERS-NEW SERIES\par
DT \tabto{0cm}Article\par
PG \tabto{0cm}30, NR 68, TC 44\par
DE \tabto{0cm}null\par
ID \tabto{0cm}DEVELOPMENT COOPERATION, FIRMS EXPORT, FOREIGN OWNERSHIP, INNOVATION, LEVEL DATA, PERFORMANCE, PRODUCTIVITY, RESEARCH-AND-DEVELOPMENT, TRADE, UNITED-KINGDOM\par
AB \tabto{0cm}This paper models the determinants of exporting (both in terms of export propensity and export intensity), with a particular emphasis on the importance of \hl{absorpti}ve \hl{capacit}y and the endogenous link between exporting and undertaking RD. Based on a merged dataset of the 2001 Community Innovation Survey and the 2000 Annual Respondents Database for the UK, our results suggest that establishment size plays a fundamental role in explaining exporting. Meanwhile, alongside other factors, undertaking RD activities and having greater \hl{absorpti}ve \hl{capacit}y (for scientific knowledge, international co-operation, and organizational structure) significantly reduce entry barriers into export markets, having controlled for self-selectivity into exporting. Nevertheless, conditional on entry into international markets, only greater \hl{absorpti}ve \hl{capacit}y (associated with scientific knowledge) seems to further boost export performance in such markets, whereas spending on RD no longer has an impact on exporting behaviour once we have taken into account its endogenous nature.\par
\clearpage

\vspace*{-2cm}
Nb \tabto{0cm}81/327 (article\_id: 408)\par
TI \tabto{0cm}Managing \hl{absorpti}ve \hl{capacit}y stocks to improve performance: Empirical evidence from the turbulent environment of Israeli hospitals\par
AU \tabto{0cm}Lev, Fiegenbaum, Shoham\par
PY \tabto{0cm}2009, SO EUROPEAN MANAGEMENT JOURNAL\par
DT \tabto{0cm}Article\par
PG \tabto{0cm}13, NR 55, TC 8\par
DE \tabto{0cm}ENVIRONMENTAL COMPETITIVENESS, HOSPITAL PERFORMANCE, KNOWLEDGE FLOWS AND STOCKS, POTENTIAL AND REALIZED \hl{ABSORPTI}VE \hl{CAPACIT}Y\par
ID \tabto{0cm}ACCUMULATION, BUSINESS, CAPABILITY, COMPETITIVE ADVANTAGE, INNOVATION, INTEGRATION, KNOWLEDGE, ORGANIZATIONS, PERSPECTIVE, RECONCEPTUALIZATION\par
AB \tabto{0cm}The current paper focuses on the management of external knowledge as a central mechanism when organizations face threats from turbulent environments. Based on \hl{absorpti}ve \hl{capacit}y (ACAP) theory, we emphasize ACAP's separation into potential and realized knowledge and suggest that each should be associated with three-dimensional stocks and distinguish between managing knowledge stocks and knowledge flows. Data from 522 managers from 12 Israli hospitals support the theoretical model. Organizations that manage both potential and realized ACAP stocks achieve better performance in a turbulent environment. The paper explores the practical and theoretical contivutions of the new suggested framework by linking environmental competitiveness, ACAP stocks, and performance. (C) 2008 Elsevier Ltd. All rights reserved.\par
\clearpage

\vspace*{-2cm}
Nb \tabto{0cm}82/327 (article\_id: 409)\par
TI \tabto{0cm}The positive effects of relationship learning and \hl{absorpti}ve \hl{capacit}y on innovation performance and competitive advantage in industrial markets\par
AU \tabto{0cm}Chen, Lin, Chang\par
PY \tabto{0cm}2009, SO INDUSTRIAL MARKETING MANAGEMENT\par
DT \tabto{0cm}Article\par
PG \tabto{0cm}7, NR 39, TC 71\par
DE \tabto{0cm}\hl{ABSORPTI}VE \hl{CAPACIT}Y, COMPETITIVE ADVANTAGE, INNOVATION PERFORMANCE, RELATIONSHIP LEARNING\par
ID \tabto{0cm}ANTECEDENTS, DEMOGRAPHY, EXTENSION, FIRM, GUANXI, MODEL, PERSPECTIVE, PRODUCT DEVELOPMENT, SUCCESS\par
AB \tabto{0cm}This study utilized structural equations modeling (SEM) to explore the positive effects of relationship learning and \hl{absorpti}ve \hl{capacit}y on competitive advantages of companies through their innovation performances in Taiwanese manufacturing industry. The results of this study showed that relationship learning and \hl{absorpti}ve \hl{capacit}y positively influence upon innovation performances of companies, and further have positive effects on competitive advantages of companies. In addition, this study divided the sample into three groups by the levels of relationship learning and \hl{absorpti}ve \hl{capacit}y and found that there was a significant difference of innovation performance among these three groups: 'Highly Capable Companies', 'Medially Capable Companies', and 'Lowly Capable Companies'. It is important for 'Lowly Capable Companies' to increase both of their relationship learning and \hl{absorpti}ve \hl{capacit}y to enhance their innovation performances. (C) 2008 Elsevier Inc. All rights reserved.\par
\clearpage

\vspace*{-2cm}
Nb \tabto{0cm}83/327 (article\_id: 410)\par
TI \tabto{0cm}Managing external knowledge flows: The moderating role of \hl{absorpti}ve \hl{capacit}y\par
AU \tabto{0cm}Escribano, Fosfuri, Tribo\par
PY \tabto{0cm}2009, SO RESEARCH POLICY\par
DT \tabto{0cm}Article\par
PG \tabto{0cm}10, NR 45, TC 123\par
DE \tabto{0cm}\hl{ABSORPTI}VE \hl{CAPACIT}Y, EXTERNAL KNOWLEDGE FLOWS, INNOVATION\par
ID \tabto{0cm}ACQUISITION, INNOVATION, MARKET, ORGANIZATIONAL FORMS, PERFORMANCE, PHARMACEUTICAL-INDUSTRY, RESEARCH-AND-DEVELOPMENT, SCIENTIFIC-RESEARCH, SPILLOVERS, TECHNOLOGY\par
AB \tabto{0cm}In this paper, we argue that those firms with higher levels of \hl{absorpti}ve \hl{capacit}y can manage external knowledge flows more efficiently, and stimulate innovative outcomes. We test this contention with a sample of 2265 Spanish firms, drawn from the Community Innovation Surveys (CIS) for 2000 and 2002, produced by the Spanish National Statistics Institute (INE). We find that \hl{absorpti}ve \hl{capacit}y is indeed an important source of competitive advantage, especially in sectors characterized by turbulent knowledge and strong intellectual property rights protection. The implications for management practice and policy are also discussed. (C) 2008 Elsevier B.V. All rights reserved.\par
\clearpage

\vspace*{-2cm}
Nb \tabto{0cm}84/327 (article\_id: 411)\par
TI \tabto{0cm}The effect of financial constraints, \hl{absorpti}ve \hl{capacit}y and complementarities on the adoption of multiple process technologies\par
AU \tabto{0cm}Gomez, Vargas\par
PY \tabto{0cm}2009, SO RESEARCH POLICY\par
DT \tabto{0cm}Article\par
PG \tabto{0cm}14, NR 69, TC 14\par
DE \tabto{0cm}\hl{ABSORPTI}VE \hl{CAPACIT}Y, ADOPTION, DIFFUSION, INTERNAL R\&D, PROCESS TECHNOLOGIES\par
ID \tabto{0cm}BANKING INDUSTRY, DETERMINANTS, E-BUSINESS, EMPIRICAL-ANALYSIS, INFORMATION-TECHNOLOGY, INNOVATION, INTRAFIRM DIFFUSION, MANUFACTURING FIRMS, MARKET, PUBLIC-POLICY\par
AB \tabto{0cm}This paper investigates the factors affecting the multiple adoption of new process technologies in manufacturing. We focus our attention on the effect of both financial resources and \hl{absorpti}ve \hl{capacit}y on the decision to introduce the technology. We argue in favour of a negative effect of financial constraints and provide reasons for a differential effect of internal and external R\&D on innovation adoption. Additionally, the methodology allows us to consider the possible complementarities arising when firms adopt several new process technologies. Our results show that financial constraints are dependent on the technology analyzed, whereas only internal R\&D investments are strong predictors of adoption. We are also able to present evidence that the three technologies analyzed (numerically controlled machines, computer aided design and robotics) are, to some extent, complementary. (C) 2008 Elsevier B.V. All rights reserved.\par
\clearpage

\vspace*{-2cm}
Nb \tabto{0cm}85/327 (article\_id: 412)\par
TI \tabto{0cm}Proof of concept processes in UK university technology transfer: an \hl{absorpti}ve \hl{capacit}y perspective\par
AU \tabto{0cm}McAdam, Brown\par
PY \tabto{0cm}2009, SO R \& D MANAGEMENT\par
DT \tabto{0cm}Article\par
PG \tabto{0cm}19, NR 55, TC 4\par
DE \tabto{0cm}null\par
ID \tabto{0cm}ACADEMIC ENTREPRENEURS, BUSINESS, COMPANIES, FACULTY, IMPACT, INNOVATION, INVENTIONS, PERFORMANCE, TRANSFER OFFICES, VENTURES\par
AB \tabto{0cm}Successful research commercialisation within the university domain is predicated upon basic research being developed into technology that will attract funding, ultimately resulting in entities such as University spin-out companies or licensing arrangements. This development process involves considerable risk and uncertainty and may require substantial resources to fund early stage operations while returns are uncertain. Hence there is a need to explore risk-minimisation approaches relating to proving the potential for development while concurrently allocating resources in an incremental manner. This paper focuses on the development of the Northern Ireland Proof of Concept (PoC) process within a University Science Park Incubator (USI) as a particular approach to addressing these challenges inherent in the United Kingdom University technology transfer. Furthermore, \hl{Absorpti}ve \hl{Capacit}y has emerged in the literature as an appropriate theoretical framework or lens for exploring the development and application of new technology. Therefore, the aim of this paper is to explore the PoC process within a USI as a means for improving the commercialisation of University technology transfer using an \hl{Absorpti}ve \hl{Capacit}y perspective. A multiple case analysis of PoC applications within a UK university is described. From the findings it emerges that \hl{Absorpti}ve \hl{Capacit}y influencing factors such as levels of R\&D investment, prior knowledge base and integration of stakeholder and technology planning all impact on PoC outcomes. In addition a number of process improvement areas for PoC are identified in relation to the influencing factors within the \hl{Absorpti}ve \hl{Capacit}y framework.\par
\clearpage

\vspace*{-2cm}
Nb \tabto{0cm}86/327 (article\_id: 413)\par
TI \tabto{0cm}\hl{Absorpti}ve \hl{capacit}y and the search for innovation\par
AU \tabto{0cm}Fabrizio\par
PY \tabto{0cm}2009, SO RESEARCH POLICY\par
DT \tabto{0cm}Article\par
PG \tabto{0cm}13, NR 92, TC 103\par
DE \tabto{0cm}\hl{ABSORPTI}VE \hl{CAPACIT}Y, INNOVATION, PHARMACEUTICAL AND BIOTECHNOLOGY, UNIVERSITY RESEARCH\par
ID \tabto{0cm}ACADEMIC RESEARCH, BASIC RESEARCH, COMPETITIVE ADVANTAGE, INDUSTRIAL-INNOVATION, PATENT CITATIONS, PUBLIC SCIENCE, RESEARCH-AND-DEVELOPMENT, SCIENTIFIC-RESEARCH, STRATEGIC ALLIANCES, TECHNOLOGICAL SEARCH\par
AB \tabto{0cm}This paper examines the link between a firm's \hl{absorpti}ve \hl{capacit}y-building activities and the search process for innovation. We propose that the enhanced access to university research enjoyed by firms that engage in basic research and collaborate with university scientists leads to superior search for new inventions and provides advantage in terms of both the timing and quality of search outcomes. Results based on a panel data of pharmaceutical and biotechnology firms support these contentions and suggest that the two research activities are mutually beneficial, but also uncover intriguing differences that suggest differing roles of internally and externally developed knowledge. (C) 2008 Elsevier B.V. All rights reserved.\par
\clearpage

\vspace*{-2cm}
Nb \tabto{0cm}87/327 (article\_id: 414)\par
TI \tabto{0cm}Search patterns and \hl{absorpti}ve \hl{capacit}y: Low- and high-technology sectors in European countries\par
AU \tabto{0cm}Grimpe, Sofka\par
PY \tabto{0cm}2009, SO RESEARCH POLICY\par
DT \tabto{0cm}Article\par
PG \tabto{0cm}12, NR 70, TC 60\par
DE \tabto{0cm}\hl{ABSORPTI}VE \hl{CAPACIT}Y, HIGH-TECHNOLOGY SECTORS, LOW-TECHNOLOGY SECTORS, MEDIUM-TECHNOLOGY SECTORS, SEARCH STRATEGIES\par
ID \tabto{0cm}CAPABILITIES, COMPETITIVE ADVANTAGE, FIRM, INNOVATION, KNOWLEDGE, MODEL, PERFORMANCE, PRODUCT, RESEARCH-AND-DEVELOPMENT, RESOURCE-BASED VIEW\par
AB \tabto{0cm}Searching for externally available knowledge has been characterised as a vital part of the innovation process. Previous research has, however, almost exclusively focused on high-technology environments, largely ignoring the substantial low- and medium-technology sectors of modern economies. We argue that firms from low- and high-technology sectors differ in their search patterns and that these mediate the relationship between innovation inputs and outputs. Based on a sample of 4500 firms from 13 European countries, we find that search patterns in low-technology industries focus on market knowledge and that they differ from technology sourcing activities in high-technology industries. (C) 2008 Elsevier B.V. All rights reserved.\par
\clearpage

\vspace*{-2cm}
Nb \tabto{0cm}88/327 (article\_id: 415)\par
TI \tabto{0cm}How do threshold firms sustain corporate entrepreneurship? The role of boards and \hl{absorpti}ve \hl{capacit}y\par
AU \tabto{0cm}Zahra, Filatotchev, Wright\par
PY \tabto{0cm}2009, SO JOURNAL OF BUSINESS VENTURING\par
DT \tabto{0cm}Article\par
PG \tabto{0cm}13, NR 50, TC 68\par
DE \tabto{0cm}\hl{ABSORPTI}VE \hl{CAPACIT}Y, BOARDS, CAPABILITY, CORPORATE ENTREPRENEURSHIP, KNOWLEDGE-BASED THEORY\par
ID \tabto{0cm}ACCOUNTABILITY, COMPETITIVE ADVANTAGE, GOVERNANCE, INNOVATION, KNOWLEDGE, MODEL, OWNERSHIP, PERFORMANCE, PERSPECTIVE, RESOURCE-BASED VIEW\par
AB \tabto{0cm}As companies move from one stage of their cycle to the next, they often have to revamp their skills and build innovative capabilities to survive, achieve profitability, and stimulate growth. Corporate entrepreneurship (CE) activities give these firms a foundation for building and exploiting these capabilities. In turn, stimulating and sustaining CE requires the infusion of resources and new knowledge into the firm's operations, using multiple external sources. In this paper, we highlight the importance of boards of directors and \hl{absorpti}ve \hl{capacit}y for gaining access to varied and current knowledge that enriches CE. We suggest that boards and \hl{absorpti}ve \hl{capacit}y complement each other in fueling CE activities. Further, boards can sometimes substitute for poor \hl{absorpti}ve \hl{capacit}y and vice versa, influencing the intensity of CE activities. Managing these complementarities (or substitutions) is crucial for sustaining CE initiatives and creating value from them. (C) 2008 Elsevier B.V. All rights reserved.\par
\clearpage

\vspace*{-2cm}
Nb \tabto{0cm}89/327 (article\_id: 416)\par
TI \tabto{0cm}AID ALLOCATION TO FRAGILE STATES: \hl{ABSORPTI}VE \hl{CAPACIT}Y CONSTRAINTS\par
AU \tabto{0cm}Feeny, Mcgillivray\par
PY \tabto{0cm}2009, SO JOURNAL OF INTERNATIONAL DEVELOPMENT\par
DT \tabto{0cm}Article\par
PG \tabto{0cm}15, NR 32, TC 7\par
DE \tabto{0cm}\hl{ABSORPTI}VE \hl{CAPACIT}Y, ECONOMIC GROWTH, FOREIGN AID, FRAGILE STATES\par
ID \tabto{0cm}FOREIGN-AID, GROWTH, POVERTY\par
AB \tabto{0cm}The international donor community has grave concerns about the effectiveness of aid to countries it classifies as 'fragile states'. The impact of aid on growth and poverty reduction and the ability to efficiently absorb additional inflows is thought to be significantly lower in these countries compared to other recipients. This paper examines this issue and suggests that a while a number of fragile states can efficiently absorb more aid than they have received, a number receive far more aid than they can efficiently absorb from a perspective based purely on per capita income growth. Policy recommendations are provided. Copyright (C) 2008 John Wiley \& Sons, Ltd.\par
\clearpage

\vspace*{-2cm}
Nb \tabto{0cm}90/327 (article\_id: 417)\par
TI \tabto{0cm}Ambidexterity in Technology Sourcing: The Moderating Role of \hl{Absorpti}ve \hl{Capacit}y\par
AU \tabto{0cm}Rothaermel, Alexandre\par
PY \tabto{0cm}2009, SO ORGANIZATION SCIENCE\par
DT \tabto{0cm}Review\par
PG \tabto{0cm}22, NR 112, TC 132\par
DE \tabto{0cm}\hl{ABSORPTI}VE \hl{CAPACIT}Y, AMBIDEXTERITY, DYNAMIC CAPABILITIES, EXPLORATION AND EXPLOITATION, INNOVATION, TECHNOLOGY SOURCING\par
ID \tabto{0cm}2 FACES, COMPLEMENTARY ASSETS, DYNAMIC CAPABILITIES, FIRM PERFORMANCE, INTERFIRM COOPERATION, ORGANIZATIONAL AMBIDEXTERITY, PRODUCT DEVELOPMENT, PROJECT PERFORMANCE, RESEARCH-AND-DEVELOPMENT, STRATEGIC MANAGEMENT\par
AB \tabto{0cm}A firm's organizational and technological boundaries are two important demarcation lines when sourcing for technology. Based on this theoretical lens, four possible combinations of exploration and exploitation emerge. Applying an ambidexterity perspective to a firm's technology sourcing strategy, we hypothesize that a curvilinear relationship exists between a firm's technology sourcing mix and its performance. We further introduce a contingency element by proposing that a firm's \hl{absorpti}ve \hl{capacit}y exerts a positive moderating effect on this relationship. We empirically test these hypotheses on a random, multi-industry sample of U. S. manufacturing companies. We find support for the notion that the relationship between technology sourcing mix and firm performance is an inverted U-shape. Moreover, higher levels of \hl{absorpti}ve \hl{capacit}y allow a firm to more fully capture the benefits resulting from ambidexterity in technology sourcing.\par
\clearpage

\vspace*{-2cm}
Nb \tabto{0cm}91/327 (article\_id: 418)\par
TI \tabto{0cm}\hl{ABSORPTI}VE \hl{CAPACIT}Y, ENVIRONMENTAL TURBULENCE, AND THE COMPLEMENTARITY OF ORGANIZATIONAL LEARNING PROCESSES (Retracted article. See vol. 56, pg. 1830, 2013)\par
AU \tabto{0cm}Lichtenthaler\par
PY \tabto{0cm}2009, SO ACADEMY OF MANAGEMENT JOURNAL\par
DT \tabto{0cm}Article; Proceedings Paper\par
PG \tabto{0cm}25, NR 91, TC 185\par
DE \tabto{0cm}null\par
ID \tabto{0cm}COMBINATIVE CAPABILITIES, COMPETITIVE ADVANTAGE, DYNAMIC CAPABILITIES, EXPLORATION, FIRM-LEVEL, INNOVATION PERFORMANCE, KNOWLEDGE TRANSFER, PRODUCT DEVELOPMENT, RESOURCE COMPLEMENTARITY, STRATEGIC ALLIANCES\par
AB \tabto{0cm}Following the process-based definition of \hl{absorpti}ve \hl{capacit}y, this article identifies technological and market knowledge as two critical components of prior knowledge in the organizational learning processes of \hl{absorpti}ve \hl{capacit}y. Data from a multi-informant survey conducted in 175 industrial firms show that exploratory, transformative, and exploitative learning have complementary effects on innovation and performance. The results emphasize the multidimensional nature of \hl{absorpti}ve \hl{capacit}y, and they help to explain interfirm discrepancies in profiting from external knowledge. Moreover, the findings underscore the importance of dynamic capabilities in contexts characterized by high degrees of technological and market turbulence.\par
\clearpage

\vspace*{-2cm}
Nb \tabto{0cm}92/327 (article\_id: 419)\par
TI \tabto{0cm}What do patent examiner inserted citations indicate for a region with low \hl{absorpti}ve \hl{capacit}y?\par
AU \tabto{0cm}Azagra-Caro, Fernandez-De-Lucio, Perruchas, Mattsson\par
PY \tabto{0cm}2009, SO SCIENTOMETRICS\par
DT \tabto{0cm}Article\par
PG \tabto{0cm}15, NR 30, TC 5\par
DE \tabto{0cm}null\par
ID \tabto{0cm}FLOWS, INNOVATION SYSTEMS, KNOWLEDGE SPILLOVERS, LOCALIZATION, SCIENCE, TECHNOLOGY\par
AB \tabto{0cm}Most studies of patents citations focus on national or international contexts, especially contexts of high \hl{absorpti}ve \hl{capacit}y, and employ examiner citations. We argue that results can vary if we take the region as the context of analysis, especially if it is a region with low \hl{absorpti}ve \hl{capacit}y, and if we study applicant citations and examiner-inserted citations separately. Using a sample from the Valencian Community (Spain), we conclude that (i) the use of examiner-inserted citations as a proxy for applicant citations, (ii) the interpretation of non-patent references as indicators of science-industry links, and (iii) the traditional results for geographical localization are not generalizable to all regions with low \hl{absorpti}ve \hl{capacit}y.\par
\clearpage

\vspace*{-2cm}
Nb \tabto{0cm}93/327 (article\_id: 420)\par
TI \tabto{0cm}\hl{Absorpti}ve \hl{capacit}y and R\&D tax policy: Are in-house and external contract R\&D substitutes or complements?\par
AU \tabto{0cm}Watkins, Paff\par
PY \tabto{0cm}2009, SO SMALL BUSINESS ECONOMICS\par
DT \tabto{0cm}Article\par
PG \tabto{0cm}21, NR 49, TC 5\par
DE \tabto{0cm}\hl{ABSORPTI}VE \hl{CAPACIT}Y, R\&D, R\&D SUBSTITUTION, TAX CREDIT, TECHNOLOGY POLICY\par
ID \tabto{0cm}BASIC RESEARCH, FIRMS, INNOVATION, KNOWLEDGE, PRODUCTIVITY\par
AB \tabto{0cm}Firms fund research and development (R\&D) to generate commercializable innovations and to increase their ability to understand and absorb knowledge from elsewhere. This dual role and opposed incentive structure of internal R\&D create a significant question for both theory and R\&D policy: Is internal R\&D a complement or substitute for external R\&D? We develop a model and novel technique for empirically estimating R\&D substitution elasticities. We focus on bio-pharmaceutical and software industries in California and Massachusetts, where tax credit rates changed differently over time for the two types of R\&D, creating a natural experiment. The effective tax prices for the two R\&D types differ from type to type, firm to firm, state to state, and year to year. This allows us to examine changes in the composition of firms' R\&D budgets between in-house R\&D and external basic research when the relative tax prices of each category of research change. We find evidence of a substitute relationship both for a sample comprising exclusively small firms as well as for a more general distribution of firm sizes.\par
\clearpage

\vspace*{-2cm}
Nb \tabto{0cm}94/327 (article\_id: 421)\par
TI \tabto{0cm}No Pain, No Gain: An R\&D Model with Endogenous \hl{Absorpti}ve \hl{Capacit}y\par
AU \tabto{0cm}Hammerschmidt\par
PY \tabto{0cm}2009, SO JOURNAL OF INSTITUTIONAL AND THEORETICAL ECONOMICS-ZEITSCHRIFT FUR DIE\par
DT \tabto{0cm}Article\par
PG \tabto{0cm}20, NR 34, TC 5\par
DE \tabto{0cm}null\par
ID \tabto{0cm}COSTS, FIRMS, INNOVATIONS, ME HALFWAY, PATENTS, PRODUCTIVITY, RESEARCH JOINT VENTURES, SPILLOVERS\par
AB \tabto{0cm}I investigate how the introduction of the \hl{absorpti}ve-\hl{capacit}y hypothesis changes the results of strategic investment games with exogenous spillover. As a novelty, this paper distinguishes between two different components of R\&D: \hl{absorpti}ve R\&D and inventive R\&D. It is demonstrated that a new effect appears: The \hl{absorpti}ve-\hl{capacit}y effect now counteracts the free-rider effect of the traditional models. The findings show that firms will invest more in R\&D to strengthen \hl{absorpti}ve \hl{capacit}y when the spillover parameter is higher. To learn from its competitor, a firm needs to develop its \hl{absorpti}ve \hl{capacit}y: no pain, no gain. (JEL O 31, L 13, C 72)\par
\clearpage

\vspace*{-2cm}
Nb \tabto{0cm}95/327 (article\_id: 422)\par
TI \tabto{0cm}The many faces of \hl{absorpti}ve \hl{capacit}y: spillovers of copper interconnect technology for semiconductor chips\par
AU \tabto{0cm}Lim\par
PY \tabto{0cm}2009, SO INDUSTRIAL AND CORPORATE CHANGE\par
DT \tabto{0cm}Article\par
PG \tabto{0cm}36, NR 47, TC 28\par
DE \tabto{0cm}null\par
ID \tabto{0cm}BASIC RESEARCH, BIOTECHNOLOGY, COMBINATIVE CAPABILITIES, DRUG DISCOVERY, INDUSTRY, INNOVATION, KNOWLEDGE, PATENT CITATIONS, RESEARCH-AND-DEVELOPMENT, SCIENTIFIC-RESEARCH\par
AB \tabto{0cm}A case study of copper interconnect technology suggests that \hl{absorpti}ve \hl{capacit}y exist in three forms: disciplinary, domain specific and encoded. Each involves different ways of managing R\&D and linking internal to external research. Disciplinary \hl{absorpti}ve \hl{capacit}y requires a firm to actively engage with the scientific community, while protecting domain-specific knowledge. Domain-specific \hl{absorpti}ve \hl{capacit}y depends upon influencing disciplinary research at universities and consortia, then capturing domain knowledge through collaboration and hiring. As technology develops, it becomes encoded, and \hl{absorpti}on depends increasingly upon integrating knowledge from suppliers. Hence, \hl{absorpti}ve \hl{capacit}y is a multifaceted construct that is heavily shaped by the type and maturity of technology absorbed.\par
\clearpage

\vspace*{-2cm}
Nb \tabto{0cm}96/327 (article\_id: 423)\par
TI \tabto{0cm}A Capability-Based Framework for Open Innovation: Complementing \hl{Absorpti}ve \hl{Capacit}y\par
AU \tabto{0cm}Lichtenthaler\par
PY \tabto{0cm}2009, SO JOURNAL OF MANAGEMENT STUDIES\par
DT \tabto{0cm}Article\par
PG \tabto{0cm}24, NR 99, TC 135\par
DE \tabto{0cm}null\par
ID \tabto{0cm}COMBINATIVE CAPABILITIES, COMPETITIVE ADVANTAGE, DYNAMIC CAPABILITIES, FIRM CAPABILITIES, INTEGRATIVE FRAMEWORK, KNOWLEDGE CREATION, ORGANIZATIONAL AMBIDEXTERITY, PRODUCT DEVELOPMENT, RESOURCE-BASED VIEW, STRATEGIC ALLIANCES\par
AB \tabto{0cm}P>We merge research into knowledge management, \hl{absorpti}ve \hl{capacit}y, and dynamic capabilities to arrive at an integrative perspective, which considers knowledge exploration, retention, and exploitation inside and outside a firm's boundaries. By complementing the concept of \hl{absorpti}ve \hl{capacit}y, we advance towards a capability-based framework for open innovation processes. We identify the following six 'knowledge \hl{capacit}ies' as a firm's critical capabilities of managing internal and external knowledge in open innovation processes: inventive, \hl{absorpti}ve, transformative, connective, innovative, and desorptive \hl{capacit}y. 'Knowledge management \hl{capacit}y' is a dynamic capability, which reconfigures and realigns the knowledge \hl{capacit}ies. It refers to a firm's ability to successfully manage its knowledge base over time. The concept may be regarded as a framework for open innovation, as a complement to \hl{absorpti}ve \hl{capacit}y, and as a move towards understanding dynamic capabilities for managing knowledge. On this basis, it contributes to explaining interfirm heterogeneity in knowledge and alliance strategies, organizational boundaries, and innovation performance.\par
\clearpage

\vspace*{-2cm}
Nb \tabto{0cm}97/327 (article\_id: 424)\par
TI \tabto{0cm}\hl{Absorpti}ve \hl{capacit}y, its determinants, and influence on innovation output: Cross-cultural validation of the structural model\par
AU \tabto{0cm}Murovec, Prodan\par
PY \tabto{0cm}2009, SO TECHNOVATION\par
DT \tabto{0cm}Review\par
PG \tabto{0cm}14, NR 113, TC 67\par
DE \tabto{0cm}\hl{ABSORPTI}VE \hl{CAPACIT}Y, CROSS-CULTURAL STUDY, INNOVATION, STRUCTURAL MODEL\par
ID \tabto{0cm}CONFIRMATORY FACTOR-ANALYSIS, DEVELOPMENT COOPERATION, EMPIRICAL-ANALYSIS, FIRMS, KNOWLEDGE, ME HALFWAY, MISSING DATA, PRODUCT INNOVATION, RESEARCH-AND-DEVELOPMENT, TECHNOLOGY-TRANSFER\par
AB \tabto{0cm}The main purpose of this study is to provide stronger quantitative evidence in the field of organizational \hl{absorpti}ve \hl{capacit}y research by using a more direct measure of \hl{absorpti}ve \hl{capacit}y and a wide range of variables in a cross-nationally tested structural model. The results show that there exist two kinds of \hl{absorpti}ve \hl{capacit}y: demand-pull and science-push. Their most important determinants proved to be internal R\&D, training of personnel, innovation co-operation and attitude toward change. Both kinds of \hl{absorpti}ve \hl{capacit}y are positively related to product and process innovation output. Therefore, \hl{absorpti}ve \hl{capacit}y is to be given more attention in the future research and innovation policy considerations. (C) 2009 Elsevier Ltd. All rights reserved.\par
\clearpage

\vspace*{-2cm}
Nb \tabto{0cm}98/327 (article\_id: 425)\par
TI \tabto{0cm}\hl{Absorpti}ve \hl{Capacit}y for RD: The Identification of Different Firm Profiles\par
AU \tabto{0cm}Rodriguez-Castellanos, Hagemeister, Ranguelov\par
PY \tabto{0cm}2010, SO EUROPEAN PLANNING STUDIES\par
DT \tabto{0cm}Article\par
PG \tabto{0cm}17, NR 55, TC 1\par
DE \tabto{0cm}null\par
ID \tabto{0cm}ASSETS, COMPETITIVE ADVANTAGE, KNOWLEDGE, MODELS, REGIONAL INNOVATION SYSTEMS, RESEARCH-AND-DEVELOPMENT, RESOURCE-BASED VIEW, TECHNOLOGY\par
AB \tabto{0cm}Being competitive requires continuously performing product and process innovations nowadays. Because of this reason, the \hl{absorpti}on of externally generated RD is increasingly important for companies. It is well known that companies differ regarding their aptitude for knowledge \hl{absorpti}on. This paper aims at the identification of different firm profiles by means of the identification and valuation of drivers that support the \hl{absorpti}on of external RD. For this, we have carried out an empirical work that is based on a random sample of companies located in the northern Spanish county of Biscay. We identify four company conglomerates according to their attitude to absorb externally generated RD.\par
\clearpage

\vspace*{-2cm}
Nb \tabto{0cm}99/327 (article\_id: 426)\par
TI \tabto{0cm}The Importance of \hl{Absorpti}ve \hl{Capacit}y in the Road to Becoming a "Giant Lion"-ASUSTek Computer Inc\par
AU \tabto{0cm}Chen, Belcher\par
PY \tabto{0cm}2010, SO GLOBAL ECONOMIC REVIEW\par
DT \tabto{0cm}Article\par
PG \tabto{0cm}25, NR 50, TC 2\par
DE \tabto{0cm}\hl{ABSORPTI}VE \hl{CAPACIT}Y, ASUS, BUSINESS CULTURE, INNOVATION\par
ID \tabto{0cm}INNOVATION\par
AB \tabto{0cm}In Taiwan, small businesses, which account for about 97\% of all firms, spawned Taiwan's economic miracle in the 1980s. For the past two decades, the information technology (IT) industry has formed the backbone of Taiwan's economy. There are many theories that have identified a plethora of factors which contribute to the success of a small business. This case study applies the theory of \hl{absorpti}ve \hl{capacit}y to ASUSTek Computer Inc., a small Taiwanese business founded on April 2, 1990 by four engineers with capital of NT\$30 million (US\$950,000) which has grown to become a global leader in the IT industry with a total revenue of NT\$755 billion (US\$24 billion) and market capitalization of NT\$49 billion (US\$1.5 billion) as of December 31, 2007. (Based on exchange rate US\$1=NT\$31.5. ASUS's revenue dropped to NT\$668 billion in 2008 mainly due to the global financial crisis.) ASUS' business significantly took off after Jonney Shih joined the company as CEO in 1993. He remains a key anchor of ASUS. ASUS' ability to not only absorb, transform, and utilize new knowledge but develop new technology in-house to create and sustain competitive advantage in the market place has ensured the company's continuing success as well as propelling it to a position of leadership in the IT industry worldwide. This paper will discuss how ASUS' in-house RD; apprenticeship-style on-the-job training; management leadership's organizational culture and structureall of which stress learning and innovation; financial resources and conservative fiscal policy; as well as pro-business government policies have provided the foundation and environment for great \hl{absorpti}ve \hl{capacit}y to develop and adapt technology, enabling ASUS to become a oGiant Liono in the global computer industry.\par
\clearpage

\vspace*{-2cm}
Nb \tabto{0cm}100/327 (article\_id: 427)\par
TI \tabto{0cm}POTENTIAL \hl{ABSORPTI}VE \hl{CAPACIT}Y OF STATE IT DEPARTMENTS: A COMPARISON OF PERCEPTIONS OF CIOs AND IT MANAGERS\par
AU \tabto{0cm}Riemenschneider, Allen, Armstrong, Reid\par
PY \tabto{0cm}2010, SO JOURNAL OF ORGANIZATIONAL COMPUTING AND ELECTRONIC COMMERCE\par
DT \tabto{0cm}Article\par
PG \tabto{0cm}23, NR 62, TC 2\par
DE \tabto{0cm}\hl{ABSORPTI}VE \hl{CAPACIT}Y, ORGANIZATIONAL CULTURE, STATE INFORMATION TECHNOLOGY DEPARTMENTS, STRATEGIC POSTURE\par
ID \tabto{0cm}ENVIRONMENTS, FIRMS, INFORMATION-SYSTEMS, INNOVATION, ORGANIZATIONAL-CHANGE, PERSPECTIVE, PUBLIC-SECTOR, RISK-TAKING, STRATEGY, TECHNOLOGY\par
AB \tabto{0cm}Public sector information technology (IT) departments are facing a myriad of challenges (e.g., budget cuts, service expansions, and political turmoil) in addition to the constant and rapid technological changes facing private sector firms. One way to meet these challenges may be through the development of the organization's \hl{absorpti}ve \hl{capacit}y. \hl{Absorpti}ve \hl{capacit}y refers to an organization's ability to recognize the value of new information, assimilate it, and use it to address organizational challenges associated with external change [6]. Few researchers have focused on \hl{absorpti}ve \hl{capacit}y in public sector organizations. The purpose of this research is to ascertain how state IT departments, specifically Chief Information Officers (CIOs) and IT managers, view their external environment and their departments' ability to absorb new information.
The findings are derived from a national survey of state IT departments in the United States and indicate that for CIOs and IT managers the external environment and organizational culture are significant in predicting potential \hl{absorpti}ve \hl{capacit}y. These variables are significant for the IT managers as a group, but for the CIOs as a group, only external environment is significant. These findings may be used by state IT management to increase the organization's ability to be aware of, identify, and take effective advantage of new knowledge and innovative technologies.\par
\clearpage

\vspace*{-2cm}
Nb \tabto{0cm}101/327 (article\_id: 428)\par
TI \tabto{0cm}\hl{Absorpti}ve \hl{Capacit}y in a Non-Market Environment\par
AU \tabto{0cm}Harvey, Skelcher, Spencer, Jas, Walshe\par
PY \tabto{0cm}2010, SO PUBLIC MANAGEMENT REVIEW\par
DT \tabto{0cm}Article\par
PG \tabto{0cm}21, NR 63, TC 13\par
DE \tabto{0cm}\hl{ABSORPTI}VE \hl{CAPACIT}Y, IMPROVEMENT, KNOWLEDGE, PERFORMANCE, PUBLIC, TURNAROUND\par
ID \tabto{0cm}DYNAMIC CAPABILITIES, FAILURE, FIRM, INNOVATION, KNOWLEDGE MANAGEMENT, PERFORMANCE, PUBLIC ORGANIZATIONS, RESOURCE-BASED VIEW, SECTOR, STRATEGIC MANAGEMENT\par
AB \tabto{0cm}Improved performance by public sector organizations is a political imperative in numerous countries. There are particular challenges in turnaround of poorly performing organizations. Theoretical explanations of the performance trajectories of public organizations, and especially the causes of failure, highlight the importance of knowledge processes, often from an organizational learning perspective. \hl{Absorpti}ve \hl{capacit}y provides an alternative way of theorizing the relationships between organizational performance and knowledge processes, derived from the resource-based view of the firm and the broader concept of dynamic capabilities. The article reviews the conceptual, theoretical, and methodological implications of applying \hl{absorpti}ve \hl{capacit}y to the performance of public organizations. It concludes that the approach has value and presents a number of propositions to be tested through empirical study, alongside some more general challenges for researchers who wish to study the concept further. The high political salience of public organizations' performance, and the costs of failure, mandates a major research effort on these issues.\par
\clearpage

\vspace*{-2cm}
Nb \tabto{0cm}102/327 (article\_id: 429)\par
TI \tabto{0cm}Consulting knowledge and organisation's \hl{absorpti}ve \hl{capacit}y: A communication chain perspective\par
AU \tabto{0cm}Lu, Su, Huang\par
PY \tabto{0cm}2010, SO SERVICE INDUSTRIES JOURNAL\par
DT \tabto{0cm}Article\par
PG \tabto{0cm}16, NR 40, TC 4\par
DE \tabto{0cm}\hl{ABSORPTI}VE \hl{CAPACIT}Y, COMMUNICATION CHAIN OF CONSULTING KNOWLEDGE, CONSULTING KNOWLEDGE\par
ID \tabto{0cm}BUSINESS, CAPABILITIES, COMPETITIVE ADVANTAGE, FIRMS, INNOVATION, MANAGEMENT, PERFORMANCE, ROLES, SYSTEM\par
AB \tabto{0cm}This study investigates the communication chain of consulting knowledge constituted by consultants and internal lecturers. We analyse the differences between consultants and internal lecturers in their capability of knowledge training and discuss its influence on organisations' \hl{absorpti}on of consulting knowledge. Based on a Mann-Whitney U test of data from 47 quality management consultants and 235 internal lecturers in Taiwan, we found that internal lecturers significantly exhibited weaker capability than consultants, especially in knowledge structure, knowledge transformation, trainee orientation and training ethics. The capability gap was disadvantageous for organisations to absorb consulting knowledge and suggestions for improving this problem were provided finally.\par
\clearpage

\vspace*{-2cm}
Nb \tabto{0cm}103/327 (article\_id: 430)\par
TI \tabto{0cm}The effects of \hl{absorpti}ve \hl{capacit}y, knowledge sourcing strategy, and alliance forms on firm performance\par
AU \tabto{0cm}Lee, Liang, Liu\par
PY \tabto{0cm}2010, SO SERVICE INDUSTRIES JOURNAL\par
DT \tabto{0cm}Article\par
PG \tabto{0cm}20, NR 64, TC 8\par
DE \tabto{0cm}\hl{ABSORPTI}VE \hl{CAPACIT}Y, ALLIANCE FORMS, KNOWLEDGE SOURCING STRATEGY\par
ID \tabto{0cm}CAPABILITIES, CHOICE, GOVERNANCE, INNOVATION, INVESTMENT, ORGANIZATIONS, PERSPECTIVES, RESEARCH-AND-DEVELOPMENT, SECTORAL DIFFERENCES, TRANSACTION COST\par
AB \tabto{0cm}This paper explores the relationship between \hl{absorpti}ve \hl{capacit}y, knowledge sourcing strategy, alliance forms, and firm performance. Based on the literature, the concept of a knowledge sourcing strategy in alliance contexts is proposed, which can be categorised into two types: a knowledge internalisation strategy and a knowledge access strategy. From an organisational learning perspective, it is argued that a firm's \hl{absorpti}ve \hl{capacit}y has a positive influence on a knowledge internalisation strategy, and accordingly a firm's choices of alliance forms are also influenced. RD performance is also included in the theoretical model in order to generate further managerial implications. Instead of using conventional regression methods, structural equation modelling (SEM) is adopted to conduct path analysis, as SEM is well suited in verifying multiple-dependent models. The arguments advanced are supported by empirical analysis of a sample of 148 alliances.\par
\clearpage

\vspace*{-2cm}
Nb \tabto{0cm}104/327 (article\_id: 431)\par
TI \tabto{0cm}Relationships between knowledge acquisition, \hl{absorpti}ve \hl{capacit}y and innovation capability: an empirical study on Taiwan's financial and manufacturing industries\par
AU \tabto{0cm}Liao, Wu, Hu, Tsui\par
PY \tabto{0cm}2010, SO JOURNAL OF INFORMATION SCIENCE\par
DT \tabto{0cm}Article\par
PG \tabto{0cm}17, NR 51, TC 16\par
DE \tabto{0cm}\hl{ABSORPTI}VE \hl{CAPACIT}Y, INNOVATION CAPABILITY, KNOWLEDGE ACQUISITION, MODERATOR ANALYSIS, STRUCTURAL EQUATION MODELLING\par
ID \tabto{0cm}COMPETITIVE ADVANTAGE, CREATION, IMPACT, MANAGEMENT, ORGANIZATIONS, PERSPECTIVE, TECHNOLOGY\par
AB \tabto{0cm}This study investigates the relationships between knowledge acquisition, \hl{absorpti}ve capability, and innovation capability on Taiwan's knowledge-intensive industries using a structural equation model, which is constructed based on the data sampled from financial and manufacturing industries, and the 362 returned valid research samples. By testing five hypotheses, the research results find that \hl{absorpti}ve \hl{capacit}y is the mediator between knowledge acquisition and innovation capability, and that knowledge acquisition has a positive effect on \hl{absorpti}ve \hl{capacit}y. In addition, we used a multi-group approach and found that industry is a moderator between knowledge acquisition and innovation capability. Finally, a conclusion including research findings, discussion, implication, and future works is presented.\par
\clearpage

\vspace*{-2cm}
Nb \tabto{0cm}105/327 (article\_id: 432)\par
TI \tabto{0cm}\hl{Absorpti}ve \hl{capacit}y and the reach of collaboration in high technology small firms\par
AU \tabto{0cm}de Jong, Freel\par
PY \tabto{0cm}2010, SO RESEARCH POLICY\par
DT \tabto{0cm}Article\par
PG \tabto{0cm}8, NR 48, TC 31\par
DE \tabto{0cm}\hl{ABSORPTI}VE \hl{CAPACIT}Y, COGNITIVE DISTANCE, COLLABORATION, GEOGRAPHICAL DISTANCE\par
ID \tabto{0cm}CAPABILITY, COMPLEMENTARITY, EMBEDDEDNESS, EXPLORATION, INNOVATION, LINKAGES, NETWORKS, PERSPECTIVE, PROXIMITY, REGIONS\par
AB \tabto{0cm}The current paper is concerned with exploring the role of \hl{absorpti}ve \hl{capacit}y in extending the reach of innovation-related collaboration in high technology small firms. Drawing on survey data from a sample of 316 Dutch high-tech small firms, engaged in 1245 collaborations, we explore the relationship between R\&D expenditure and distance to collaboration partners. In general terms, we find most partners to be 'local'. However, controlling for a variety of potential influences, higher R\&D expenditure is positively related to collaboration with more distant organizations. The implications of our results for policy, practice and future research are discussed. (C) 2009 Elsevier B.V. All rights reserved.\par
\clearpage

\vspace*{-2cm}
Nb \tabto{0cm}106/327 (article\_id: 433)\par
TI \tabto{0cm}Building \hl{absorpti}ve \hl{capacit}y to organise inbound open innovation in traditional industries\par
AU \tabto{0cm}Spithoven, Clarysse, Knockaert\par
PY \tabto{0cm}2010, SO TECHNOVATION\par
DT \tabto{0cm}Article\par
PG \tabto{0cm}12, NR 60, TC 66\par
DE \tabto{0cm}\hl{ABSORPTI}VE \hl{CAPACIT}Y, OPEN INNOVATION, TECHNOLOGY INTERMEDIATION\par
ID \tabto{0cm}COMPETITIVE ADVANTAGE, EMPIRICAL-ANALYSIS, FIRMS, INTERMEDIATION, KNOWLEDGE, PERFORMANCE, RECONCEPTUALIZATION, RESOURCES, TECHNOLOGY-TRANSFER, TRANSFORMATION\par
AB \tabto{0cm}The discussion on open innovation suggests that the ability to absorb external knowledge has become a major driver for competition. For R\&D intensive large firms, the concept of open innovation in relation to \hl{absorpti}ve \hl{capacit}y is relatively well understood. Little attention has: however, been paid to how both small firms and firms, which operate in traditional sectors, engage in open innovation activities. The latter two categories of firms often dispose of no, or at most a relatively low level of, \hl{absorpti}ve \hl{capacit}y. Open innovation has two faces. In the case of inbound open innovation, companies screen their environment to search for technology and knowledge and do not exclusively rely on in-house R\&D. A key pre-condition is that firms dispose of "\hl{absorpti}ve \hl{capacit}y" to internalise external knowledge. SMEs and firms in traditional industries might need assistance in building \hl{absorpti}ve \hl{capacit}y. This paper focuses on the role of collective research centres in building \hl{absorpti}ve \hl{capacit}y at the inter-organisational level. In order to do so, primary data was collected through interviews with CEOs of these technology intermediaries and their member firms and analysed in combination with secondary data. The technology intermediaries discussed are created to help firms to take advantage of technological developments. The paper demonstrates that the openness of the innovation process forces firms lacking \hl{absorpti}ve \hl{capacit}y to search for alternative ways to engage in inbound open innovation. The paper highlights the multiple activities of which \hl{absorpti}ve \hl{capacit}y in intermediaries is made up; defines the concept of \hl{absorpti}ve \hl{capacit}y as a pre-condition to open innovation; and demonstrates how firms lacking \hl{absorpti}ve \hl{capacit}y collectively cope with distributed knowledge and innovation. (C) 2009 Elsevier Ltd. All rights reserved.\par
\clearpage

\vspace*{-2cm}
Nb \tabto{0cm}107/327 (article\_id: 434)\par
TI \tabto{0cm}\hl{Absorpti}ve \hl{Capacit}y and Social Capital in Regional Innovation Systems: The Case of the Lahti Region in Finland\par
AU \tabto{0cm}Kallio, Harmaakorpi, Pihkala\par
PY \tabto{0cm}2010, SO URBAN STUDIES\par
DT \tabto{0cm}Article\par
PG \tabto{0cm}17, NR 54, TC 8\par
DE \tabto{0cm}null\par
ID \tabto{0cm}CAPABILITY, IDEAS, KNOWLEDGE TRANSFER, PERFORMANCE, PERSPECTIVE, STRENGTH, WEAK TIES\par
AB \tabto{0cm}The recent theories of innovation suggest that there is great potential for innovation in the structural holes and weak links of the innovation system. Higher \hl{absorpti}ve \hl{capacit}y enables an easier crossing over of structural holes in the innovation system, aided by social capital that is located in the social relationships of actors. However, the level of human and social interaction in regional innovation systems has been largely neglected as a research topic. Empirical research on a sample in the Lahti region in Finland suggested three forms of social capital: organisational bonding social capital, regional bridging social capital and personal creative social capital. Further analysis revealed three groups of actors' interaction behaviour: Missionaries, House Mice and the Passive Resistance.\par
\clearpage

\vspace*{-2cm}
Nb \tabto{0cm}108/327 (article\_id: 435)\par
TI \tabto{0cm}Competing pressures of risk and \hl{absorpti}ve \hl{capacit}y potential on commitment and information sharing in global supply chains\par
AU \tabto{0cm}Arnold, Benford, Hampton, Sutton\par
PY \tabto{0cm}2010, SO EUROPEAN JOURNAL OF INFORMATION SYSTEMS\par
DT \tabto{0cm}Article\par
PG \tabto{0cm}19, NR 45, TC 6\par
DE \tabto{0cm}\hl{ABSORPTI}VE \hl{CAPACIT}Y, E-COMMERCE RISK, GLOBAL SUPPLY CHAIN, INFORMATION SHARING, INTERORGANIZATIONAL RELATIONSHIPS, RELATIONSHIP COMMITMENT\par
ID \tabto{0cm}ADOPTION, FIT, IMPACT, INNOVATION, MANAGEMENT, MODEL, NETWORKS, PERSPECTIVE, TECHNOLOGY\par
AB \tabto{0cm}Organizations' competitiveness and success are no longer dependent solely on their own performance, but rather are dependent on the competitiveness of the supply chains in which they participate. Increasingly, these supply chains are globally distributed introducing the possibility of greater benefits, as well as greater risk. This study examines the countervailing impact of a global supply chain partner's business-to-business e-commerce business risk and \hl{absorpti}ve \hl{capacit}y on an organization's willingness to commit to and share information with that supply chain partner. We survey 207 organizations on their perceptions of specific offshore outsourcing and supply chain partners across dimensions of risk, \hl{absorpti}ve \hl{capacit}y, commitment, and information sharing. The results support the theorized relationships indicating that a supply chain partner's increased levels of perceived risk has a strong negative effect on an organization's commitment and information sharing; conjointly, increases in a supply chain partner's \hl{absorpti}ve \hl{capacit}y has a strong positive effect on commitment and information sharing. For both risk and \hl{absorpti}ve \hl{capacit}y, commitment partially mediates the relationship with information sharing. Testing for systemic effects from geographical/cultural location on the relationship factors provides no evidence of a regional effect on measured items. European Journal of Information Systems (2010) 19, 134-152. doi: 10.1057/ejis.2009.49; published online 19 January 2010\par
\clearpage

\vspace*{-2cm}
Nb \tabto{0cm}109/327 (article\_id: 436)\par
TI \tabto{0cm}On the implementation of a 'global' environmental strategy: The role of \hl{absorpti}ve \hl{capacit}y\par
AU \tabto{0cm}Pinkse, Kuss, Hoffmann\par
PY \tabto{0cm}2010, SO INTERNATIONAL BUSINESS REVIEW\par
DT \tabto{0cm}Article\par
PG \tabto{0cm}18, NR 63, TC 12\par
DE \tabto{0cm}\hl{ABSORPTI}VE \hl{CAPACIT}Y, ENVIRONMENTAL MANAGEMENT, HEADQUARTERS-SUBSIDIARY RELATIONSHIP, KNOWLEDGE SHARING, SUSTAINABLE DEVELOPMENT\par
ID \tabto{0cm}ALLIANCES, CAPABILITIES, DETERMINANTS, ENTERPRISES, FIRM, KNOWLEDGE TRANSFER, MULTINATIONAL-CORPORATION, PERSPECTIVE, RECONCEPTUALIZATION, TECHNOLOGY\par
AB \tabto{0cm}This paper sheds light on factors influencing to what extent MNCs are able to implement a global environmental strategy. We apply the concept of \hl{absorpti}ve \hl{capacit}y to analyze what role the uptake and integration of external knowledge plays in implementing an environmental strategy and propose to make a distinction between shared and unit-specific levels of \hl{absorpti}ve \hl{capacit}y. Based on an in-depth investigation of the multinational chemical company BASF, we derive three propositions about the influence of \hl{absorpti}ve \hl{capacit}y on the implementation of global environmental practices on the regional headquarters and subsidiary level. The main finding is that a shared level of \hl{absorpti}ve \hl{capacit}y across subsidiaries facilitates a common understanding and use of environment-related knowledge, but, as environment-related knowledge often applies to a specific context only, there is also a need to build unit-specific \hl{absorpti}ve \hl{capacit}y on a subsidiary level. By allowing subsidiaries to build their \hl{absorpti}ve \hl{capacit}y, MNCs can more efficiently adapt global environmental practices and lower the cost of implementing a global environmental standard. (C) 2009 Elsevier Ltd. All rights reserved.\par
\clearpage

\vspace*{-2cm}
Nb \tabto{0cm}110/327 (article\_id: 437)\par
TI \tabto{0cm}An Examination of the Relationship Between \hl{Absorpti}ve \hl{Capacit}y and Organizational Learning, and a Proposed Integration\par
AU \tabto{0cm}Sun, Anderson\par
PY \tabto{0cm}2010, SO INTERNATIONAL JOURNAL OF MANAGEMENT REVIEWS\par
DT \tabto{0cm}Review\par
PG \tabto{0cm}21, NR 102, TC 29\par
DE \tabto{0cm}null\par
ID \tabto{0cm}DYNAMIC CAPABILITIES, EMPLOYEE CREATIVITY, INNOVATION, JOINT VENTURES, KNOWLEDGE TRANSFER, LIFE-CYCLE, MANAGEMENT, PERSPECTIVE, STRATEGIC ALLIANCES, WORK COMMITMENT\par
AB \tabto{0cm}Since its inception, the concept of \hl{absorpti}ve \hl{capacit}y has been closely linked with notions of organizational learning. Yet the precise nature of the relationship between these two concepts has never been established. This relationship is examined in a variety of ways, and it is suggested that the literature on these two concepts shares a conceptual affinity which needs to be delineated. It is suggested that \hl{absorpti}ve \hl{capacit}y (a dynamic capability) is a concrete example of organizational learning that concerns an organization's relationship with new external knowledge. Using the 4I Model for organizational learning (Crossan, M.M., Lane, H.W. and White, R.E. (1999). An organizational learning framework: from intuition to institution. Academy of Management Review, 24, 522-537) and Zahra and George's conceptualization of \hl{absorpti}ve \hl{capacit}y (Zahra, S.A. and George, G. (2002). \hl{Absorpti}ve \hl{capacit}y: a review, reconceptualization, and extension. Academy of Management Review, 27, 185-203), this paper proposes an integration of the two concepts.\par
\clearpage

\vspace*{-2cm}
Nb \tabto{0cm}111/327 (article\_id: 438)\par
TI \tabto{0cm}Knowledge arbitrage in global pharma: a synthetic view of \hl{absorpti}ve \hl{capacit}y and open innovation\par
AU \tabto{0cm}Hughes, Wareham\par
PY \tabto{0cm}2010, SO R \& D MANAGEMENT\par
DT \tabto{0cm}Article\par
PG \tabto{0cm}20, NR 89, TC 22\par
DE \tabto{0cm}null\par
ID \tabto{0cm}DYNAMIC CAPABILITIES, INTERORGANIZATIONAL COLLABORATION, LIFE SCIENCES, NEW-MODEL, PERFORMANCE, RECONCEPTUALIZATION, RESEARCH-AND-DEVELOPMENT, STRATEGY, TECHNOLOGY, TOOLKITS\par
AB \tabto{0cm}This case study examines a global pharmaceutical company widely using open innovation (OI). Three main research questions are addressed: (1) what OI concepts are salient in their innovation portfolio?, (2) what OI concepts are used in the strategy formulation? and (3) what other concepts are present that augment OI? Interviews with 120 managers and archival documents were analyzed using thematic analysis. Two concepts prominent in the literature, (i) value capture models and (ii) technology evaluation criteria, were not present in this portfolio. By contrast, we found a focus on OI capability building, external information sharing and uncertain knowledge arbitrage in networks. Finally, we discuss these capabilities in relation to \hl{absorpti}ve \hl{capacit}y, proposing a simple, but important bi-directional perspective to embrace OI.\par
\clearpage

\vspace*{-2cm}
Nb \tabto{0cm}112/327 (article\_id: 439)\par
TI \tabto{0cm}Knowledge \hl{absorpti}ve \hl{capacit}y: New insights for its conceptualization and measurement\par
AU \tabto{0cm}Camison, Fores\par
PY \tabto{0cm}2010, SO JOURNAL OF BUSINESS RESEARCH\par
DT \tabto{0cm}Article\par
PG \tabto{0cm}9, NR 64, TC 46\par
DE \tabto{0cm}\hl{ABSORPTI}VE \hl{CAPACIT}Y, DYNAMIC COMPETENCES, KNOWLEDGE MANAGEMENT, MULTI-ITEM MEASUREMENT SCALES, STRUCTURAL EQUATION MODELING\par
ID \tabto{0cm}ANTECEDENTS, COMBINATIVE CAPABILITIES, COMPETITIVE ADVANTAGE, CONSTRUCTS, FIRM, INNOVATION, INSTRUMENTS, PERFORMANCE, RECONCEPTUALIZATION, TECHNOLOGY\par
AB \tabto{0cm}The processes for absorbing external knowledge become an essential element for innovation in firms and in adapting to changes in the competitive environment. Despite the huge growth in the \hl{absorpti}ve \hl{capacit}y literature, a methodological gap still remains about a certain ambiguity in the definition of the construct specifying its theoretical domain and dimensionalization, and a lack of validation of the construct in most studies. The aim of this paper is to contribute to the literature on \hl{absorpti}ve \hl{capacit}y through the creation and validation of two scales, justified with a thorough analysis of the literature, to measure the key components of the \hl{absorpti}ve \hl{capacit}y construct: potential and realized \hl{absorpti}ve \hl{capacit}ies. The study includes confirmatory factor analysis on a sample of 952 Spanish firms to verify that the scales meet the psychometric properties the literature requires. The study results confirm the validity of the proposed scales and support their consolidation as a commonly used instrument with which to measure \hl{absorpti}ve \hl{capacit}y. (C) 2009 Elsevier Inc. All rights reserved.\par
\clearpage

\vspace*{-2cm}
Nb \tabto{0cm}113/327 (article\_id: 440)\par
TI \tabto{0cm}Absorbing the Concept of \hl{Absorpti}ve \hl{Capacit}y: How to Realize Its Potential in the Organization Field\par
AU \tabto{0cm}Volberda, Foss, Lyles\par
PY \tabto{0cm}2010, SO ORGANIZATION SCIENCE\par
DT \tabto{0cm}Article\par
PG \tabto{0cm}21, NR 114, TC 164\par
DE \tabto{0cm}\hl{ABSORPTI}VE \hl{CAPACIT}Y, KNOWLEDGE MANAGEMENT, MICROFOUNDATIONS OF \hl{ABSORPTI}VE \hl{CAPACIT}Y, MULTIPLE ANALYTICAL LEVELS, ORGANIZATIONAL CAPABILITIES\par
ID \tabto{0cm}COMBINATIVE CAPABILITIES, DOMINANT LOGIC, DYNAMIC CAPABILITIES, EMPIRICAL-TEST, FORTUNE FAVORS, INTERNATIONAL-JOINT-VENTURES, KNOWLEDGE TRANSFER, RESEARCH-AND-DEVELOPMENT, STRATEGIC RENEWAL, TECHNOLOGY-TRANSFER\par
AB \tabto{0cm}The purpose of this perspective paper is to advance understanding of \hl{absorpti}ve \hl{capacit}y, its underlying dimensions, its multilevel antecedents, its impact on firm performance, and the contextual factors that affect \hl{absorpti}ve \hl{capacit}y. Twenty years after the Cohen and Levinthal 1990 paper, the field is characterized by a wide array of theoretical perspectives and a wealth of empirical evidence. In this paper, we first review these underlying theories and empirical studies of \hl{absorpti}ve \hl{capacit}y. Given the size and diversity of the \hl{absorpti}ve \hl{capacit}y literature, we subsequently map the existing terrain of research through a bibliometric analysis. The resulting bibliometric cartography shows the major discrepancies in the organization field, namely that (1) most attention so far has been focused on the tangible outcomes of \hl{absorpti}ve \hl{capacit}y; (2) organizational design and individual level antecedents have been relatively neglected in the \hl{absorpti}ve \hl{capacit}y literature; and (3) the emergence of \hl{absorpti}ve \hl{capacit}y from the actions and interactions of individual, organizational, and interorganizational antecedents remains unclear. Building on the bibliometric analysis, we develop an integrative model that identifies the multilevel antecedents, process dimensions, and outcomes of \hl{absorpti}ve \hl{capacit}y as well as the contextual factors that affect \hl{absorpti}ve \hl{capacit}y. We argue that realizing the potential of the \hl{absorpti}ve \hl{capacit}y concept requires more research that shows how "micro-antecedents" and " macro-antecedents" influence future outcomes such as competitive advantage, innovation, and firm performance. In particular, we identify conceptual gaps that may guide future research to fully exploit the \hl{absorpti}ve \hl{capacit}y concept in the organization field and to explore future fruitful extensions of the concept.\par
\clearpage

\vspace*{-2cm}
Nb \tabto{0cm}114/327 (article\_id: 441)\par
TI \tabto{0cm}Technology Transfer across Organizational Boundaries: \hl{ABSORPTI}VE \hl{CAPACIT}Y AND DESORPTIVE \hl{CAPACIT}Y\par
AU \tabto{0cm}Lichtenthaler\par
PY \tabto{0cm}2010, SO CALIFORNIA MANAGEMENT REVIEW\par
DT \tabto{0cm}Article\par
PG \tabto{0cm}18, NR 54, TC 25\par
DE \tabto{0cm}null\par
ID \tabto{0cm}ANTECEDENTS, DYNAMIC CAPABILITIES, INDUSTRY, INTELLECTUAL PROPERTY, KNOWLEDGE TRANSFER, LICENSING STRATEGIES, MODEL, OPEN INNOVATION, PERFORMANCE, PERSPECTIVE\par
AB \tabto{0cm}In light of the trend towards open innovation, interorganizational technology transfer by means of alliances and licensing has often become a key component of open innovation processes. Inbound open innovation describes inward technology transfer, whereas outbound open innovation refers to outward technology transfer. Traditionally, inward technology transfer has received considerable attention because most practitioners and academics focus on technology recipient's \hl{absorpti}ve \hl{capacit}y. In contrast, the role of the technology source has been relatively neglected. This article addresses the concept of desorptive \hl{capacit}y, which refers to a firm's ability to identify technology transfer opportunities and to transfer technology to the recipient. The notion of market knowledge in the concept of desorptive \hl{capacit}y deepens our understanding of many firms' managerial difficulties in implementing active technology transfer strategies. Thus, desorptive \hl{capacit}y enriches our understanding of the dynamics of outward technology transfer. It provides new insights into the success or failure of interorganizational technology transactions.\par
\clearpage

\vspace*{-2cm}
Nb \tabto{0cm}115/327 (article\_id: 442)\par
TI \tabto{0cm}FDI SPILLOVERS IN AN EMERGING MARKET: THE ROLE OF FOREIGN FIRMS' COUNTRY ORIGIN DIVERSITY AND DOMESTIC FIRMS' \hl{ABSORPTI}VE \hl{CAPACIT}Y\par
AU \tabto{0cm}Zhang, Li, Zhou\par
PY \tabto{0cm}2010, SO STRATEGIC MANAGEMENT JOURNAL\par
DT \tabto{0cm}Article\par
PG \tabto{0cm}21, NR 64, TC 47\par
DE \tabto{0cm}\hl{ABSORPTI}VE \hl{CAPACIT}Y, EMERGING MARKET, FDI SPILLOVERS, TECHNOLOGY GAP, THE DIVERSITY OF FDI COUNTRY ORIGINS\par
ID \tabto{0cm}CHINA, COMPETITION, DIRECT-INVESTMENT, INNOVATION, MANUFACTURING SECTOR, PERFORMANCE, PRODUCTIVITY SPILLOVERS, RESEARCH-AND-DEVELOPMENT, STRATEGY, TECHNOLOGY SPILLOVERS\par
AB \tabto{0cm}Prior literature on foreign direct investment (FDI) spillovers has mainly focused on how the presence of FDI affects the productivity of domestic firms. In this study, we advance the literature by examining the effect of the diversity of FDI country origins on the productivity of domestic firms. We propose that the diversity of FDI country origins can facilitate FDI spillovers by increasing the variety of technologies and management practices brought by foreign firms, to which domestic firms are exposed and that they can potentially utilize. Further, the extent to which domestic firms can utilize these technologies and practices depends upon their \hl{absorpti}ve \hl{capacit}y. Using panel data on Chinese manufacturing firms during the period 1998-2003, our results support these propositions. We,find that the diversity of FDI country origins in an industry has a positive relationship with the productivity of domestic firms in the industry. This positive relationship is stronger when domestic firms are larger, and when the technology gap between FDI and the domestic firms is intermediate. Copyright (C) 2010 John Wiley \& Sons, Ltd.\par
\clearpage

\vspace*{-2cm}
Nb \tabto{0cm}116/327 (article\_id: 443)\par
TI \tabto{0cm}\hl{Absorpti}ve \hl{Capacit}y in R\&D Project Teams: A Conceptualization and Empirical Test\par
AU \tabto{0cm}Nemanich, Keller, Vera, Chin\par
PY \tabto{0cm}2010, SO IEEE TRANSACTIONS ON ENGINEERING MANAGEMENT\par
DT \tabto{0cm}Article\par
PG \tabto{0cm}15, NR 85, TC 11\par
DE \tabto{0cm}\hl{ABSORPTI}VE \hl{CAPACIT}Y, AUTONOMY, COMPUTER INDUSTRY, EXTERNAL KNOWLEDGE, PRIOR KNOWLEDGE, R\&D TEAMS, SHARED MENTAL MODELS\par
ID \tabto{0cm}CENTRIPETAL FORCES, DYNAMIC CAPABILITIES, FIRMS, INDUSTRY, INNOVATION, KNOWLEDGE CREATION, ORGANIZATIONS, PERFORMANCE, PRODUCT DEVELOPMENT, TECHNOLOGY\par
AB \tabto{0cm}The purpose of this study is to answer a call for the rejuvenation of the \hl{absorpti}ve \hl{capacit}y (ACAP) construct by offering a novel conceptualization and empirical test of a multidimensional model of R\&D project team ACAP that portrays it as a capability distinct from prior knowledge, specifies each dimension's level of analysis, distinguishes between individual and collective assimilation, and considers the moderating effects of team structure. Using a dataset from survey and archival sources on 100 innovations by R\&D project teams, we find that the capability of R\&D team members to evaluate external knowledge is related to their ability to assimilate it and that both individual assimilation capabilities and collective assimilation capabilities, in the form of ability to reach a shared understanding, are important to the team's ability to apply external knowledge. We also find that prior knowledge negatively moderates the relationship between individual assimilation and application ability and that team autonomy positively moderates this relationship. By clarifying levels of analysis and encompassing multiple dimensions of ACAP, this work leads to a more fine-grained understanding of the complex nature of ACAP. Implications of these findings for future research and R\&D team management are presented.\par
\clearpage

\vspace*{-2cm}
Nb \tabto{0cm}117/327 (article\_id: 444)\par
TI \tabto{0cm}An exploratory study of Principal Investigator roles in UK university Proof-of-Concept processes: an \hl{Absorpti}ve \hl{Capacit}y perspective\par
AU \tabto{0cm}McAdam, Galbraith, Miller\par
PY \tabto{0cm}2010, SO R \& D MANAGEMENT\par
DT \tabto{0cm}Article\par
PG \tabto{0cm}19, NR 54, TC 7\par
DE \tabto{0cm}null\par
ID \tabto{0cm}ACADEMIC ENTREPRENEURS, BIOTECHNOLOGY, INNOVATION, KNOWLEDGE, LINKAGES, PERFORMANCE, RECONCEPTUALIZATION, SCIENCE, TECHNOLOGY-TRANSFER OFFICES, VENTURES\par
AB \tabto{0cm}The increasing emphasis on academic entrepreneurship, technology transfer and research commercialisation within UK universities is predicated on basic research being developed by academics into commercial entities such as university spin-off companies or licensing arrangements. However, this process is fraught with challenges and risks, given the degree of uncertainty regarding future returns. In an attempt to minimise such risks, the Proof-of-Concept (PoC) process has been developed within University Science Park Incubators (USIs) to test the technological, business and market potential of embryonic technology. The key or the pivotal stakeholder within the PoC is the Principal Investigator (PI), who is usually the lead academic responsible for the embryonic technology. Within the current literature, there appears to be a lack of research pertaining to the role of the PI in the PoC process. Moreover, \hl{Absorpti}ve \hl{Capacit}y (ACAP) has emerged within the literature as a theoretical framework or lens for exploring the development and application of new knowledge and technology, where the USI is the organisation considered in the current study. Therefore, the aim of this paper is to explore the role and influence of the PI in the PoC process within a USI setting using an ACAP perspective. The research involved a multiple case analysis of PoC applications within a UK university USI. The results demonstrate the role of the PI in developing practices and routines within the PoC process. These practices and processes were initially tacit and informal in nature but became more explicit and formal over time so that knowledge was retained within the USI after the PIs had completed the PoC process.\par
\clearpage

\vspace*{-2cm}
Nb \tabto{0cm}118/327 (article\_id: 445)\par
TI \tabto{0cm}Learning processes in municipal broadband projects: An \hl{absorpti}ve \hl{capacit}y perspective\par
AU \tabto{0cm}Techatassanasoontorn, Tapia, Powell\par
PY \tabto{0cm}2010, SO TELECOMMUNICATIONS POLICY\par
DT \tabto{0cm}Article\par
PG \tabto{0cm}24, NR 82, TC 2\par
DE \tabto{0cm}\hl{ABSORPTI}VE \hl{CAPACIT}Y, BROADBAND TECHNOLOGY, INFRASTRUCTURE DEVELOPMENT, MUNICIPAL WIRELESS NETWORK, PUBLIC ORGANIZATION, WI-FI\par
ID \tabto{0cm}AGILITY, CAPABILITY, FIRM PERFORMANCE, INFORMATION-SYSTEMS DEVELOPMENT, INFRASTRUCTURE, INNOVATION, INTERNATIONAL JOINT VENTURES, KNOWLEDGE TRANSFER, PERSONNEL, TECHNOLOGY\par
AB \tabto{0cm}Effective knowledge management is important to the success of information technology projects. This research applies the integrated lens of the \hl{absorpti}ve \hl{capacit}y theory and the social process model of information system development projects to examine the dynamic of knowledge activities concerning broadband infrastructure development in the context of municipal broadband networks. The research questions are: (1) What is the extent of the dynamic of knowledge activities involved in the development process?, (2) What are the events that trigger knowledge activities in municipal broadband development?, and (3) How does a city create and utilize new knowledge in the development process? This study examines municipal wireless projects in three cities (Chaska, MN; Hermosa Beach, CA; and Fredericton, Canada). Events that trigger knowledge activities are assignment of personnel, physical system construction, performance problems, resistance, and reassignment of organizational roles. Four factors that influence knowledge activities and project performance are the dynamic of technology development, partnership commitments, limitation of external knowledge and learning-by-doing, and political dynamics. The study has policy implications for cities that are in the process of planning and deployment. A good project planning, user expectation management, systematic performance evaluation, a careful partner selection process, and the use of service level agreements are important to project success. In addition, cities need to take into consideration that the technology is not a plug and play technology and that considerable efforts are needed to integrate the technology with other solutions to deliver broadband services as well as to configure the system according to topologies, street conditions, buildings, density of trees, among others. (C) 2010 Elsevier Ltd. All rights reserved.\par
\clearpage

\vspace*{-2cm}
Nb \tabto{0cm}119/327 (article\_id: 446)\par
TI \tabto{0cm}\hl{Absorpti}ve \hl{capacit}y, foreign direct investment-linked spillovers and economic growth in Vietnam\par
AU \tabto{0cm}Anwar, Lan\par
PY \tabto{0cm}2010, SO ASIAN BUSINESS \& MANAGEMENT\par
DT \tabto{0cm}Article\par
PG \tabto{0cm}18, NR 52, TC 4\par
DE \tabto{0cm}\hl{ABSORPTI}VE \hl{CAPACIT}Y, ECONOMIC GROWTH, FDI-LINKED HORIZONTAL AND VERTICAL LINKAGES, FIXED EFFECT POOLED REGRESSION, FOREIGN DIRECT INVESTMENT, VIETNAM\par
ID \tabto{0cm}COUNTRIES, DOMESTIC FIRMS, EFFICIENCY, FDI, INDUSTRIES, LINKAGES, MULTINATIONAL-ENTERPRISES, PARTICIPATION, PRODUCTIVITY SPILLOVERS, TRADE\par
AB \tabto{0cm}Before the introduction of the economic reform process in the late 1980s, the Vietnamese economy was operating under a centrally planned system. The reform process resulted in the integration of Vietnam into the world economy, which led to significant increase in foreign investment. By making use of panel data on 22 manufacturing industries over the period 1995-2005, this article examines the impact of foreign direct investment (FDI)-generated spillovers on manufacturing sector growth in Vietnam. Unlike most existing studies, which are limited to considering the effect of FDI on economic growth, this article focuses on the impact of FDI-linked spillovers that take place through both horizontal and vertical linkages between domestic and foreign firms. The empirical results presented here suggest that FDI-generated spillovers have made a significant contribution to manufacturing sector growth in Vietnam through vertical-backward linkages. The positive impact of vertical-backward linkages on manufacturing sector growth is strengthened by the stock of human capital. Specifically, manufacturing industries with a larger stock of human capital have experienced a higher level of technological advancement and hence stronger economic growth. Asian Business \& Management (2010) 9, 553-570. doi: 10.1057/abm.2010.28\par
\clearpage

\vspace*{-2cm}
Nb \tabto{0cm}120/327 (article\_id: 447)\par
TI \tabto{0cm}Dollars Dollars Everywhere, Nor Any Dime to Lend: Credit Limit Constraints on Financial Sector \hl{Absorpti}ve \hl{Capacit}y\par
AU \tabto{0cm}Khwaja, Mian, Zia\par
PY \tabto{0cm}2010, SO REVIEW OF FINANCIAL STUDIES\par
DT \tabto{0cm}Article\par
PG \tabto{0cm}43, NR 32, TC 2\par
DE \tabto{0cm}E22, E44, G21\par
ID \tabto{0cm}CASH FLOW SENSITIVITIES, FIRMS, INVESTMENT, LIQUIDITY, MANAGEMENT, MARKET, MONETARY-POLICY, SHOCKS, TRANSMISSION\par
AB \tabto{0cm}We exploit an unexpected inflow of liquidity in an emerging market to study how capital is intermediated to firms. We find that backward-looking credit limit constraints imposed by banks make it difficult for firms to borrow, despite readily available bank liquidity, healthy aggregate demand, and a sharply falling cost of capital. The resulting aggregate failure to extend and retain capital in the economy suggests that agency costs that force banks to rely on sticky balance-sheet-based credit limits prevent emerging economies from effectively intermediating capital.\par
\clearpage

\vspace*{-2cm}
Nb \tabto{0cm}121/327 (article\_id: 448)\par
TI \tabto{0cm}Social capital, internationalization and \hl{absorpti}ve \hl{capacit}y: The electronics and ICT cluster of the Basque Country\par
AU \tabto{0cm}Valdaliso, Elola, Aranguren, Lopez\par
PY \tabto{0cm}2011, SO ENTREPRENEURSHIP AND REGIONAL DEVELOPMENT\par
DT \tabto{0cm}Article\par
PG \tabto{0cm}27, NR 107, TC 13\par
DE \tabto{0cm}\hl{ABSORPTI}VE \hl{CAPACIT}Y, CLUSTERS, INTERNATIONALIZATION, KNOWLEDGE, SOCIAL CAPITAL\par
ID \tabto{0cm}COLLECTIVE LEARNING-PROCESSES, COMPETITIVENESS, EVOLUTION, FIRM, INDUSTRIAL DISTRICTS, INNOVATION, KNOWLEDGE DEVELOPMENT, NETWORKS, PERSPECTIVE, REGIONAL CLUSTERS\par
AB \tabto{0cm}This article analyses the case of a successful young high technology cluster in an old industrialized European region, the electronics and information and communications technology cluster in the Basque Country (Spain). Based on the findings of this case study, we propose that social capital and internationalization play an important role in increasing the \hl{absorpti}ve \hl{capacit}y of clusters (thus, the \hl{capacit}y of a cluster to absorb, diffuse and creatively exploit extra-cluster knowledge), and hence, in sustaining their growth and dynamism. \hl{Absorpti}ve \hl{capacit}y depends on the \hl{capacit}y of firms to establish intra-and extra-cluster knowledge linkages. We put forward in this article the fact that social capital fosters intra-cluster knowledge linkages, and cluster's internationalization the extra-cluster knowledge ones. Therefore, social capital and internationalization are key elements to increase the \hl{absorpti}ve \hl{capacit}y of a cluster and its growth. Given the accumulative, path-and place-dependent nature of social capital and knowledge creation and accumulation, we employed a largely qualitative and historical analysis, combining statistical and qualitative cluster data and interviews with key actors.\par
\clearpage

\vspace*{-2cm}
Nb \tabto{0cm}122/327 (article\_id: 449)\par
TI \tabto{0cm}A Functional Perspective on Learning and Innovation: Investigating the Organization of \hl{Absorpti}ve \hl{Capacit}y\par
AU \tabto{0cm}Bogers, Lhuillery\par
PY \tabto{0cm}2011, SO INDUSTRY AND INNOVATION\par
DT \tabto{0cm}Article\par
PG \tabto{0cm}30, NR 117, TC 10\par
DE \tabto{0cm}\hl{ABSORPTI}VE \hl{CAPACIT}Y, FUNCTIONAL STRATEGIES, OPEN INNOVATION, PROCESS INNOVATION, PRODUCT INNOVATION\par
ID \tabto{0cm}DEVELOPMENT COOPERATION, EMPIRICAL-EVIDENCE, KNOWLEDGE TRANSFER, MARKET ORIENTATION, PRODUCT DEVELOPMENT, PROJECT PERFORMANCE, RESEARCH-AND-DEVELOPMENT, RESOURCE-BASED VIEW, STRATEGIC ALLIANCES, UK MANUFACTURING FIRMS\par
AB \tabto{0cm}We investigate the intra-organizational antecedents of firm-level \hl{absorpti}ve \hl{capacit}y (AC). Specifically, we examine how the functional areas of R\&D, manufacturing and marketing contribute to the \hl{absorpti}on of knowledge coming from different external knowledge sources. The econometric results on a representative sample of Swiss firms show that non-R\&D-based AC plays a significantly different role compared to the standard R\&D-based one that is typically considered in studies on AC. We also reveal that AC is organized through a specialization of external knowledge \hl{absorpti}on across functional areas. In particular, we find: (1) R\&D is particularly important as an absorber of knowledge from public research organizations for product innovation; (2) manufacturing is important as an absorber of supplier knowledge for product innovation and of competitor knowledge for process innovation; and (3) marketing helps to absorb customer knowledge for product and process innovation as well as competitor knowledge for product innovation. We further investigate the differences between product and process innovation and find that marketing-based AC is more important for the former, although the overall analysis of these differences is less conclusive. In short, we show how functional areas play a role in the organization of AC and that firms may need an ambidextrous strategy to innovate effectively based on both upstream-and downstream-based AC.\par
\clearpage

\vspace*{-2cm}
Nb \tabto{0cm}123/327 (article\_id: 450)\par
TI \tabto{0cm}The complementary effect of internal learning \hl{capacit}y and \hl{absorpti}ve \hl{capacit}y on performance: the mediating role of innovation \hl{capacit}y\par
AU \tabto{0cm}Fores, Camison\par
PY \tabto{0cm}2011, SO INTERNATIONAL JOURNAL OF TECHNOLOGY MANAGEMENT\par
DT \tabto{0cm}Article\par
PG \tabto{0cm}26, NR 76, TC 3\par
DE \tabto{0cm}BUSINESS PERFORMANCE, DYNAMIC \hl{CAPACIT}IES, KNOWLEDGE MANAGEMENT, STRUCTURAL EQUATION MODELLING\par
ID \tabto{0cm}COMBINATIVE CAPABILITIES, COMPETITIVE ADVANTAGE, CONSTRUCT, EXTERNAL KNOWLEDGE ACQUISITION, FIRM INNOVATION, INDUSTRY, MARKET, ORGANIZATIONS, PRODUCT DEVELOPMENT, TECHNOLOGY\par
AB \tabto{0cm}Organisations are finding it increasingly more difficult to keep abreast with the pace of change. The continuous rise in the number of business opportunities and the increase in global competition require firms to combine internal and external learning processes to renew and reconfigure existing capabilities and knowledge to enable them to meet environmental demands and to innovate. This study aims to unravel the complex linkage between internal learning \hl{capacit}y and \hl{absorpti}ve \hl{capacit}y and at exploring the joint effect of both knowledge generation processes on innovation \hl{capacit}y. This study also proposes innovation \hl{capacit}y as an antecedent of business performance. Using data from 952 industrial Spanish firms and the technique of structural equation modelling, we provide evidence on the joint effect of internal learning \hl{capacit}y and \hl{absorpti}ve \hl{capacit}y on innovation \hl{capacit}y. We also show that innovation \hl{capacit}y acts as a catalyst for the effect of learning \hl{capacit}ies on business performance.\par
\clearpage

\vspace*{-2cm}
Nb \tabto{0cm}124/327 (article\_id: 451)\par
TI \tabto{0cm}Linking properties of knowledge with innovation performance: the moderate role of \hl{absorpti}ve \hl{capacit}y\par
AU \tabto{0cm}Wang, Han\par
PY \tabto{0cm}2011, SO JOURNAL OF KNOWLEDGE MANAGEMENT\par
DT \tabto{0cm}Article\par
PG \tabto{0cm}18, NR 56, TC 8\par
DE \tabto{0cm}\hl{ABSORPTI}VE \hl{CAPACIT}Y, CHINA, INNOVATION, INNOVATION PERFORMANCE, KNOWLEDGE ECONOMY, KNOWLEDGE MANAGEMENT, MANAGEMENT INNOVATION, MODERATED MULTIPLE REGRESSION, PROPERTIES OF KNOWLEDGE, SMALL TO MEDIUM-SIZED ENTERPRISES, TECHNOLOGY INNOVATION\par
ID \tabto{0cm}AMBIGUITY, CAPABILITIES, COMPETITIVE ADVANTAGE, CREATION, FIRM, KNOW-HOW, NETWORK POSITION, ORGANIZATIONS, STRATEGIC ALLIANCES, TECHNOLOGY-TRANSFER\par
AB \tabto{0cm}Purpose - The purpose of this paper is to unravel the complex linkages between properties of knowledge, firm's \hl{absorpti}ve \hl{capacit}ies, and innovation performance in China small and medium-sized enterprises (SMEs).
Design/methodology/approach - The authors utilize moderated multiple regression (MMR) to explore the complex relationships by using the empirical data of Chinese information technology and electro-communication firms during 2005-2008.
Findings - The results show that only a few knowledge properties have negative effects on innovation performance. Most knowledge properties have a positive effect on innovation. The results also claim that the relationship between properties of knowledge and innovation performance is more pronounced when the firm has higher \hl{absorpti}ve \hl{capacit}y
Originality/value - The study indicates that properties of knowledge and \hl{absorpti}ve \hl{capacit}y are two inseparable determinants for innovation performance, and \hl{absorpti}ve \hl{capacit}y moderates the relationship between properties of knowledge and innovation performance.\par
\clearpage

\vspace*{-2cm}
Nb \tabto{0cm}125/327 (article\_id: 452)\par
TI \tabto{0cm}Knowledge \hl{absorpti}ve \hl{capacit}y and innovation performance in KIBS\par
AU \tabto{0cm}Tseng, Pai, Hung\par
PY \tabto{0cm}2011, SO JOURNAL OF KNOWLEDGE MANAGEMENT\par
DT \tabto{0cm}Article\par
PG \tabto{0cm}13, NR 41, TC 20\par
DE \tabto{0cm}INNOVATION, INNOVATION PERFORMANCE, KNOWLEDGE \hl{ABSORPTI}VE \hl{CAPACIT}Y, KNOWLEDGE INPUT, KNOWLEDGE MANAGEMENT, KNOWLEDGE SPILLOVER, KNOWLEDGE-INTENSIVE BUSINESS SERVICES (KIBS)\par
ID \tabto{0cm}FIRM, INDICATORS, PATENT, PERSPECTIVE, RESEARCH-AND-DEVELOPMENT, SPILLOVERS, TECHNOLOGY\par
AB \tabto{0cm}Purpose - The purpose of this paper is to discuss whether the three knowledge sources, knowledge input, knowledge spillover and knowledge \hl{absorpti}ve \hl{capacit}y really increase the innovation performance of firms in the Taiwan IC design industry, one of the most important knowledge-intensive business services (KIBS) industries in Taiwan.
Design/methodology/approach - Based on the knowledge-based theory, this study uses pooled regression analysis and tests with fixed effect model to analyze the influence of three knowledge sources on innovation performance in the KIBS sector.
Findings - The results demonstrate that: knowledge input is positively related to innovation performance; knowledge spillover effect is partial positively to innovation performance; and knowledge \hl{absorpti}ve \hl{capacit}y is positively related to innovation performance.
Originality/value - The paper advances the concept of \hl{absorpti}ve \hl{capacit}y by defining it as the interactions between knowledge input and knowledge spillover and refines the measurement of \hl{absorpti}ve \hl{capacit}y as the multiplication of knowledge input and knowledge spillover effects. Moreover, knowledge spillover effects and knowledge \hl{absorpti}ve \hl{capacit}y are both divided into four kinds which help us distinguish clearly different sources of knowledge spillover and \hl{absorpti}ve \hl{capacit}y In addition to that, this study also contributes to the empirical evidence to innovation activities by using firm-level micro data.\par
\clearpage

\vspace*{-2cm}
Nb \tabto{0cm}126/327 (article\_id: 453)\par
TI \tabto{0cm}Multilevel \hl{absorpti}ve \hl{capacit}y and interorganizational new product development: A process study\par
AU \tabto{0cm}Newey, Verreynne\par
PY \tabto{0cm}2011, SO JOURNAL OF MANAGEMENT \& ORGANIZATION\par
DT \tabto{0cm}Article\par
PG \tabto{0cm}17, NR 36, TC 5\par
DE \tabto{0cm}\hl{ABSORPTI}VE \hl{CAPACIT}Y, CROSS-LEVEL INTERACTIONS, INTERORGANIZATIONAL NEW PRODUCT DEVELOPMENT SYSTEMS, LONGITUDINAL CASE STUDY, THEORY-BUILDING\par
ID \tabto{0cm}ADVANTAGE, BIOTECHNOLOGY, FIRMS, INNOVATION, KNOWLEDGE, NETWORKS, PERSPECTIVE, RECONCEPTUALIZATION, RESEARCH-AND-DEVELOPMENT, STRATEGIC ALLIANCES\par
AB \tabto{0cm}The objective of this paper is to deepen our understanding of the role of \hl{absorpti}ve \hl{capacit}y in enabling interorganizational new product development (INPD). We contend that despite what is known about the benefits of \hl{absorpti}ve \hl{capacit}y to innovating firms, this research is dominated by firm-level analyses using cross-sectional data. The accumulated knowledge about \hl{absorpti}ve \hl{capacit}y thus does not help with understanding how \hl{absorpti}ve \hl{capacit}y unfolds as a process within and between firms such as when firms collaborate in new product development. Using a longitudinal case study, we build new theory at the INPD system level of analysis that leads us to shed new light on the cross-level interactions between firm- and alliance-level \hl{absorpti}ve \hl{capacit}ies.\par
\clearpage

\vspace*{-2cm}
Nb \tabto{0cm}127/327 (article\_id: 454)\par
TI \tabto{0cm}BENEFITING FROM SUPPLIER OPERATIONAL INNOVATIVENESS: THE INFLUENCE OF SUPPLIER EVALUATIONS AND \hl{ABSORPTI}VE \hl{CAPACIT}Y\par
AU \tabto{0cm}Azadegan\par
PY \tabto{0cm}2011, SO JOURNAL OF SUPPLY CHAIN MANAGEMENT\par
DT \tabto{0cm}Article\par
PG \tabto{0cm}16, NR 85, TC 25\par
DE \tabto{0cm}\hl{ABSORPTI}VE \hl{CAPACIT}Y, SUPPLIER EVALUATION, SUPPLIER INNOVATION\par
ID \tabto{0cm}BUSINESS MARKETS, BUYER-SELLER RELATIONSHIPS, CAPABILITIES, COMPETITIVE ADVANTAGE, INTEGRATION, MANAGEMENT, PERFORMANCE, PERSPECTIVE, PRODUCT DEVELOPMENT, VALUE CREATION\par
AB \tabto{0cm}As suppliers take on more important roles in manufacturing and designing products, their operational innovativeness becomes an important source of value. We use the relational view theory to hypothesize for a positive association between operational innovativeness of established suppliers and manufacturer performance. Furthermore, we posit that the manufacturer can enhance this value by ensuring that the supplier is performing as expected (i.e., through supplier evaluation programs), and by focusing on learning from the supplier (i.e., through \hl{absorpti}ve \hl{capacit}y). We then develop hypotheses as to how the influence of these two approaches differs when working with suppliers assigned to different types of tasks. We test the hypotheses using survey responses from 136 manufacturers evaluating 272 of their suppliers. Results show that supplier evaluation programs and \hl{absorpti}ve \hl{capacit}y are both effective means of augmenting the benefits of supplier operational innovativeness. Contrary to theoretical predictions, benefits of operational innovativeness of suppliers with knowledge-intensive tasks are enhanced through increased \hl{absorpti}ve \hl{capacit}y and through increased supplier evaluation programs.\par
\clearpage

\vspace*{-2cm}
Nb \tabto{0cm}128/327 (article\_id: 455)\par
TI \tabto{0cm}Differences in knowledge acquisition mechanisms between IJVs with Western vs Japanese parents Focus on factors comprising \hl{absorpti}ve \hl{capacit}y\par
AU \tabto{0cm}Park\par
PY \tabto{0cm}2011, SO MANAGEMENT DECISION\par
DT \tabto{0cm}Article\par
PG \tabto{0cm}22, NR 65, TC 7\par
DE \tabto{0cm}ACQUISITIONS AND MERGERS, JAPAN, JOINT VENTURES, KOREA, MULTINATIONAL COMPANIES\par
ID \tabto{0cm}COMPETITIVE ADVANTAGE, EMPIRICAL-TEST, FOREIGN PARENTS, INNOVATION, INTERNATIONAL-JOINT-VENTURES, KNOW-HOW, MANAGERIAL KNOWLEDGE, NETWORKS, PERFORMANCE, STRATEGIC ALLIANCES\par
AB \tabto{0cm}Purpose - The main objective of this study is as follows: while "knowledge acquisition in international joint ventures (IJVs)" has been widely in the limelight, the question of whether learning mechanisms in IJVs with Western vs Japanese parents are different has not yet been answered. In order to fill the current gap in the literature, this research seeks to answer the question by focusing on the \hl{absorpti}ve \hl{capacit}y perspective.
Design/methodology/approach - The data were obtained by survey. A total of 1,207 questionnaires were posted to the CEOs of IJVs in Korea and 288 were returned, 42 of which were unusable, thus giving a response rate of 20.38 percent.
Findings - By using OLS regressions, two key findings are reported. First, the importance of \hl{absorpti}ve \hl{capacit}y of IJVs in order to acquire foreign technology from parent firms is confirmed. Second, the results indicate that IJVs with Japanese multinational firms do not show different patterns of technology acquisition compared with IJVs with Western firms. Based on the findings, it is concluded that the learning mechanisms facilitating technology acquisition in IJVs is not highly influenced by foreign origins.
Originality/value - To reiterate, "knowledge acquisition in IJVs" has been widely in the limelight. However, no one has empirically analyzed the distinctions in learning mechanisms in IJVs with Western vs Japanese parents. This research contributes to the current literature by confirming the minimal substantial difference between them.\par
\clearpage

\vspace*{-2cm}
Nb \tabto{0cm}129/327 (article\_id: 456)\par
TI \tabto{0cm}Microfoundations of Internal and External \hl{Absorpti}ve \hl{Capacit}y Routines\par
AU \tabto{0cm}Lewin, Massini, Peeters\par
PY \tabto{0cm}2011, SO ORGANIZATION SCIENCE\par
DT \tabto{0cm}Article\par
PG \tabto{0cm}18, NR 171, TC 79\par
DE \tabto{0cm}\hl{ABSORPTI}VE \hl{CAPACIT}Y, CAPABILITIES, COMPLEMENTARITIES, IMITATORS, INNOVATORS, MICROFOUNDATIONS, ROUTINES\par
ID \tabto{0cm}APPLYING ORGANIZATIONAL ROUTINES, COMBINATIVE CAPABILITIES, COMPETITIVE ADVANTAGE, DYNAMIC CAPABILITIES, KNOWLEDGE TRANSFER, MARKET ORIENTATION, PRODUCT DEVELOPMENT, RESEARCH-AND-DEVELOPMENT, RESOURCE-BASED VIEW, STRATEGIC MANAGEMENT RESEARCH\par
AB \tabto{0cm}The 20 years following the introduction of the seminal construct of \hl{absorpti}ve \hl{capacit}y (AC) by Cohen and Levinthal (Cohen, W. M., D. A. Levinthal. 1989. Innovation and learning: The two faces of R\&D. Econom. J. 99(397) 569-596; Cohen, W. M., D. A. Levinthal. 1990. \hl{Absorpti}ve \hl{capacit}y: A new perspective on learning and innovation. Admin. Sci. Quart. 35(1) 128-152) have seen the proliferation of a vast literature citing the AC construct in over 10,000 published papers, chapters, and books, and interpreting it or applying it in many areas of organization science research, including organization theory, strategic management, and economics. However, with very few exceptions, the specific organizational routines and processes that constitute AC capabilities remain a black box. In this paper, we propose a routine-based model of AC as a first step toward the operationalization of the AC construct. Our intent is to direct attention to the importance of balancing internal knowledge creating processes with the identification, acquisition, and assimilation of new knowledge originating in the external environment. We decompose the construct of AC into two components, internal and external AC capabilities, and identify the configuration of metaroutines underlying these two components. These higher-level routines are expressed within organizations by configurations of empirically observable practiced routines that are idiosyncratic and firm specific. Therefore, we conceptualize metaroutines as the foundations of practiced routines. The ability of organizations to discover and implement complementarities between AC routines may explain why some firms are successful early adopters and most firms are imitators. Success as an early adopter of a new management practice or an innovation is expected to depend on the extent to which an organization evolves, adapts, and implements the configuration of its internal and external \hl{absorpti}ve \hl{capacit}y routines.\par
\clearpage

\vspace*{-2cm}
Nb \tabto{0cm}130/327 (article\_id: 457)\par
TI \tabto{0cm}Knowledge diversity as a moderator: inter-firm relationships, R \& D investment and \hl{absorpti}ve \hl{capacit}y\par
AU \tabto{0cm}Lin\par
PY \tabto{0cm}2011, SO TECHNOLOGY ANALYSIS \& STRATEGIC MANAGEMENT\par
DT \tabto{0cm}Article\par
PG \tabto{0cm}13, NR 57, TC 6\par
DE \tabto{0cm}ACQUISITIONS, ALLIANCES, INTER-FIRM GOVERNANCE, KNOWLEDGE DIVERSITY\par
ID \tabto{0cm}COMPETITIVE ADVANTAGE, DIVERSIFICATION, FIRM PERFORMANCE, INDUSTRY, INNOVATION, INTENSIVE FIRMS, JAPANESE FIRMS, STRATEGIC ALLIANCES, TECHNOLOGY, TRANSFERRING KNOWLEDGE\par
AB \tabto{0cm}The study examines how knowledge diversity moderates the effects of RD investment, strategic alliances, and acquisitions on firm performance using a sample of 2404 firm-year data from US technology firms. Results confirm that the main effect of knowledge diversity on firm growth is not significant and it indeed plays a role of a moderator. The theory of \hl{absorpti}ve \hl{capacit}y provides a good explanation that for firms with high knowledge diversity, strategic alliances and acquisitions are more effective while for firms with low knowledge diversity, internal RD investment is more effective. These findings point to an important research direct that the characteristics of a firm's knowledge portfolio play a critical role in determining the effectiveness of knowledge sourcing as well as interfirm partnership strategies.\par
\clearpage

\vspace*{-2cm}
Nb \tabto{0cm}131/327 (article\_id: 458)\par
TI \tabto{0cm}Building \hl{absorpti}ve \hl{capacit}y to organise inbound open innovation in traditional industries\par
AU \tabto{0cm}Spithoven, Clarysse, Knockaert\par
PY \tabto{0cm}2011, SO TECHNOVATION\par
DT \tabto{0cm}Article; Proceedings Paper\par
PG \tabto{0cm}12, NR 60, TC 54\par
DE \tabto{0cm}\hl{ABSORPTI}VE \hl{CAPACIT}Y, OPEN INNOVATION, TECHNOLOGY INTERMEDIATION\par
ID \tabto{0cm}COMPETITIVE ADVANTAGE, EMPIRICAL-ANALYSIS, FIRMS, INTERMEDIATION, KNOWLEDGE, PERFORMANCE, RECONCEPTUALIZATION, RESOURCES, TECHNOLOGY-TRANSFER, TRANSFORMATION\par
AB \tabto{0cm}The discussion on open innovation suggests that the ability to absorb external knowledge has become a major driver for competition. For R\&D intensive large firms, the concept of open innovation in relation to \hl{absorpti}ve \hl{capacit}y is relatively well understood. Little attention has; however, been paid to how both small firms and firms, which operate in traditional sectors, engage in open innovation activities. The latter two categories of firms often dispose of no, or at most a relatively low level of, \hl{absorpti}ve \hl{capacit}y. Open innovation has two faces. In the case of inbound open innovation, companies screen their environment to search for technology and knowledge and do not exclusively rely on in-house R\&D. A key pre-condition is that firms dispose of "\hl{absorpti}ve \hl{capacit}y" to internalise external knowledge. SMEs and firms in traditional industries might need assistance in building \hl{absorpti}ve \hl{capacit}y. This paper focuses on the role of collective research centres in building \hl{absorpti}ve \hl{capacit}y at the inter-organisational level. In order to do so, primary data was collected through interviews with CEOs of these technology intermediaries and their member firms and analysed in combination with secondary data. The technology intermediaries discussed are created to help firms to take advantage of technological developments. The paper demonstrates that the openness of the innovation process forces firms lacking \hl{absorpti}ve \hl{capacit}y to search for alternative ways to engage in inbound open innovation. The paper highlights the multiple activities of which \hl{absorpti}ve \hl{capacit}y in intermediaries is made up; defines the concept of \hl{absorpti}ve \hl{capacit}y as a pre-condition to open innovation; and demonstrates how firms lacking \hl{absorpti}ve \hl{capacit}y collectively cope with distributed knowledge and innovation. (C) 2009 Elsevier Ltd. All rights reserved.\par
\clearpage

\vspace*{-2cm}
Nb \tabto{0cm}132/327 (article\_id: 459)\par
TI \tabto{0cm}New Product Development and \hl{Absorpti}ve \hl{Capacit}y in Industrial Districts: A Multidimensional Approach\par
AU \tabto{0cm}Exposito-Langa, Molina-Morales, Capo-Vicedo\par
PY \tabto{0cm}2011, SO REGIONAL STUDIES\par
DT \tabto{0cm}Article\par
PG \tabto{0cm}13, NR 76, TC 9\par
DE \tabto{0cm}\hl{ABSORPTI}VE \hl{CAPACIT}Y, INDUSTRIAL DISTRICT, INNOVATION, KNOWLEDGE, NEW PRODUCT DEVELOPMENT\par
ID \tabto{0cm}CAPABILITIES, CLUSTERS, EMBEDDEDNESS, ENVIRONMENT, FIRM, INNOVATION, KNOWLEDGE TRANSFER, ORGANIZATIONS, PERFORMANCE, TECHNOLOGY\par
AB \tabto{0cm}Exposito-Langa M., Molina-Morales F. X. and Capo-Vicedo J. New product development and \hl{absorpti}ve \hl{capacit}y in industrial districts: a multidimensional approach, Regional Studies. This research studies to what extent the \hl{absorpti}ve \hl{capacit}y of a firm influences its \hl{capacit}y to exploit new opportunities through new products, particularly in a specific context of industrial districts. A multidimensional approach to the \hl{absorpti}ve \hl{capacit}y concept is used to distinguish between identification, assimilation, and exploitation of external knowledge. The population of companies belonging to a Spanish textile district is studied. Findings suggest that information and knowledge that a company receives from external sources provide the company with the necessary abilities to innovate. In the present case, the greater the \hl{absorpti}ve \hl{capacit}y, the greater the innovation \hl{capacit}y for the company. [image omitted] Exposito-Langa M., Molina-Morales F. X. et Capo-Vicedo J. La mise au point des produits nouveaux et la \hl{capacit}e d'\hl{absorpti}on dans les districts industriels: une facon multi-dimensionnelle, Regional Studies. Cette recherche etudie jusqu'a quel point la \hl{capacit}e d'\hl{absorpti}on d'une entreprise influe sur sa \hl{capacit}e d'exploiter de nouvelles possibilites par moyen des produits nouveaux, notamment dans le contexte des districts industriels. On applique une facon multidimensionnelle a la notion de \hl{capacit}e d'\hl{absorpti}on afin de distinguer entre l'identification, l'assimilation et l'exploitation de la connaissance externe. On etudie le parc d'entreprises implante sur un district du textile en Espagne. Les resultats laissent supposer que l'information et la connaissance que recoit une entreprise des sources externes fournissent ce dont elles ont besoin pour innover. Il s'avere de la presente etude que la plus grande est la \hl{capacit}e d'\hl{absorpti}on, la plus grande est la \hl{capacit}e d'innovation. \hl{Capacit}e d'\hl{absorpti}on Connaissance District industriel Innovation Mise au point des produits nouveaux Exposito-Langa M., Molina-Morales F. X. und Capo-Vicedo J. Die Entwicklung neuer Produkte und die \hl{absorpti}ve Kapazitat in Industriebezirken: ein multidimensionaler Ansatz, Regional Studies. In diesem Beitrag untersuchen wir, in welchem Ausmass sich insbesondere im spezifischen Kontext der Industriebezirke die \hl{absorpti}ve Kapazitat einer Firma auf ihre Kapazitat auswirkt, neue Gelegenheiten durch neue Produkte zu nutzen. Zur Unterscheidung zwischen der Identifizierung, der Assimilation und der Nutzung von externem Wissen wird ein multidimensionaler Ansatz fur die \hl{absorpti}ve Kapazitat herangezogen. Untersucht wurden die Firmen eines Textilbezirks in Spanien. Aus den Ergebnissen geht hervor, dass die Informationen und das Wissen, die eine Firma von externen Quellen bezieht, der Firma die notigen Fahigkeiten zur Innovation verleihen. Im vorliegenden Fall gilt, dass die innovative Kapazitat einer Firma um so grosser ausfallt, je grosser ihre \hl{absorpti}ve Kapazitat ist. \hl{Absorpti}ve Kapazitat Wissen Industriebezirk Innovation Entwicklung neuer Produkte Exposito-Langa M., Molina-Morales F. X. y Capo-Vicedo J. Desarrollo de nuevos productos y capacidad de absorcion en distritos industriales. Una aproximacion multidimensional, Regional Studies. El presente trabajo estudia el efecto de la capacidad de absorcion en la empresa sobre el desarrollo de nuevos productos, dentro de un contexto de distrito industrial.
Utilizamos una aproximacion multidimensional para el concepto de la capacidad de absorcion con el objeto de distinguir entre la identificacion, asimilacion y explotacion deconocimiento. Hemos trabajado con empresas pertenecientes al distrito industrial textil espanol. Los resultados sugieren que las externalidades que la empresa recibe en forma de informacion y conocimiento de su entorno, junto a las habilidades internas necesarias, benefician su proceso de innovacion. De forma particular, el desarrollo de la capacidad de absorcion en la empresa favorece su capacidad innovadora. Capacidad de absorcion Conocimiento Distrito industrial Innovacion Desarrollo de nuevos productos.\par
\clearpage

\vspace*{-2cm}
Nb \tabto{0cm}133/327 (article\_id: 460)\par
TI \tabto{0cm}Distances, Knowledge Brokerage and \hl{Absorpti}ve \hl{Capacit}y in Enhancing Regional Innovativeness: A Qualitative Case Study of Lahti Region, Finland\par
AU \tabto{0cm}Parjanen, Melkas, Uotila\par
PY \tabto{0cm}2011, SO EUROPEAN PLANNING STUDIES\par
DT \tabto{0cm}Article\par
PG \tabto{0cm}28, NR 58, TC 9\par
DE \tabto{0cm}null\par
ID \tabto{0cm}CAPABILITY, PERSPECTIVE, PROXIMITY, RECONCEPTUALIZATION, SOCIAL-STRUCTURE, SYSTEMS\par
AB \tabto{0cm}Scholars researching innovation are unanimous about the huge innovation potential in combining different fields of knowledge. Structural holes in innovation networks are especially fruitful in fostering new ideas and innovations. One problem in utilizing the innovation potential in structural holes stems from diversity or "distance" between the innovating partners. This study focuses on the concepts of distances, proximities, \hl{absorpti}ve \hl{capacit}y and knowledge brokerage in relation to innovativeness in regional innovation networks. Knowledge brokers' own perceptions concerning their functions and roles in innovation policy are investigated by means of a case analysis of Lahti region in Finland. This study uses the experiences of the knowledge brokers to answer the question of how regional innovativeness could be skilfully enhanced by brokerage functions-in particular, by utilizing distances and proximities. As a result of this study, five central roles are defined for knowledge brokers. Knowledge brokers' roles and functions are demanding as recognized by the brokers themselves. Successful brokerage and the related improvement of \hl{absorpti}ve \hl{capacit}y require a holistic approach to entire innovation processes and their wider environment.\par
\clearpage

\vspace*{-2cm}
Nb \tabto{0cm}134/327 (article\_id: 461)\par
TI \tabto{0cm}KNOWLEDGE SPILLOVERS, \hl{ABSORPTI}VE \hl{CAPACIT}Y, AND SKILL INTENSITY OF CHILEAN MANUFACTURING PLANTS\par
AU \tabto{0cm}Saito, Gopinath\par
PY \tabto{0cm}2011, SO JOURNAL OF REGIONAL SCIENCE\par
DT \tabto{0cm}Article\par
PG \tabto{0cm}19, NR 43, TC 6\par
DE \tabto{0cm}null\par
ID \tabto{0cm}ECONOMIC-DEVELOPMENT, EDUCATION, EXTERNALITIES, FOREIGN DIRECT-INVESTMENT, INDUSTRY, INNOVATION, LEVEL, LINKAGES, PRODUCTIVITY, UNITED-STATES\par
AB \tabto{0cm}Knowledge spillovers are an important source of economic growth. In this study, we identify a mechanism through which knowledge spillovers occur among plants in the Chilean manufacturing industry. A plant-level production function is estimated with the \hl{absorpti}ve-\hl{capacit}y hypothesis, that is, employment of skilled workers is a key channel through which knowledge is transmitted across plants. Results show that a plant's productivity from spillovers increases with its skill intensity, which is measured by the share of skilled workers in total employment. We also find that plants in a region with a large knowledge stock increase their skill intensity to benefit more from spillovers. Our results suggest that an increase in regional knowledge stock is the most effective policy to improve a plant's productivity. However, policies that encourage a plant to employ high skill-intensive production also enhance its productivity.\par
\clearpage

\vspace*{-2cm}
Nb \tabto{0cm}135/327 (article\_id: 462)\par
TI \tabto{0cm}Gaining from interactions with universities: Multiple methods for nurturing \hl{absorpti}ve \hl{capacit}y\par
AU \tabto{0cm}Bishop, D'Este, Neely\par
PY \tabto{0cm}2011, SO RESEARCH POLICY\par
DT \tabto{0cm}Article\par
PG \tabto{0cm}11, NR 38, TC 33\par
DE \tabto{0cm}\hl{ABSORPTI}VE \hl{CAPACIT}Y, BENEFITS, EXPLOITATION, EXPLORATION, UNIVERSITY-INDUSTRY INTERACTIONS\par
ID \tabto{0cm}ACADEMIC RESEARCH, ALLIANCES, INDUSTRIAL-INNOVATION, RESEARCH-AND-DEVELOPMENT, TECHNOLOGY\par
AB \tabto{0cm}This paper examines the various methods through which firms benefit from interactions with universities, arguing that such benefits are instrumental in nurturing the multiple facets of a firm's \hl{absorpti}ve \hl{capacit}y. We bring together data collected from a survey of UK firms that collaborated with universities, and firm-level data on past partnerships with universities. The results show that benefits from interactions with universities are multifaceted, including enhancement of the firm's explorative and exploitative capabilities. Results also indicate that firms' R\&D commitments, geographical proximity to and research quality of university partners have a distinct impact on the different types of benefits from interactions with universities. We find geographical proximity is crucial for assessing problem-solving as an important benefit, while interactions with top quality universities have a positive influence on the benefits associated with firms' downstream activities. We discuss the implications of these findings for research and policy. (C) 2010 Elsevier B.V. All rights reserved.\par
\clearpage

\vspace*{-2cm}
Nb \tabto{0cm}136/327 (article\_id: 463)\par
TI \tabto{0cm}Under the Tip of the Iceberg: \hl{Absorpti}ve \hl{Capacit}y, Environmental Strategy, and Competitive Advantage\par
AU \tabto{0cm}Delmas, Hoffmann, Kuss\par
PY \tabto{0cm}2011, SO BUSINESS \& SOCIETY\par
DT \tabto{0cm}Article\par
PG \tabto{0cm}39, NR 90, TC 28\par
DE \tabto{0cm}\hl{ABSORPTI}VE \hl{CAPACIT}Y, CHEMICAL INDUSTRY, COMPETITIVE ADVANTAGE, ENVIRONMENTAL STRATEGY, STRUCTURAL EQUATION MODELING\par
ID \tabto{0cm}CHEMICAL-INDUSTRY, CORPORATE SOCIAL PERFORMANCE, COVARIANCE-STRUCTURES, ELECTRIC UTILITY INDUSTRY, FINANCIAL PERFORMANCE, GREEN MANAGEMENT, MANAGEMENT STANDARDS, NATURAL-ENVIRONMENT, RESOURCE-BASED VIEW, RESPONSIBLE CARE\par
AB \tabto{0cm}Although existing research evaluates how the adoption of proactive environmental strategies affects corporate performance, there is little understanding of the organizational mechanisms that link such strategies to competitive advantage. It is, therefore, unclear how environmental strategies relate to other management strategies that could lead to a competitive advantage. In this article, we analyze the organizational capabilities that underlie a firm's ability to generate competitive advantage from the adoption of proactive environmental strategies. We develop and test a model where \hl{absorpti}ve \hl{capacit}y facilitates the development of proactive environmental strategies that result in competitive advantage. Results from a survey of 157 German chemical firms strongly support the model.\par
\clearpage

\vspace*{-2cm}
Nb \tabto{0cm}137/327 (article\_id: 464)\par
TI \tabto{0cm}Knowledge creation and \hl{absorpti}ve \hl{capacit}y: The effect of intra-district shared competences\par
AU \tabto{0cm}Camison, Fores\par
PY \tabto{0cm}2011, SO SCANDINAVIAN JOURNAL OF MANAGEMENT\par
DT \tabto{0cm}Article\par
PG \tabto{0cm}21, NR 158, TC 16\par
DE \tabto{0cm}\hl{ABSORPTI}VE \hl{CAPACIT}Y, INDUSTRIAL DISTRICT, KNOWLEDGE CREATION \hl{CAPACIT}Y, KNOWLEDGE SPILLOVERS, ORGANISATIONAL LEARNING, SHARED COMPETENCES\par
ID \tabto{0cm}COMPETITIVE ADVANTAGE, DEVELOPMENT SPILLOVERS, DYNAMIC CAPABILITIES, FIRM PERFORMANCE, INDUSTRIAL CLUSTERS, INNOVATION, ORGANIZATIONAL RESEARCH, PUBLIC-POLICY, SOCIAL-STRUCTURE, STRATEGIC MANAGEMENT\par
AB \tabto{0cm}This paper takes a cross-level approach in contributing to defining the competences accumulated and shared in an industrial district, and to explaining how they differ from firm-specific, knowledge-based \hl{capacit}ies. From a dataset of 952 Spanish firms and 35 industrial districts, we provide empirical evidence that industrial districts are spaces with dense networks of information and knowledge transfer, inter-personnel relationships and a strong specialised stock of human capital, which are accessible and shared by all firms embedded in such a district. However, we explain the complementarity between district and firm-specific \hl{capacit}ies in order to develop the notion of \hl{absorpti}ve \hl{capacit}y, by indicating that the diffusion of shared competences is neither easy nor direct and that it requires a firm's internal learning effort to better absorb localised knowledge spillovers. Results enable us to shed new light on how firms' knowledge creation and diffusion processes benefit from these external knowledge flows. (C) 2010 Elsevier Ltd. All rights reserved.\par
\clearpage

\vspace*{-2cm}
Nb \tabto{0cm}138/327 (article\_id: 465)\par
TI \tabto{0cm}Managerial ties, knowledge acquisition, realized \hl{absorpti}ve \hl{capacit}y and new product market performance of emerging multinational companies: A case of China\par
AU \tabto{0cm}Kotabe, Jiang, Murray\par
PY \tabto{0cm}2011, SO JOURNAL OF WORLD BUSINESS\par
DT \tabto{0cm}Article\par
PG \tabto{0cm}11, NR 93, TC 26\par
DE \tabto{0cm}DECISION-MAKING, EMERGING ECONOMY, KNOWLEDGE ACQUISITION, MANAGERIAL TIES, NEW PRODUCT MARKET PERFORMANCE, REALIZED \hl{ABSORPTI}VE \hl{CAPACIT}Y\par
ID \tabto{0cm}COMPETITIVE ADVANTAGE, DYNAMIC CAPABILITIES, EMPIRICAL-TEST, FIRM PERFORMANCE, INNOVATION, INTEGRATION, ORGANIZATIONAL RESEARCH, RESEARCH-AND-DEVELOPMENT, TECHNOLOGY, WEAK TIES\par
AB \tabto{0cm}Many firms rely on external sources to acquire knowledge that is critical for enhancing new product market performance. Using a sample of 121 emerging multinational corporations (EMNCs) from China, we explore the effects of managerial ties with government officials and foreign MNC partners on knowledge acquisition and investigate how the acquired knowledge affects firms' new product market performance. Our results indicate that knowledge acquisition could only enhance new product market performance with the presence of realized \hl{absorpti}ve \hl{capacit}y. Our study suggests that managers' decisions on knowledge acquisition from external sources may not increase firms' new product market performance. Instead, managerial prowess in integrating and transforming knowledge becomes paramount in enhancing new product market performance. Published by Elsevier Inc.\par
\clearpage

\vspace*{-2cm}
Nb \tabto{0cm}139/327 (article\_id: 466)\par
TI \tabto{0cm}A measure of \hl{absorpti}ve \hl{capacit}y: Scale development and validation\par
AU \tabto{0cm}Flatten, Engelen, Zahra, Brettel\par
PY \tabto{0cm}2011, SO EUROPEAN MANAGEMENT JOURNAL\par
DT \tabto{0cm}Article\par
PG \tabto{0cm}19, NR 141, TC 46\par
DE \tabto{0cm}\hl{ABSORPTI}VE \hl{CAPACIT}Y, SCALE DEVELOPMENT, SURVEY-BASED RESEARCH\par
ID \tabto{0cm}ANTECEDENTS, FIRM, IMPACT, INDUSTRIES, INNOVATION PERFORMANCE, KNOWLEDGE TRANSFER, MANAGEMENT, ORGANIZATIONS, PRODUCT DEVELOPMENT, TECHNOLOGICAL ACQUISITIONS\par
AB \tabto{0cm}Academic interest in \hl{absorpti}ve \hl{capacit}y (ACAP), which has grown rapidly over the past two decades, has focused on ACAP's effect on organizational learning, knowledge sharing, innovation, capability building, and firm performance. Even though Cohen and Levinthal's work (1990) highlights the multidimensionality of ACAP, researchers have measured it as a uni-dimensional construct, often using a firm's R\&D spending intensity as a proxy for this construct. This practice raises questions about the veracity of the claims made in the literature about the nature and contributions of ACAP. The present study develops and validates a multidimensional measure of ACAP, building on relevant prior literature, a series of pre-tests, and two large survey-based studies of German companies. (C) 2010 Elsevier Ltd. All rights reserved.\par
\clearpage

\vspace*{-2cm}
Nb \tabto{0cm}140/327 (article\_id: 467)\par
TI \tabto{0cm}Validation of an instrument to measure \hl{absorpti}ve \hl{capacit}y\par
AU \tabto{0cm}Jimenez-Barrionuevo, Garcia-Morales, Molina\par
PY \tabto{0cm}2011, SO TECHNOVATION\par
DT \tabto{0cm}Article\par
PG \tabto{0cm}13, NR 78, TC 24\par
DE \tabto{0cm}ASSESSMENT INSTRUMENT, POTENTIAL AND REALIZED \hl{ABSORPTI}VE \hl{CAPACIT}Y, RESOURCE-BASED VIEW\par
ID \tabto{0cm}COMPETITIVE ADVANTAGE, DYNAMIC CAPABILITIES, FIRM PERFORMANCE, INNOVATION, INTERNATIONAL JOINT VENTURES, KNOWLEDGE TRANSFER, RECONCEPTUALIZATION, RESOURCE-BASED VIEW, SAMPLE-SIZE, STRATEGIC ALLIANCES\par
AB \tabto{0cm}\hl{Absorpti}ve \hl{capacit}y is an ability firms should develop if they wish to adapt to changes in an increasingly competitive and changing environment and to achieve and sustain competitive advantage. Despite the increase in literature on \hl{absorpti}ve \hl{capacit}y, some ambiguity remains in determining the dimensions that shape the construct. Thus, no measurement instrument can be adapted to these dimensions. The aim of this paper is to contribute to the literature on \hl{absorpti}ve \hl{capacit}y by using a resource-based view to present an alternative measurement instrument for \hl{absorpti}ve \hl{capacit}y. This instrument differentiates between the phases of acquisition. assimilation, transformation and exploitation of knowledge, as well as between the two dimensions of \hl{absorpti}ve \hl{capacit}y (potential and realized), to reduce the problem of measuring and identifying the dimensions that shape this important construct. The instrument's validity and reliability are guaranteed and have been tested using data from 168 Spanish organizations. (C) 2010 Elsevier Ltd. All rights reserved.\par
\clearpage

\vspace*{-2cm}
Nb \tabto{0cm}141/327 (article\_id: 468)\par
TI \tabto{0cm}UNPACKING \hl{ABSORPTI}VE \hl{CAPACIT}Y: A STUDY OF KNOWLEDGE UTILIZATION FROM ALLIANCE PORTFOLIOS\par
AU \tabto{0cm}Vasudeva, Anand\par
PY \tabto{0cm}2011, SO ACADEMY OF MANAGEMENT JOURNAL\par
DT \tabto{0cm}Article\par
PG \tabto{0cm}13, NR 70, TC 48\par
DE \tabto{0cm}null\par
ID \tabto{0cm}BIOTECHNOLOGY, FIRMS, LOCAL SEARCH, MARKET VALUE, NETWORKS, PATENT CITATIONS, PERFORMANCE, RESEARCH-AND-DEVELOPMENT, STRATEGIC MANAGEMENT, TECHNOLOGICAL-INNOVATION\par
AB \tabto{0cm}To understand how firms facing technological discontinuities utilize knowledge from alliance portfolios, we unpack \hl{absorpti}ve \hl{capacit}y into "latitudinal" and "longitudinal" components, corresponding to use of diverse and distant knowledge, respectively. We find that a moderate burden on firms' latitudinal \hl{absorpti}ve \hl{capacit}y, corresponding to medium diversity in their portfolios, contributes to optimal knowledge utilization. Simultaneously increasing the demand on firms' longitudinal \hl{absorpti}ve \hl{capacit}y affects this relationship negatively. Highlighting important trade-offs between latitudinal and longitudinal \hl{absorpti}ve \hl{capacit}ies, our findings reveal two portfolio strategies, "telescopic" and "panoptic" searches, that optimize knowledge utilization. We address important dialectics concerning \hl{absorpti}ve \hl{capacit}y constraints and knowledge utilization.\par
\clearpage

\vspace*{-2cm}
Nb \tabto{0cm}142/327 (article\_id: 469)\par
TI \tabto{0cm}Sources of External Technology, \hl{Absorpti}ve \hl{Capacit}y, and Innovation Capability in Chinese State-Owned High-Tech Enterprises\par
AU \tabto{0cm}Li\par
PY \tabto{0cm}2011, SO WORLD DEVELOPMENT\par
DT \tabto{0cm}Article\par
PG \tabto{0cm}9, NR 53, TC 24\par
DE \tabto{0cm}\hl{ABSORPTI}VE \hl{CAPACIT}Y, DOMESTIC TECHNOLOGY PURCHASE, FOREIGN TECHNOLOGY IMPORT, INNOVATION CAPABILITY, R\&D\par
ID \tabto{0cm}2 FACES, AFFECT ECONOMIC-GROWTH, COUNT DATA, FIRM-LEVEL, FOREIGN TECHNOLOGY, INDUSTRIES, PANEL-DATA, PRODUCTIVITY GROWTH, RESEARCH-AND-DEVELOPMENT, SPILLOVERS\par
AB \tabto{0cm}This paper examines the pattern of innovation and learning among state-owned enterprises in Chinese high-tech sectors and empirically estimates the impact of three types of investment for acquiring technological knowledge-in-house R\&D, importing foreign technology, and purchasing domestic technology-on the innovation capabilities of firms. Based on a panel dataset consisting of 21 high-tech sectors during the period 1995-2004, an augmented knowledge production function is estimated. The results show that importing foreign technology alone does not facilitate innovation in Chinese state-owned high-tech enterprises, unless in-house R\&D is also conducted. Domestic technology purchases, however, are found to have a favorable direct impact on innovation, suggesting that firms have less difficulty in absorbing domestic technological knowledge than utilizing foreign technology and that \hl{absorpti}ve \hl{capacit}y is contingent upon the source or nature of the external knowledge. (C) 2011 Elsevier Ltd. All rights reserved.\par
\clearpage

\vspace*{-2cm}
Nb \tabto{0cm}143/327 (article\_id: 470)\par
TI \tabto{0cm}\hl{Absorpti}ve \hl{Capacit}y and Firm Performance in SMEs: The Mediating Influence of Strategic Alliances\par
AU \tabto{0cm}Flatten, Greve, Brettel\par
PY \tabto{0cm}2011, SO EUROPEAN MANAGEMENT REVIEW\par
DT \tabto{0cm}Article\par
PG \tabto{0cm}16, NR 101, TC 21\par
DE \tabto{0cm}\hl{ABSORPTI}VE \hl{CAPACIT}Y, DYNAMIC CAPABILITIES, MODERATED MEDIATION, STRATEGIC ALLIANCES\par
ID \tabto{0cm}ANTECEDENTS, BUSINESS UNIT, HIGH-TECHNOLOGY, INNOVATION, JOINT VENTURES, KNOWLEDGE TRANSFER, MANAGEMENT, MARKET ORIENTATION, START-UP, STRUCTURAL EQUATION MODELS\par
AB \tabto{0cm}\hl{Absorpti}ve \hl{capacit}y (ACAP) is a firm's ability to innovate by identifying, assimilating, and exploiting knowledge available in its environment. ACAP has been widely researched, but this research has not sufficiently analyzed the influence of ACAP on an interfirm level, especially regarding the multidimensional character of this construct. The present study intends to reveal whether the relationship between ACAP and firm performance in small and mediun sized enterprises (SMEs) is mediated by strategic alliances. Furthermore, different moderating characteristics such as age and size of the companies are taken into consideration. The findings indicate that strategic alliances of SMEs mediate both the relationship between ACAP and firm performance and the relationship between each dimension of ACAP and firm performance. However, these results might not be valid under certain circumstances since strategic alliances have no mediating influence when it comes to young SMEs, for example.\par
\clearpage

\vspace*{-2cm}
Nb \tabto{0cm}144/327 (article\_id: 471)\par
TI \tabto{0cm}Innovations in a relational context: Mechanisms to connect learning processes of \hl{absorpti}ve \hl{capacit}y\par
AU \tabto{0cm}Knoppen, Saenz, Johnston\par
PY \tabto{0cm}2011, SO MANAGEMENT LEARNING\par
DT \tabto{0cm}Article\par
PG \tabto{0cm}20, NR 51, TC 10\par
DE \tabto{0cm}\hl{ABSORPTI}VE \hl{CAPACIT}Y, INNOVATION, LEARNING MECHANISM, RELATIONAL CONTEXT\par
ID \tabto{0cm}COMPETITIVENESS, EXPLOITATION, FIRM, KNOWLEDGE DEVELOPMENT, ORGANIZATIONS, PERFORMANCE, PERSPECTIVE, PRODUCT DEVELOPMENT, RECONCEPTUALIZATION, SUPPLIER RELATIONSHIPS\par
AB \tabto{0cm}Companies increasingly regard relationships with other companies as a source of competitive advantage. Relationships constitute a context in which the firm may learn and build \hl{absorpti}ve \hl{capacit}y. This study provides an in-depth explanation of the key mechanisms that interlace the different learning processes leading to innovations in a relational context. A theoretical elaboration of these mechanisms precedes their empirical study within four customer-supplier dyads, centred on two focal customer organizations. The article contributes by discussing how the mechanisms act and interact to create \hl{absorpti}ve \hl{capacit}y for a focal firm across relationships. We find that structural learning mechanisms, while necessary are not sufficient to explain variation in the presence of \hl{absorpti}ve \hl{capacit}y across different learning contexts. Cultural, psychological and policy learning mechanisms complement the picture. From the empirical analysis we derive propositions to guide further research into the creation of \hl{absorpti}ve \hl{capacit}y in a relational context.\par
\clearpage

\vspace*{-2cm}
Nb \tabto{0cm}145/327 (article\_id: 472)\par
TI \tabto{0cm}\hl{Absorpti}ve \hl{capacit}y, efficiency effect and competitors' spillovers\par
AU \tabto{0cm}Lhuillery\par
PY \tabto{0cm}2011, SO JOURNAL OF EVOLUTIONARY ECONOMICS\par
DT \tabto{0cm}Article\par
PG \tabto{0cm}15, NR 27, TC 2\par
DE \tabto{0cm}\hl{ABSORPTI}VE \hl{CAPACIT}Y, INNOVATION SURVEYS, INTRA-INDUSTRY SPILLOVERS, R\&D\par
ID \tabto{0cm}HETEROGENEITY, IMITATION, INNOVATION, PERFORMANCE, RESEARCH-AND-DEVELOPMENT\par
AB \tabto{0cm}Standard innovation surveys do not consider incoming spillovers for non-innovative firms. As a consequence, empirical works may overestimate the \hl{absorpti}ve \hl{capacit}y effect, particularly among competitors. The Swiss innovation surveys presented here measure the importance of knowledge for both innovating and non-innovating firms. This original feature enables us to show that knowledge from rivals actually deters manufacturing firms from engaging in R\&D activities. We therefore provide stronger evidence that the efficiency effect due to intra-industry spillovers is larger than that generally estimated by data from standard surveys. The R\&D based \hl{absorpti}ve \hl{capacit}y is weaker than expected, and non-innovative firms as well as non-R\&D firms heavily rely on their rivals' knowledge to maintain their technological \hl{capacit}ies.\par
\clearpage

\vspace*{-2cm}
Nb \tabto{0cm}146/327 (article\_id: 473)\par
TI \tabto{0cm}The effects of interpartner resource alignment and \hl{absorpti}ve \hl{capacit}y on knowledge transfer performance\par
AU \tabto{0cm}Tsai, Wu\par
PY \tabto{0cm}2011, SO AFRICAN JOURNAL OF BUSINESS MANAGEMENT\par
DT \tabto{0cm}Article\par
PG \tabto{0cm}12, NR 72, TC 1\par
DE \tabto{0cm}\hl{ABSORPTI}VE \hl{CAPACIT}Y, INTERPARTNER RESOURCE ALIGNMENT, KNOWLEDGE TRANSFER PERFORMANCE, UNIVERSITY-INDUSTRY (U-I) INTERACTION\par
ID \tabto{0cm}COMPETITIVE ADVANTAGE, FIRM, INDUSTRY, JOINT VENTURES, NETWORKS, PERSPECTIVE, STRATEGIC ALLIANCES, TECHNOLOGY, TRUST, US\par
AB \tabto{0cm}Against a background of increasing international competition and rapid technological change, universities and industry have often engaged in collaboration as a means of improving cross-unit transfers of the resources, capabilities, and knowledge. Following this frontier, this study examines the effects of university and firm interpartner resource alignment and \hl{absorpti}ve \hl{capacit}y on knowledge transfer performance from the university-industry (U-I) perspective. Regression analysis is used to test the hypotheses in a sample of 120 Taiwanese firms. The research findings suggest that, interpartner resource utilization is positively related to U-I interaction; that when the effect of a firm's \hl{absorpti}ve \hl{capacit}y is higher, U-I interaction is more favorable; and that U-I interaction is positively related to knowledge transfer performance. Our empirical results support the process-oriented view and indicate that U-I interaction plays the mediating role between resource alignment, \hl{absorpti}ve \hl{capacit}y and knowledge transfer performance. The implications of this study are two-fold. First, it emphasizes the value of creative financial and general performance in knowledge transfer by incorporating the U-I interaction perspective. Second, the research results offer support for suggestions that interpartner resource alignment provides a necessary element for \hl{capacit}y and resource exchange. Hence, one interesting further research would analyze practical cases of some Taiwanese companies with the described approach. This is an important practical issue that should be examined in the future.\par
\clearpage

\vspace*{-2cm}
Nb \tabto{0cm}147/327 (article\_id: 474)\par
TI \tabto{0cm}Social Capital of Young Technology Firms and Their IPO Values: The Complementary Role of Relevant \hl{Absorpti}ve \hl{Capacit}y\par
AU \tabto{0cm}Xiong, Bharadwaj\par
PY \tabto{0cm}2011, SO JOURNAL OF MARKETING\par
DT \tabto{0cm}Article\par
PG \tabto{0cm}18, NR 82, TC 9\par
DE \tabto{0cm}\hl{ABSORPTI}VE \hl{CAPACIT}Y, BUSINESS-TO-BUSINESS RELATIONSHIPS, INITIAL PUBLIC OFFERING VALUE, MARKETING-FINANCE INTERFACE, SOCIAL CAPITAL, STOCHASTIC FRONTIER ESTIMATION\par
ID \tabto{0cm}COMPETITIVE ADVANTAGE, EMPIRICAL-ANALYSIS, IMPACT, INITIAL PUBLIC OFFERINGS, MANAGEMENT, PRODUCT DEVELOPMENT, RESEARCH-AND-DEVELOPMENT, STRATEGIC ALLIANCES, SUPPLIER RELATIONSHIPS, VALUE CREATION\par
AB \tabto{0cm}The strategic importance of business-to-business (B2B) relationships is well recognized, but their financial impact remains equivocal. This study links social capital from three types of B2B networks of young technology firms with their initial public offering (IPO) value. The authors identify three relevant types of \hl{absorpti}ve \hl{capacit}y that facilitate the transformation of B2B social capital into IPO value. For the transformation to occur, the authors find that young firms need not only the opportunity to access the resources provided by B2B relationships but also the ability to leverage them through the complementary capability-namely, \hl{absorpti}ve \hl{capacit}y. They test the hypotheses on a sample of 177 IPOs, and the results are robust to endogeneity concerns and alternative measures. As one of the first studies in marketing-finance interface to focus on young firms, the findings provide novel insights, such as the deleterious financial consequence of having marketing and research-and-development B2B relationships without the relevant \hl{absorpti}ve \hl{capacit}y. The authors conclude with a discussion of managerial implications regarding communicating the value of \hl{absorpti}ve \hl{capacit}y, disclosure of marketing-related information, and the importance of marketing for young technology firms.\par
\clearpage

\vspace*{-2cm}
Nb \tabto{0cm}148/327 (article\_id: 475)\par
TI \tabto{0cm}\hl{Absorpti}ve \hl{Capacit}y of Project Networks\par
AU \tabto{0cm}Unsal, Taylor\par
PY \tabto{0cm}2011, SO JOURNAL OF CONSTRUCTION ENGINEERING AND MANAGEMENT-ASCE\par
DT \tabto{0cm}Article\par
PG \tabto{0cm}9, NR 41, TC 3\par
DE \tabto{0cm}\hl{ABSORPTI}VE \hl{CAPACIT}Y, INNOVATION, LEARNING, ORGANIZATIONAL ISSUES, PRODUCTIVITY, PROJECT NETWORKS\par
ID \tabto{0cm}CONSTRUCTION-INDUSTRY, FIRMS, INNOVATION, KNOWLEDGE TRANSFER, PERFORMANCE, PERSPECTIVE, PRODUCTIVITY, SIMULATION\par
AB \tabto{0cm}\hl{Absorpti}ve \hl{capacit}y is a firm's ability to value, assimilate, and utilize new external knowledge and apply it to commercial ends. Much of the prior research on \hl{absorpti}ve \hl{capacit}y focuses on characterizing the factors that influence \hl{absorpti}ve \hl{capacit}y within organizations. However, the mechanism of how related factors affect \hl{absorpti}ve \hl{capacit}y across interdependent organizations in project networks remains less explored. This paper extends a simulation model of project network learning to explore the \hl{absorpti}ve \hl{capacit}y of project networks where periodic external innovations exist. This model is utilized in a series of simulation experiments to untangle the effects of varying types of innovation and degrees of relational instability in a project network. We establish a measure of project network \hl{absorpti}ve \hl{capacit}y and develop an argument that relational instability moderates the project network's \hl{absorpti}ve \hl{capacit}y for different types of innovation. These findings have significant implications for assessing and developing strategies to improve a project organizational network's \hl{capacit}y to absorb and, hence, profit from innovation. DOI: 10.1061/(ASCE)CO.1943-7862.0000361. (C) 2011 American Society of Civil Engineers.\par
\clearpage

\vspace*{-2cm}
Nb \tabto{0cm}149/327 (article\_id: 476)\par
TI \tabto{0cm}\hl{Absorpti}ve \hl{capacit}y, innovation, and financial performance\par
AU \tabto{0cm}Kostopoulos, Papalexandris, Papachroni, Ioannou\par
PY \tabto{0cm}2011, SO JOURNAL OF BUSINESS RESEARCH\par
DT \tabto{0cm}Article\par
PG \tabto{0cm}9, NR 86, TC 47\par
DE \tabto{0cm}\hl{ABSORPTI}VE \hl{CAPACIT}Y, EXTERNAL KNOWLEDGE INFLOWS, FINANCIAL PERFORMANCE, INNOVATION, TIME-LAGGED MEASURES\par
ID \tabto{0cm}COMPLEMENTARITY, DYNAMIC CAPABILITIES, FIRMS, KNOWLEDGE ACQUISITION, MARKET ORIENTATION, MULTINATIONAL-CORPORATIONS, ORGANIZATIONAL ANTECEDENTS, PRODUCT DEVELOPMENT, RESEARCH-AND-DEVELOPMENT, STOCK RETURNS\par
AB \tabto{0cm}This study here examines the role of \hl{absorpti}ve \hl{capacit}y as both a mechanism to identify and translate external knowledge inflows into tangible benefits, as well as a means of achieving superior innovation and time-lagged financial performance. Using path analysis in a sample of 461 Greek enterprises participating in the third Community Innovation Survey, this study demonstrates that external knowledge inflows are directly related to \hl{absorpti}ve \hl{capacit}y and indirectly related to innovation. \hl{Absorpti}ve \hl{capacit}y contributes, directly and indirectly, to innovation and financial performance but in different time spans. This study, therefore, contributes to the understanding of \hl{absorpti}ve \hl{capacit}y's antecedents and outcomes by providing empirical evidence of longitudinal form that offers important research and practical implications. (C) 2010 Elsevier Inc. All rights reserved.\par
\clearpage

\vspace*{-2cm}
Nb \tabto{0cm}150/327 (article\_id: 477)\par
TI \tabto{0cm}\hl{Absorpti}ve \hl{capacit}y: a proposed operationalization\par
AU \tabto{0cm}Noblet, Simon, Parent\par
PY \tabto{0cm}2011, SO KNOWLEDGE MANAGEMENT RESEARCH \& PRACTICE\par
DT \tabto{0cm}Article\par
PG \tabto{0cm}11, NR 60, TC 14\par
DE \tabto{0cm}KNOWLEDGE MANAGEMENT PRACTICE, KNOWLEDGE TRANSFER, SYSTEMS THINKING, TACIT KNOWLEDGE\par
ID \tabto{0cm}ALLIANCES, COMPETITIVE ADVANTAGE, DYNAMIC CAPABILITIES, FIRM, INNOVATION, KNOWLEDGE TRANSFER, MANAGEMENT, NETWORKS, PERFORMANCE, STICKINESS\par
AB \tabto{0cm}The concept of \hl{absorpti}ve \hl{capacit}y has already been considerably studied from a theoretical perspective, but few, if any, attempts at operationalizing the concept have been studied in ways that would allow its full assessment. The more specific focus provided by the four dimensions identified in some recent literature - acquisition, assimilation, transformation and exploitation - opens up some promising avenues for operationalizing the concept. This exploratory research studies and describes case studies of ten innovative companies using a cross-sectional research design. In the first part of the article, we re-examine the concept of \hl{absorpti}ve \hl{capacit}y in terms of dynamic capabilities and provide a review of the relevant literature. The second part describes the work accomplished to operationalize the concept of dynamic capability and analyses the possible relationship between the business strategies adopted by the companies studied and their particular strategic \hl{capacit}y. Knowledge Management Research \& Practice (2011) 9, 367-377. doi:10.1057/kmrp.2011.26\par
\clearpage

\vspace*{-2cm}
Nb \tabto{0cm}151/327 (article\_id: 478)\par
TI \tabto{0cm}Openness, \hl{absorpti}ve \hl{capacit}y, and regional innovation in China\par
AU \tabto{0cm}Yang, Lin\par
PY \tabto{0cm}2012, SO ENVIRONMENT AND PLANNING A\par
DT \tabto{0cm}Article\par
PG \tabto{0cm}23, NR 62, TC 4\par
DE \tabto{0cm}\hl{ABSORPTI}VE \hl{CAPACIT}Y, INNOVATION, PATENT, SPILLOVER\par
ID \tabto{0cm}2 FACES, ECONOMIC-GROWTH, FIRM-LEVEL, FOREIGN DIRECT-INVESTMENT, IN-HOUSE, INDUSTRY, KNOWLEDGE SPILLOVERS, PRODUCTIVITY GROWTH, RESEARCH-AND-DEVELOPMENT, TECHNOLOGY\par
AB \tabto{0cm}This paper systematically investigates the impacts of openness on regional innovation in China, especially the role of \hl{absorpti}ve \hl{capacit}y in mediating the spillover effect brought about by openness. Based on provincial-level data over the period 1997-2007 and adopting patents as indicator of innovation, the empirical results show that the estimated patents - R\&D elasticity is lower than that for OECD countries, while there is a significant R\&D spillover effect across regions in China. Technology import overall has no significant influence on fostering innovations, but it is significantly positive for coastal regions. Openness to trade, in terms of FDI and high-tech product exports, is witnessed to exhibit a significantly positive impact on promoting regional innovation. Crucially, the effects of technology imports and FDI on innovations vary significantly between coastal and non-coastal regions. Using human capital as a proxy for \hl{absorpti}ve \hl{capacit}y, we find it helps in learning external sources of knowledge and then contributes to innovations. Moreover, \hl{absorpti}ve \hl{capacit}y is expected to play an important role in mediating the spillover effect brought about by openness to trade.\par
\clearpage

\vspace*{-2cm}
Nb \tabto{0cm}152/327 (article\_id: 479)\par
TI \tabto{0cm}The impact of regional \hl{absorpti}ve \hl{capacit}y on spatial knowledge spillovers: the Cohen and Levinthal model revisited\par
AU \tabto{0cm}Caragliu, Nijkamp\par
PY \tabto{0cm}2012, SO APPLIED ECONOMICS\par
DT \tabto{0cm}Article\par
PG \tabto{0cm}12, NR 33, TC 9\par
DE \tabto{0cm}null\par
ID \tabto{0cm}DETERMINANTS, ECONOMIC-GROWTH, INNOVATION, PATENT CITATIONS\par
AB \tabto{0cm}We design a conceptual framework for linking two approaches: \hl{absorpti}ve \hl{capacit}y and spatial Knowledge Spillovers (KSs). Regions produce new knowledge, but only part of it is efficiently adopted in the economy; the share of efficiently adopted technology depends on cognitive capital. Our dataset is based on a panel of European regions over the period 1999 to 2006, combining data from EUROSTAT and the European Values Study (EVS). We test the hypothesis that insufficient levels of cognitive capital hamper the capability of regions to fully exploit new knowledge. Results show that a lower regional \hl{absorpti}ve \hl{capacit}y increases KS towards surrounding areas, hampering the regions' capability to decode and efficiently exploit new knowledge, both locally produced and originating from outside.\par
\clearpage

\vspace*{-2cm}
Nb \tabto{0cm}153/327 (article\_id: 480)\par
TI \tabto{0cm}Asymmetrically realized \hl{absorpti}ve \hl{capacit}y and relationship durability\par
AU \tabto{0cm}Andersen, Kask\par
PY \tabto{0cm}2012, SO MANAGEMENT DECISION\par
DT \tabto{0cm}Article\par
PG \tabto{0cm}15, NR 48, TC 7\par
DE \tabto{0cm}CUSTOMER RELATIONSHIP MANAGEMENT, KNOWLEDGE MANAGEMENT, PARTNERS\par
ID \tabto{0cm}COMPETITIVE ADVANTAGE, EVOLUTION, FIRM, INNOVATION, INTERNATIONAL JOINT VENTURES, LINK ALLIANCES, MARKETING RELATIONSHIPS, NETWORKS, PERFORMANCE, STRATEGIC ALLIANCES\par
AB \tabto{0cm}Purpose - Absorbing knowledge from partner firms is a key feature of marketing relationships. Recent publications have called for more dynamic and cognitive approaches in marketing relationship research. Also, established definitions of \hl{absorpti}ve \hl{capacit}ies have been questioned. This article aims to address propositions that take these overlooked and questioned elements into consideration, which can help explain conducts and dependencies, and affect relationship durability.
Design/methodology/approach - The authors put forward four propositions by combining literature on interfirm relationships and managerial cognition with evolutionary ideas from marketing and management literature.
Findings - The authors embrace a redefinition of potential \hl{absorpti}ve \hl{capacit}y (the disposed \hl{capacit}y to absorb knowledge) and realized \hl{absorpti}ve \hl{capacit}y (the \hl{absorpti}on of knowledge actually performed). This distinction can, to some extent, be explained by the degree of cognitive attention given to the marketing relationship. Moreover, asymmetrically realized \hl{absorpti}ve \hl{capacit}y vis-a-vis a partner substantially influences the dynamics of partners' conduct and dependency, which may vary the risk that the relationship will end.
Practical implications - The propositions illustrate how a motivated partner that gives more attention to the relationship is more likely to absorb more knowledge than its counterpart, which can threaten the durability of a relationship. Thus, managers need to be able to understand possible long-term consequences of the partner's conduct in order to avoid losses of joint strategic resources and relational benefits.
Originality/value - By advocating an evolutionary approach, an impetus for more dynamism in marketing relationship research is presented. This study also shows the importance of including the longitudinal dimension in analysis if one wants to understand change in - and durability of - marketing relationships.\par
\clearpage

\vspace*{-2cm}
Nb \tabto{0cm}154/327 (article\_id: 481)\par
TI \tabto{0cm}Six sigma, \hl{absorpti}ve \hl{capacit}y and organisational learning orientation\par
AU \tabto{0cm}Gutierrez, Bustinza, Molina\par
PY \tabto{0cm}2012, SO INTERNATIONAL JOURNAL OF PRODUCTION RESEARCH\par
DT \tabto{0cm}Article\par
PG \tabto{0cm}15, NR 97, TC 9\par
DE \tabto{0cm}\hl{ABSORPTI}VE \hl{CAPACIT}Y, ORGANISATIONAL LEARNING ORIENTATION, PROCESS MANAGEMENT, SIX SIGMA, TEAMWORK\par
ID \tabto{0cm}6-SIGMA IMPLEMENTATION, COMPETITIVE ADVANTAGE, GOAL-THEORETIC PERSPECTIVE, INNOVATION, INTERNATIONAL-JOINT-VENTURES, KNOWLEDGE CREATION, PERFORMANCE, QUALITY MANAGEMENT, SHARED-VISION, SIX-SIGMA\par
AB \tabto{0cm}The importance of the six sigma methodology in industry is growing constantly. However, there are few empirical studies that analyse the advantages of this methodology and its positive effects on organisational performance. The purpose of this paper is to extend understanding of the success of six sigma quality management initiatives by investigating the effects of six sigma teamwork and process management on \hl{absorpti}ve \hl{capacit}y. It also seeks to understand the relation between \hl{absorpti}ve \hl{capacit}y and organisational learning as two sources of sustainable competitive advantage. The information used comes from a larger study, the data for which was collected from a random sample of 237 European firms. Of these 237 organisations, 58 are six sigma organisations. Structural equation modelling (SEM) was used to test the hypotheses. The main findings show that six sigma teamwork and process management positively affect the development of \hl{absorpti}ve \hl{capacit}y. A positive and significant relationship is also observed between \hl{absorpti}ve \hl{capacit}y and organisational learning orientation. The findings of this study justify six sigma implementation in firms. This study provides us with an in-depth understanding of some structural elements that characterise the six sigma methodology, enabling us to provide an explanation for its success.\par
\clearpage

\vspace*{-2cm}
Nb \tabto{0cm}155/327 (article\_id: 482)\par
TI \tabto{0cm}The role of relative \hl{absorpti}ve \hl{capacit}y in improving suppliers' operational performance\par
AU \tabto{0cm}Nagati, Rebolledo\par
PY \tabto{0cm}2012, SO INTERNATIONAL JOURNAL OF OPERATIONS \& PRODUCTION MANAGEMENT\par
DT \tabto{0cm}Article\par
PG \tabto{0cm}20, NR 80, TC 6\par
DE \tabto{0cm}BUYER-SUPPLIER RELATIONSHIPS, CANADA, COLLABORATION, KNOWLEDGE MANAGEMENT\par
ID \tabto{0cm}ANTECEDENTS, CHAIN, FIRMS, IMPROVEMENT, INTERNATIONAL STRATEGIC ALLIANCES, JOINT VENTURES, KNOWLEDGE TRANSFER, MANAGEMENT, NETWORK, ORGANIZATIONAL CAPABILITY\par
AB \tabto{0cm}Purpose - The purpose of this paper is to examine the link between relative \hl{absorpti}ve \hl{capacit}y and suppliers' operational performance.
Design/methodology/approach - The paper uses structural equation modelling of survey data from 218 Canadian manufacturers referring to a particular relationship with one of their customers.
Findings - Results suggest that only the first dimension of the relative \hl{absorpti}ve \hl{capacit}y - knowledge sharing routines - influences the knowledge transferred from the customer to the supplier. Knowledge transfer acts as a mediator between knowledge sharing routines and the supplier's operational performance improvement.
Research limitations/implications - The absence of a significant association between the second dimension of relative \hl{absorpti}ve \hl{capacit}y - overlapped knowledge bases - and knowledge transfer is a surprising result that should be further investigated.
Originality/value - This appears to be the first study to operationalise and empirically test relative \hl{absorpti}ve \hl{capacit}y and its consequences in the particular context of customer-supplier relationships.\par
\clearpage

\vspace*{-2cm}
Nb \tabto{0cm}156/327 (article\_id: 483)\par
TI \tabto{0cm}Improving the \hl{absorpti}ve \hl{capacit}y through unlearning context: an empirical investigation in hospital-in-the-home units\par
AU \tabto{0cm}Cepeda-Carrion, Navarro, Martinez-Caro\par
PY \tabto{0cm}2012, SO SERVICE INDUSTRIES JOURNAL\par
DT \tabto{0cm}Article\par
PG \tabto{0cm}20, NR 79, TC 1\par
DE \tabto{0cm}CHANGE MANAGEMENT, IMPLEMENTATION FAILURE, KNOWLEDGE PROCESSES\par
ID \tabto{0cm}CARE, DYNAMIC CAPABILITIES, INNOVATION, KNOWLEDGE, MARKET ORIENTATION, ORGANIZATIONAL ROUTINES, PARTIAL LEAST-SQUARES, PERFORMANCE, PERSPECTIVE, STRUCTURAL EQUATION MODELS\par
AB \tabto{0cm}The Spanish healthcare system has undergone important changes, particularly in the development of new homecare services. In practice, however, results have been mixed. Some homecare services have been successful, but implementation failures are common and the intended patients are frequently reluctant to use the homecare services. A possible explanation for efficiency and effectiveness gaps of services provided by hospital-in-the-home units (HHUs) may relate to the advantages and disadvantages of the knowledge processes that these units highlight as a result of their different structural properties. This study examines the impact of an unlearning (forgetting) context on the HHU's ability to challenge basic beliefs and to implement processes that are explicitly or tacitly helpful in the reception of new ideas (\hl{absorpti}ve \hl{capacit}y). These relationships are examined through an empirical investigation of 54 doctors and 62 nurses belonging to 44 HHUs. The results show that the unlearning context plays a key role in managing the tension between potential \hl{absorpti}ve \hl{capacit}y and realized \hl{absorpti}ve \hl{capacit}y.\par
\clearpage

\vspace*{-2cm}
Nb \tabto{0cm}157/327 (article\_id: 484)\par
TI \tabto{0cm}The role of a firm's \hl{absorpti}ve \hl{capacit}y and the technology transfer process in clusters: How effective are technology centres in low-tech clusters?\par
AU \tabto{0cm}Hervas-Oliver, Albors-Garrigos, De-Miguel, Hidalgo\par
PY \tabto{0cm}2012, SO ENTREPRENEURSHIP AND REGIONAL DEVELOPMENT\par
DT \tabto{0cm}Article\par
PG \tabto{0cm}37, NR 209, TC 10\par
DE \tabto{0cm}\hl{ABSORPTI}VE \hl{CAPACIT}Y, CLUSTERS, INDUSTRIAL RESEARCH INSTITUTES, SMES, TECHNOLOGY CENTRES\par
ID \tabto{0cm}BASQUE COUNTRY, EMPIRICAL-ANALYSIS, ENTERPRISES SMES, INDUSTRIAL DISTRICTS, LOCAL KNOWLEDGE, PRODUCTIVITY GROWTH, REGIONAL INNOVATION SYSTEMS, RESEARCH-AND-DEVELOPMENT, SECTORAL PATTERNS, SOCIAL NETWORK ANALYSIS\par
AB \tabto{0cm}This paper analyses how the internal resources of small-and medium-sized enterprises determine access (learning processes) to technology centres (TCs) or industrial research institutes (innovation infrastructure) in traditional low-tech clusters. These interactions basically represent traded (market-based) transactions, which constitute important sources of knowledge in clusters. The paper addresses the role of TCs in low-tech clusters, and uses semi-structured interviews with 80 firms in a manufacturing cluster. The results point out that producer-user interactions are the most frequent; thus, the higher the sector knowledge-intensive base, the more likely the utilization of the available research infrastructure becomes. Conversely, the sectors with less knowledge-intensive structures, i.e. less \hl{absorpti}ve \hl{capacit}y (AC), present weak linkages to TCs, as they frequently prefer to interact with suppliers, who act as transceivers of knowledge. Therefore, not all the firms in a cluster can fully exploit the available research infrastructure, and their AC moderates this engagement. In addition, the existence of TCs is not sufficient since the active role of a firm's search strategies to undertake interactions and conduct openness to available sources of knowledge is also needed. The study has implications for policymakers and academia.\par
\clearpage

\vspace*{-2cm}
Nb \tabto{0cm}158/327 (article\_id: 485)\par
TI \tabto{0cm}Constituents and outcomes of \hl{absorpti}ve \hl{capacit}y - appropriability regime changing the game\par
AU \tabto{0cm}Hurmelinna-Laukkanen\par
PY \tabto{0cm}2012, SO MANAGEMENT DECISION\par
DT \tabto{0cm}Article\par
PG \tabto{0cm}22, NR 68, TC 5\par
DE \tabto{0cm}\hl{ABSORPTI}VE \hl{CAPACIT}Y, APPROPRIABILITY, INNOVATION, INNOVATION PERFORMANCE, ORGANIZATIONAL INNOVATION, RESILIENCE\par
ID \tabto{0cm}CAPABILITY, COMPETITIVE ADVANTAGE, DRUG DISCOVERY, EMPIRICAL-TEST, INNOVATION, KNOWLEDGE TRANSFER, MODERATING ROLE, PERFORMANCE, SPILLOVERS, TECHNOLOGY\par
AB \tabto{0cm}Purpose - Recent research and practice have put a great deal of effort into finding efficient ways of managing and organizing to promote innovation within organizations. This study aims to continue this trend in addressing issues related to knowledge transfer and protection through examining roles of \hl{absorpti}ve \hl{capacit}y and appropriability regimes and the interplay between them. An appropriability regime can play a dual role when external knowledge and the knowledge-base of the firm form the basis for \hl{absorpti}ve \hl{capacit}y, which then contributes to innovation performance.
Design/methodology/approach - The study provides an empirical examination of the direct and moderating roles of appropriability regime regarding the above-mentioned dual role. Data collected from 335 firms was utilized to perform regression analyses.
Findings - The empirical evidence suggests, first, that the strength of the appropriability regime has a positive effect on \hl{absorpti}ve \hl{capacit}y (especially the acquisition of knowledge) together with good connectedness to external knowledge sources and high levels of internal R\&D. In addition, support can be found for the idea of \hl{absorpti}ve \hl{capacit}y and the appropriability regime being positively related to innovation performance. Both direct and moderating effects can be found, but they are slightly different for knowledge acquisition and application.
Originality/value - This study contributes to prior studies by producing empirical evidence on the relationships described above. An important issue is also that it departs from prior works by viewing an appropriability regime as a factor that can be affected by the firm (i.e. as a strategic tool), and not as a purely environmental or external factor.\par
\clearpage

\vspace*{-2cm}
Nb \tabto{0cm}159/327 (article\_id: 486)\par
TI \tabto{0cm}The role of \hl{absorpti}ve \hl{capacit}y in partnership retention\par
AU \tabto{0cm}Lee, Woo, Joshi\par
PY \tabto{0cm}2012, SO ASIAN JOURNAL OF TECHNOLOGY INNOVATION\par
DT \tabto{0cm}Article\par
PG \tabto{0cm}15, NR 66, TC 0\par
DE \tabto{0cm}\hl{ABSORPTI}VE \hl{CAPACIT}Y, ALLIANCE STABILITY, PARTNERSHIP, RESOURCE COMPLEMENTARITY, RESOURCE UNIQUENESS\par
ID \tabto{0cm}CAPABILITIES, COMPETITIVE ADVANTAGE, EXCHANGE, INTERNATIONALIZATION, JOINT VENTURES, PERFORMANCE, RESOURCE COMPLEMENTARITY, STRATEGIC ALLIANCES, TECHNOLOGY-BASED FIRMS, TRANSACTION COST\par
AB \tabto{0cm}This study investigates the moderating role of \hl{absorpti}ve \hl{capacit}y in the relationship between the attributes of partner resources and the partnership retention intentions of focal firms. C-level executives from US software small- and medium-sized enterprises (SMEs) were surveyed to obtain data for the study. The results show that \hl{absorpti}ve \hl{capacit}y of SMEs positively moderates the relationship between the uniqueness of partner resources and the intention of SMEs to retain the partnership. However, \hl{absorpti}ve \hl{capacit}y of SMEs negatively moderates the relationship between the complementarity of partner resources and partnership retention intentions. The results of the study contribute to understanding the role of the \hl{absorpti}ve \hl{capacit}y of a focal firm in partnership stability.\par
\clearpage

\vspace*{-2cm}
Nb \tabto{0cm}160/327 (article\_id: 487)\par
TI \tabto{0cm}\hl{Absorpti}ve \hl{capacit}y, learning processes and combinative capabilities as determinants of strategic innovation\par
AU \tabto{0cm}Gebauer, Worch, Truffer\par
PY \tabto{0cm}2012, SO EUROPEAN MANAGEMENT JOURNAL\par
DT \tabto{0cm}Article\par
PG \tabto{0cm}17, NR 41, TC 30\par
DE \tabto{0cm}\hl{ABSORPTI}VE \hl{CAPACIT}Y, COMBINATIVE CAPABILITIES, ELECTRICITY UTILITIES, KNOWLEDGE ACQUISITION AND ASSIMILATION, KNOWLEDGE TRANSFORMATION AND EXPLOITATION, LEARNING PROCESSES, STRATEGIC INNOVATION\par
ID \tabto{0cm}DELIBERATE, DYNAMIC CAPABILITIES, FIRM, KNOWLEDGE, PERFORMANCE, PERSPECTIVE, RECONCEPTUALIZATION\par
AB \tabto{0cm}The current paper focuses on \hl{absorpti}ve \hl{capacit}y in the context of strategic innovation. Strategic innovation aims at a re-conceptualisation of business models, the creation of uncontested market spaces, and leaps in customer value. By using the learning-process perspective of \hl{absorpti}ve \hl{capacit}y (exploratory, assimilative, transformative, and exploitative learning processes), we suggest that transformative learning processes in particular, play a key role in strategic innovation. In addition, a follower strategy and participative role in the knowledge network, instead of a first-mover strategy and a dominant role in the knowledge network, do indeed promote strategic innovation. Companies should not only manage the accumulation of external knowledge, but also adapt their combinative capabilities (systematisation, coordination, and socialisation of knowledge) in order to succeed with strategic innovation. The findings yield a set of research propositions for further academic and managerial consideration. Two longitudinal case studies of European electricity providers form the empirical background. (C) 2011 Elsevier Ltd. All rights reserved.\par
\clearpage

\vspace*{-2cm}
Nb \tabto{0cm}161/327 (article\_id: 488)\par
TI \tabto{0cm}The combined influence of top and middle management leadership styles on \hl{absorpti}ve \hl{capacit}y\par
AU \tabto{0cm}Sun, Anderson\par
PY \tabto{0cm}2012, SO MANAGEMENT LEARNING\par
DT \tabto{0cm}Article\par
PG \tabto{0cm}27, NR 59, TC 3\par
DE \tabto{0cm}\hl{ABSORPTI}VE \hl{CAPACIT}Y, EXPLOITATIVE LEARNING, EXPLORATORY, ORGANIZATIONAL LEARNING, TRANSACTIONAL LEADERSHIP, TRANSFORMATIONAL LEADERSHIP\par
ID \tabto{0cm}CONTEXT, CREATIVITY, EXPLOITATION, INNOVATION, LEARNING-PROCESSES, ORGANIZATIONAL AMBIDEXTERITY, PERFORMANCE, PRODUCT DEVELOPMENT, RECONCEPTUALIZATION, TRANSACTIONAL LEADERSHIP\par
AB \tabto{0cm}\hl{Absorpti}ve \hl{capacit}y is an important organizational capability constituted by exploratory, transformative, and exploitative learning processes. Leadership has been shown to affect such processes, but little is known about how the combined leadership styles of top and middle management influence \hl{absorpti}ve \hl{capacit}y. This theory-building, exploratory qualitative case study discusses the need for top and middle management to be ambidextrous and to change their styles to better facilitate the three different learning processes. We found that an exploratory learning process was facilitated when both top and middle management used a transformational style, a transformative learning process was facilitated when top management used a transformational style while middle management used a transactional style, and an exploitative learning process was facilitated when both top and middle management used a transactional style. Furthermore, for each of the three learning processes, the leadership styles of top and middle management operated more effectively when certain attributes of the organizational context were emphasized.\par
\clearpage

\vspace*{-2cm}
Nb \tabto{0cm}162/327 (article\_id: 489)\par
TI \tabto{0cm}The Effect of \hl{Absorpti}ve \hl{Capacit}y on Innovativeness: Context and Information Systems Capability as Catalysts\par
AU \tabto{0cm}Cepeda-Carrion, Cegarra-Navarro, Jimenez-Jimenez\par
PY \tabto{0cm}2012, SO BRITISH JOURNAL OF MANAGEMENT\par
DT \tabto{0cm}Review\par
PG \tabto{0cm}20, NR 105, TC 26\par
DE \tabto{0cm}null\par
ID \tabto{0cm}ANTECEDENTS, DYNAMIC CAPABILITIES, FRAMEWORK, KNOWLEDGE MANAGEMENT, MARKET ORIENTATION, ORGANIZATIONAL PERFORMANCE, PERSPECTIVE, PRODUCT DEVELOPMENT, TACIT KNOWLEDGE, TECHNOLOGY\par
AB \tabto{0cm}The purpose of this study is to examine the relationship between \hl{absorpti}ve \hl{capacit}y and company innovativeness and to identify potential contexts and capabilities that can act as catalysts for these relationships. We also examine the relationship between \hl{absorpti}ve \hl{capacit}y and the existence and enhancement of innovativeness. These relationships are examined through an empirical investigation of 286 large Spanish companies. Our results show that \hl{absorpti}ve \hl{capacit}y is an important dynamic determinant for developing a company's innovativeness. Moreover, this relationship is best explained by two related constructs. First, the company's unlearning context is a crucial determinant for both potential \hl{capacit}y and realized \hl{absorpti}ve \hl{capacit}y. Second, the results also indicate a tangible means for managers to enhance their \hl{absorpti}ve \hl{capacit}y through information systems capabilities.\par
\clearpage

\vspace*{-2cm}
Nb \tabto{0cm}163/327 (article\_id: 490)\par
TI \tabto{0cm}Enhancing effects of manufacturing flexibility through operational \hl{absorpti}ve \hl{capacit}y and operational ambidexterity\par
AU \tabto{0cm}Patel, Terjesen, Li\par
PY \tabto{0cm}2012, SO JOURNAL OF OPERATIONS MANAGEMENT\par
DT \tabto{0cm}Article\par
PG \tabto{0cm}20, NR 105, TC 26\par
DE \tabto{0cm}ENVIRONMENTAL UNCERTAINTY, LATENT MODERATED STRUCTURAL EQUATIONS, MANUFACTURING FLEXIBILITY, OPERATIONAL \hl{ABSORPTI}VE \hl{CAPACIT}Y, OPERATIONAL AMBIDEXTERITY\par
ID \tabto{0cm}COMPETITIVE ADVANTAGE, CONGRUENCE RESEARCH, ENVIRONMENTAL UNCERTAINTY, FUTURE-DIRECTIONS, LATENT-VARIABLES, MAXIMUM-LIKELIHOOD-ESTIMATION, MEDIUM-SIZED FIRMS, ORGANIZATIONAL AMBIDEXTERITY, STRUCTURAL EQUATION MODELS, SUPPLY CHAIN MANAGEMENT\par
AB \tabto{0cm}A large body of research investigates how manufacturing flexibility in uncertain environments leads to firm performance, with mixed results. The mixed findings could be due to differences across firms in terms of the capabilities to acquire, assimilate, and transform knowledge and to simultaneously pursue both the exploitation of existing operational capabilities and the exploration for new operational capabilities. Building on the literature that suggests that manufacturing flexibility mediates the relationship between environmental uncertainty and firm performance, we explore the applicability of two organizational learning contingencies to the operations environment: operational \hl{absorpti}ve capability and operational ambidexterity. \hl{Absorpti}ve \hl{capacit}y enables the recognition and assimilation of new knowledge. Ambidexterity determines whether this knowledge will be applied for both exploration and exploitation. Using a sample of 852 manufacturing firms, we find that environmental uncertainty affects firm performance directly and indirectly through manufacturing flexibility. Furthermore, both operational \hl{absorpti}ve \hl{capacit}y and operational ambidexterity moderate the relationship between environmental uncertainty and manufacturing flexibility and the relationship between manufacturing flexibility and firm performance. Theoretical and practical implications are discussed. (C) 2011 Elsevier B.V. All rights reserved.\par
\clearpage

\vspace*{-2cm}
Nb \tabto{0cm}164/327 (article\_id: 491)\par
TI \tabto{0cm}\hl{Absorpti}ve \hl{capacit}y and localized spillovers: focal firms as technological gatekeepers in industrial districts\par
AU \tabto{0cm}Munari, Sobrero, Malipiero\par
PY \tabto{0cm}2012, SO INDUSTRIAL AND CORPORATE CHANGE\par
DT \tabto{0cm}Article\par
PG \tabto{0cm}34, NR 86, TC 9\par
DE \tabto{0cm}O30, O32\par
ID \tabto{0cm}CLUSTERS, COMPETITION, DETERMINANTS, GEOGRAPHY, INNOVATION, KNOWLEDGE SPILLOVERS, PATENT CITATIONS, PATTERNS, PRODUCTION NETWORKS, SEMICONDUCTOR INDUSTRY\par
AB \tabto{0cm}Despite the diffusion of communication tools and boundary spanning technologies, knowledge flows in innovation processes retain a distinct localized nature in many industries, and geographical clusters emerge as critical areas to foster technological diffusion. In this article, we focus on the knowledge mediating role, as technological "gatekeepers," of focal firms in industrial districts. Based on a longitudinal dataset of 720 patents granted by the United States Patent and Trademark Office (USPTO) between 1990 and 2003 to firms in the automatic packaging machinery industrial district in Northern Italy, and controlling for the uneven geographical distribution of patenting activities, we show that: (i) firms within the district use local knowledge to a greater extent and more rapidly than knowledge from outside the district, (ii) focal firms use external knowledge to a greater extent than other firms operating in the district, and (iii) other (nonfocal) firms within the district rely on knowledge generated by focal firms to a greater extent than would be expected, given the geographic distribution of innovative activity in the industry. Implications for research on innovation in localized economic systems and firm-level strategic differentiation are discussed.\par
\clearpage

\vspace*{-2cm}
Nb \tabto{0cm}165/327 (article\_id: 492)\par
TI \tabto{0cm}Measuring \hl{absorpti}ve \hl{capacit}y constraints to foreign aid\par
AU \tabto{0cm}Feeny, de Silva\par
PY \tabto{0cm}2012, SO ECONOMIC MODELLING\par
DT \tabto{0cm}Article\par
PG \tabto{0cm}9, NR 61, TC 5\par
DE \tabto{0cm}\hl{ABSORPTI}VE \hl{CAPACIT}Y, DEVELOPING COUNTRIES, ECONOMIC GROWTH, FOREIGN AID\par
ID \tabto{0cm}AFRICA, ALLOCATION, GOVERNANCE, GROWTH, INTERNATIONAL AID, OPPORTUNITY, PERFORMANCE, POVERTY REDUCTION, PROLIFERATION, VOLATILITY\par
AB \tabto{0cm}To assist with progress towards the United Nations Millennium Development Goals (MDGs) in developing countries, the international community is scaling-up foreign aid to record levels. Concurrently, there are concerns that additional aid will not be used effectively due to a problem of \hl{absorpti}ve \hl{capacit}y in recipient countries. Empirical studies lend support to these concerns with many finding that there are diminishing returns to foreign aid. This paper reviews the extensive aid effectiveness literature to identify the various dimensions of \hl{absorpti}ve \hl{capacit}y. It proceeds by devising a composite index of \hl{absorpti}ve \hl{capacit}y for individual recipient countries which can assist policymakers in guiding the allocation of their aid. The relevance of the index is confirmed through its employment in a standard empirical model of aid effectiveness. The paper highlights the developing countries that currently receive high levels of aid relative to their estimated level of \hl{absorpti}ve \hl{capacit}y. (C) 2012 Elsevier B.V. All rights reserved.\par
\clearpage

\vspace*{-2cm}
Nb \tabto{0cm}166/327 (article\_id: 493)\par
TI \tabto{0cm}The alliance innovation performance of R\&D alliances-the \hl{absorpti}ve \hl{capacit}y perspective\par
AU \tabto{0cm}Lin, Wu, Chang, Wang, Lee\par
PY \tabto{0cm}2012, SO TECHNOVATION\par
DT \tabto{0cm}Article\par
PG \tabto{0cm}11, NR 89, TC 30\par
DE \tabto{0cm}\hl{ABSORPTI}VE \hl{CAPACIT}Y, ALLIANCE PORTFOLIO, CO-PATENTING, R\&D ALLIANCE, R\&D INTENSITY, TECHNOLOGICAL DISTANCE\par
ID \tabto{0cm}FIRM PERFORMANCE, INTENSIVE FIRMS, INTERNAL CAPABILITIES, INTERORGANIZATIONAL COLLABORATION, KNOWLEDGE, MANUFACTURING-INDUSTRY, NETWORKS, PRODUCT INTRODUCTIONS, STRATEGIC ALLIANCES, TECHNOLOGICAL DIVERSITY\par
AB \tabto{0cm}In this work we explore the role of interfirm R\&D alliances as a vital mechanism for creating new technological knowledge. Drawing on the \hl{absorpti}ve \hl{capacit}y perspective, we argue that firms with a high level of such \hl{capacit}y seem to benefit more from their alliances. Specifically, three indicators of technology strategy relevant to \hl{absorpti}ve \hl{capacit}y, including proportion of R\&D alliances in an alliance portfolio, technological distance, and R\&D intensity are explored to examine their impacts on innovation performance. Using alliance data from the Securities Data Company (SDC), patent data from the United States Patent and Trademark Office (USPTO), firm data from S\&P COMPUSTAT, and co-patents granted as a proxy for the alliance innovation performance, these results show that while alliance networks potentially provide a firm with access to various benefits that can help in creating new technologies, R\&D alliances in particular are more suitable than other types of partnerships to achieve this aim. Furthermore, given that information transfer and learning are key benefits of R\&D alliances, moderate technological distance is needed if such alliances are to be successful. In particular, the innovation performance peaks at the moderate level of technological distance with alliance partners when this interacts with the proportion of R\&D alliances in a firm's alliance portfolio. Finally, R\&D alliances should be regarded as a complement to rather than a substitute for a firm's internal R\&D. (C) 2012 Elsevier Ltd. All rights reserved.\par
\clearpage

\vspace*{-2cm}
Nb \tabto{0cm}167/327 (article\_id: 494)\par
TI \tabto{0cm}Managing open incremental process innovation: \hl{Absorpti}ve \hl{Capacit}y and distributed learning\par
AU \tabto{0cm}Robertson, Casali, Jacobson\par
PY \tabto{0cm}2012, SO RESEARCH POLICY\par
DT \tabto{0cm}Article\par
PG \tabto{0cm}11, NR 91, TC 16\par
DE \tabto{0cm}\hl{ABSORPTI}VE \hl{CAPACIT}Y, CAPABILITIES, INCREMENTAL INNOVATION, NETWORKS, OPEN INNOVATION, PROCESS INNOVATION\par
ID \tabto{0cm}COMPETITIVE ADVANTAGE, DYNAMIC CAPABILITIES, EVOLUTION, EXTERNAL TECHNOLOGY INTEGRATION, FIRM PERFORMANCE, KNOWLEDGE MANAGEMENT, MODEL, ORGANIZATION, PRODUCT, STRATEGIC ALLIANCES\par
AB \tabto{0cm}In this conceptual article, we extend earlier work on Open Innovation and \hl{Absorpti}ve \hl{Capacit}y. We suggest that the literature on \hl{Absorpti}ve \hl{Capacit}y does not place sufficient emphasis on distributed knowledge and learning or on the application of innovative knowledge. To accomplish physical transformations, organisations need specific Innovative \hl{Capacit}ies that extend beyond knowledge management. Accessive \hl{Capacit}y is the ability to collect, sort and analyse knowledge from both internal and external sources. Adaptive \hl{Capacit}y is needed to ensure that new pieces of equipment are suitable for the organisation's own purposes even though they may have been originally developed for other uses. Integrative \hl{Capacit}y makes it possible for a new or modified piece of equipment to be fitted into an existing production process with a minimum of inessential and expensive adjustment elsewhere in the process. These Innovative \hl{Capacit}ies are controlled and coordinated by Innovative Management \hl{Capacit}y, a higher-order dynamic capability. (C) 2012 Elsevier B.V. All rights reserved.\par
\clearpage

\vspace*{-2cm}
Nb \tabto{0cm}168/327 (article\_id: 495)\par
TI \tabto{0cm}Strategic management of a family-owned airline: Analysing the \hl{absorpti}ve \hl{capacit}y of Cimber Sterling Group A/S\par
AU \tabto{0cm}Boyd, Hollensen\par
PY \tabto{0cm}2012, SO JOURNAL OF FAMILY BUSINESS STRATEGY\par
DT \tabto{0cm}Article\par
PG \tabto{0cm}9, NR 74, TC 4\par
DE \tabto{0cm}\hl{ABSORPTI}VE \hl{CAPACIT}Y, AVIATION INDUSTRY, COMPETITIVE ADVANTAGES, FAMILY FIRMS, LOW-COST AIRLINES, STRATEGIC MANAGEMENT\par
ID \tabto{0cm}BUSINESSES, CAPABILITIES, COMPETITIVE ADVANTAGE, INTERNATIONALIZATION, OPPORTUNITIES, PERSPECTIVE\par
AB \tabto{0cm}The concept of \hl{absorpti}ve \hl{capacit}y (ACAP) observing a firm's ability to value, assimilate and utilise new external knowledge is applied in this paper. This case study analysis focuses on the strategic management processes and competitiveness of the Cimber Sterling airline. The aim is to discover resources and capabilities that lead to competitive advantages within the aviation industry. From an ACAP perspective, Cimber Sterling Group A/S was analysed by interviewing selected owners, managers and employees of the airline. A comparison within the airline industry regarding external factors and the strategic management of other selected low-cost airlines is part of the ACAP concept. The analysis shows to what extent Cimber Sterling Group A/S, as a Danish family business, copes with increasing competition and critical situations, such as the recent volcanic ash cloud and the financial crisis. Identifying the potential and realised \hl{capacit}y of the strategic management of airlines was revealed as a source of strategic competitiveness. The ACAP was improved through stakeholder experience, strategic flexibility, networking capabilities and customer orientation, leading to a competitive advantage realisation in the low-cost airline market. (c) 2012 Elsevier Ltd. All rights reserved.\par
\clearpage

\vspace*{-2cm}
Nb \tabto{0cm}169/327 (article\_id: 496)\par
TI \tabto{0cm}\hl{ABSORPTI}VE \hl{CAPACIT}Y AND INFORMATION SYSTEMS RESEARCH: REVIEW, SYNTHESIS, AND DIRECTIONS FOR FUTURE RESEARCH\par
AU \tabto{0cm}Roberts, Galluch, Dinger, Grover\par
PY \tabto{0cm}2012, SO MIS QUARTERLY\par
DT \tabto{0cm}Review\par
PG \tabto{0cm}24, NR 116, TC 32\par
DE \tabto{0cm}\hl{ABSORPTI}VE \hl{CAPACIT}Y, INFORMATION SYSTEMS, IT CAPABILITY, KNOWLEDGE, ORGANIZATIONAL LEARNING\par
ID \tabto{0cm}BUSINESS, COMPETITIVE ADVANTAGE, INNOVATION, KNOWLEDGE TRANSFER, LEARNING OUTCOMES, ORGANIZATIONS, PERFORMANCE, PRODUCT DEVELOPMENT, RESOURCE-BASED ANALYSIS, TECHNOLOGY PROFESSIONALS\par
AB \tabto{0cm}\hl{Absorpti}ve \hl{capacit}y is a firm's ability to identify assimilate, transform, and apply valuable external knowledge. iris considered an imperative for business success. Modern information technologies perform a critical role in the development and maintenance of a firm's \hl{absorpti}ve \hl{capacit}y. We provide an assessment of \hl{absorpti}ve \hl{capacit}y in the information systems literature. IS scholars have used the \hl{absorpti}ve \hl{capacit}y construct in diverse and often contradictory ways. Confusion surrounds how \hl{absorpti}ve \hl{capacit}y should be conceptualized, its appropriate level of analysis, and how it can be measured. Our aim in reviewing this construct is to reduce such confusion by improving our understanding of \hl{absorpti}ve \hl{capacit}y and guiding its effective use in IS research. We trace the evolution of the \hl{absorpti}ve \hl{capacit}y construct in the broader organizational literature and pay special attention to its conceptualization, assumptions, and relationship to organizational learning. Following this, we investigate how \hl{absorpti}ve \hl{capacit}y has been conceptualized, measured, and used in IS research. We also examine how \hl{absorpti}ve \hl{capacit}y fits into distinct IS themes and facilitates understanding of various IS phenomena. Based on our analysis, we provide a framework through which IS researchers can more fully leverage the rich aspects of \hl{absorpti}ve \hl{capacit}y when investigating the role of information technology in organizations.\par
\clearpage

\vspace*{-2cm}
Nb \tabto{0cm}170/327 (article\_id: 497)\par
TI \tabto{0cm}The Social Dynamics Underpinning Telecentres in Nepal: Feedback and \hl{Absorpti}ve \hl{Capacit}y in a National Innovation System\par
AU \tabto{0cm}Turpin, Ghimire\par
PY \tabto{0cm}2012, SO SCIENCE TECHNOLOGY AND SOCIETY\par
DT \tabto{0cm}Article\par
PG \tabto{0cm}20, NR 46, TC 0\par
DE \tabto{0cm}null\par
ID \tabto{0cm}DEVELOPING-COUNTRIES, TECHNOLOGY\par
AB \tabto{0cm}This article offers a conceptual analysis of two aspects of openness: extended feedback and \hl{absorpti}ve \hl{capacit}y in the context of a developing country national innovation system. Extended feedback is defined as the \hl{capacit}y of national agencies, responsible for telecentre development, to learn and share learning about the practices, ideas and information demands of people using their telecentres. This allows for a greater diffusion of knowledge across the national innovation system (NIS). \hl{Absorpti}ve \hl{capacit}y is defined as the \hl{capacit}y of users to access and share information and apply it for productive practices. These concepts are used in this article to help explain the ways that social networks can be consolidated and extended, horizontally and vertically, through telecentres in rural Nepal and consequently contribute to social and economic development. The argument is developed that telecentres, providing there is extended feedback at the institutional level and \hl{absorpti}ve \hl{capacit}y at both local and central levels, extend access to 'bridging social capital'. The paper concludes by identifying some potential indicators for monitoring and evaluating the impact of ICT (Information Communication Technologies) using these concepts.\par
\clearpage

\vspace*{-2cm}
Nb \tabto{0cm}171/327 (article\_id: 498)\par
TI \tabto{0cm}EXPATRIATE KNOWLEDGE TRANSFER, SUBSIDIARY \hl{ABSORPTI}VE \hl{CAPACIT}Y, AND SUBSIDIARY PERFORMANCE\par
AU \tabto{0cm}Chang, Gong, Peng\par
PY \tabto{0cm}2012, SO ACADEMY OF MANAGEMENT JOURNAL\par
DT \tabto{0cm}Article\par
PG \tabto{0cm}22, NR 79, TC 50\par
DE \tabto{0cm}null\par
ID \tabto{0cm}CULTURAL DISTANCE, FIRM, MANAGERS, METHOD VARIANCE, MODEL, MULTINATIONAL-CORPORATIONS, ORGANIZATIONS, PERSPECTIVE, RELATIONAL EMBEDDEDNESS, SELF-REPORTED AFFECT\par
AB \tabto{0cm}In this study, we theoretically identify three dimensions of expatriate competencies ability, motivation, and opportunity seeking for knowledge transfer. Integrating the ability-motivation-opportunity framework and the \hl{absorpti}ve \hl{capacit}y perspective, we propose that expatriate competencies in knowledge transfer influence a subsidiary's performance through the knowledge received by the subsidiary, but that this indirect effect is stronger when subsidiary \hl{absorpti}ve \hl{capacit}y is greater. We collected multi-source, time-lagged data from 162 British subsidiaries of Taiwanese multinational firms. The results support our hypotheses. Overall, we contribute to expatriation theory and research by revealing specific expatriate competencies as well as identifying boundary conditions for successful expatriate knowledge transfer.\par
\clearpage

\vspace*{-2cm}
Nb \tabto{0cm}172/327 (article\_id: 499)\par
TI \tabto{0cm}\hl{Absorpti}ve \hl{capacit}y and post-acquisition inventor productivity\par
AU \tabto{0cm}Hussinger\par
PY \tabto{0cm}2012, SO JOURNAL OF TECHNOLOGY TRANSFER\par
DT \tabto{0cm}Article\par
PG \tabto{0cm}18, NR 42, TC 3\par
DE \tabto{0cm}\hl{ABSORPTI}VE \hl{CAPACIT}Y, INVENTOR PRODUCTIVITY, M \& AS\par
ID \tabto{0cm}ACQUISITIONS, BIOTECHNOLOGY, FIRM PERFORMANCE, IMPACT, INNOVATION, KNOWLEDGE, MARKET, MERGERS, SAMPLE SELECTION, TECHNOLOGY\par
AB \tabto{0cm}Inventors often experience a low productivity after their company has been subject to a merger or acquisition (M\&As). It is of central managerial interest to identify factors facilitating the integration of new inventive staff and thereby counteracting innovation declines after M\&As. This paper provides empirical evidence into the role of acquiring firms' \hl{absorpti}ve \hl{capacit}y for the post-merger patent productivity of the acquired inventors. Based on a sample of 544 inventors employed by European acquisition targets in the period 2000-2001 it is shown that the post-merger productivity of acquired inventors is significantly higher within acquiring firms with a distinct \hl{absorpti}ve \hl{capacit}y. It can be concluded that \hl{absorpti}ve \hl{capacit}y is a firm capability that enhances the integration of inventors after firm takeovers.\par
\clearpage

\vspace*{-2cm}
Nb \tabto{0cm}173/327 (article\_id: 500)\par
TI \tabto{0cm}Relative \hl{absorpti}ve \hl{capacit}y: a research profiling\par
AU \tabto{0cm}Martinez, Jaime, Camacho\par
PY \tabto{0cm}2012, SO SCIENTOMETRICS\par
DT \tabto{0cm}Article\par
PG \tabto{0cm}18, NR 53, TC 3\par
DE \tabto{0cm}BIBLIOMETRICS, PUBLICATION ANALYSIS, RELATIVE \hl{ABSORPTI}VE \hl{CAPACIT}Y, RESEARCH PROFILING\par
ID \tabto{0cm}BIBLIOMETRIC ANALYSIS, CAPABILITIES, COLLABORATION, FIRM, INNOVATION, KNOWLEDGE, SCIENCE, STRATEGIC ALLIANCES, TECHNOLOGY, TOOL\par
AB \tabto{0cm}This paper provides the profiling on the 'relative \hl{absorpti}ve \hl{capacit}y of knowledge' research to provide insights of the field based on data collected from the ISI Web of science database during the years 2001-2010. The analysis is established in three phases, namely, the general publication, the subject area, and the topic profiling. The study obtains patterns, characteristics, and attributes at country, institutions, journals, author, and core reference levels. It shows the increase of the research activity in the field, based on the publication productivity during the years mentioned. Most of these publications are classified in the subject areas of business and economics, engineering, and operations research and management science. We highlight the nascent interest of the computer science subject area as a way to operationalize the different studies conducted. We found a lack of contribution from African and Latin-American countries despite the importance of the field for them. Our results are useful in terms of science strategy, science and technology policy, research agendas, research alliances, and research networks according to the special interest of specific actors at the individual, institutional, and national levels.\par
\clearpage

\vspace*{-2cm}
Nb \tabto{0cm}174/327 (article\_id: 501)\par
TI \tabto{0cm}Enriching \hl{Absorpti}ve \hl{Capacit}y through Social Interaction\par
AU \tabto{0cm}Hotho, Becker-Ritterspach, Saka-Helmhout\par
PY \tabto{0cm}2012, SO BRITISH JOURNAL OF MANAGEMENT\par
DT \tabto{0cm}Article\par
PG \tabto{0cm}19, NR 64, TC 11\par
DE \tabto{0cm}null\par
ID \tabto{0cm}CAPABILITIES, FIRM, INNOVATION, KNOWLEDGE TRANSFER, MANAGING KNOWLEDGE, MULTINATIONAL-CORPORATIONS, NETWORKS, ORGANIZATION, PERFORMANCE, SUBSIDIARIES\par
AB \tabto{0cm}\hl{Absorpti}ve \hl{capacit}y is frequently highlighted as a key determinant of knowledge transfer within multinational enterprises. But how individual behaviour translates into \hl{absorpti}ve \hl{capacit}y at the subsidiary level, and how this is contingent on subsidiaries' social context, remains under-addressed. This not only limits our understanding of the relationship between individual- and organizational-level \hl{absorpti}ve \hl{capacit}y, but also hampers further research on potentially relevant managerial and organizational antecedents, and limits the implications we can draw for practitioners who seek to increase their organization's \hl{capacit}y to put new knowledge to use. To address this shortcoming we conduct an in-depth comparative case study of a headquarters-initiated knowledge transfer at two subsidiaries of the same multinational enterprise. The findings demonstrate that social interaction is a prerequisite for subsidiary \hl{absorpti}ve \hl{capacit}y as it enables employees to participate in the transformation of new knowledge to the local context and the development of local applications. The findings also illustrate how organizational conditions at the subsidiary level can impact subsidiary \hl{absorpti}ve \hl{capacit}y by enabling or constraining local interaction patterns. These insights contribute to the \hl{absorpti}ve \hl{capacit}y literature by demonstrating the scale and scope of social interaction as a key link between individual- and organizational-level \hl{absorpti}ve \hl{capacit}y.\par
\clearpage

\vspace*{-2cm}
Nb \tabto{0cm}175/327 (article\_id: 502)\par
TI \tabto{0cm}The temporal effects of relative and firm-level \hl{absorpti}ve \hl{capacit}y on interorganizational learning\par
AU \tabto{0cm}Schildt, Keil, Maula\par
PY \tabto{0cm}2012, SO STRATEGIC MANAGEMENT JOURNAL\par
DT \tabto{0cm}Article\par
PG \tabto{0cm}20, NR 77, TC 12\par
DE \tabto{0cm}\hl{ABSORPTI}VE \hl{CAPACIT}Y, ALLIANCE, INTERORGANIZATIONAL LEARNING, PATENTS, TIME\par
ID \tabto{0cm}ACQUISITION PERFORMANCE, CAPABILITIES, EMPIRICAL-EXAMINATION, INTERNATIONAL JOINT VENTURES, KNOWLEDGE TRANSFER, LOCAL SEARCH, PATENT CITATIONS, RESEARCH-AND-DEVELOPMENT, STRATEGIC ALLIANCES, TECHNOLOGICAL DIVERSIFICATION\par
AB \tabto{0cm}We examine how determinants of \hl{absorpti}ve \hl{capacit}y influence learning in alliances over time. Using longitudinal patent cross-citation data, we find an inverted U-shaped pattern over time that is influenced by firm-level and relational factors. Technological similarity only modestly increases learning in the initial stages of a relationship, but moderate levels substantially increase knowledge flows later in the alliance. High technological diversity is related to higher initial learning rates, but the effects diminish over time. Somewhat surprisingly, research and development intensity is negatively related to initial learning rates but has a considerable positive effect later in the relationship. We suggest that initial learning rates in alliances may be constrained by the \hl{capacit}y to absorb knowledge, while later-stage outcomes are constrained by exploitation \hl{capacit}y. Copyright (c) 2012 John Wiley \& Sons, Ltd.\par
\clearpage

\vspace*{-2cm}
Nb \tabto{0cm}176/327 (article\_id: 503)\par
TI \tabto{0cm}\hl{Absorpti}ve \hl{Capacit}y and ERP Implementation in Indian Medium-Sized Firms\par
AU \tabto{0cm}Sharma, Daniel, Gray\par
PY \tabto{0cm}2012, SO JOURNAL OF GLOBAL INFORMATION MANAGEMENT\par
DT \tabto{0cm}Article\par
PG \tabto{0cm}26, NR 101, TC 1\par
DE \tabto{0cm}\hl{ABSORPTI}VE \hl{CAPACIT}Y, DEVELOPING COUNTRIES, ENTERPRISE RESOURCE PLANNING (ERP), KNOWLEDGE, SMALL-MEDIUM ENTERPRISES (SMES)\par
ID \tabto{0cm}ANTECEDENTS, CRITICAL SUCCESS FACTORS, DYNAMIC CAPABILITIES, ENTERPRISE SYSTEMS, INFORMATION-SYSTEMS, JOINT VENTURES, KNOWLEDGE LIFE-CYCLE, PERSPECTIVE, STRATEGIC MANAGEMENT, SUPPLY CHAINS\par
AB \tabto{0cm}Whilst \hl{absorpti}ve \hl{capacit}y has been identified as an important contributor to the effective implementation of IT systems, previous studies have failed to explicitly consider the contribution of individual and organizational knowledge processes. Nine case studies of Enterprise Resource Planning (ERP) implementation were undertaken. The case studies were all undertaken in SMEs in a developing country since this is an important but under researched area for the application of the concept of \hl{absorpti}ve \hl{capacit}y. A particular implication of the findings is that firms lacking knowledge of IT implementation cannot simply seek this from external sources but must develop internal organizational knowledge processes if their implementations of IT systems are to be effective. This finding is particularly pertinent to the developing country and SME context of this study, where low levels of experience within the firm and the loss of experienced staff are found to impact on the development of \hl{absorpti}ve \hl{capacit}y.\par
\clearpage

\vspace*{-2cm}
Nb \tabto{0cm}177/327 (article\_id: 504)\par
TI \tabto{0cm}Diversification and Innovation Revisited: An \hl{Absorpti}ve \hl{Capacit}y View of Technological Knowledge Creation\par
AU \tabto{0cm}Sugheir, Phan, Hasan\par
PY \tabto{0cm}2012, SO IEEE TRANSACTIONS ON ENGINEERING MANAGEMENT\par
DT \tabto{0cm}Article\par
PG \tabto{0cm}10, NR 84, TC 5\par
DE \tabto{0cm}\hl{ABSORPTI}VE \hl{CAPACIT}Y, INNOVATION, PATENTS, PRODUCT DIVERSIFICATION, TECHNOLOGY-BASED FIRMS\par
ID \tabto{0cm}COMPETITIVE ADVANTAGE, DYNAMIC THEORY, ENTROPY MEASURE, FIRM PERFORMANCE, MULTINATIONAL-CORPORATION, MULTIPRODUCT FIRMS, ORGANIZATIONAL KNOWLEDGE, PRODUCT DIVERSIFICATION, RESEARCH-AND-DEVELOPMENT, RESOURCE-BASED VIEW\par
AB \tabto{0cm}The relationship between innovation and product diversification in firms has been studied and debated for decades. Early articles proposed a positive relationship, while subsequent research supported a negative influence on innovation from product diversification based on observable reductions in research and development expenditures. Such findings also suggest a negative influence on \hl{absorpti}ve \hl{capacit}y from increasing product diversification. This paper uses an \hl{absorpti}ve \hl{capacit}y perspective to revisit the relationship. Together with related literature on knowledge creation and transfer processes, a positive association between related product diversification by firms and the quantity of created technological knowledge is suggested. Evidence to support such a relationship is provided using patent data from technology-based firms in a sample of 1997 firm years between 1990 and 2006. Some evidence of a negative association between knowledge creation and very high levels of unrelated diversification is indicated, qualifying and supporting the "M-form" hypothesis. The findings more closely align understandings of the relationship between product diversification and innovation with the relationship between product diversification and firm performance.\par
\clearpage

\vspace*{-2cm}
Nb \tabto{0cm}178/327 (article\_id: 505)\par
TI \tabto{0cm}Foreign direct investment and income inequality: Does the relationship vary with \hl{absorpti}ve \hl{capacit}y?\par
AU \tabto{0cm}Wu, Hsu\par
PY \tabto{0cm}2012, SO ECONOMIC MODELLING\par
DT \tabto{0cm}Article\par
PG \tabto{0cm}7, NR 38, TC 2\par
DE \tabto{0cm}\hl{ABSORPTI}VE \hl{CAPACIT}Y, FOREIGN DIRECT INVESTMENT, INCOME INEQUALITY, THRESHOLD REGRESSION\par
ID \tabto{0cm}COUNTRIES, ECONOMIC-GROWTH, FDI, OPENNESS, PANEL, POOR, SPILLOVERS, TESTS\par
AB \tabto{0cm}There has been little systematic empirical literature on the linkage between income inequality and FDI (Basu and Guariglia, 2007; Tsai, 1995). This paper analyzes the effects of foreign direct investment (FDI) on income inequality and asks whether the relationship depends on \hl{absorpti}ve \hl{capacit}y or not, by using a cross-sectional dataset taken from 54 countries over the period 1980-2005. We adopt the endogenous threshold regression model proposed by Hansen (2000) and Caner and Hansen (2004) and find strong evidence of a two-regime split in our sample. That is, FDI is likely to be harmful to the income distribution of those host countries with low levels of \hl{absorpti}ve \hl{capacit}y. By contrast, our results support the perspective that FDI has little effect on income inequality in the case of countries with better \hl{absorpti}ve \hl{capacit}y. It is also shown that international trade can lead to more equal income distribution. Crown Copyright (C) 2012 Published by Elsevier B.V. All rights reserved.\par
\clearpage

\vspace*{-2cm}
Nb \tabto{0cm}179/327 (article\_id: 506)\par
TI \tabto{0cm}How do young firms manage product portfolio complexity? The role of \hl{absorpti}ve \hl{capacit}y and ambidexterity\par
AU \tabto{0cm}Fernhaber, Patel\par
PY \tabto{0cm}2012, SO STRATEGIC MANAGEMENT JOURNAL\par
DT \tabto{0cm}Article\par
PG \tabto{0cm}24, NR 104, TC 17\par
DE \tabto{0cm}\hl{ABSORPTI}VE \hl{CAPACIT}Y, AMBIDEXTERITY, ORGANIZATIONAL LEARNING, PRODUCT PORTFOLIO COMPLEXITY, TECHNOLOGY\par
ID \tabto{0cm}CAPABILITIES, COMPETITIVE ADVANTAGE, CONGRUENCE RESEARCH, EXPLOITATION, INNOVATION, ORGANIZATIONAL AMBIDEXTERITY, PERFORMANCE, RESEARCH-AND-DEVELOPMENT, SOCIAL INTEGRATION, STRATEGIC FLEXIBILITY\par
AB \tabto{0cm}Building a complex portfolio of products can be beneficial for young firms due to increased sales growth and competitiveness. Yet, the benefits from product portfolio complexity (PPC) are often outweighed by rising costs, leading to an inverted U-shaped relationship between PPC and performance. Recent research has called for an increased understanding of how firms are able to better manage higher levels of PPC. We suggest that \hl{absorpti}ve \hl{capacit}y and ambidexterity are vital to enhancing the benefits and mitigating the costs of increasing PPC. Using a sample of 215 young high technology firms, we find support for positive moderating effects of \hl{absorpti}ve \hl{capacit}y and ambidexterity on the inverted U-shaped relationship between PPC and firm performance. Copyright (C) 2012 John Wiley \& Sons, Ltd.\par
\clearpage

\vspace*{-2cm}
Nb \tabto{0cm}180/327 (article\_id: 507)\par
TI \tabto{0cm}Orchestrating R\&D networks: \hl{Absorpti}ve \hl{capacit}y, network stability, and innovation appropriability\par
AU \tabto{0cm}Hurmelinna-Laukkanen, Olander, Blomqvist, Panfilii\par
PY \tabto{0cm}2012, SO EUROPEAN MANAGEMENT JOURNAL\par
DT \tabto{0cm}Article\par
PG \tabto{0cm}12, NR 94, TC 3\par
DE \tabto{0cm}KNOWLEDGE PROTECTION, KNOWLEDGE-SHARING, NETWORKS, ORCHESTRATION, R\&D, STABILITY\par
ID \tabto{0cm}COMPETITIVE ADVANTAGE, EMPIRICAL-TEST, JOINT VENTURES, KNOWLEDGE, PRODUCT INNOVATION, RELATIONAL GOVERNANCE, STRATEGIC ALLIANCES, STRUCTURAL HOLES, TRUST, VALUE CREATION\par
AB \tabto{0cm}Networked R\&D has faced upheaval over the last decade. However, in order to fully benefit from collaboration, firms need to embrace paradoxes that are inherent in R\&D networks. We therefore investigate how orchestration (rather than traditional management) of relationships by improving \hl{absorpti}ve \hl{capacit}y, network stability, and innovation appropriability contribute to the success of both the network and the individual firm, from the firm's point of view. We use survey data collected from 213 R\&D-intensive firms. The results indicate that diverging effects emerge regarding alliance and firm success; stability and \hl{absorpti}ve \hl{capacit}y are most relevant for the former, and \hl{absorpti}ve \hl{capacit}y and appropriability for the latter. (C) 2012 Elsevier Ltd. All rights reserved.\par
\clearpage

\vspace*{-2cm}
Nb \tabto{0cm}181/327 (article\_id: 508)\par
TI \tabto{0cm}Assessing the roles that \hl{absorpti}ve \hl{capacit}y and economic distance play in the foreign direct investment-productivity growth nexus\par
AU \tabto{0cm}Bodman, Le\par
PY \tabto{0cm}2013, SO APPLIED ECONOMICS\par
DT \tabto{0cm}Article\par
PG \tabto{0cm}13, NR 36, TC 3\par
DE \tabto{0cm}\hl{ABSORPTI}VE \hl{CAPACIT}Y, FDI, HUMAN CAPITAL, INTERNATIONAL SPILLOVERS, PRODUCTIVITY, R\&D, TRADE\par
ID \tabto{0cm}BRAIN-DRAIN, COINTEGRATION, DEVELOPMENT SPILLOVERS, HETEROGENEOUS PANELS, RESEARCH-AND-DEVELOPMENT, TECHNOLOGY DIFFUSION, TRADE\par
AB \tabto{0cm}We further examine the channels through which Foreign Direct Investment (FDI) develops the national productivity of host countries. We investigate whether FDI is an effective channel of technological transfer across borders and whether that technology transfer is bi-directional: from an investing country to a host country and vice versa. In particular, an analysis is provided of whether FDI helps channel more resources towards the promotion of education activities and hence augments economic growth indirectly through augmenting the host country's \hl{absorpti}ve \hl{capacit}y. Also, the analysis uses a novel approach to take into account the possibility that physical distances can act as a barrier to economic and technological interactions amongst countries, by embedding a measure of geographical distance into two specific channels: international trade and FDI. Empirical results obtained all lend strong support to these hypotheses.\par
\clearpage

\vspace*{-2cm}
Nb \tabto{0cm}182/327 (article\_id: 509)\par
TI \tabto{0cm}Territory's \hl{Absorpti}ve \hl{Capacit}y\par
AU \tabto{0cm}Schillaci, Romano, Nicotra\par
PY \tabto{0cm}2013, SO ENTREPRENEURSHIP RESEARCH JOURNAL\par
DT \tabto{0cm}Article\par
PG \tabto{0cm}18, NR 85, TC 2\par
DE \tabto{0cm}\hl{ABSORPTI}VE \hl{CAPACIT}Y, KNOWLEDGE GATEKEEPERS, SYSTEM OF INNOVATION, TERRITORY\par
ID \tabto{0cm}null\par
AB \tabto{0cm}In the paper, the concept of \hl{absorpti}ve \hl{capacit}y, already successfully applied to firms, is extended to territories. Territory's \hl{Absorpti}ve \hl{Capacit}y is here defined as the ability of a Region to identify, assimilate, and exploit external knowledge.
From a theoretical point of view, the concept of Territory's \hl{Absorpti}ve \hl{Capacit}y integrates two fields of studies, those of innovation systems and regional science.
Three propositions are developed starting from the consideration that territory's \hl{absorpti}ve \hl{capacit}y is dependent on increasing investments in human resource learning in a Region, promoting R\&D activities and, above all, integrating knowledge flows through the action of territorial knowledge gatekeepers.\par
\clearpage

\vspace*{-2cm}
Nb \tabto{0cm}183/327 (article\_id: 510)\par
TI \tabto{0cm}Comment on "Territory's \hl{Absorpti}ve \hl{Capacit}y"\par
AU \tabto{0cm}Parada\par
PY \tabto{0cm}2013, SO ENTREPRENEURSHIP RESEARCH JOURNAL\par
DT \tabto{0cm}Editorial Material\par
PG \tabto{0cm}6, NR 17, TC 0\par
DE \tabto{0cm}\hl{ABSORPTI}VE \hl{CAPACIT}Y, DETRACTED REGIONS, KNOLWEDGE GATEKEEPERS\par
ID \tabto{0cm}null\par
AB \tabto{0cm}\hl{Absorpti}ve \hl{capacit}y has been widely studied in organizational contexts. Few studies, however, deal with broader contexts, despite the fact that some authors suggest that organizations may build inter-firm networks to keep knowledge until they need to absorb it internally. Especially relevant to the role of human capital and knowledge gatekeepers as mentioned in Schillaci, Romano and Nicotra's article, this commentary build on these two propositions arguing the need for a more integrative framework to understand \hl{absorpti}ve \hl{capacit}y in detracted regions, not forgetting the role of the environment.\par
\clearpage

\vspace*{-2cm}
Nb \tabto{0cm}184/327 (article\_id: 511)\par
TI \tabto{0cm}IMPORTED CAPITAL INPUT, \hl{ABSORPTI}VE \hl{CAPACIT}Y, AND FIRM PERFORMANCE: EVIDENCE FROM FIRM-LEVEL DATA\par
AU \tabto{0cm}Yasar\par
PY \tabto{0cm}2013, SO ECONOMIC INQUIRY\par
DT \tabto{0cm}Article\par
PG \tabto{0cm}13, NR 50, TC 2\par
DE \tabto{0cm}null\par
ID \tabto{0cm}DEVELOPMENT SPILLOVERS, ECONOMIC-GROWTH, INTERMEDIATE INPUTS, INTERNATIONAL TECHNOLOGY DIFFUSION, INVESTMENTS, MANUFACTURING FIRMS, PLANT-LEVEL, PRODUCTIVITY, RESEARCH-AND-DEVELOPMENT, TRADE\par
AB \tabto{0cm}Importing capital inputs has been recognized as a critical channel for technology transfer across countries. We examine whether and to what extent the productive impact of imported capital varies with firms' abilities to absorb new technologies using ordinary least squares, instrumental variable, and threshold regression estimators. We find that firms with higher \hl{absorpti}ve \hl{capacit}y gain significantly more from importing foreign capital. Our results also suggest a threshold for such benefits. Furthermore, the productive contribution of skilled labor is significantly higher in firms that import foreign capital. Developing policies to augment \hl{absorpti}ve \hl{capacit}y will help firms in developing countries to realize benefits associated with imported capital. (JEL F14, D24, L24, O33)\par
\clearpage

\vspace*{-2cm}
Nb \tabto{0cm}185/327 (article\_id: 512)\par
TI \tabto{0cm}Incremental and Radical Innovation in Coopetition-The Role of \hl{Absorpti}ve \hl{Capacit}y and Appropriability\par
AU \tabto{0cm}Ritala, Hurmelinna-Laukkanen\par
PY \tabto{0cm}2013, SO JOURNAL OF PRODUCT INNOVATION MANAGEMENT\par
DT \tabto{0cm}Article\par
PG \tabto{0cm}16, NR 83, TC 26\par
DE \tabto{0cm}null\par
ID \tabto{0cm}COMPETITIVE ADVANTAGE, COOPERATION, EMPIRICAL-ANALYSIS, FIRM PERFORMANCE, KNOWLEDGE TRANSFER, PRODUCT DEVELOPMENT, PROTECTION, RESEARCH-AND-DEVELOPMENT, STRATEGIC ALLIANCES, TECHNOLOGICAL INNOVATION\par
AB \tabto{0cm}This study examines why some firms are better able than others to reap benefits from collaborating with their competitors in innovation. Whereas on the general level, collaborative innovation has been studied widely, and firm-specific success factors in collaboration between competitors (i.e., coopetition) have not been exhaustively addressed. Earlier literature describes coopetition as a risky but potentially rewarding relationship in which sharing, learning, and protection of knowledge are recognized as the key issues determining the possible benefits and hazards. This study provides evidence of factors related to this, suggesting that the firm's ability to acquire knowledge from external sources (potential \hl{absorpti}ve \hl{capacit}y) and to protect its innovations and core knowledge against imitation (appropriability regime) are relevant in increasing the innovation outcomes of collaborating with its competitors. This study also distinguishes between incremental and radical innovations as an outcome of coopetition, and provides differing implications for the two innovation types. The empirical evidence for the study was gathered from a cross-industry survey conducted on Finnish markets. The data are analyzed with multivariate multiple regression analysis. The results of the analysis suggest that (1) potential \hl{absorpti}ve \hl{capacit}y and appropriability regime of the firm both have a positive effect in the pursuit of incremental innovations in coopetition, and (2) in the case of radical innovations, appropriability regime has a positive effect, while the effect of \hl{absorpti}ve \hl{capacit}y is not statistically significant. However, the results also indicate that there is a moderating relationship between these variables, in that the potential \hl{absorpti}ve \hl{capacit}y is positively associated with creation of radical innovations within high levels of appropriability regime. These results yield important theoretical and managerial implications. As a whole, the results presented in this study provide new evidence on which types of firms can reap success in the challenging task of collaborative innovation with rivals. In the case of incremental innovation, a firm-level emphasis on knowledge sharing and learning will positively affect the results of coopetition, as will an emphasis on knowledge protection. Thus, when incremental developments are pursued in coopetition, firms should not only seek to exchange knowledge to create value but also remember to secure the firm-specific core knowledge within the firm's borders to stay competitive. On the other hand, when the firm is pursuing radical innovation with its rivals, the heaviest emphasis should be on protecting its existing core knowledge and also emerging novel innovations and market opportunities. Capabilities in knowledge acquisition are also beneficial in these cases, but the full benefits of knowledge exchange realize only when the firm's knowledge protection mechanisms are sufficiently strong, allowing for safe knowledge exchange between rivals.\par
\clearpage

\vspace*{-2cm}
Nb \tabto{0cm}186/327 (article\_id: 513)\par
TI \tabto{0cm}External knowledge sourcing from innovation cooperation and the role of \hl{absorpti}ve \hl{capacit}y: empirical evidence from Norway and Sweden\par
AU \tabto{0cm}Clausen\par
PY \tabto{0cm}2013, SO TECHNOLOGY ANALYSIS \& STRATEGIC MANAGEMENT\par
DT \tabto{0cm}Article\par
PG \tabto{0cm}14, NR 33, TC 9\par
DE \tabto{0cm}\hl{ABSORPTI}VE \hl{CAPACIT}Y, OPEN INNOVATION, R\&D, TRAINING\par
ID \tabto{0cm}FIRMS, LINKS, RESEARCH-AND-DEVELOPMENT, TECHNOLOGY\par
AB \tabto{0cm}Open innovation (OI) is an innovation paradigm which argues that firms should use external knowledge in order to succeed in the innovation process. It is currently unclear however how firms are able to source external knowledge from external actors when attempting to innovate. In this paper we examine the relationship between \hl{absorpti}ve \hl{capacit}y and firms' ability to enter into innovation cooperation with external actors. Our econometric results show that internal R\&D, training and an educated workforce, as core aspects of firms' \hl{absorpti}ve \hl{capacit}y, are positively associated with (the intensity of) innovation cooperation. An implication is that external knowledge does not enter the firm freely. The costs firms must invoke in order to be able to source external knowledge in the OI context is considerable. Without investing in internal R\&D, training and recruiting workers with good educational qualifications, firms may not be able to follow the open approach to innovation.\par
\clearpage

\vspace*{-2cm}
Nb \tabto{0cm}187/327 (article\_id: 514)\par
TI \tabto{0cm}The impact of ubiquitous decision support systems on decision quality through individual \hl{absorpti}ve \hl{capacit}y and perceived usefulness\par
AU \tabto{0cm}Seo, Lee\par
PY \tabto{0cm}2013, SO ONLINE INFORMATION REVIEW\par
DT \tabto{0cm}Article\par
PG \tabto{0cm}13, NR 35, TC 0\par
DE \tabto{0cm}\hl{ABSORPTI}VE \hl{CAPACIT}Y, CONTEXT AWARENESS, DECISION QUALITY, DECISION SUPPORT SYSTEMS, INFORMATION SYSTEMS, PERCEIVED USEFULNESS, UBIQUITOUS MOBILITY, UDSS MDS\par
ID \tabto{0cm}ANTECEDENTS, COMPUTER, INNOVATION, MANAGEMENT\par
AB \tabto{0cm}Purpose - The purpose of this study is to examine a mobile delivery system as a working ubiquitous decision support system (UDSS) and determine whether it would improve decision quality.
Design/methodology/approach - Ubiquitous mobility and context awareness are the two core functions of the UDSS. Hence the authors examined how they might influence individual \hl{absorpti}ve \hl{capacit}y and perceived usefulness. Moreover the authors investigated how individual \hl{absorpti}ve \hl{capacit}y and perceived usefulness might be related to decision quality. A total of 174 completed questionnaires were collected from delivery workers, and a financial incentive was provided to participants. To test the hypotheses the research model was analysed with the partial least square method.
Findings - The results reveal that all paths are statistically valid. Individual \hl{absorpti}ve \hl{capacit}y and perceived usefulness were positively influenced by ubiquitous mobility and context awareness. In addition individual \hl{absorpti}ve \hl{capacit}y and perceived usefulness have positive effects on decision quality.
Research limitations/implications - This research model did not consider all the capabilities enabled by the UDSS. Future study should pay attention to nomadicity, proactiveness, invisibility, and portability as relevant antecedents within the model.
Originality/value - In the field of IS studies the impact of the UDSS on users' decision quality has remained unclear to date. The authors adopted a mobile delivery system as a working UDSS and applied it in their study. Thereby the authors found the mediating effects of perceived usefulness and \hl{absorpti}ve \hl{capacit}y under a ubiquitous environment.\par
\clearpage

\vspace*{-2cm}
Nb \tabto{0cm}188/327 (article\_id: 515)\par
TI \tabto{0cm}Tools without skills: Exploring the moderating effect of \hl{absorpti}ve \hl{capacit}y on the relationship between e-purchasing tools and category performance\par
AU \tabto{0cm}Kauppi, Brandon-Jones, Ronchi, van Raaij\par
PY \tabto{0cm}2013, SO INTERNATIONAL JOURNAL OF OPERATIONS \& PRODUCTION MANAGEMENT\par
DT \tabto{0cm}Article\par
PG \tabto{0cm}30, NR 98, TC 4\par
DE \tabto{0cm}\hl{ABSORPTI}VE \hl{CAPACIT}Y, E-PURCHASING TOOLS, PURCHASE CATEGORY PERFORMANCE\par
ID \tabto{0cm}COMPETITIVE ADVANTAGE, E-BUSINESS TECHNOLOGIES, E-PROCUREMENT, FIRM PERFORMANCE, INFORMATION-TECHNOLOGY, JOB-SATISFACTION, MANUFACTURING TECHNOLOGY, OPERATIONAL PERFORMANCE, REVERSE AUCTIONS, SUPPLY CHAIN MANAGEMENT\par
AB \tabto{0cm}Purpose - The paper examines the moderating role of a purchasing function's \hl{absorpti}ve \hl{capacit}y (AC) on the relationship between the use of electronic purchasing tools and category, level purchasing performance. The authors argue that an e-purchasing tool may not in itself positively influence performance unless combined with AC as a human interface to maximise its information and transactional improvement potential.
Design/methodology/approach - Survey data collected from 297 procurement executives of large companies in ten countries are analysed using confirmatory factor analysis (CFA) and hierarchical moderated regression.
Findings - The results demonstrate few significant direct effects of e-purchasing tools on category performance. All performance measures studied are enhanced when dimensions of AC and their interactions with the e-purchasing tools are added. Specifically, buyer competence, manager competence and communications climate have performance-enhancing effects. In some cases, AC on its own appears to increase performance more than e-tools.
Originality/value - This paper is the first to study the moderating effects-of AC on the relationship between e-purchasing tool usage and category performance. Its findings support the view that simply implementing technology does not lead to performance improvements, but that a human interface is required to maximise the information and transactional improvement potential of e-purchasing tools.\par
\clearpage

\vspace*{-2cm}
Nb \tabto{0cm}189/327 (article\_id: 516)\par
TI \tabto{0cm}Export market location decision and performance The role of external networks and \hl{absorpti}ve \hl{capacit}y\par
AU \tabto{0cm}He, Wei\par
PY \tabto{0cm}2013, SO INTERNATIONAL MARKETING REVIEW\par
DT \tabto{0cm}Article\par
PG \tabto{0cm}32, NR 130, TC 3\par
DE \tabto{0cm}EXPORT MARKET LOCATION, EXTERNAL NETWORKS, NETWORK THEORY, PERFORMANCE, RESOURCE-BASED VIEW, STRATEGIC FIT\par
ID \tabto{0cm}CULTURAL DISTANCE, EMERGING MARKET, FIRM PERFORMANCE, INTERNATIONAL-BUSINESS RESEARCH, MANAGERIAL TIES, ORGANIZATIONAL PERFORMANCE, PSYCHIC DISTANCE, RESOURCE-BASED VIEW, STRUCTURAL EQUATION MODELS, SUBSIDIARY PERFORMANCE\par
AB \tabto{0cm}Purpose - Drawing on the resource-based view and network theory, the purpose of this paper is to investigate the role of external networks (ENs) and \hl{absorpti}ve \hl{capacit}y (AC) in export market location decision of emerging economy firms (EEFs) and the performance implication of this decision.
Design/methodology/approach - This study employs structural equation modeling to test three hypotheses: first, ENs influence an EEF manager's propensity to enter culturally/psychically distant markets for exports. Distant markets are more likely to be chosen by managers of firms with abundant ENs. Second, AC moderates this network-market location relationship. Third, superior performance results from the fit between managers' propensity to enter a market and firms' levels of ENs and AC.
Findings - An analysis of 196 Chinese exporting firms supports the hypotheses.
Research limitations/implications Though the theoretical discussion is general, the empirical context is specific to Chinese export manufacturers. Replicating the study is necessary in different contexts.
Practical implications - The study identifies to managers the importance of external (i.e. ENs) and internal resources and capabilities (i.e. AC) and linkages between resources and capabilities, strategy and performance.
Originality/value - The study is novel in conceptually addressing the role of ENs and AC in firms' decision making and performance and in testing hypotheses with robust methodology and data.\par
\clearpage

\vspace*{-2cm}
Nb \tabto{0cm}190/327 (article\_id: 517)\par
TI \tabto{0cm}DIFFERENCES OF \hl{ABSORPTI}VE \hl{CAPACIT}Y BETWEEN FIRMS WITHIN A CLUSTER\par
AU \tabto{0cm}Eiriz, Barbosa, Lima\par
PY \tabto{0cm}2013, SO TRANSFORMATIONS IN BUSINESS \& ECONOMICS\par
DT \tabto{0cm}Article\par
PG \tabto{0cm}12, NR 30, TC 0\par
DE \tabto{0cm}\hl{ABSORPTI}VE \hl{CAPACIT}Y, CLUSTER, EXPORT, PORTUGAL, POSITION, SIZE\par
ID \tabto{0cm}INNOVATION, KNOWLEDGE TRANSFER\par
AB \tabto{0cm}Firms located within a cluster have access to tacit, complex and specific local knowledge which allow them to develop competitive advantage. However, firms have no equal ability to access and to apply that knowledge, meaning that not all have a similar knowledge \hl{absorpti}ve \hl{capacit}y. Using a sample of the largest Portuguese firms within a footwear cluster, this paper examine whether there are significant differences in firm's \hl{absorpti}ve \hl{capacit}y and whether such differences within a cluster are related to firms' specific characteristics. The results suggest that \hl{absorpti}ve \hl{capacit}y is significantly associated with the firms' characteristics, namely size, export intensity and position within the cluster.\par
\clearpage

\vspace*{-2cm}
Nb \tabto{0cm}191/327 (article\_id: 518)\par
TI \tabto{0cm}\hl{Absorpti}ve \hl{capacit}y constituents in knowledge-intensive industries in Serbia\par
AU \tabto{0cm}Levi-Jaksic, Radovanovic, Radojicic\par
PY \tabto{0cm}2013, SO ZBORNIK RADOVA EKONOMSKOG FAKULTETA U RIJECI-PROCEEDINGS OF RIJEKA\par
DT \tabto{0cm}Article\par
PG \tabto{0cm}26, NR 77, TC 0\par
DE \tabto{0cm}\hl{ABSORPTI}VE \hl{CAPACIT}Y, INNOVATION CAPABILITY, KNOWLEDGE DISSEMINATION, KNOWLEDGE-INTENSIVE INDUSTRIES, RELEVANT KNOWLEDGE\par
ID \tabto{0cm}FIRM, FRAMEWORK, INNOVATION, INTRANET, MANAGEMENT, ORGANIZATIONS, PERSPECTIVE, STRATEGY, TECHNOLOGY, VENTURES\par
AB \tabto{0cm}The aim of this paper was to explore the determining constituents of \hl{absorpti}ve \hl{capacit}y and their influence on the overall \hl{absorpti}ve \hl{capacit}y level in the organizations from knowledge-intensive industries in the Republic of Serbia. The research was conducted in order to analyze how does the capability of the organizations from Serbian knowledge-intensive industries to acquire external knowledge, develop their knowledge base, communicate and exploit the absorbed knowledge affect their \hl{absorpti}ve \hl{capacit}y. The methodology included quantitative and qualitative research method based on a questionnaire. The data collected were analyzed with the \hl{absorpti}ve \hl{capacit}y evaluation model and by applying discriminant function analysis as a statistical method. The basic result shows that the \hl{absorpti}ve \hl{capacit}y of organizations from knowledge-intensive industries in Serbia is clearly dependent on the scope of the available relevant knowledge. The fundamental conclusion as the result of this research indicates that the enrichment of the organizational knowledge bases with the relevant new content strongly enhances the \hl{absorpti}ve \hl{capacit}y of organizations from knowledge-intensive industries in Serbia. Further research on this topic should be directed towards investigating the relationship between the existing knowledge base and the innovation capability in organizations from knowledge-intensive industries.\par
\clearpage

\vspace*{-2cm}
Nb \tabto{0cm}192/327 (article\_id: 519)\par
TI \tabto{0cm}An \hl{absorpti}ve \hl{capacit}y theory of knowledge spillover entrepreneurship\par
AU \tabto{0cm}Qian, Acs\par
PY \tabto{0cm}2013, SO SMALL BUSINESS ECONOMICS\par
DT \tabto{0cm}Article\par
PG \tabto{0cm}13, NR 46, TC 18\par
DE \tabto{0cm}\hl{ABSORPTI}VE \hl{CAPACIT}Y, ENTREPRENEURSHIP, HUMAN CAPITAL, KNOWLEDGE SPILLOVER\par
ID \tabto{0cm}CITIES, CONSTRUCTION, CREATIVITY, ECONOMIC-GROWTH, FIRMS, INNOVATION, PATENTS, RESEARCH-AND-DEVELOPMENT, TALENT\par
AB \tabto{0cm}The knowledge spillover theory of entrepreneurship identifies new knowledge as a source of entrepreneurial opportunities, and suggests that entrepreneurs play an important role in commercializing new knowledge developed in large incumbent firms or research institutions. This paper argues that, knowledge spillover entrepreneurship depends not only on new knowledge but more importantly on entrepreneurial \hl{absorpti}ve \hl{capacit}y that allows entrepreneurs to understand new knowledge, recognize its value, and commercialize it by creating a firm. This \hl{absorpti}ve \hl{capacit}y theory of knowledge spillover entrepreneurship is tested using data based on U.S. metropolitan areas.\par
\clearpage

\vspace*{-2cm}
Nb \tabto{0cm}193/327 (article\_id: 520)\par
TI \tabto{0cm}The impact of IT capabilities on firm performance: The mediating roles of \hl{absorpti}ve \hl{capacit}y and supply chain agility\par
AU \tabto{0cm}Liu, Ke, Wei, Hua\par
PY \tabto{0cm}2013, SO DECISION SUPPORT SYSTEMS\par
DT \tabto{0cm}Article\par
PG \tabto{0cm}11, NR 86, TC 24\par
DE \tabto{0cm}\hl{ABSORPTI}VE \hl{CAPACIT}Y, DYNAMIC CAPABILITIES, IT ASSIMILATION, IT INFRASTRUCTURE, SUPPLY CHAIN AGILITY\par
ID \tabto{0cm}DYNAMIC CAPABILITIES, E-BUSINESS, EXPLORATORY ANALYSIS, FUTURE-RESEARCH, INFORMATION-SYSTEMS RESEARCH, MANAGEMENT CAPABILITY, MANUFACTURING PRACTICES, OPERATIONAL CAPABILITIES, ORGANIZATIONAL CAPABILITIES, RESOURCE-BASED VIEW\par
AB \tabto{0cm}Researchers and practitioners regard information technology (IT) as a competitive tool. However, current knowledge on IT capability mechanisms that affect firm performance remains unclear. Based on the dynamic capabilities perspective and the view of a hierarchy of capabilities, this article proposes a model to examine how IT capabilities (i.e., flexible IT infrastructure and IT assimilation) affect firm performance through \hl{absorpti}ve \hl{capacit}y and supply chain agility in the supply chain context. Survey data show that \hl{absorpti}ve \hl{capacit}y and supply chain agility fully mediate the influences of IT capabilities on firm performance. In addition to the direct effects, \hl{absorpti}ve \hl{capacit}y also has indirect effects on firm performance by shaping supply chain agility. We conclude with implications and suggestions for future research. (C) 2012 Elsevier B.V. All rights reserved.\par
\clearpage

\vspace*{-2cm}
Nb \tabto{0cm}194/327 (article\_id: 521)\par
TI \tabto{0cm}Social Capital and Effective Innovation in Industrial Districts: Dual Effect of \hl{Absorpti}ve \hl{Capacit}y\par
AU \tabto{0cm}Parra-Requena, Ruiz-Ortega, Garcia-Villaverde\par
PY \tabto{0cm}2013, SO INDUSTRY AND INNOVATION\par
DT \tabto{0cm}Article\par
PG \tabto{0cm}23, NR 68, TC 4\par
DE \tabto{0cm}\hl{ABSORPTI}VE \hl{CAPACIT}Y, INDUSTRIAL DISTRICTS, INNOVATION, KNOWLEDGE, SOCIAL CAPITAL\par
ID \tabto{0cm}ADVANTAGE, CLUSTERS, COMBINATIVE CAPABILITIES, CREATION, FIRMS, KNOWLEDGE ACQUISITION, NETWORKS, PERFORMANCE, STRATEGIC ALLIANCES, TECHNOLOGY\par
AB \tabto{0cm}This paper deals with the factors that affect the heterogeneity in the access to knowledge and its exploitation through innovation in firms located in industrial districts. The aim of the study is to analyze the moderating role of the components of the \hl{absorpti}ve \hl{capacit}y identification and combination in the process that leads firms in industrial districts with social capital to obtain effective innovations through the knowledge acquisition. We have developed the empirical analysis on a sample of 166 firms located in the industrial districts of the footwear industry in Spain. Findings suggest that the firms in industrial districts improve the acquisition of novel and valuable knowledge from external networks of information when they have identification capabilities to explore their potential. The results also indicate combinative capability strengthens the acquired new knowledge to develop and exploit successful innovations.\par
\clearpage

\vspace*{-2cm}
Nb \tabto{0cm}195/327 (article\_id: 522)\par
TI \tabto{0cm}Firm heterogeneity and technology transfers to local suppliers: Disentangling the effects of foreign ownership, technology gap and \hl{absorpti}ve \hl{capacit}y\par
AU \tabto{0cm}Jordaan\par
PY \tabto{0cm}2013, SO JOURNAL OF INTERNATIONAL TRADE \& ECONOMIC DEVELOPMENT\par
DT \tabto{0cm}Article\par
PG \tabto{0cm}23, NR 50, TC 1\par
DE \tabto{0cm}null\par
ID \tabto{0cm}DIRECT-INVESTMENT, DOMESTIC FIRMS, ECONOMY, EMPIRICAL-EVIDENCE, EXTERNALITIES, FDI, MEXICAN MANUFACTURING-INDUSTRIES, PRODUCTIVITY, SPILLOVERS, VENEZUELA\par
AB \tabto{0cm}In this paper, I present novel microeconomic evidence on the effects of firm heterogeneity on the creation and impact of technology transfers from foreign direct investment (FDI) to local suppliers in a developing country setting. The main findings are threefold. First, FDI firms are significantly more involved in knowledge transfer activities than domestic producer firms. In particular, FDI firms offer more technological support, support with a direct positive impact on production processes of local suppliers. Second, the type of ownership also influences the effect of the technology gap on technology transfers. A large technology gap between a producer firm and its suppliers lowers the provision of support; however, FDI firms offer more technological support to their suppliers of material inputs when the technology gap is large. Independent of the support that the suppliers receive, foreign ownership of client firms and a large technology gap make it more likely that suppliers experience large positive impacts. Third, the level of \hl{absorpti}ve \hl{capacit}y of local suppliers is also important for the impact of the technology transfers, confirming the notion that heterogeneity among both producer firms and local suppliers affect the level, nature and impact of local technology transfers.\par
\clearpage

\vspace*{-2cm}
Nb \tabto{0cm}196/327 (article\_id: 523)\par
TI \tabto{0cm}\hl{Absorpti}ve \hl{capacit}y and R\&D strategy in mixed duopoly with labor-managed and profit-maximizing firms\par
AU \tabto{0cm}Luo\par
PY \tabto{0cm}2013, SO ECONOMIC MODELLING\par
DT \tabto{0cm}Article\par
PG \tabto{0cm}7, NR 24, TC 3\par
DE \tabto{0cm}\hl{ABSORPTI}VE \hl{CAPACIT}Y, LABOR-MANAGED FIRM, OLIGARCH COMPETITION, R\&D, STRATEGIC INTERACTION\par
ID \tabto{0cm}COMPETITION, ENTREPRENEURIAL, INFORMATION, INNOVATION, INVESTMENT, ME HALFWAY, POLICY, SPILLOVERS\par
AB \tabto{0cm}In a mixed duopoly with a labor-managed firm and a profit-maximizing firm coexisting, this paper explores the effect of technology \hl{absorpti}ve \hl{capacit}y on firms' output decision, R\&D investment decision and social welfare. Firstly, we develop a two-stage R\&D game model on the basis of cost-reducing R\&D with spillover and \hl{absorpti}ve \hl{capacit}y. Secondly, we explore the strategic interactions of output, R\&D investment and social welfare respectively in the mixed duopoly with a labor-managed and a profit-maximizing firms. Finally we analyze the effects of \hl{absorpti}ve \hl{capacit}y on output decision, strategic investment decision and social welfare respectively. The research suggests that labor-managed firms employ less workers and produce less outputs while they invest more in R\&D than that of profit-maximizing firms. Whether the effect of \hl{absorpti}ve \hl{capacit}y increases R\&D investment of labor-managed firms or not depends on the returns to scale. However, it bears no relationship to the returns to scale of the profit-maximizing firm. (C) 2012 Elsevier B.V. All rights reserved.\par
\clearpage

\vspace*{-2cm}
Nb \tabto{0cm}197/327 (article\_id: 524)\par
TI \tabto{0cm}Nurturing employee market knowledge \hl{absorpti}ve \hl{capacit}y through unified internal communication and integrated information technology\par
AU \tabto{0cm}Jimenez-Castillo, Sanchez-Perez\par
PY \tabto{0cm}2013, SO INFORMATION \& MANAGEMENT\par
DT \tabto{0cm}Article\par
PG \tabto{0cm}11, NR 96, TC 2\par
DE \tabto{0cm}\hl{ABSORPTI}VE \hl{CAPACIT}Y, INFORMATION DISSEMINATION, INFORMATION TECHNOLOGY, INTERNAL COMMUNICATION, MARKET KNOWLEDGE\par
ID \tabto{0cm}DISSEMINATION, IMPACT, INNOVATION PERFORMANCE, INTELLIGENCE, INTERRATER AGREEMENT, MANAGEMENT-SYSTEMS, ORGANIZATION, ORIENTATION, PERSPECTIVE, RECONCEPTUALIZATION\par
AB \tabto{0cm}Organizations that are actively engaged in the dissemination of market information frequently question whether this effort improves employee information processing. We examined how the adoption of two integrative dissemination mechanisms, unified internal communication and information technology integration, is critical to enhancing employee market knowledge \hl{absorpti}ve \hl{capacit}y. Using data from 211 industrial firms, we found that the existence of a greater market knowledge base and explicit market knowledge within firms determines the use of these mechanisms, which in turn increases employee \hl{absorpti}ve \hl{capacit}y. Indeed, the mechanisms serve as full mediators for this ability, thus accentuating their value for knowledge, information technology, and innovation management. (C) 2013 Elsevier B.V. All rights reserved.\par
\clearpage

\vspace*{-2cm}
Nb \tabto{0cm}198/327 (article\_id: 525)\par
TI \tabto{0cm}The dynamics of national innovation systems: A panel cointegration analysis of the coevolution between innovative capability and \hl{absorpti}ve \hl{capacit}y\par
AU \tabto{0cm}Castellacci, Natera\par
PY \tabto{0cm}2013, SO RESEARCH POLICY\par
DT \tabto{0cm}Article\par
PG \tabto{0cm}16, NR 46, TC 9\par
DE \tabto{0cm}\hl{ABSORPTI}VE \hl{CAPACIT}Y, COEVOLUTION, ECONOMIC GROWTH AND DEVELOPMENT, INNOVATIVE CAPABILITY, NATIONAL SYSTEMS OF INNOVATION, PANEL COINTEGRATION ANALYSIS\par
ID \tabto{0cm}COMPETITIVENESS, CONVERGENCE, DETERMINANTS, DIFFUSION, ECONOMIC-DEVELOPMENT, EVOLUTIONARY, GROWTH, INSTITUTIONS, STAGNATION, TECHNOLOGY\par
AB \tabto{0cm}This paper investigates the idea that the dynamics of national innovation systems is driven by the coevolution of two main dimensions: innovative capability and \hl{absorpti}ve \hl{capacit}y. The empirical analysis employs a broad set of indicators measuring national innovative capabilities and \hl{absorpti}ve \hl{capacit}y for a panel of 87 countries in the period 1980-2007, and makes use of panel cointegration analysis to investigate long-run relationships and coevolution patterns among these variables. The results indicate that the dynamics of national systems of innovation is driven by the coevolution of three innovative capability variables (innovative input, scientific output and technological output), on the one hand, and three \hl{absorpti}ve \hl{capacit}y factors (infrastructures, international trade and human capital), on the other. This general result does however differ and take specific patterns in national systems characterized by different levels of development. (C) 2012 Elsevier B.V. All rights reserved,\par
\clearpage

\vspace*{-2cm}
Nb \tabto{0cm}199/327 (article\_id: 526)\par
TI \tabto{0cm}Too much of a good thing? \hl{Absorpti}ve \hl{capacit}y, firm performance, and the moderating role of entrepreneurial orientation\par
AU \tabto{0cm}Wales, Parida, Patel\par
PY \tabto{0cm}2013, SO STRATEGIC MANAGEMENT JOURNAL\par
DT \tabto{0cm}Article\par
PG \tabto{0cm}12, NR 45, TC 12\par
DE \tabto{0cm}\hl{ABSORPTI}VE \hl{CAPACIT}Y, CURVILINEARITY, ENTREPRENEURIAL ORIENTATION, FIRM FINANCIAL PERFORMANCE, FIRM STRATEGY\par
ID \tabto{0cm}CONSTRUCT, EXTENSION, FUTURE, GROWTH, INFORMATION, INNOVATION, MODEL, RECONCEPTUALIZATION, RESOURCES, SUGGESTIONS\par
AB \tabto{0cm}\hl{Absorpti}ve \hl{capacit}y (ACAP) refers to a firm's ability to acquire, assimilate, transform, and exploit new knowledge. Research has yet to acknowledge the possibility of limits to the financial returns of this important strategic construct. This study suggests an inverted-U shaped relationship between ACAP and financial performance. Based on data from 285 technology-based small and medium enterprises, we observe gains within three prospective, secondary measures of growth to diminish beyond lower levels of ACAP, even turning negative and becoming harmful beyond intermediate levels. We find that entrepreneurial orientation (EO) moderates the ACAP-performance relationship, enhancing financial gains at lower levels of ACAP and mitigating the decline in financial performance at higher levels of ACAP. Further, with higher EO, higher ACAP can be achieved before financial returns diminish.Copyright (c) 2012 John Wiley \& Sons, Ltd.\par
\clearpage

\vspace*{-2cm}
Nb \tabto{0cm}200/327 (article\_id: 527)\par
TI \tabto{0cm}How does organisational \hl{absorpti}ve \hl{capacit}y matter in the assimilation of enterprise information systems?\par
AU \tabto{0cm}Saraf, Liang, Xue, Hu\par
PY \tabto{0cm}2013, SO INFORMATION SYSTEMS JOURNAL\par
DT \tabto{0cm}Article\par
PG \tabto{0cm}23, NR 71, TC 4\par
DE \tabto{0cm}\hl{ABSORPTI}VE \hl{CAPACIT}Y, ENTERPRISE SYSTEMS, ERP ASSIMILATION, INSTITUTIONAL INFLUENCES, IT ASSIMILATION, ORGANISATIONAL LEARNING\par
ID \tabto{0cm}ADVANTAGE, ANTECEDENTS, CHINA, DIFFUSION, ERP IMPLEMENTATION, MANAGEMENT, PERSPECTIVE, PROCESS INNOVATIONS, RECONCEPTUALIZATION, TECHNOLOGY\par
AB \tabto{0cm}Extant literature offers two mostly distinct perspectives on enterprise systems assimilation driven either by internal expertise and learning capability or by external institutional pressures. This study combines the two perspectives and subscribes to the view that organisations' learning capability moderates their acquiescence to institutional pressures. The study then anchors organisational learning capability to the concept of \hl{absorpti}ve \hl{capacit}y and proposes that its two dimensions potential \hl{absorpti}ve \hl{capacit}y (PACAP) and realised \hl{absorpti}ve \hl{capacit}y (RACAP) affect enterprise systems assimilation through different pathways. Our survey-based empirical study of Enterprise Resource Planning (ERP) systems in the post-implementation stage reveals that while both PACAP and RACAP have a positive direct impact on assimilation, PACAP positively moderates the impact of mimetic (institutional) pressures, but not normative (institutional) pressures, on assimilation; whereas RACAP positively moderates the impact of normative pressures, but not mimetic pressures, on assimilation. Thus, our theoretical contribution lies in understanding the distinct ways in which PACAP and RACAP moderate the influence of external institutional pressures on enterprise systems assimilation.\par
\clearpage

\vspace*{-2cm}
Nb \tabto{0cm}201/327 (article\_id: 528)\par
TI \tabto{0cm}Towards an empirical typology of buyer-supplier relationships based on \hl{absorpti}ve \hl{capacit}y\par
AU \tabto{0cm}Revilla, Saenz, Knoppen\par
PY \tabto{0cm}2013, SO INTERNATIONAL JOURNAL OF PRODUCTION RESEARCH\par
DT \tabto{0cm}Article\par
PG \tabto{0cm}17, NR 99, TC 1\par
DE \tabto{0cm}\hl{ABSORPTI}VE \hl{CAPACIT}Y, BUYERSUPPLIER RELATIONSHIPS, LEARNING PROCESSES, PERFORMANCE, TAXONOMY\par
ID \tabto{0cm}CHAIN MANAGEMENT, COMBINATIVE CAPABILITIES, COMPETITIVE ADVANTAGE, DYNAMIC CAPABILITIES, INNOVATION PERFORMANCE, KNOWLEDGE TRANSFER, LEARNING-PROCESSES, MANUFACTURING STRATEGY, ORGANIZATIONAL ANTECEDENTS, PRODUCT DEVELOPMENT\par
AB \tabto{0cm}This paper develops a taxonomy of buyersupplier relationships (BSRs), based on the supplier's \hl{absorpti}ve \hl{capacit}y (AC). AC encompasses three learning processes: exploration, assimilation, and exploitation. The aim is to develop a taxonomy that can predict a firm's performance with regard to innovation and operational efficiency. This research complements the literature, which presently focuses on descriptive rather than predictive taxonomies. Data from 153 firms were collected through survey research. Confirmatory factor analysis was used to assess the quality of data and calculate composite scores to be used in the cluster analysis to develop the BSRs patterns. Analysis of variance was used to explore the relationships between BSR type and firm performance. Finally, semi-structured interviews aided interpretation of the proposed taxonomy. Findings support the identification of groups of dyads through different combinations of the learning processes underlying AC. The different combinations are typified through AC strength and AC reinforcement. The results provide evidence of a significant relationship between AC strength and firm performance. Surprisingly, we did not find empirical support for the relationship between AC reinforcement and performance.\par
\clearpage

\vspace*{-2cm}
Nb \tabto{0cm}202/327 (article\_id: 529)\par
TI \tabto{0cm}The antecedents and innovation effects of domestic and offshore R\&D outsourcing: The contingent impact of cognitive distance and \hl{absorpti}ve \hl{capacit}y\par
AU \tabto{0cm}Bertrand, Mol\par
PY \tabto{0cm}2013, SO STRATEGIC MANAGEMENT JOURNAL\par
DT \tabto{0cm}Article\par
PG \tabto{0cm}10, NR 37, TC 11\par
DE \tabto{0cm}\hl{ABSORPTI}VE \hl{CAPACIT}Y, COGNITIVE DISTANCE, INNOVATION, OFFSHORING, OUTSOURCING\par
ID \tabto{0cm}CAPABILITIES, EXPLORATION, EXTERNAL KNOWLEDGE ACQUISITION, FIRMS, HETEROGENEITY, INDUSTRY, PERFORMANCE, PERSPECTIVE, SEARCH, TECHNOLOGY\par
AB \tabto{0cm}This paper analyzes differences in the antecedents and performance consequences of domestic and offshore R\&D outsourcing. Offshore outsourcing is characterized by larger cognitive distance. We find that \hl{absorpti}ve \hl{capacit}y from internal R\&D allows for more offshore outsourcing and that offshore outsourcing leads to more positive innovation outcomes, especially product innovation. Copyright (C) 2012 John Wiley \& Sons, Ltd.\par
\clearpage

\vspace*{-2cm}
Nb \tabto{0cm}203/327 (article\_id: 530)\par
TI \tabto{0cm}The Driving Forces of Subsidiary \hl{Absorpti}ve \hl{Capacit}y\par
AU \tabto{0cm}Schleimer, Pedersen\par
PY \tabto{0cm}2013, SO JOURNAL OF MANAGEMENT STUDIES\par
DT \tabto{0cm}Article\par
PG \tabto{0cm}27, NR 87, TC 4\par
DE \tabto{0cm}MARKETING STRATEGY, MNC ORGANIZATIONAL MECHANISMS, SUBSIDIARY \hl{ABSORPTI}VE \hl{CAPACIT}Y, SUBSIDIARY MARKETS\par
ID \tabto{0cm}CONCEPTUAL-FRAMEWORK, DYNAMIC CAPABILITIES, EMPIRICAL-EVIDENCE, INNOVATION, KNOWLEDGE TRANSFER, MARKET ORIENTATION, MULTINATIONAL-CORPORATIONS, ORGANIZATIONAL PRACTICES, PERFORMANCE, STRATEGY\par
AB \tabto{0cm}The study investigates how a multinational corporation (MNC) can promote the \hl{absorpti}ve \hl{capacit}y of its subsidiaries. The focus is on what drives the MNC subsidiary's ability to absorb marketing strategies that are initiated by the MNC parent, as well as how the subsidiary enacts on this \hl{absorpti}ve \hl{capacit}y in order to compete in its focal market. The dual embeddedness of MNC subsidiaries plays a key role in this investigation, as subsidiaries belong to the MNC network and are simultaneously embedded in their host country environment. We argue that subsidiary \hl{absorpti}ve \hl{capacit}y is formed as a purposeful response to this dual embeddedness. An analysis of marketing strategy \hl{absorpti}ons undertaken by 213 subsidiaries reveals that MNCs can assist their subsidiaries to compete in competitive and dynamic focal markets by forming specific organizational mechanisms that are conducive to the development of subsidiary \hl{absorpti}ve \hl{capacit}y. The findings hold important theoretical and practical implications.\par
\clearpage

\vspace*{-2cm}
Nb \tabto{0cm}204/327 (article\_id: 531)\par
TI \tabto{0cm}Core Knowledge Employee Creativity and Firm Performance: The Moderating Role of Riskiness Orientation, Firm Size, and Realized \hl{Absorpti}ve \hl{Capacit}y\par
AU \tabto{0cm}Gong, Zhou, Chang\par
PY \tabto{0cm}2013, SO PERSONNEL PSYCHOLOGY\par
DT \tabto{0cm}Review\par
PG \tabto{0cm}40, NR 126, TC 14\par
DE \tabto{0cm}null\par
ID \tabto{0cm}ELECTRONICS INDUSTRY, FINANCIAL PERFORMANCE, HUMAN-RESOURCE MANAGEMENT, INNOVATION, JOB DEMANDS, ORGANIZATIONAL SIZE, SELF-EFFICACY, SERVICE CONTEXT, SUBJECTIVE MEASURES, WORK\par
AB \tabto{0cm}In this study, we examine when creativity is positively or negatively related to firm performance. Building on the creationimplementation tension theorized in the literature and the attention \hl{capacit}y perspective, we argue that the relationship between creativity and firm performance is contingent on riskiness orientation, firm size, and realized \hl{absorpti}ve \hl{capacit}y. Data were collected from 761 core knowledge employees, 148 CEOs, and 148 HR executives from 148 high-technology firms. The results indicated that core knowledge employee creativity was negatively related to firm performance when riskiness orientation was high. The relationship was positive when realized \hl{absorpti}ve \hl{capacit}y was high. Finally, the relationship was more positive in small firms than in large firms. We discuss the implications of our findings for creativity research and managerial practices.\par
\clearpage

\vspace*{-2cm}
Nb \tabto{0cm}205/327 (article\_id: 532)\par
TI \tabto{0cm}An empirical study of firm's \hl{absorpti}ve \hl{capacit}y dimensions, supplier involvement and new product development performance\par
AU \tabto{0cm}Tavani, Sharifi, Soleimanof, Najmi\par
PY \tabto{0cm}2013, SO INTERNATIONAL JOURNAL OF PRODUCTION RESEARCH\par
DT \tabto{0cm}Article\par
PG \tabto{0cm}19, NR 76, TC 5\par
DE \tabto{0cm}\hl{ABSORPTI}VE \hl{CAPACIT}Y, NEW PRODUCT DEVELOPMENT, SUPPLIER INVOLVEMENT\par
ID \tabto{0cm}COLLABORATIVE NETWORKS, COMPETITIVE ADVANTAGE, DEVELOPMENT COOPERATION, DEVELOPMENT-PROJECTS, INNOVATION PERFORMANCE, INTERNATIONAL-JOINT-VENTURES, KNOWLEDGE, MARKET ORIENTATION, RESEARCH-AND-DEVELOPMENT, TECHNOLOGY\par
AB \tabto{0cm}Firms' performance in their new product development (NPD) is believed to be positively related with involving suppliers in the process of new product development, and also with the organisation's \hl{capacit}y and capability to absorb external and internal knowledge, namely \hl{absorpti}ve \hl{capacit}y (AC). Addressing a gap in the literature, this study adopts the definition and structure for AC suggested by Tu et al. (2006) to examine relationships between AC's sub-dimensions with NPD performance, and also their moderating effects on the relationship between supplier involvement and new product development performance, on both financial and nonfinancial aspects. Data from a survey of 161 manufacturing firms are used to test the developed hypotheses using structural equation modelling and hierarchical regression. Direct and contingent effects of supplier involvement and AC on new product development performance are studied. As a result factors determining AC are found of different level of effects on financial and nonfinancial performance of new products, which will have implication for theory and practice.\par
\clearpage

\vspace*{-2cm}
Nb \tabto{0cm}206/327 (article\_id: 533)\par
TI \tabto{0cm}Strategic Renewal Over Time: The Enabling Role of Potential \hl{Absorpti}ve \hl{Capacit}y in Aligning Internal and External Rates of Change\par
AU \tabto{0cm}Ben-Menahem, Kwee, Volberda, Van den Bosch\par
PY \tabto{0cm}2013, SO LONG RANGE PLANNING\par
DT \tabto{0cm}Article\par
PG \tabto{0cm}20, NR 76, TC 8\par
DE \tabto{0cm}null\par
ID \tabto{0cm}CAPABILITIES, COEVOLUTION, FIT, HIGH-VELOCITY ENVIRONMENTS, INNOVATION, KNOWLEDGE, MANAGEMENT RESEARCH, ORGANIZATIONAL-CHANGE, PERFORMANCE, RETROSPECTIVE ACCOUNTS\par
AB \tabto{0cm}Top managers of multinational corporations are increasingly confronted with an accelerating rate of change in the external environment. Yet strategic renewal literature has devoted limited attention to the organizational mechanisms enabling firms to align internal with external rates of change, so as to achieve a dynamic firm-environment fit over time. This paper addresses that gap by taking a knowledge-based perspective. We develop a framework clarifying how a firm's potential \hl{absorpti}ve \hl{capacit}y enables it to align internal with external rates of change. We illustrate the framework empirically by analyzing the rate of change in strategic renewal actions of Royal Dutch Shell as an indicator of the company's internal rate of change in the period 1980-2007, and by comparing it with external rates of change in the oil industry over the same period. The findings show that Shell's potential \hl{absorpti}ve \hl{capacit}y was positively related to the alignment of internal and external rates of change. In addition, we find evidence that the degree of alignment was positively related to the company's performance during the observation period. Our study implies that managers who are aiming to align internal and external rates of change over time should: 1) monitor external rates of change through environmental scanning and boundary spanning, 2) create shared understanding of the long-term implications of change, 3) identify drivers of internal rates of change and understand how to pace the rate of strategic renewal actions, and finally, 4) maintain baseline levels of potential \hl{absorpti}ve \hl{capacit}y, since increasing potential \hl{absorpti}ve \hl{capacit}y takes time and requires a long-term perspective. (C) 2012 Elsevier Ltd. All rights reserved.\par
\clearpage

\vspace*{-2cm}
Nb \tabto{0cm}207/327 (article\_id: 534)\par
TI \tabto{0cm}Greening logistics and its impact on environmental performance: an \hl{absorpti}ve \hl{capacit}y perspective\par
AU \tabto{0cm}Abareshi, Molla\par
PY \tabto{0cm}2013, SO INTERNATIONAL JOURNAL OF LOGISTICS-RESEARCH AND APPLICATIONS\par
DT \tabto{0cm}Article\par
PG \tabto{0cm}18, NR 53, TC 1\par
DE \tabto{0cm}\hl{ABSORPTI}VE \hl{CAPACIT}Y, AUSTRALIA, ENVIRONMENTAL PERFORMANCE, GREEN LOGISTICS\par
ID \tabto{0cm}BENEFITS, COMPETITIVE ADVANTAGE, DYNAMIC CAPABILITIES, FIRM, KNOWLEDGE, MANAGEMENT, RECONCEPTUALIZATION, RESOURCE-BASED VIEW, TRANSPORT\par
AB \tabto{0cm}This study investigates the role of \hl{absorpti}ve \hl{capacit}y in implementing green logistics practices and the impact of the implementation on green logistics performance (GLP). Data were collected from a survey of 279 Australian Logistics and Transport operators and analysed using structural equation modelling. The findings indicate that enhancing green logistics knowledge exploitation is important to improve GLP. This can be achieved through changing the logistics operations and incorporating new knowledge into green practices in a way that can reduce CO2 emission, fuel consumption, or the cost of environmental compliance. The findings also show that addressing environmental concerns requires a process in which environmental information, through a wide range of channels and practices, is acquired, assimilated, transformed, and exploited. The paper reports an original research that contributes to the understanding of the value of green practices and routines to the environmental performance of firms in the Logistics and Transport sector. The results provide practitioners with insights that facilitate the transformation towards greener logistics practices and routines.\par
\clearpage

\vspace*{-2cm}
Nb \tabto{0cm}208/327 (article\_id: 535)\par
TI \tabto{0cm}Procedural Justice, Not \hl{Absorpti}ve \hl{Capacit}y, Matters in Multinational Enterprise ICT Transfers\par
AU \tabto{0cm}Verbeke, Bachor, Nguyen\par
PY \tabto{0cm}2013, SO MANAGEMENT INTERNATIONAL REVIEW\par
DT \tabto{0cm}Article\par
PG \tabto{0cm}20, NR 55, TC 3\par
DE \tabto{0cm}\hl{ABSORPTI}VE \hl{CAPACIT}Y, INFORMATION AND COMMUNICATIONS TECHNOLOGY, KNOWLEDGE MANAGEMENT, KNOWLEDGE TACITNESS, MULTINATIONAL ENTERPRISES, PROCEDURAL JUSTICE\par
ID \tabto{0cm}FIRM, INNOVATION, KNOWLEDGE, ORGANIZATIONAL PRACTICES, PERSPECTIVE, RATIONALITY, STICKINESS, STRATEGIC DECISION-MAKING, SUBSIDIARIES, TECHNOLOGY\par
AB \tabto{0cm}This paper empirically tests the effectiveness of information and communications technology (ICT) knowledge transfer and adoption in the multinational enterprise (MNE) as an issue of critical importance to contemporary MNE functioning. In contrast to mainstream thinking on \hl{absorpti}ve \hl{capacit}y, but in line with prevailing international business theory, our research supports the proposition that perceptions of procedural justice, rather than \hl{absorpti}ve \hl{capacit}y, determine effectiveness, especially in cases of high tacit knowledge transfers.
Data was collected from senior ICT representatives in 86 Canadian subsidiaries of foreign owned MNEs. Each of these subsidiaries recently experienced a significant ICT transfer imposed by the parent organization.
Support was found for the main propositions: Procedural justice significantly predicted successful ICT transfer and adoption, while \hl{absorpti}ve \hl{capacit}y was not significant. These findings are consistent even when knowledge tacitness was high.
The perceived success of the ICT transfer as well as its adoption varied widely across these firms. The potential reasons for this divergence in effectiveness are manifold, but our findings suggest that in situations of substantial knowledge tacitness, a higher level of procedural justice, rather than a higher level of \hl{absorpti}ve \hl{capacit}y, is critical to effective transfer and adoption.\par
\clearpage

\vspace*{-2cm}
Nb \tabto{0cm}209/327 (article\_id: 536)\par
TI \tabto{0cm}Imports and TFP at the firm level: the role of \hl{absorpti}ve \hl{capacit}y\par
AU \tabto{0cm}Augier, Cadot, Dovis\par
PY \tabto{0cm}2013, SO CANADIAN JOURNAL OF ECONOMICS-REVUE CANADIENNE D ECONOMIQUE\par
DT \tabto{0cm}Article\par
PG \tabto{0cm}26, NR 63, TC 5\par
DE \tabto{0cm}null\par
ID \tabto{0cm}CAPITAL GOODS, DEVELOPMENT SPILLOVERS, INCREASING RETURNS, INTERNATIONAL-TRADE, MANUFACTURING FIRMS, PRODUCTIVITY, PROPENSITY SCORE, RESEARCH-AND-DEVELOPMENT, SPANISH FIRMS, TECHNOLOGY\par
AB \tabto{0cm}This paper estimates the effect of the decision to import intermediate goods and capital equipment on Total Factor Productivity (TFP) at the firm level on a panel of Spanish firms (1991-2002). We use two alternative approaches. In the first, we estimate TFP and apply a diff-in-diff estimator with a control group constructed by propensity-score matching. In the second, direct method, we estimate TFP with imported inputs as a state variable in one stage. Both approaches show that the effect of a firm's decision to source intermediates and capital equipment abroad on its TFP depends critically on its \hl{capacit}y to absorb technology, measured by the proportion of skilled labour.\par
\clearpage

\vspace*{-2cm}
Nb \tabto{0cm}210/327 (article\_id: 537)\par
TI \tabto{0cm}The Relationship between Organizational Culture, \hl{Absorpti}ve \hl{Capacit}y, and Performance of Korea's International Logistics Service Providers: Verification of the Mediation Effect\par
AU \tabto{0cm}Cho, Pak, Hur, Lee\par
PY \tabto{0cm}2013, SO JOURNAL OF KOREA TRADE\par
DT \tabto{0cm}Article\par
PG \tabto{0cm}21, NR 40, TC 0\par
DE \tabto{0cm}\hl{ABSORPTI}VE \hl{CAPACIT}Y, INTERNATIONAL LOGISTICS SERVICE PROVIDERS (ILSP), MEDIATION EFFECT, ORGANIZATIONAL CULTURE, PERFORMANCE\par
ID \tabto{0cm}JOINT VENTURES, MANAGEMENT, NATIONAL CULTURE, RECONCEPTUALIZATION\par
AB \tabto{0cm}With the advent of globalization - a 21st century phenomenon - many corporations have been developing inroads into overseas markets. As international competition continues to intensify, some companies and researchers have taken a profound interest in the social cognitions, reactions, backgrounds, and behaviors of members in organizations. In line with these tendencies, organizational culture, along with \hl{absorpti}ve \hl{capacit}y, is becoming one of the decisive factors. This study aims to use empirical analysis to identify the relationships between organizational culture, \hl{absorpti}ve \hl{capacit}y, and performance of international logistics service providers (ILSP) in Korea.
The empirical research shows that (1) there is no direct relationship between the construct of power distance and performance; (2) \hl{absorpti}ve \hl{capacit}y has no mediation effect on the relationship between the construct of uncertainty avoidance and business performance; (3) in the case of the constructs of masculinity and femininity and individualism and collectivism, there is a full mediation effect only on financial performance; (4) there is partial mediation between long-term orientation and non-financial performance.
Overall, the results of this study suggest that managers of ILSP should bear in mind that the constructs of organizational culture have considerable influence over an organization's \hl{absorpti}ve \hl{capacit}y and business performance. These firms should acknowledge these influences and apply appropriate strategies, focusing on known weaknesses, to effectively strengthen their competitive market positions and long-term survival and success.\par
\clearpage

\vspace*{-2cm}
Nb \tabto{0cm}211/327 (article\_id: 538)\par
TI \tabto{0cm}\hl{ABSORPTI}VE \hl{CAPACIT}Y, KNOWLEDGE FLOWS, AND INNOVATION IN US METROPOLITAN AREAS\par
AU \tabto{0cm}Mukherji, Silberman\par
PY \tabto{0cm}2013, SO JOURNAL OF REGIONAL SCIENCE\par
DT \tabto{0cm}Article\par
PG \tabto{0cm}26, NR 50, TC 4\par
DE \tabto{0cm}null\par
ID \tabto{0cm}2 FACES, CITIES, EUROPE, GEOGRAPHY, GROWTH, MODELS, PANEL-DATA, PATENT CITATIONS, RESEARCH-AND-DEVELOPMENT, SPILLOVERS\par
AB \tabto{0cm}High growth and progressive regions possess a culture that promotes innovation. Innovation depends on a region's ability to use its own existing knowledge and knowledge generated elsewhere. This paper demonstrates the importance of the ability to absorb external knowledge in explaining innovation productivity for 106 U. S. metropolitan areas. Using a spatial interaction model of patent citation flows with origin and destination dependence, the destination fixed-effects coefficients provides a measure of a region's \hl{absorpti}ve \hl{capacit}y. We identify local conditions that shape a region's \hl{absorpti}ve \hl{capacit}y and demonstrate it has a positive and significant impact on innovation productivity.\par
\clearpage

\vspace*{-2cm}
Nb \tabto{0cm}212/327 (article\_id: 539)\par
TI \tabto{0cm}\hl{Absorpti}ve \hl{capacit}y and openness of small biopharmaceutical firms - a European Union-United States comparison\par
AU \tabto{0cm}Xia\par
PY \tabto{0cm}2013, SO R \& D MANAGEMENT\par
DT \tabto{0cm}Article\par
PG \tabto{0cm}19, NR 144, TC 3\par
DE \tabto{0cm}null\par
ID \tabto{0cm}DEDICATED BIOTECHNOLOGY FIRMS, DEVELOPMENT PROJECT PERFORMANCE, INDUSTRY, KNOWLEDGE FLOWS, MANUFACTURING FIRMS, MEDIUM-SIZED ENTERPRISES, OPEN INNOVATION, PRODUCT DEVELOPMENT, RESEARCH-AND-DEVELOPMENT, STRATEGIC ALLIANCES\par
AB \tabto{0cm}The complementarities between internal capabilities and external linkages have been widely acknowledged in the open innovation literature, yet little is known about the extent to which internal capabilities affect firms' openness within different institutional contexts. This paper therefore empirically explores the relationship between \hl{absorpti}ve \hl{capacit}y (ACAP) and openness in the United States and European biopharmaceutical sectors. Based on analysis of data from a large-scale international survey of 349 biopharmaceutical firms in the United States, the United Kingdom, France and Germany, the results suggest that exploratory openness depends more strongly on the research and development (R\&D) aspect of firms' potential \hl{absorpti}ve \hl{capacit}y, whereas exploitative openness is more conditional on firms' realized \hl{absorpti}ve \hl{capacit}y (RACAP). The results also highlight the major differences between firms' openness and ACAP in the United States and Europe - in the United States, firms' skill levels prove more significant in contributing to firms' engagement with exploratory relationships, whereas in Europe, continuity of R\&D proves more important. Engagement with exploitative relationships, however, is more conditional on firms' RACAP in Europe only.\par
\clearpage

\vspace*{-2cm}
Nb \tabto{0cm}213/327 (article\_id: 540)\par
TI \tabto{0cm}The effects of Information Technology on \hl{absorpti}ve \hl{capacit}y and organisational performance\par
AU \tabto{0cm}Bolivar-Ramos, Garcia-Morales, Martin-Rojas\par
PY \tabto{0cm}2013, SO TECHNOLOGY ANALYSIS \& STRATEGIC MANAGEMENT\par
DT \tabto{0cm}Article\par
PG \tabto{0cm}18, NR 59, TC 0\par
DE \tabto{0cm}INFORMATION TECHNOLOGY, INTERDEPENDENT TASKS, ORGANISATIONAL PERFORMANCE, POTENTIAL \hl{ABSORPTI}VE \hl{CAPACIT}Y, REALISED \hl{ABSORPTI}VE \hl{CAPACIT}Y, TECHNICAL IT SKILLS\par
ID \tabto{0cm}ANTECEDENTS, BUSINESS, CAPABILITIES, COMPETITIVE ADVANTAGE, FIRM PERFORMANCE, INNOVATION, KNOWLEDGE MANAGEMENT, PERSPECTIVE, RESOURCE-BASED ANALYSIS, SUPPLY CHAIN\par
AB \tabto{0cm}Information Technology (IT) offers many opportunities for firms to succeed. The aim of this paper is to present a model to reflect how technical IT skills and the use of IT in interdependent tasks may influence the development of organisational \hl{absorpti}ve \hl{capacit}y, both potential and realised, which also affects organisational performance. Since knowledge constitutes one of the main resources for organisations to gain competitive advantages and helps firms to improve their organisational performance, \hl{absorpti}ve \hl{capacit}y is a key factor in success. This model was tested empirically using a sample of 160 European technological firms. The results of our analysis suggest that the mastery of technical IT skills and the use of IT in interdependent tasks positively affect potential and realised \hl{absorpti}ve \hl{capacit}ies, which in turn enhances organisational performance. The study concludes by presenting some theoretical and practical implications, limitations, and future research lines.\par
\clearpage

\vspace*{-2cm}
Nb \tabto{0cm}214/327 (article\_id: 541)\par
TI \tabto{0cm}Firm R\&D, \hl{Absorpti}ve \hl{Capacit}y and Learning by Exporting: Firm-level Evidence from China\par
AU \tabto{0cm}Dai, Yu\par
PY \tabto{0cm}2013, SO WORLD ECONOMY\par
DT \tabto{0cm}Article\par
PG \tabto{0cm}15, NR 25, TC 2\par
DE \tabto{0cm}null\par
ID \tabto{0cm}2 FACES, DYNAMICS, INDUSTRY, INNOVATION, PERFORMANCE, PRODUCTIVITY\par
AB \tabto{0cm}The \hl{absorpti}ve \hl{capacit}y of firms developed through R\&D promotes learning by exporting. In this paper, we estimate the instantaneous and long-run productivity effects of exporting on the universe of Chinese manufacturing firms. We find that exporting has very different productivity effects for firms with different pre-export R\&D status. It has large and lasting productivity effects for firms with pre-export R\&D, while it has little effects for firms without pre-export R\&D. Furthermore, the effect of exporting increases with the number of years of pre-export R\&D investment.\par
\clearpage

\vspace*{-2cm}
Nb \tabto{0cm}215/327 (article\_id: 542)\par
TI \tabto{0cm}How information systems help create OM capabilities: Consequents and antecedents of operational \hl{absorpti}ve \hl{capacit}y\par
AU \tabto{0cm}Setia, Patel\par
PY \tabto{0cm}2013, SO JOURNAL OF OPERATIONS MANAGEMENT\par
DT \tabto{0cm}Article\par
PG \tabto{0cm}23, NR 161, TC 6\par
DE \tabto{0cm}ENVIRONMENTAL COMPLEXITY, INFORMATION SYSTEMS CAPABILITIES, OPERATIONAL \hl{ABSORPTI}VE \hl{CAPACIT}Y, STRATEGIC IT ALIGNMENT, TOBIN'S Q\par
ID \tabto{0cm}COMPETITIVE ADVANTAGE, DECISION-MAKING, DEVELOPMENT-PROJECTS, FIRM PERFORMANCE, KNOWLEDGE ACQUISITION, ORGANIZATIONAL PERFORMANCE, PRODUCT DEVELOPMENT, RELATIONAL VALUE, SUPPLY CHAIN MANAGEMENT, TOBINS-Q\par
AB \tabto{0cm}In contemporary business environments, the ability to manage operational knowledge is an important predictor of organizational competitiveness. Organizations invest large sums in various types of information technologies (ITs) to manage operational knowledge. Because of their superior storage, processing and communication capabilities, ITs offer technical platforms to build knowledge management (KM) capabilities. However, merely acquiring ITs are not sufficient, and organizations must structure information system (IS) designs to leverage ITs for building KM capabilities. We study how technical and strategic IS designs enhance operational \hl{absorpti}ve \hl{capacit}y (OAC) - the KM capability of an operations management (OM) department. Specifically, we use a capabilities perspective of \hl{absorpti}ve \hl{capacit}y to examine potential \hl{absorpti}ve \hl{capacit}y (POAC) and realized \hl{absorpti}ve \hl{capacit}y (ROAC) capabilities - the two OAC capabilities that create and utilize knowledge, respectively. Our theory proposes that integrated IS capability, - an aspect of technical IS design - is an antecedent of POAC and ROAC capabilities, and business-IT alignment - an aspect of strategic IS design - moderates the relationship between integrated IS capability and ROAC capability. Combining data gleaned from a multi-respondent survey with archival data from COMPUSTAT, we test our hypotheses using a dataset from 153 manufacturing organizations. By proposing that IS design enables an OM department's KM processes, i.e., the POAC and ROAC capabilities, our interdisciplinary theoretical framework opens the "black box" of OAC and contributes to improved understanding of IS and OM synergies. We offer a detailed discussion of our contributions to the literature at the IS-OM interface and implications for practitioners. (C) 2013 Elsevier B.V. All rights reserved.\par
\clearpage

\vspace*{-2cm}
Nb \tabto{0cm}216/327 (article\_id: 543)\par
TI \tabto{0cm}Technological upgrading in Taiwan's TFT-LCD industry: signs of a deeper \hl{absorpti}ve \hl{capacit}y?\par
AU \tabto{0cm}Chuang, Hobday\par
PY \tabto{0cm}2013, SO TECHNOLOGY ANALYSIS \& STRATEGIC MANAGEMENT\par
DT \tabto{0cm}Article\par
PG \tabto{0cm}22, NR 73, TC 1\par
DE \tabto{0cm}LATECOMER \hl{ABSORPTI}VE \hl{CAPACIT}Y, LATECOMER FIRMS, TAIWAN, TECHNOLOGICAL CAPABILITY, TFT-LCD, TRANSITION, UPGRADING\par
ID \tabto{0cm}CAPABILITY ACCUMULATION, COMBINATIVE CAPABILITIES, DEVELOPING-COUNTRY, DISPLAY INDUSTRY, INNOVATION CAPABILITY, INTERNATIONAL JOINT VENTURES, KNOWLEDGE TRANSFER, LATECOMER FIRMS, RESEARCH-AND-DEVELOPMENT, SEMICONDUCTOR INDUSTRY\par
AB \tabto{0cm}Firms from several advanced developing countries have successfully upgraded to higher levels of capability and competitiveness on the international stage. This study explores how leading Taiwanese latecomer firms acquired strong capabilities in the display (thin film transistor-liquid crystal display, TFT-LCD) industry. We identify how these firms acquired the technology base to catch up rapidly and move from one set of advanced products and technologies to another, focusing on both the accumulation of technological capabilities and the underlying \hl{absorpti}ve \hl{capacit}y of each firm. Empirically, we distinguish three phases of capability building, namely pre-entry, entry, and innovation and diversification. Theoretically, we suggest that the notion of latecomer' \hl{absorpti}ve \hl{capacit}y, centred on engineering and design rather than R\&D, might explain both the rapidity of recent technological catch up and the diversification across products and technologies as latecomer firms approach the technology frontier.\par
\clearpage

\vspace*{-2cm}
Nb \tabto{0cm}217/327 (article\_id: 544)\par
TI \tabto{0cm}An exploratory analysis of the relationship between \hl{absorpti}ve \hl{capacit}y and business strategy\par
AU \tabto{0cm}Flor, Oltra\par
PY \tabto{0cm}2013, SO TECHNOLOGY ANALYSIS \& STRATEGIC MANAGEMENT\par
DT \tabto{0cm}Article\par
PG \tabto{0cm}15, NR 35, TC 1\par
DE \tabto{0cm}BUSINESS STRATEGY, POTENTIAL \hl{ABSORPTI}VE \hl{CAPACIT}Y, REALISED \hl{ABSORPTI}VE \hl{CAPACIT}Y\par
ID \tabto{0cm}ANTECEDENTS, FIRM, INNOVATION, ORGANIZATIONAL PERFORMANCE, ORIENTATION\par
AB \tabto{0cm}In this study, we argue that a firm's \hl{absorpti}ve \hl{capacit}y will vary depending on the strategy it adopts. To examine this, based on the fact that \hl{absorpti}ve \hl{capacit}y is developed through the ability to acquire, assimilate, transform and exploit externally-generated knowledge, we look at the importance of each of these dimensions at firms pursuing different business strategies. To reflect business strategy, we draw on Miles and Snow's typology. The information has been obtained based on a sample made up of 81 Spanish small and medium-sized enterprises. Results show that knowledge acquisition \hl{capacit}y is greater at prospectors than at defenders and analysers, and that transformation and exploitation \hl{capacit}ies are greater at prospectors than at defenders. No differences in knowledge assimilation \hl{capacit}y are observed.\par
\clearpage

\vspace*{-2cm}
Nb \tabto{0cm}218/327 (article\_id: 545)\par
TI \tabto{0cm}Innovating not-for-profit social ventures: Exploring the microfoundations of internal and external \hl{absorpti}ve \hl{capacit}y routines\par
AU \tabto{0cm}Chalmers, Balan-Vnuk\par
PY \tabto{0cm}2013, SO INTERNATIONAL SMALL BUSINESS JOURNAL\par
DT \tabto{0cm}Article\par
PG \tabto{0cm}26, NR 100, TC 1\par
DE \tabto{0cm}\hl{ABSORPTI}VE \hl{CAPACIT}Y, DYNAMIC CAPABILITIES, ORGANISATIONAL ROUTINES, SOCIAL ENTREPRENEURSHIP, SOCIAL INNOVATION\par
ID \tabto{0cm}CHALLENGES, COMPETITION, CREATION, DYNAMIC CAPABILITIES, ENTREPRENEURSHIP, KNOWLEDGE TRANSFER, ORGANIZATIONAL ROUTINES, ORIENTATION, PERFORMANCE, SENSEMAKING\par
AB \tabto{0cm}Research into the phenomenon of social innovation has long focused on what it is and why people become engaged in this form of behaviour. However, another piece of the theoretical jigsaw requires understanding how this type of innovation is enacted by organisations. This article looks at the means by which not-for-profit ventures pursuing socially innovative activities develop the necessary capabilities to innovate. The multidimensional theoretical construct of \hl{absorpti}ve \hl{capacit}y and the evolutionary economics concept of organisational routines are used to analyse 14 case studies of innovative not-for-profit ventures in Australia and the UK. The results show that these organisations have a unique mediating function in the social innovation process by configuring internal and external \hl{absorpti}ve \hl{capacit}y routines to combine user and technological knowledge flows. The article concludes by proposing some research directions for those taking forward the study of social innovation.\par
\clearpage

\vspace*{-2cm}
Nb \tabto{0cm}219/327 (article\_id: 546)\par
TI \tabto{0cm}Flexibility-Oriented HRM Systems, \hl{Absorpti}ve \hl{Capacit}y, and Market Responsiveness and Firm Innovativeness\par
AU \tabto{0cm}Chang, Gong, Way, Jia\par
PY \tabto{0cm}2013, SO JOURNAL OF MANAGEMENT\par
DT \tabto{0cm}Article\par
PG \tabto{0cm}28, NR 92, TC 10\par
DE \tabto{0cm}\hl{ABSORPTI}VE \hl{CAPACIT}Y, FIRM INNOVATIVENESS, FLEXIBILITY-ORIENTED HRM SYSTEMS, MARKET RESPONSIVENESS\par
ID \tabto{0cm}CAPABILITIES, COMPETITIVE ADVANTAGE, HUMAN-RESOURCE MANAGEMENT, KNOWLEDGE, LEARNING-PROCESSES, MATTER, PERFORMANCE, PRODUCTIVITY, TECHNOLOGY, WORK PRACTICES\par
AB \tabto{0cm}Although market responsiveness and firm innovativeness are important aspects of firm performance, little is known about which human resource management (HRM) systems foster these performance aspects and how. Building on prior research, we delineate flexibility-oriented human resource management (FHRM) systems in terms of resource- and coordination-flexibility-oriented HRM subsystems. In addition, we draw on organizational learning theory and the concept of \hl{absorpti}ve \hl{capacit}y (AC) to articulate the mechanisms through which these systems might influence market responsiveness and firm innovativeness. We develop and validate measures of FHRM systems using a series of four independent samples. Our findings based on a sample of high-technology firms indicate that FHRM systems are positively associated with firm-level potential and realized AC and that potential AC, in turn, is positively associated with market responsiveness and firm innovativeness.\par
\clearpage

\vspace*{-2cm}
Nb \tabto{0cm}220/327 (article\_id: 547)\par
TI \tabto{0cm}Knowledge creation capability, \hl{absorpti}ve \hl{capacit}y, and product innovativeness\par
AU \tabto{0cm}Su, Ahlstrom, Li, Cheng\par
PY \tabto{0cm}2013, SO R \& D MANAGEMENT\par
DT \tabto{0cm}Article\par
PG \tabto{0cm}13, NR 44, TC 4\par
DE \tabto{0cm}null\par
ID \tabto{0cm}CHINA, DIMENSIONS, EXISTING KNOWLEDGE, FIRM PERFORMANCE, IMPACT, LEVEL, MARKET, PERSPECTIVE, RESEARCH-AND-DEVELOPMENT\par
AB \tabto{0cm}This study focuses on the impact of knowledge creation capability and \hl{absorpti}ve \hl{capacit}y on product innovativeness. Capabilities contribute through their uniqueness, their integration into effective configurations, and their deployment in response to external environment changes. Therefore, this study examines the individual (uniqueness) and interactive (integration) effects of knowledge creation capability and \hl{absorpti}ve \hl{capacit}y on product innovativeness as well as how these effects vary in differing technologically turbulent contexts (deployment). Based on a survey of 212 Chinese firms, this study finds that in addition to their individually positive effects, knowledge creation capability and \hl{absorpti}ve \hl{capacit}y have a synergistic effect on product innovativeness. Moreover, the individual effect of knowledge creation capability and the synergistic effect become stronger as technological turbulence increases, whereas the impact of \hl{absorpti}ve \hl{capacit}y tends to be dampened by technological turbulence.\par
\clearpage

\vspace*{-2cm}
Nb \tabto{0cm}221/327 (article\_id: 548)\par
TI \tabto{0cm}\hl{Absorpti}ve \hl{Capacit}y and the Growth and Investment Effects of Regional Transfers: A Regression Discontinuity Design with Heterogeneous Treatment Effects\par
AU \tabto{0cm}Becker, Egger, von Ehrlich\par
PY \tabto{0cm}2013, SO AMERICAN ECONOMIC JOURNAL-ECONOMIC POLICY\par
DT \tabto{0cm}Article\par
PG \tabto{0cm}49, NR 58, TC 3\par
DE \tabto{0cm}null\par
ID \tabto{0cm}CORRUPTION, COUNTRIES, ECONOMIC-DEVELOPMENT, EMPIRICS, FOREIGN-AID, GOVERNMENT, IDENTIFICATION, INFERENCE, LEAST-SQUARES, SKILL COMPLEMENTARITY\par
AB \tabto{0cm}Researchers often estimate average treatment effects of programs without investigating heterogeneity across units. Yet, individuals, firms, regions, or countries vary in their ability to utilize transfers. We analyze Objective 1 transfers of the EU to regions below a certain income level by way of a regression discontinuity design with systematically varying heterogeneous treatment effects. Only about 30 percent and 21 percent of the regions-those with sufficient human capital and good-enough institutions-are able to turn transfers into faster per capita income growth and per capita investment, respectively. In general, the variance of the treatment effect is much bigger than its mean.\par
\clearpage

\vspace*{-2cm}
Nb \tabto{0cm}222/327 (article\_id: 549)\par
TI \tabto{0cm}An Advantage of Newness: Vicarious Learning Despite Limited \hl{Absorpti}ve \hl{Capacit}y\par
AU \tabto{0cm}Posen, Chen\par
PY \tabto{0cm}2013, SO ORGANIZATION SCIENCE\par
DT \tabto{0cm}Article\par
PG \tabto{0cm}16, NR 68, TC 2\par
DE \tabto{0cm}\hl{ABSORPTI}VE \hl{CAPACIT}Y, ENTRY, EXPERIENTIAL LEARNING, VICARIOUS LEARNING\par
ID \tabto{0cm}COMPETITION, EXPERIENCE, FIRMS, HOTEL INDUSTRY, INFORMATION, KNOWLEDGE, MARKET ENTRY, ORGANIZATIONS, PATTERNS, PERFORMANCE\par
AB \tabto{0cm}Entrants are often viewed as suffering from a "liability of newness"-at founding, they rarely possess the knowledge and capabilities necessary to compete and survive. They can overcome this liability by learning vicariously from the knowledge of incumbent firms. But how can entrants learn from external knowledge when they lack the prior related knowledge that forms the basis of \hl{absorpti}ve \hl{capacit}y? We theorize that the process of internal experiential learning facilitates learning from external knowledge, particularly for entrants. To test this theory, we examine learning using a comprehensive set of U.S. commercial banking firms, including a full census of entrants. Our estimates suggest that the share of vicarious learning realized in the process of experiential learning is twice as large for entrants as for incumbents. In this sense, entrants enjoy an "advantage of newness" in learning.\par
\clearpage

\vspace*{-2cm}
Nb \tabto{0cm}223/327 (article\_id: 550)\par
TI \tabto{0cm}The impact of local linkages, international linkages, and \hl{absorpti}ve \hl{capacit}y on innovation for foreign firms operating in an emerging economy\par
AU \tabto{0cm}Liao, Yu\par
PY \tabto{0cm}2013, SO JOURNAL OF TECHNOLOGY TRANSFER\par
DT \tabto{0cm}Article\par
PG \tabto{0cm}19, NR 57, TC 4\par
DE \tabto{0cm}\hl{ABSORPTI}VE \hl{CAPACIT}Y, INNOVATION, INTERNATIONAL LINKAGES, LOCAL LINKAGES\par
ID \tabto{0cm}ANTECEDENTS, CHINESE FIRMS, COMPLEMENTARITY, EMPIRICAL-EVALUATION, INDUSTRY, LEGITIMACY, MANAGERIAL TIES, NETWORKS, ORGANIZATION, PERFORMANCE\par
AB \tabto{0cm}This study analyzes the impact of local linkages, international linkages, and \hl{absorpti}ve \hl{capacit}y on firm innovation for firms based in one emerging economy while operating in another emerging economy. Testing research hypotheses derived from the institutional and organizational learning perspectives on a sample of 102 Taiwanese manufacturing firms operating in China, we find that the impact of international linkages is greater than that of local linkages, while both local and international linkages have a positive impact on innovation. Further analysis confirms that \hl{absorpti}ve \hl{capacit}y has a stronger moderating effect on the relationship between local linkages and innovation than it does on the relationship between international linkages and innovation.\par
\clearpage

\vspace*{-2cm}
Nb \tabto{0cm}224/327 (article\_id: 551)\par
TI \tabto{0cm}Market knowledge \hl{absorpti}ve \hl{capacit}y: a measurement scale\par
AU \tabto{0cm}Jimenez-Castillo, Sanchez-Perez\par
PY \tabto{0cm}2013, SO INFORMATION RESEARCH-AN INTERNATIONAL ELECTRONIC JOURNAL\par
DT \tabto{0cm}Article\par
PG \tabto{0cm}15, NR 68, TC 0\par
DE \tabto{0cm}null\par
ID \tabto{0cm}ANTECEDENTS, COMPLEMENTARITY, CONCEPTUALIZATION, IMPACT, INFORMATION, INNOVATION PERFORMANCE, INTERRATER AGREEMENT, ORIENTATION, PRODUCT PERFORMANCE, RECONCEPTUALIZATION\par
AB \tabto{0cm}Introduction. Managers frequently question whether the effort in collecting market information and disseminating it within their firm leads to improved processing. The purpose of this paper is to provide a useful indicator to evaluate the ability of employees to process and exploit market knowledge. To this end, we develop and validate a measurement scale of 'market knowledge \hl{absorpti}ve \hl{capacit}y'.
Method. We utilize deductive and inductive approaches for item generation and structural equation modelling for detailed assessment of the dimensionality, reliability, and validity of the measurement scale.
Analysis. We performed the analyses with LISREL 8.80 statistical package software using data from 211 industrial firms.
Results. The results validate a parsimonious four-dimensional scale consisting of acquisition, assimilation, transformation, and exploitation \hl{capacit}ies. The psychometric properties of the scale are successfully assessed. Nomological validity is confirmed by demonstrating a positive relationship between market knowledge \hl{absorpti}ve \hl{capacit}y and innovation performance.
Conclusions. The proposed multidimensional scale can be used as a diagnostic tool for firms to evaluate the \hl{capacit}y of employees to effectively absorb market knowledge. It can also be considered a valuable extension to the existing research on \hl{absorpti}ve \hl{capacit}y as it helps overcome the limited nature of previous measures.\par
\clearpage

\vspace*{-2cm}
Nb \tabto{0cm}225/327 (article\_id: 552)\par
TI \tabto{0cm}Coping with rivals' \hl{absorpti}ve \hl{capacit}y in innovation activities\par
AU \tabto{0cm}Hurmelinna-Laukkanen, Olander\par
PY \tabto{0cm}2014, SO TECHNOVATION\par
DT \tabto{0cm}Article\par
PG \tabto{0cm}9, NR 84, TC 3\par
DE \tabto{0cm}APPROPRIABILITY REGIME, INNOVATION PERFORMANCE, KNOWLEDGE PROTECTION, RIVAL \hl{ABSORPTI}VE \hl{CAPACIT}Y\par
ID \tabto{0cm}APPROPRIABILITY, COMPETITIVE ADVANTAGE, EMPIRICAL-EVIDENCE, INDUSTRY, INTELLECTUAL PROPERTY, KNOWLEDGE SPILLOVERS, PERFORMANCE, PRODUCT INNOVATION, RESEARCH-AND-DEVELOPMENT, STRATEGY\par
AB \tabto{0cm}Two factors jointly determine the likelihood of a firm's competitors obtaining information on its intangible assets and using it to damage the firm's innovation performance. Those factors are the \hl{absorpti}ve \hl{capacit}y of the rival firm and the appropriability regime of the innovating firm. However, the precise roles of the two factors in affecting performance outcomes are not well documented. Furthermore, we lack knowledge of the interplay between an appropriability regime and \hl{absorpti}ve \hl{capacit}y, although they clearly have the \hl{capacit}y to exert positive and negative effects both on each other and on innovativeness. This study presents findings derived from theoretical discussion and an empirical examination of 155 firms that suggest that while competitors' \hl{absorpti}ve \hl{capacit}y does not play a direct negative or positive role on the innovation performance of a firm, an appropriability regime exerts a strong positive influence. Nevertheless, high rival \hl{absorpti}ve \hl{capacit}y is not without importance, since the significant interaction effects suggest that a strong appropriability regime has positive effects on innovation performance especially in the context of a rival having high \hl{absorpti}ve \hl{capacit}y. (C) 2013 Elsevier Ltd. All rights reserved.\par
\clearpage

\vspace*{-2cm}
Nb \tabto{0cm}226/327 (article\_id: 553)\par
TI \tabto{0cm}A study of contingency relationships between supplier involvement, \hl{absorpti}ve \hl{capacit}y and agile product innovation\par
AU \tabto{0cm}Tavani, Sharifi, Ismail\par
PY \tabto{0cm}2014, SO INTERNATIONAL JOURNAL OF OPERATIONS \& PRODUCTION MANAGEMENT\par
DT \tabto{0cm}Article\par
PG \tabto{0cm}28, NR 89, TC 2\par
DE \tabto{0cm}\hl{ABSORPTI}VE \hl{CAPACIT}Y, AGILE PRODUCT INNOVATION, AGILITY, SUPPLIER INVOLVEMENT\par
ID \tabto{0cm}BUSINESS PERFORMANCE, COLLABORATIVE NETWORKS, DEVELOPMENT SUCCESS, EMPIRICAL-TEST, FIRM PERFORMANCE, KNOWLEDGE TRANSFER, MANUFACTURING STRATEGY, MODERATING ROLE, ORGANIZATIONAL PERFORMANCE, STRUCTURAL EQUATION MODELS\par
AB \tabto{0cm}Purpose - The purpose of this paper is to employ agility concept to develop a contingency perspective of relationship between suppliers' involvement, \hl{absorpti}ve \hl{capacit}y (AC) and product innovation (PI). While the moderating effect of AC on the relationship between supplier involvement and PI performance is investigated, a firm's agility in PI is entered as one dimension of the firm's performance to accommodate a multidimensional perspective.
Design/methodology/approach - The paper formulates six hypotheses extracted from the relevant literature. The survey was conducted over the internet by using web-based questionnaire. A sampling frame of 1,200 manufacturing UK-based companies provided 233 usable responses. A confirmatory factor analysis was used to test a validity and reliability of constructs and further the paper employed hierarchical multiple regression to test the research hypotheses.
Findings - The results while reaffirm some of the existing theories of the subject provide some differing view of the issues allowing projection of new insight on the approach to PI and involvement of suppliers. The results support the proposition of PI performance multidimensionality where achievements beyond financial and market-related factors play a critical role. Furthermore, research findings suggest AC as a competitive factor that can provide the grounds for proactively winning in the PI game through increasing agility capabilities.
Research limitations/implications - This study uses a random sample of UK manufacturing companies, which could be extended to firms from outside the UK too.
Originality/value - The paper provides a new insight into the existing literature on "new product innovation" and its relationship with suppliers' involvement as well as the firm's AC by employing agility perspective, as a leading theory to explain dynamics and uncertainties in the business environment.\par
\clearpage

\vspace*{-2cm}
Nb \tabto{0cm}227/327 (article\_id: 554)\par
TI \tabto{0cm}MNC knowledge transfer, subsidiary \hl{absorpti}ve \hl{capacit}y and HRM\par
AU \tabto{0cm}Minbaeva, Pedersen, Bjorkman, Fey, Park\par
PY \tabto{0cm}2014, SO JOURNAL OF INTERNATIONAL BUSINESS STUDIES\par
DT \tabto{0cm}Article\par
PG \tabto{0cm}14, NR 48, TC 1\par
DE \tabto{0cm}\hl{ABSORPTI}VE \hl{CAPACIT}Y, HRM, KNOWLEDGE TRANSFER\par
ID \tabto{0cm}HUMAN-RESOURCE MANAGEMENT, IMPACT, INTERNATIONAL-JOINT-VENTURES, MANUFACTURING PERFORMANCE, MOTIVATION, MULTINATIONAL-CORPORATIONS, ORGANIZATIONAL PERFORMANCE, PERSPECTIVE, PRODUCTIVITY, SYSTEMS\par
AB \tabto{0cm}Based on a sample of 169 subsidiaries of multinational corporations (MNCs) operating in the USA, Russia, and Finland, this paper investigates the relationship between MNC subsidiary HRM practices, \hl{absorpti}ve \hl{capacit}y and knowledge transfer. First, we examine the relationship between the application of specific HRM practices and the level of the \hl{absorpti}ve \hl{capacit}y. Second, we suggest that \hl{absorpti}ve \hl{capacit}y should be conceptualized as being comprised of both employees' ability and motivation. Further, results indicate that both ability and motivation (\hl{absorpti}ve \hl{capacit}y) are needed to facilitate the transfer of knowledge from other parts of the MNC.\par
\clearpage

\vspace*{-2cm}
Nb \tabto{0cm}228/327 (article\_id: 555)\par
TI \tabto{0cm}A retrospective on: MNC knowledge transfer, subsidiary \hl{absorpti}ve \hl{capacit}y, and HRM\par
AU \tabto{0cm}Minbaeva, Pedersen, Bjorkman, Fey\par
PY \tabto{0cm}2014, SO JOURNAL OF INTERNATIONAL BUSINESS STUDIES\par
DT \tabto{0cm}Article\par
PG \tabto{0cm}11, NR 81, TC 6\par
DE \tabto{0cm}\hl{ABSORPTI}VE \hl{CAPACIT}Y, DECADE AWARD, HUMAN RESOURCE MANAGEMENT (HRM), KNOWLEDGE TRANSFER\par
ID \tabto{0cm}COMPETITIVE ADVANTAGE, DETERMINANTS, HUMAN-RESOURCE MANAGEMENT, IMPACT, INNOVATION, INTERNATIONAL-JOINT-VENTURES, MANAGING KNOWLEDGE, MECHANISMS, MULTINATIONAL-CORPORATIONS, PERFORMANCE\par
AB \tabto{0cm}In this retrospective, we revisit the goals of the original paper, and we review the studies that have used our paper to discuss the "concept" and the "development" of \hl{absorpti}ve \hl{capacit}y. We also propose directions for future research, stressing the need to develop thorough theoretical and empirical models of \hl{absorpti}ve \hl{capacit}y as a multi-level and dynamic construct that is contingent on the context in which it is embedded.\par
\clearpage

\vspace*{-2cm}
Nb \tabto{0cm}229/327 (article\_id: 556)\par
TI \tabto{0cm}Subsidiary \hl{absorpti}ve \hl{capacit}y and knowledge transfer within multinational corporations\par
AU \tabto{0cm}Song\par
PY \tabto{0cm}2014, SO JOURNAL OF INTERNATIONAL BUSINESS STUDIES\par
DT \tabto{0cm}Article\par
PG \tabto{0cm}12, NR 74, TC 5\par
DE \tabto{0cm}\hl{ABSORPTI}VE \hl{CAPACIT}Y, DECADE AWARD, HRM PRACTICES, KNOWLEDGE TRANSFER, MOTIVATION, MULTINATIONAL CORPORATIONS (MNCS) AND ENTERPRISES (MNES)\par
ID \tabto{0cm}CREATION, FIRMS, FLOWS, FOREIGN SUBSIDIARIES, INNOVATION, MANAGEMENT, MANAGING KNOWLEDGE, MNC, PERFORMANCE, RESEARCH-AND-DEVELOPMENT\par
AB \tabto{0cm}The paper reviews extant literature on subsidiary \hl{absorpti}ve \hl{capacit}y and knowledge transfer within multinational corporations (MNCs), and proposes an agenda for future research on the relationship between these two constructs. It suggests that motivation should be viewed as a moderating factor between subsidiary \hl{absorpti}ve \hl{capacit}y and MNC knowledge transfer, and that future research should make a clear distinction between the choices of MNC headquarters and those of subsidiaries regarding knowledge transfer. The paper proposes that a more comprehensive, multi-level framework and dynamic model of the determinants of subsidiary \hl{absorpti}ve \hl{capacit}y and MNC knowledge transfer be developed in future studies.\par
\clearpage

\vspace*{-2cm}
Nb \tabto{0cm}230/327 (article\_id: 557)\par
TI \tabto{0cm}\hl{Absorpti}ve \hl{capacit}y and network orchestration in innovation communities - promoting service innovation\par
AU \tabto{0cm}Natti, Hurmelinna-Laukkanen, Johnston\par
PY \tabto{0cm}2014, SO JOURNAL OF BUSINESS \& INDUSTRIAL MARKETING\par
DT \tabto{0cm}Article\par
PG \tabto{0cm}12, NR 85, TC 2\par
DE \tabto{0cm}\hl{ABSORPTI}VE \hl{CAPACIT}Y, INNOVATION COMMUNITY, INNOVATION NETWORK, NETWORK ORCHESTRATION, SERVICE INNOVATION\par
ID \tabto{0cm}BUSINESS, CAPABILITIES, COLLABORATION, FIRM, KNOWLEDGE TRANSFER, ORGANIZATIONS, PERFORMANCE, RECONCEPTUALIZATION, STRATEGIC ALLIANCES, VALUE CREATION\par
AB \tabto{0cm}Purpose - The purpose of this study is to increase understanding of service innovation in networks. Especially the most loosely coupled forms of innovation networks, innovation communities, can be valuable in service innovation, but may not be manageable in the traditional sense. Rather, they may require orchestration characterized by discreet guidance that also accommodates the specific nature of services. Through informed orchestration, it is possible to deal with several contingencies, and influence the \hl{absorpti}ve \hl{capacit}y at the network level to generate new service innovations.
Design/methodology/approach - These issues are examined through literature review and a case study.
Findings - The findings suggest that individual orchestration mechanisms may be more closely connected to certain contingencies than others, and that both orchestration mechanisms and contingency factors have a role in \hl{absorpti}ve \hl{capacit}y development within service innovation networks.
Research limitations/implications - While the case study approach limits the possibility to make wide generalizations, the in-depth insights provide valuable knowledge.
Practical implications - There has been a shift from inter-firm competition towards competition between networks of organizations, increasing relevance of \hl{absorpti}ve \hl{capacit}y at the network level.
Originality/value - Despite the recent increase in service innovation literature, research on service innovation taking place in networks is scant. Knowledge on some aspects can be derived from more traditional notions on technological innovation, but both the distinctive features of services and central characteristics of innovation networks make it necessary to reconsider some of the established views. In particular, managing - or rather orchestrating - service innovation is still a challenging area.\par
\clearpage

\vspace*{-2cm}
Nb \tabto{0cm}231/327 (article\_id: 558)\par
TI \tabto{0cm}Technology Spillovers of Foreign Direct Investment in Coastal Regions of East China: A Perspective on Technology \hl{Absorpti}ve \hl{Capacit}y\par
AU \tabto{0cm}Xu, Wan, Sun\par
PY \tabto{0cm}2014, SO EMERGING MARKETS FINANCE AND TRADE\par
DT \tabto{0cm}Article\par
PG \tabto{0cm}11, NR 20, TC 3\par
DE \tabto{0cm}COASTAL REGIONS OF EAST CHINA, FDI, HUMAN RESOURCE, INSTITUTIONAL CHANGE, TECHNOLOGY \hl{ABSORPTI}VE \hl{CAPACIT}Y, TECHNOLOGY SPILLOVERS\par
ID \tabto{0cm}ECONOMIC-GROWTH, FDI, IMPACT, INDUSTRIES\par
AB \tabto{0cm}By establishing a foreign direct investment (FDI) technology spillover estimation model based on technology \hl{absorpti}ve \hl{capacit}y and using provincial panel data of the coastal regions of East China from 2001 to 2010, we empirically conclude that FDI technology spillover effect in the coastal regions of East China is not significant. However, technology \hl{absorpti}ve \hl{capacit}y is the determining factor of FDI technology spillover effect. Further analysis shows that technology \hl{absorpti}ve capabilities respectively represented by institutional change and human resources have different effects on the endogenous economic growth and the FDI technology spillovers. To be more specific, the effect of technology \hl{absorpti}ve \hl{capacit}y represented by the level of human resources on the economic growth and FDI technology spillovers is much more significant than that represented by institutional change.\par
\clearpage

\vspace*{-2cm}
Nb \tabto{0cm}232/327 (article\_id: 559)\par
TI \tabto{0cm}Environmental turbulence, \hl{absorpti}ve \hl{capacit}y and external knowledge search among Chinese SMEs\par
AU \tabto{0cm}Guo, Wang\par
PY \tabto{0cm}2014, SO CHINESE MANAGEMENT STUDIES\par
DT \tabto{0cm}Article\par
PG \tabto{0cm}15, NR 65, TC 1\par
DE \tabto{0cm}\hl{ABSORPTI}VE \hl{CAPACIT}Y, EXTERNAL KNOWLEDGE SEARCH, SMES\par
ID \tabto{0cm}BEHAVIOR, CLUSTER, DETERMINANTS, INFORMATION, INNOVATION PERFORMANCE, INTEGRATIVE FRAMEWORK, MODERATING ROLE, RECONCEPTUALIZATION, STRATEGY, UNCERTAINTY\par
AB \tabto{0cm}Purpose - This paper tests which theoretical perspective(s) can better explain firms' external knowledge search behavior. Information processing and resource-based view theories propose a positive relationship between environmental turbulence and knowledge search breadth, whereas transaction cost economics and managerial attention theoretical perspectives posit that knowledge search breadth will be negatively influenced by environmental turbulence. In the context of Chinese small- and medium-sized enterprises (SMEs), this study examines the direct effect of environmental turbulence and the interactive effect of environmental turbulence and \hl{absorpti}ve \hl{capacit}y (ACAP) on external knowledge search breadth.
Design/methodology/approach - This study adopted firm-level data collected via questionnaires from SMEs within the manufacturing sector in China. The partial least squares method was used to explore the determinants of the external knowledge search breadth of Chinese SMEs.
Findings - The results reveal that external search breadth tends to increase with an increase in a firm's perceived environmental turbulence. In addition, the interaction between ACAP and environmental turbulence will be negatively related to external knowledge search breadth. The empirical evidence indicates information processing and resource-based view theories are more powerful in explaining the external knowledge search behavior of Chinese SMEs.
Originality/value - Unlike most of the innovation search literature, which have focused on the effect on performance of external search, this study focuses on the antecedents of firms' innovation search behavior. The study contributes to the understanding of the relationship between environmental turbulence and knowledge search breadth as well as the understanding of the influence of ACAP on external knowledge search in the context of SMEs from emerging economies.\par
\clearpage

\vspace*{-2cm}
Nb \tabto{0cm}233/327 (article\_id: 560)\par
TI \tabto{0cm}Exploring the impact of empowering leadership on knowledge sharing, \hl{absorpti}ve \hl{capacit}y and team performance in IT service\par
AU \tabto{0cm}Lee, Park\par
PY \tabto{0cm}2014, SO INFORMATION TECHNOLOGY \& PEOPLE\par
DT \tabto{0cm}Article\par
PG \tabto{0cm}21, NR 70, TC 0\par
DE \tabto{0cm}\hl{ABSORPTI}VE \hl{CAPACIT}Y, IT PROJECT MANAGEMENT, KNOWLEDGE WORKER PERFORMANCE, LEADERSHIP\par
ID \tabto{0cm}DEVELOPMENT-PROJECTS, INFORMATION-SYSTEMS DEVELOPMENT, INTEGRATION, ORGANIZATIONS, PERSPECTIVE, PROCESS INNOVATION, PSYCHOLOGICAL EMPOWERMENT, RECONCEPTUALIZATION, TECHNOLOGY, VALIDATION\par
AB \tabto{0cm}Purpose - The purpose of this paper is to empirically investigate the mechanism through which empowering leadership of a team leader might influence the team performance in IT service.
Design/methodology/approach - The data of 315 individuals collected from 85 different IT projects through online survey is used to empirically test the hypotheses.
Findings - The results confirm that team leader's empowering leadership raises the level of knowledge sharing among team members and increase the \hl{absorpti}ve \hl{capacit}y of the team, and lead to better team performance.
Research limitations/implications - This research theoretically presented and demonstrated the middle-and long-term impacts of empowering leadership resulting from the development of \hl{absorpti}ve \hl{capacit}y as the effects of knowledge sharing in an IT project team are produced through \hl{absorpti}ve \hl{capacit}y.
Practical implications - The findings indicate that more effective in increasing the performance of IT project teams can be to strengthen empowering leadership than to promote traditional charisma or directive leadership. Knowledge sharing at a team level has the direct effect of improving project performance by providing information and knowledge regarding the related project, but on the other hand it contributes to making stronger the path of associating \hl{absorpti}ve \hl{capacit}y with project performance.
Originality/value - The impact of empowering leaderships on team performance of IT project has received less research attention. Little prior research has carried out such an integrated analysis in IT service context. This study contributes to knowledge management research by identifying a key antecedent of knowledge sharing.\par
\clearpage

\vspace*{-2cm}
Nb \tabto{0cm}234/327 (article\_id: 561)\par
TI \tabto{0cm}Innovation with IS usage: individual \hl{absorpti}ve \hl{capacit}y as a mediator\par
AU \tabto{0cm}Wang, Liu, Feng\par
PY \tabto{0cm}2014, SO INDUSTRIAL MANAGEMENT \& DATA SYSTEMS\par
DT \tabto{0cm}Article\par
PG \tabto{0cm}21, NR 57, TC 0\par
DE \tabto{0cm}\hl{ABSORPTI}VE \hl{CAPACIT}Y, BUSINESS INTELLIGENCE SYSTEMS, FAIRNESS OF REWARDS, INFORMATION SYSTEM INNOVATION, JOB AUTONOMY\par
ID \tabto{0cm}ASSIMILATION, ENTERPRISE SYSTEMS, INFORMATION-TECHNOLOGY, KNOWLEDGE TRANSFER, MODEL, ORGANIZATIONAL CULTURE, PERSPECTIVE, SYSTEM USE, TOP MANAGEMENT, USER ACCEPTANCE\par
AB \tabto{0cm}Purpose - After information systems (IS) implementation, many organizations report that system underutilization causes the failure to meet expected IS investment returns. It is imperative to understand the way to leverage employees' fullest potential in the IS usage. The paper aims to discuss these issues.
Design/methodology/approach - Anchoring on \hl{absorpti}ve \hl{capacit}y (ACAP) theory, the authors develop an employee innovation model. Using survey data and structural equation modeling, this research investigates how perceived organizational levers affect innovation with IS usage (INVU) by introducing individual ACAP as a mediator.
Findings - The authors find general support for the research model through a survey of 205 employees using SAP business intelligence systems in China. The empirical data shows that three interrelated components of individual ACAP significantly contribute to INVU. The findings also suggest that, both fairness of reward and job autonomy are key organizational levers for the utility of individual ACAP. Furthermore, their effects on INVU can be fully mediated by individual ACAP.
Originality/value - The authors empirically unpack and validate individual ACAP in IS innovation situation. The findings provide academics and practitioners with an understanding of how management can inspire employees' potential in implemented system innovation.\par
\clearpage

\vspace*{-2cm}
Nb \tabto{0cm}235/327 (article\_id: 562)\par
TI \tabto{0cm}\hl{Absorpti}ve \hl{Capacit}y, Proximity in Cooperation and Integration Mechanisms. Empirical Evidence from CIS Data\par
AU \tabto{0cm}Franco, Marzucchi, Montresor\par
PY \tabto{0cm}2014, SO INDUSTRY AND INNOVATION\par
DT \tabto{0cm}Article\par
PG \tabto{0cm}26, NR 60, TC 0\par
DE \tabto{0cm}\hl{ABSORPTI}VE \hl{CAPACIT}Y, HUMAN CAPITAL, INNOVATION COOPERATION\par
ID \tabto{0cm}ALLIANCES, FIRM, IMPACT, INNOVATION PERFORMANCE, INTERNATIONAL-JOINT-VENTURES, MNC KNOWLEDGE TRANSFER, ORGANIZATION, RECONCEPTUALIZATION, RESOURCE MANAGEMENT-PRACTICES, SEARCH\par
AB \tabto{0cm}The paper extends available findings on the antecedents and impact of the firm's \hl{absorpti}ve \hl{capacit}y. Innovation cooperation is recognized as a driver of its potential side (PAC). Considering different forms of proximity, we expect to find a higher impact for interactions occurring between close partners. Human capital (HC) is expected to be as important as other organizational mechanisms for the innovation impact of PAC. An empirical application with Community Innovation Survey data confirms these arguments only partially. The firm's cooperation with geographically closer partners (i.e., in the same country) increases its PAC, but it is cooperation with institutionally distant ones (e.g., research organizations) that augments it. Among the integration mechanisms of external knowledge, those increasing the firm's HC are the only ones that positively moderate the innovation impact of PAC.\par
\clearpage

\vspace*{-2cm}
Nb \tabto{0cm}236/327 (article\_id: 563)\par
TI \tabto{0cm}Exploring the moderating effects of \hl{absorpti}ve \hl{capacit}y on the relationship between social networks and innovation\par
AU \tabto{0cm}Ahlin, Drnovsek, Hisrich\par
PY \tabto{0cm}2014, SO JOURNAL FOR EAST EUROPEAN MANAGEMENT STUDIES\par
DT \tabto{0cm}Article\par
PG \tabto{0cm}23, NR 75, TC 0\par
DE \tabto{0cm}\hl{ABSORPTI}VE \hl{CAPACIT}Y, CROSS-CULTURAL, ENTREPRENEUR NETWORKS, INNOVATION, MODERATING EFFECT, SMES\par
ID \tabto{0cm}COLLABORATION, ECONOMIC-DEVELOPMENT, ENTREPRENEURS, INFORMATION, KNOWLEDGE, MANUFACTURING FIRMS, PERFORMANCE, R-AND-D, SMES, SUCCESS\par
AB \tabto{0cm}Although networks in prior research have been highlighted as a remedy to resource constraints experienced by smaller firms, little attention has been paid to understanding mechanisms through which smaller firms benefit from networks for effective innovation performance. In this paper we develop a conceptual model of direct and moderated network effects on a firm's innovation. We test this model on a large sample of small and medium-sized firms from a post-transitional and developed economy. We show that \hl{absorpti}ve \hl{capacit}y moderates the relationship between networks and innovation outcomes, but with some country level variations. The implications of these results in relation to entrepreneurship theory and practice are discussed.\par
\clearpage

\vspace*{-2cm}
Nb \tabto{0cm}237/327 (article\_id: 564)\par
TI \tabto{0cm}Balancing on a tightrope: Managing the boundaries of a firm-sponsored OSS community and its impact on innovation and \hl{absorpti}ve \hl{capacit}y\par
AU \tabto{0cm}Teigland, Di Gangi, Flaten, Giovacchini, Pastorino\par
PY \tabto{0cm}2014, SO INFORMATION AND ORGANIZATION\par
DT \tabto{0cm}Article\par
PG \tabto{0cm}23, NR 62, TC 2\par
DE \tabto{0cm}\hl{ABSORPTI}VE \hl{CAPACIT}Y, BOUNDARY MANAGEMENT, COMMUNITY MANAGEMENT, INNOVATION, OPEN SOURCE, PRIVATE-COLLECTIVE COMMUNITIES, SOFTWARE\par
ID \tabto{0cm}ARCHITECTURE, GOVERNANCE, MANAGEMENT, MECHANISMS, MODEL, MOTIVATION, ONLINE COMMUNITIES, OPEN-SOURCE SOFTWARE, ORGANIZATION SCIENCE, PERFORMANCE\par
AB \tabto{0cm}Realizing the innovation potential of OSS communities, firms now create or sponsor their own open source software (OSS) communities, generally as part of an open innovation strategy. However, maximizing the innovation capability of a sponsored OSS community is a challenging task since firms cannot rely on traditional hierarchical authority to control community members. Furthermore, a firm's efforts to manage its sponsored community may also impact the firm's \hl{absorpti}ve \hl{capacit}y, or its ability to effectively absorb and leverage the valuable knowledge created by the community. Thus, the purpose of this article is to investigate two research questions: 1) How does the boundary management of a firm-sponsored OSS community impact the community's innovation \hl{capacit}y? and 2) How does the boundary management of a firm-sponsored OSS community impact the firm's \hl{absorpti}ve \hl{capacit}y? Using the results from our qualitative analysis of eZ Systems and its successfully sponsored OSS community - eZ Publish we develop a theoretical model depicting how the boundary management of a firm-sponsored OSS community influences both the community's innovation \hl{capacit}y and the \hl{absorpti}ve \hl{capacit}y of the firm. In addition, the results of our study highlight the central importance of an integrative IT platform in boundary management activities. (C) 2014 Elsevier Ltd. All rights reserved.\par
\clearpage

\vspace*{-2cm}
Nb \tabto{0cm}238/327 (article\_id: 565)\par
TI \tabto{0cm}Beyond \hl{absorpti}ve \hl{capacit}y: in-house R\&D as a driver of innovative complementarities\par
AU \tabto{0cm}Catozzella, Vivarelli\par
PY \tabto{0cm}2014, SO APPLIED ECONOMICS LETTERS\par
DT \tabto{0cm}Article\par
PG \tabto{0cm}4, NR 11, TC 1\par
DE \tabto{0cm}COMPLEMENTARITY, INNOVATION, O31, R\&D\par
ID \tabto{0cm}null\par
AB \tabto{0cm}This article shows how in-house R\&D may play a role in affecting innovative output beyond its direct impact and its indirect effect through \hl{absorpti}ve \hl{capacit}y, generating additional synergies that amplify the impacts of innovative inputs other than R\&D itself.\par
\clearpage

\vspace*{-2cm}
Nb \tabto{0cm}239/327 (article\_id: 566)\par
TI \tabto{0cm}Realised \hl{absorpti}ve \hl{capacit}y, technology acquisition and performance in international collaborative formations: an empirical examination in the Korean context\par
AU \tabto{0cm}Buckley, Park\par
PY \tabto{0cm}2014, SO ASIA PACIFIC BUSINESS REVIEW\par
DT \tabto{0cm}Article\par
PG \tabto{0cm}27, NR 61, TC 2\par
DE \tabto{0cm}INTERNATIONAL COLLABORATIVE FORMATIONS, KOREA, PERFORMANCE, REALISED \hl{ABSORPTI}VE \hl{CAPACIT}Y, TECHNOLOGY ACQUISITION\par
ID \tabto{0cm}CAPABILITIES, CHINESE JOINT VENTURES, FOREIGN PARENTS, KNOWLEDGE TRANSFER, PERSPECTIVE, STRATEGIC ALLIANCES, TRANSITION ECONOMY, VIETNAM\par
AB \tabto{0cm}The research objectives of this paper are twofold. First, it attempts to identify the critical factors facilitating the acquisition of technology from foreign partner firms and performance enhancement in international collaborative formations. Second, it compares these factors across Western and Japanese sub-samples. A new concept, realised \hl{absorpti}ve \hl{capacit}y, is employed to achieve these objectives. Using a sample collected through a questionnaire-based survey, our results show that trust and communication are critical components in transforming new knowledge, while active managerial involvement of the foreign firm and the participation of foreign expatriates are the keys to applying it for business operations. In particular, components consisting of the ability to exploit new knowledge (i.e. active managerial involvement of foreign firm, participation of foreign expatriates and provision of training) are closely associated with performance enhancement. In addition, our results confirm that there clearly exist different patterns of learning mechanisms and incompatible behaviours in improving performance between the two sub-samples. Based on the results, the authors suggest implications and future research avenues.\par
\clearpage

\vspace*{-2cm}
Nb \tabto{0cm}240/327 (article\_id: 567)\par
TI \tabto{0cm}\hl{Absorpti}ve \hl{capacit}y development in Indonesian exporting firms: How do institutions matter?\par
AU \tabto{0cm}Gunawan, Rose\par
PY \tabto{0cm}2014, SO INTERNATIONAL BUSINESS REVIEW\par
DT \tabto{0cm}Article\par
PG \tabto{0cm}10, NR 57, TC 3\par
DE \tabto{0cm}\hl{ABSORPTI}VE \hl{CAPACIT}Y, EXPORT, INDONESIA, INTERNATIONALIZATION, LEARNING\par
ID \tabto{0cm}ANTECEDENTS, CLUSTER, IMPACT, INNOVATION PERFORMANCE, INTERNATIONALIZATION PROCESS, ISOMORPHISM, KNOWLEDGE, MARKETS, PERSPECTIVE, STRATEGIES\par
AB \tabto{0cm}This paper addresses how firms from an emerging market characterized by a challenging and variable institutional environment learn about internationalizing. Building on the organizational learning and institutional literatures, and the concept of \hl{absorpti}ve \hl{capacit}y (AC), and using a sample of Indonesian manufacturing-sector exporters we identify two dimensions of internationalization-related AC: international market and international strategic operation. Unlike previous literature, we find that indirect, or second-hand, experience contributes more than the firm's own experience to the development of international market AC. Furthermore, the second-hand experience feeds Indonesian manufacturing exporters' learning in both positive (e.g., buyers) and negative (e.g., suppliers and foreign multinationals in Indonesia) ways. In contrast, the development of international operation strategy AC appears to be driven internally, with minimal contribution from either first- or second-hand experience. We posit that these outcomes are influenced by the rapid and substantial changes in the domestic institutional environment faced by the Indonesian manufacturers. (C) 2013 Elsevier Ltd. All rights reserved.\par
\clearpage

\vspace*{-2cm}
Nb \tabto{0cm}241/327 (article\_id: 568)\par
TI \tabto{0cm}Connections count: How relational embeddedness and relational empowerment foster \hl{absorpti}ve \hl{capacit}y\par
AU \tabto{0cm}Ebers, Maurer\par
PY \tabto{0cm}2014, SO RESEARCH POLICY\par
DT \tabto{0cm}Article\par
PG \tabto{0cm}15, NR 114, TC 5\par
DE \tabto{0cm}\hl{ABSORPTI}VE \hl{CAPACIT}Y, INNOVATION, RELATIONAL EMBEDDEDNESS, RELATIONAL EMPOWERMENT\par
ID \tabto{0cm}FIRM, KNOWLEDGE TRANSFER, MEDIATING ROLE, MODEL, ORGANIZATIONAL RESEARCH, PERFORMANCE, RESEARCH-AND-DEVELOPMENT, SOCIAL-STRUCTURE, TECHNOLOGICAL-INNOVATION, WEAK TIES\par
AB \tabto{0cm}While research has produced ample evidence showing that \hl{absorpti}ve \hl{capacit}y affects innovation and organizational performance outcomes, we still know little about why some organizations possess greater \hl{absorpti}ve \hl{capacit}y than others. This study extends previous research by showing how \hl{absorpti}ve \hl{capacit}y emerges as an unintended consequence from organizational boundary spanners' external and internal relational embeddedness and their relational empowerment. Drawing upon survey data from 218 inter-organizational projects in the German engineering industry, we propose and find empirically that potential and realized \hl{absorpti}ve \hl{capacit}y have partially distinct antecedents. Moreover, we show that the two components of \hl{absorpti}ve \hl{capacit}y unfold not only separate but also complementary effects on innovation, implying that the whole of \hl{absorpti}ve \hl{capacit}y is greater than its parts. In examining how different components of \hl{absorpti}ve \hl{capacit}y emerge and unfold their effects, this study addresses critical limitations of the literature on \hl{absorpti}ve \hl{capacit}y. (C) 2013 Elsevier B.V. All rights reserved.\par
\clearpage

\vspace*{-2cm}
Nb \tabto{0cm}242/327 (article\_id: 569)\par
TI \tabto{0cm}\hl{ABSORPTI}VE \hl{CAPACIT}Y IN BUYER-SUPPLIER RELATIONSHIPS:EMPIRICAL EVIDENCE OF ITS MEDIATING ROLE\par
AU \tabto{0cm}Saenz, Revilla, Knoppen\par
PY \tabto{0cm}2014, SO JOURNAL OF SUPPLY CHAIN MANAGEMENT\par
DT \tabto{0cm}Article\par
PG \tabto{0cm}23, NR 92, TC 0\par
DE \tabto{0cm}\hl{ABSORPTI}VE \hl{CAPACIT}Y, BUYER-SUPPLIER RELATIONSHIPS, EFFICIENCY, INNOVATION, LEARNING PROCESSES, ORGANIZATIONAL COMPATIBILITY, STRUCTURAL EQUATION MODELING\par
ID \tabto{0cm}CHAIN MANAGEMENT, COMPETITIVE ADVANTAGE, DYNAMIC-CAPABILITIES, INFORMATION-TECHNOLOGY, KNOWLEDGE DEVELOPMENT, LEARNING-PROCESSES, MEASUREMENT INVARIANCE, PERFORMANCE, PRODUCT DEVELOPMENT, STRATEGIC ALLIANCES\par
AB \tabto{0cm}Companies increasingly depend upon the knowledge of supply chain partners to deliver superior value to customers with ever shifting preferences. This transference requires \hl{absorpti}ve \hl{capacit}y (AC), which allows an organization to identify external knowledge and convert it into value for the firm. Based on an approach of dynamic capabilities, AC encompasses three related learning processes: exploration, assimilation, and exploitation. Within the particular context of buyer-supplier relationships (BSR), the aim of this research is to examine AC, one of its most relevant antecedents - organizational compatibility - and its outcomes. Two samples of 153 and 199 companies, operating as key suppliers of two focal buyers, a European multinational retail chain and an American multinational spare parts distributor, respectively, constitute the empirical base of the study. Results derived from structural equation modeling and, more precisely, multi-group confirmatory factor analysis and a formal test of mediation, strongly indicate for both samples that AC mediates between organizational compatibility on the one hand and innovation and efficiency performance on the other hand. Results also indicate that the mediating effect of AC related to innovation increases with demand uncertainty. This paper thus suggests that managers must be aware that the selection of supply chain partners based on their compatibility alone is not enough. AC is necessary to achieve
sustainable performance improvement.\par
\clearpage

\vspace*{-2cm}
Nb \tabto{0cm}243/327 (article\_id: 570)\par
TI \tabto{0cm}The effects of MNC parent effort and social structure on subsidiary \hl{absorpti}ve \hl{capacit}y\par
AU \tabto{0cm}Schleimer, Pedersen\par
PY \tabto{0cm}2014, SO JOURNAL OF INTERNATIONAL BUSINESS STUDIES\par
DT \tabto{0cm}Article\par
PG \tabto{0cm}18, NR 118, TC 2\par
DE \tabto{0cm}\hl{ABSORPTI}VE \hl{CAPACIT}Y, HEADQUARTERS-SUBSIDIARY ROLES AND RELATIONS, MARKETING STRATEGY, ORGANIZATIONAL LEARNING, STRUCTURAL EQUATION MODELING\par
ID \tabto{0cm}COMBINATIVE CAPABILITIES, FOREIGN SUBSIDIARIES, GLOBAL STRATEGY, KNOWLEDGE TRANSFER, MULTINATIONAL-CORPORATIONS, MULTIUNIT ORGANIZATION, ORGANIZATIONAL PRACTICES, PERFORMANCE, R-AND-D, WEAK TIES\par
AB \tabto{0cm}Although the literature provides ample evidence that the global transfer and local implementation of knowledge represents a key advantage for multinational corporations (MNCs), we lack comparable understanding as to whether knowledge-creating MNC parents can actively expand the \hl{absorpti}ve \hl{capacit}y of their subsidiaries. Using a teacher-student lens, this study examines the combined impact of specific structural mechanisms and motivational processes by MNC parents on the ability of 216 subsidiaries to absorb parent-initiated marketing strategies. The findings reveal that MNC parents can indeed cultivate subsidiaries' ability to appropriate marketing knowledge through a combination of adopting specific social structures and investing in particular efforts. However, the effect of social structure on subsidiary \hl{absorpti}ve \hl{capacit}y is indirect, and accounted for by the parents' intensity of effort. A number of theoretical implications emerge from the findings for research on \hl{absorpti}ve \hl{capacit}y in relation to the role of the knowledge source, the need to examine organizational influences in relation to one another, and validating the original \hl{absorpti}ve \hl{capacit}y dimensions. For managers in the global marketplace, the findings lead to the suggestion that MNCs devote attention to nurturing the \hl{absorpti}ve \hl{capacit}ies at different organizational levels in order to optimize the global transfer of knowledge.\par
\clearpage

\vspace*{-2cm}
Nb \tabto{0cm}244/327 (article\_id: 571)\par
TI \tabto{0cm}\hl{Absorpti}ve \hl{capacit}y from foreign direct investment in Spanish manufacturing firms\par
AU \tabto{0cm}Sanchez-Sellero, Rosell-Martinez, Garcia-Vazquez\par
PY \tabto{0cm}2014, SO INTERNATIONAL BUSINESS REVIEW\par
DT \tabto{0cm}Article\par
PG \tabto{0cm}11, NR 92, TC 5\par
DE \tabto{0cm}\hl{ABSORPTI}VE \hl{CAPACIT}Y, DYNAMIC CAPABILITIES, FOREIGN DIRECT INVESTMENT, INTERNATIONALIZATION, PRODUCTIVITY\par
ID \tabto{0cm}2 FACES, COMPETITIVE ADVANTAGE, DEVELOPMENT SUBSIDIES, EMPIRICAL-ANALYSIS, JOINT VENTURES, KNOWLEDGE TRANSFER, MULTINATIONAL-CORPORATIONS, PANEL-DATA, PRODUCT DEVELOPMENT, RESEARCH-AND-DEVELOPMENT\par
AB \tabto{0cm}This paper deals with the determinants of \hl{absorpti}ve \hl{capacit}y from foreign direct investment (FDI) spillovers. We study how firm behavior, capabilities, and structure drive \hl{absorpti}ve \hl{capacit}y such as research and development (R\&D) activities and expenditures, R\&D results, internal organization of innovation, external relationships of innovation, human-capital quality, family management, business complexity, and market concentration. Our results enhance and complement previous evidence of the determinants of \hl{absorpti}ve \hl{capacit}y, particularly with different approaches to innovative activities as mediators of the capability. (C) 2013 The Authors. Published by Elsevier Ltd. All rights reserved.\par
\clearpage

\vspace*{-2cm}
Nb \tabto{0cm}245/327 (article\_id: 572)\par
TI \tabto{0cm}Counter-knowledge and realised \hl{absorpti}ve \hl{capacit}y\par
AU \tabto{0cm}Cegarra-Navarro, Eldridge, Wensley\par
PY \tabto{0cm}2014, SO EUROPEAN MANAGEMENT JOURNAL\par
DT \tabto{0cm}Article\par
PG \tabto{0cm}12, NR 84, TC 5\par
DE \tabto{0cm}COUNTER-KNOWLEDGE, REALISED \hl{ABSORPTI}VE \hl{CAPACIT}Y, UNLEARNING CONTEXT\par
ID \tabto{0cm}ANTECEDENTS, CONSTRUCTION, EXTENSION, IN-SERVICE ORGANIZATIONS, INNOVATION, PERFORMANCE, PERSPECTIVE, PROJECTS, ROUTINES, TECHNOLOGY\par
AB \tabto{0cm}In the following paper we investigate the concept of counter-knowledge and how its effects may be mitigated in an organisational context. Counter-knowledge may be acquired unwittingly from unreliable or inaccurate sources such as gossip, lies, exaggeration and partial truths. We consider that if counter-knowledge is present then specific actions are required to stimulate realised \hl{absorpti}ve \hl{capacit}y and, hence, provide for the creation and assimilation of new knowledge and new knowledge structures. Thus, in this paper, we focus on intentional unlearning as a method to counteract the problem of counter-knowledge. We have analysed the relationships between an unlearning context and counter-knowledge using an empirical study of 164 Spanish hospitality companies in order to identify whether the impact of unlearning on RACAP can be strength. A model is tested in which counter-knowledge is simultaneously a hindrance and a challenge stressor. Our results confirm that counter-knowledge is a variable that, when controlled, has the effect of strengthening the relationship between unlearning and RACAP. However, when left uncontrolled, the relationship between unlearning and RACAP is weaker than it otherwise would be. (C) 2013 Elsevier Ltd. All rights reserved.\par
\clearpage

\vspace*{-2cm}
Nb \tabto{0cm}246/327 (article\_id: 573)\par
TI \tabto{0cm}\hl{Absorpti}ve \hl{capacit}y and innovation: when is it better to cooperate?\par
AU \tabto{0cm}Egbetokun, Savin\par
PY \tabto{0cm}2014, SO JOURNAL OF EVOLUTIONARY ECONOMICS\par
DT \tabto{0cm}Article\par
PG \tabto{0cm}22, NR 45, TC 2\par
DE \tabto{0cm}\hl{ABSORPTI}VE \hl{CAPACIT}Y, COGNITIVE DISTANCE, INNOVATION, INTER-FIRM COOPERATION, KNOWLEDGE SPILLOVERS\par
ID \tabto{0cm}COLLABORATION, EVOLUTION, EXPLORATION, FIRMS, KNOWLEDGE, NETWORKS, OPTIMAL COGNITIVE DISTANCE, STRATEGIC ALLIANCES\par
AB \tabto{0cm}Cooperation can benefit and hurt firms at the same time. An important question then is: when is it better to cooperate? And, once the decision to cooperate is made, how can an appropriate partner be selected? In this paper we present a model of inter-firm cooperation driven by cognitive distance, appropriability conditions and external knowledge. \hl{Absorpti}ve \hl{capacit}y of firms develops as an outcome of the interaction between \hl{absorpti}ve R\&D and cognitive distance from voluntary and involuntary knowledge spillovers. Thus, we offer a revision of the original model by Cohen and Levinthal (Econ J 99(397):569-596, 1989), accounting for recent empirical findings and explicitly modeling \hl{absorpti}ve \hl{capacit}y within the framework of interactive learning. We apply that to the analysis of firms' cooperation and R\&D investment preferences. The results show that cognitive distance and appropriability conditions between a firm and its cooperation partner have an ambiguous effect on the profit generated by the firm. Thus, a firm chooses to cooperate and selects a partner conditional on the investments in \hl{absorpti}ve \hl{capacit}y it is willing to make to solve the understandability/novelty trade-off.\par
\clearpage

\vspace*{-2cm}
Nb \tabto{0cm}247/327 (article\_id: 574)\par
TI \tabto{0cm}Social media assimilation in firms: Investigating the roles of \hl{absorpti}ve \hl{capacit}y and institutional pressures\par
AU \tabto{0cm}Bharati, Zhang, Chaudhury\par
PY \tabto{0cm}2014, SO INFORMATION SYSTEMS FRONTIERS\par
DT \tabto{0cm}Article\par
PG \tabto{0cm}16, NR 80, TC 2\par
DE \tabto{0cm}\hl{ABSORPTI}VE \hl{CAPACIT}Y, INFORMATION SYSTEMS ASSIMILATION, INNOVATION, INSTITUTIONAL THEORY, SOCIAL MEDIA, WEB 2.0\par
ID \tabto{0cm}ADOPTION, DIFFUSION, INFORMATION-TECHNOLOGY, INNOVATION, ISOMORPHISM, ORGANIZATIONAL ASSIMILATION, PERFORMANCE, PERSPECTIVE, SYSTEMS, TOP MANAGEMENT\par
AB \tabto{0cm}Firms are increasingly employing social media to manage relationships with partner organizations, yet the role of institutional pressures in social media assimilation has not been studied. We investigate social media assimilation in firms using a model that combines the two theoretical streams of IT adoption: organizational innovation and institutional theory. The study uses a composite view of \hl{absorpti}ve \hl{capacit}y that includes both previous experience with similar technology and the general ability to learn and exploit new technologies. We find that institutional pressures are an important antecedent to \hl{absorpti}ve \hl{capacit}y, an important measure of organizational learning capability. The paper augments theory in finding the role and limits of institutional pressures. Institutional pressures are found to have no direct effect on social media assimilation but to impact \hl{absorpti}ve \hl{capacit}y, which mediates its influence on assimilation.\par
\clearpage

\vspace*{-2cm}
Nb \tabto{0cm}248/327 (article\_id: 575)\par
TI \tabto{0cm}Preparing for distant collaboration: Antecedents to potential \hl{absorpti}ve \hl{capacit}y in cross-industry innovation\par
AU \tabto{0cm}Enkel, Heil\par
PY \tabto{0cm}2014, SO TECHNOVATION\par
DT \tabto{0cm}Article\par
PG \tabto{0cm}19, NR 171, TC 4\par
DE \tabto{0cm}\hl{ABSORPTI}VE \hl{CAPACIT}Y, CASE STUDY ANALYSIS, COGNITIVE DISTANCE, CROSS-INDUSTRY INNOVATION, EXPLORATORY INNOVATION, NETWORK ANALYSIS, RADICAL INNOVATION\par
ID \tabto{0cm}COMBINATIVE CAPABILITIES, DYNAMIC CAPABILITIES, ESTABLISHED FIRMS, EXPLORATORY INNOVATION, OPTIMAL COGNITIVE DISTANCE, ORGANIZATIONAL AMBIDEXTERITY, RELATIONAL EMBEDDEDNESS, SECTORAL PATTERNS, STRATEGIC ALLIANCES, TECHNOLOGICAL-INNOVATION\par
AB \tabto{0cm}Cross-industry innovation entails distinctive innovation opportunities and challenges according to the knowledge heterogeneity between the collaborating firms. This heterogeneity yields increases in organizational-level cognitive distance. Whereas recent theory suggests cognitive distance is positively related to exploratory innovation, too much distance can hinder efficient knowledge \hl{absorpti}on and results in a reduced effect on novelty value. This paper focuses on the research question of how to build potential \hl{absorpti}ve \hl{capacit}y for distant collaboration beyond established industry boundaries to gain radical rather than incremental results. To address this question, we mapped a cross-industry network using survey data on 215 bilateral cross-industry collaborations between firms from a variety of industries and captured cognitive proximity (the inverse of distance) in terms of overall knowledge redundancy between firms. This approach introduces a new method to infer organizational-level cognitive distance from network analysis. Subsequently, based on results from the network analysis, we examined coordination antecedents to potential \hl{absorpti}ve \hl{capacit}y for cross-industry innovation with partners at moderate and high distance applying case study analysis. Our study revealed three alternative approaches to coordination antecedents that drive a firm's potential \hl{absorpti}ve \hl{capacit}y for distant collaboration. These findings extend research on \hl{absorpti}ve \hl{capacit}y to the field of cross-industry innovation. (C) 2014 Elsevier Ltd. All rights reserved.\par
\clearpage

\vspace*{-2cm}
Nb \tabto{0cm}249/327 (article\_id: 576)\par
TI \tabto{0cm}Product portfolio decision-making and \hl{absorpti}ve \hl{capacit}y: A simulation study\par
AU \tabto{0cm}Makinen, Vilkko\par
PY \tabto{0cm}2014, SO JOURNAL OF ENGINEERING AND TECHNOLOGY MANAGEMENT\par
DT \tabto{0cm}Article\par
PG \tabto{0cm}16, NR 27, TC 0\par
DE \tabto{0cm}\hl{ABSORPTI}VE \hl{CAPACIT}Y, DECISION-MAKING, PRODUCT PORTFOLIO, SYSTEM DYNAMICS\par
ID \tabto{0cm}DESIGN, DYNAMIC CAPABILITIES, EXPLOITATION, EXPLORATION, INERTIA, INNOVATION, MODEL, RECONCEPTUALIZATION\par
AB \tabto{0cm}The central decision any firm must make is determining which markets to serve with what products. This core managerial decision-making involves balancing between exploration and exploitation of new technology-market knowledge. We construct a history-friendly empirically grounded system dynamics simulation setting that explores the dynamics of managing firm's product portfolio. Our simulation findings illustrate how managerial decision-making regarding the \hl{absorpti}ve \hl{capacit}y may influence the evolution of the product portfolio. The results pave the way for multiple fruitful research avenues for future studies. (C) 2013 Elsevier B.V. All rights reserved.\par
\clearpage

\vspace*{-2cm}
Nb \tabto{0cm}250/327 (article\_id: 577)\par
TI \tabto{0cm}\hl{Absorpti}ve \hl{capacit}y, innovation and cultural barriers: A Conditional mediation model\par
AU \tabto{0cm}Leal-Rodriguez, Ariza-Montes, Roldan, Leal-Millan\par
PY \tabto{0cm}2014, SO JOURNAL OF BUSINESS RESEARCH\par
DT \tabto{0cm}Article\par
PG \tabto{0cm}6, NR 27, TC 3\par
DE \tabto{0cm}\hl{ABSORPTI}VE \hl{CAPACIT}Y, CULTURAL BARRIERS, INNOVATION, ORGANIZATIONAL CULTURE, PARTIAL LEAST SQUARES\par
ID \tabto{0cm}FIRMS, KNOWLEDGE, ORGANIZATIONAL INNOVATION\par
AB \tabto{0cm}The construct of \hl{absorpti}ve \hl{capacit}y has two dimensions: potential \hl{absorpti}ve \hl{capacit}y (PACAP) and realized \hl{absorpti}ve \hl{capacit}y (RACAP). This study addresses these two dimensions separately, and analyzes their influence on innovation outcomes (10) in organizations. The study also examines the mediating role of RACAP in the relationship between PACAP and IO. Furthermore, the paper contains a discussion on the moderating role of cultural barriers (CB) in decreasing the PACAP-RACAP link. Consequently, this study builds and tests a conditional process model. Data comes from a sample of 110 firms from the Spanish automotive components manufacturing sector. Results from variance-based structural equation modeling and the PROCESS tool show that RACAP fully mediates the influence of PACAP on IO, and that CB negatively conditions this indirect effect. This study provides evidence that when CB attains medium-to-high values, this indirect influence is not different from zero. (C) 2013 Elsevier Inc. All rights reserved.\par
\clearpage

\vspace*{-2cm}
Nb \tabto{0cm}251/327 (article\_id: 578)\par
TI \tabto{0cm}R\&D Cooperation in Regular Networks with Endogenous \hl{Absorpti}ve \hl{Capacit}y\par
AU \tabto{0cm}Correani, Garofalo, Pugliesi\par
PY \tabto{0cm}2014, SO REVIEW OF NETWORK ECONOMICS\par
DT \tabto{0cm}Article\par
PG \tabto{0cm}36, NR 55, TC 0\par
DE \tabto{0cm}\hl{ABSORPTI}VE \hl{CAPACIT}Y, OLIGOPOLY, R\&D NETWORK\par
ID \tabto{0cm}APPROPRIABILITY, DEVELOPMENT SPILLOVERS, FIRMS, INNOVATION, ME HALFWAY, OLIGOPOLY, OPTIMAL COGNITIVE DISTANCE, PERFORMANCE, RESEARCH JOINT VENTURES, STABLE NETWORKS\par
AB \tabto{0cm}In this paper we analyze how firms' R\&D investment decisions, firms' profits and social welfare are affected by \hl{absorpti}ve \hl{capacit}y; that is, the ability of a firm to learn from other collaborating firms. The model developed is a strategic regular network where firms have the opportunity to form pair-wise collaborative links with other firms and then compete a la Cournot. Different to the existing literature, we find that firms' R\&D efforts could increase or decrease with the degree of the network, depending on the level of \hl{absorpti}ve \hl{capacit}y, the market size and the network dimension. In particular, in the case of small market size and low learning effect, the connection between firms drives up research investments. Moreover, if \hl{absorpti}ve \hl{capacit}y is sufficiently low, the research collaboration between firms turns out not to be desirable from a private point of view while, in line with the existing literature, social efficiency requires a complete or intermediate level of collaborative activity. We also show that the complete network is pair-wise stable and socially optimal for an intermediate level of spillover intensity, while the empty network maximizes firms' profits when \hl{absorpti}ve \hl{capacit}y is small, yet it is not pair-wise stable.\par
\clearpage

\vspace*{-2cm}
Nb \tabto{0cm}252/327 (article\_id: 579)\par
TI \tabto{0cm}Technological variables and \hl{absorpti}ve \hl{capacit}y's influence on performance through corporate entrepreneurship\par
AU \tabto{0cm}Garcia-Morales, Bolivar-Ramos, Martin-Rojas\par
PY \tabto{0cm}2014, SO JOURNAL OF BUSINESS RESEARCH\par
DT \tabto{0cm}Article\par
PG \tabto{0cm}10, NR 68, TC 3\par
DE \tabto{0cm}\hl{ABSORPTI}VE \hl{CAPACIT}Y, CORPORATE ENTREPRENEURSHIP, ORGANIZATIONAL PERFORMANCE, TECHNOLOGICAL DISTINCTIVE COMPETENCIES, TECHNOLOGICAL SKILL, TOP MANAGEMENT SUPPORT\par
ID \tabto{0cm}CAPABILITIES, COMPETENCES, ECONOMY, ENVIRONMENT, FIRM PERFORMANCE, INFORMATION-TECHNOLOGY, INNOVATION, KNOWLEDGE MANAGEMENT, MODEL, SYSTEMS\par
AB \tabto{0cm}Technology and corporate entrepreneurship constitute an important source of competitive advantage for organizations, as they enable the development and exploitation of new opportunities. This study proposes a model to analyze the effects of top management support for technology on the promotion of technological skills, \hl{absorpti}ve \hl{capacit}y, and technological distinctive competencies. The research also considers the impact of technological skills and \hl{absorpti}ve \hl{capacit}y on the development of technological distinctive competencies, analyzing the influence of these variables on organizational performance through corporate entrepreneurship. The study tests these relationships empirically using 160 European technology firms. The paper ends with discussion of the findings and provides several theoretical and practical implications for future research. (C) 2013 Elsevier Inc. All rights reserved.\par
\clearpage

\vspace*{-2cm}
Nb \tabto{0cm}253/327 (article\_id: 580)\par
TI \tabto{0cm}From potential \hl{absorpti}ve \hl{capacit}y to innovation outcomes in project teams: The conditional mediating role of the realized \hl{absorpti}ve \hl{capacit}y in a relational learning context\par
AU \tabto{0cm}Leal-Rodriguez, Roldan, Ariza-Montes, Leal-Millan\par
PY \tabto{0cm}2014, SO INTERNATIONAL JOURNAL OF PROJECT MANAGEMENT\par
DT \tabto{0cm}Article\par
PG \tabto{0cm}14, NR 84, TC 6\par
DE \tabto{0cm}\hl{ABSORPTI}VE \hl{CAPACIT}Y, POTENTIAL \hl{ABSORPTI}VE \hl{CAPACIT}Y, PROJECT TEAMS, REALIZED \hl{ABSORPTI}VE \hl{CAPACIT}Y INNOVATION OUTCOMES, RELATIONAL LEARNING\par
ID \tabto{0cm}ANTECEDENTS, DYNAMIC THEORY, FIRM, KNOWLEDGE MANAGEMENT, MODELS, MULTIDIMENSIONAL CONSTRUCTS, ORGANIZATIONAL PERFORMANCE, SYSTEMS\par
AB \tabto{0cm}Starting from the construct \hl{absorpti}ve \hl{capacit}y, this study separately treats its two dimensions - potential \hl{absorpti}ve \hl{capacit}y (PACAP) and realized \hl{absorpti}ve \hl{capacit}y (RACAP) - and analyzes their influence on innovation outcomes (IO) in project teams. We also examine potential \hl{absorpti}ve \hl{capacit}y as an antecedent of realized \hl{absorpti}ve \hl{capacit}y. In addition, we propose that relational learning (RL) will play a moderator role reinforcing the PACAP and RACAP link. Consequently, this paper builds and tests a conditional process model. Data was collected from a sample of 110 project managers of firms belonging to the Spanish automotive components manufacturing sector. Results from variance-based structural equation modeling and PROCESS tool show that RACAP fully mediates the influence of the PACAP on IO, and this indirect effect is positively conditioned by RL. This paper provides evidence that when RL achieves a low value, this indirect influence is not different from zero. (C) 2014 Elsevier Ltd. APM and IPMA. All rights reserved.\par
\clearpage

\vspace*{-2cm}
Nb \tabto{0cm}254/327 (article\_id: 581)\par
TI \tabto{0cm}The moderating effect of human resource management practices on the relationship between knowledge \hl{absorpti}ve \hl{capacit}y and project performance in project-oriented companies\par
AU \tabto{0cm}Popaitoon, Siengthai\par
PY \tabto{0cm}2014, SO INTERNATIONAL JOURNAL OF PROJECT MANAGEMENT\par
DT \tabto{0cm}Article\par
PG \tabto{0cm}13, NR 77, TC 6\par
DE \tabto{0cm}HRM PRACTICES, KNOWLEDGE \hl{ABSORPTI}VE \hl{CAPACIT}Y, PROJECT PERFORMANCE, PROJECT-ORIENTED COMPANIES (POCS), PROJECT-TO-PROJECT\par
ID \tabto{0cm}CAPABILITIES, FIRM PERFORMANCE, HRM PRACTICES, INNOVATION PERFORMANCE, INTENSIVE FIRMS, ORGANIZATIONS, PRODUCT DEVELOPMENT, SUBSIDIARIES, SUCCESS FACTORS, SYSTEMS\par
AB \tabto{0cm}In response to recent calls for research on human resource management (HRM) in project management, this research investigates the links between HRM practices, the project team's knowledge \hl{absorpti}ve \hl{capacit}y (ACAP) and project performance in project-oriented companies (POCs). Based on survey data from 198 projects in multinational companies (MNCs) in the Thai automotive industry, this research finds that HRM practices moderate the effects of a project team's knowledge ACAP on project performance, in particular of potential ACAP on long-run project performance. In addition, HRM practices covary with a project team's realized ACAP, the other dimension of ACAP, to affect short-run project performance. This research sheds light on the different roles that HRM practices play in a project, finding that HRM practices not only facilitate knowledge management from the current project to future projects but also strengthen the relationship between a project team's knowledge ACAP and long-term project performance. This research contributes to the understanding of HRM in the literature of project management. (C) 2013 Elsevier Ltd. APM and IPMA. All rights reserved.\par
\clearpage

\vspace*{-2cm}
Nb \tabto{0cm}255/327 (article\_id: 582)\par
TI \tabto{0cm}Non-R\&D SMEs: external knowledge, \hl{absorpti}ve \hl{capacit}y and product innovation\par
AU \tabto{0cm}Moilanen, Ostbye, Woll\par
PY \tabto{0cm}2014, SO SMALL BUSINESS ECONOMICS\par
DT \tabto{0cm}Article\par
PG \tabto{0cm}16, NR 80, TC 1\par
DE \tabto{0cm}\hl{ABSORPTI}VE \hl{CAPACIT}Y, EXTERNAL KNOWLEDGE, NON-R\&D INNOVATORS, PERIPHERAL REGIONS\par
ID \tabto{0cm}CAPABILITIES, MANAGEMENT, MEDIUM-SIZED ENTERPRISES, NETWORKS, ORGANIZATIONS, PERFORMANCE, PUBLIC-POLICY, RECONCEPTUALIZATION, SMALL FIRMS, TECHNOLOGY\par
AB \tabto{0cm}The relationship between external knowledge, \hl{absorpti}ve \hl{capacit}y (AC) and innovative performance for small and medium-sized enterprises (SMEs) is investigated empirically. Using data from a survey on firms located in North Norway, we ask whether AC plays a mediating role between different external knowledge inflows and innovative performance. The results are consistent with AC as an important mediator for transforming external knowledge inflows into higher innovative performance if we include all SMEs in the sample. However, this result is not robust when considering the sub-sample of non-R\&D SMEs only. External knowledge inflows have a much stronger direct effect on innovation performance for non-R\&D firms and leave a weak mediating effect of AC. Our findings suggest that measures of AC should be developed further in order to make AC a more relevant concept for empirical studies of SMEs without in-house R\&D.\par
\clearpage

\vspace*{-2cm}
Nb \tabto{0cm}256/327 (article\_id: 583)\par
TI \tabto{0cm}SOCIAL CAPITAL AND LEARNING ADVANTAGES: A PROBLEM OF \hl{ABSORPTI}VE \hl{CAPACIT}Y\par
AU \tabto{0cm}Hughes, Morgan, Ireland\par
PY \tabto{0cm}2014, SO STRATEGIC ENTREPRENEURSHIP JOURNAL\par
DT \tabto{0cm}Article\par
PG \tabto{0cm}20, NR 54, TC 3\par
DE \tabto{0cm}\hl{ABSORPTI}VE \hl{CAPACIT}Y, ENTREPRENEURIAL FIRMS, LEARNING ADVANTAGES OF NEWNESS, NETWORK-BASED LEARNING, SOCIAL CAPITAL\par
ID \tabto{0cm}COMMON METHOD VARIANCE, COMPETITIVE ADVANTAGE, ENTREPRENEURIAL VENTURES, GROWTH, HETEROGENEITY, INNOVATION, NETWORKS, PERFORMANCE, STRATEGIC ALLIANCES, TECHNOLOGY-BASED FIRMS\par
AB \tabto{0cm}Theoretically, social capital allows entrepreneurial firms to capitalize on learning advantages of newness and gain access to knowledge as the foundation for improved performance. But this understates its complexity. We consider whether learning through social capital relationships has a direct effect on performance and whether \hl{absorpti}ve \hl{capacit}y mediates and moderates this relationship. We find that network-based learning has no direct relationship with performance, but this is mediated in each instance by \hl{absorpti}ve \hl{capacit}y and is moderated twice. Our findings challenge the learning advantages of newness thesis and reveal how \hl{absorpti}ve \hl{capacit}y can enable business performance from a firm's network relationships. Copyright (C) 2014 Strategic Management Society.\par
\clearpage

\vspace*{-2cm}
Nb \tabto{0cm}257/327 (article\_id: 584)\par
TI \tabto{0cm}Toward a choreography of congregating: A practice-based perspective on organizational \hl{absorpti}ve \hl{capacit}y in a semiconductor industry consortium\par
AU \tabto{0cm}Muller-Seitz, Guttel\par
PY \tabto{0cm}2014, SO MANAGEMENT LEARNING\par
DT \tabto{0cm}Article\par
PG \tabto{0cm}21, NR 53, TC 1\par
DE \tabto{0cm}\hl{ABSORPTI}VE \hl{CAPACIT}Y, CHOREOGRAPHY OF CONGREGATING, CONGREGATING, INTERORGANIZATIONAL NETWORK, PRACTICE, SEMICONDUCTOR INDUSTRY\par
ID \tabto{0cm}CONFERENCES, DESORPTIVE \hl{CAPACIT}Y, EXTENSION, FIELD, INNOVATION, INTERORGANIZATIONAL NETWORKS, KNOWLEDGE, ROUTINES, SEMATECH, TECHNOLOGIES\par
AB \tabto{0cm}Previous studies on \hl{absorpti}ve \hl{capacit}y focus either upon dyadic interorganizational constellations or upon an organization's environment in general, thus neglecting other managerially relevant interorganizational constellations. Furthermore, despite significant advancements in our understanding, we still know little about the way \hl{absorpti}ve \hl{capacit}y actually unfolds on an organizational level, as many studies predominantly take \hl{absorpti}ve \hl{capacit}y as a quantitatively measurable phenomenon. We address these two shortcomings by following recent calls for a practice perspective on \hl{absorpti}ve \hl{capacit}y. In particular, we reveal the choreography of knowledge \hl{absorpti}on practices in an interorganizational network. Based on an empirical study in the semiconductor industry, we put forward the thesis that this can be achieved by congregating, that is, repeatedly exchanging face-to-face ideas at network-wide venues such as conferences or workshops. We illustrate how a lead firm in this industry carefully choreographs congregating, which helps the organization acquire knowledge from network partners and utilize the knowledge internally. Herein, we contribute a practice perspective on how to influence organizational \hl{absorpti}ve \hl{capacit}y when engaging with an interorganizational network as a distinctive organizational form. A practice lens also entails being sensitive to the political dimension of \hl{absorpti}ve \hl{capacit}y. Moreover, choreographies of bundles of \hl{absorpti}ve \hl{capacit}y-relevant practices represent a concept to inform future research.\par
\clearpage

\vspace*{-2cm}
Nb \tabto{0cm}258/327 (article\_id: 585)\par
TI \tabto{0cm}Influence of the Adoption and Use of Social Media Tools on \hl{Absorpti}ve \hl{Capacit}y in New Product Development\par
AU \tabto{0cm}Peltola, Makinen\par
PY \tabto{0cm}2014, SO EMJ-ENGINEERING MANAGEMENT JOURNAL\par
DT \tabto{0cm}Article\par
PG \tabto{0cm}7, NR 42, TC 0\par
DE \tabto{0cm}\hl{ABSORPTI}VE \hl{CAPACIT}Y, NEW PRODUCT DEVELOPMENT, SOCIAL MEDIA\par
ID \tabto{0cm}CAPABILITIES, FUZZY FRONT-END, INNOVATION, KNOWLEDGE, MANAGEMENT, PERFORMANCE, SOFTWARE, TECHNOLOGY\par
AB \tabto{0cm}In this article, we discuss how the adoption and use of social media tools influences new product development (NPD). Our aim is specifically to consider how an organization's \hl{absorpti}ve \hl{capacit}y is influenced by using social media tools. Based on a two-phase data collection process comparing the situation before and after adoption of social media tools in three organizations, we conclude that the amount of accessible knowledge and the number of ideas increases as an organization's ability to find and access various sources of intra-organizational expertise increases, thus, increasing knowledge acquisition and assimilation. Consequently, we infer that organizational \hl{absorpti}ve \hl{capacit}y rises and is linked to improvement in new product development.\par
\clearpage

\vspace*{-2cm}
Nb \tabto{0cm}259/327 (article\_id: 586)\par
TI \tabto{0cm}Firm Heterogeneity in Biotech: \hl{Absorpti}ve \hl{Capacit}y, Strategies and Local-Regional Connections\par
AU \tabto{0cm}Bagchi-Sen, Smith\par
PY \tabto{0cm}2014, SO EUROPEAN PLANNING STUDIES\par
DT \tabto{0cm}Article\par
PG \tabto{0cm}19, NR 52, TC 0\par
DE \tabto{0cm}null\par
ID \tabto{0cm}ALLIANCES, BIOSCIENCE MEGACENTRES, CAPABILITIES, COEVOLUTION, EMPIRICAL-ANALYSIS, INDUSTRY, INNOVATION, KNOWLEDGE, RESEARCH-AND-DEVELOPMENT, SCIENCE\par
AB \tabto{0cm}This paper focuses on the characteristics of biotech firms with various levels of research and development (R\&D) activity. It is done by exploring the relationship between R\&D intensity, alliances and the extent of regionalization of firms' activities using evidence from a survey of US-based biotechnology firms. We profile two firm prototypes: research-oriented firms and productoriented firms, focusing on their characteristics, strategies and operations. These include activities devoted to exploration and exploitation through alliances with universities (more exploration) and with pharmaceutical companies (exploration and exploitation), and locational needs which facilitate both exploration and exploitation.\par
\clearpage

\vspace*{-2cm}
Nb \tabto{0cm}260/327 (article\_id: 587)\par
TI \tabto{0cm}Sources of Variation in the Efficiency of Adopting Management Innovation: The Role of \hl{Absorpti}ve \hl{Capacit}y Routines, Managerial Attention and Organizational Legitimacy\par
AU \tabto{0cm}Peeters, Massini, Lewin\par
PY \tabto{0cm}2014, SO ORGANIZATION STUDIES\par
DT \tabto{0cm}Article\par
PG \tabto{0cm}29, NR 87, TC 2\par
DE \tabto{0cm}\hl{ABSORPTI}VE \hl{CAPACIT}Y ROUTINES, GLOBAL SOURCING OF BUSINESS SERVICES, MANAGEMENT INNOVATION, MANAGERIAL ATTENTION, ORGANIZATIONAL LEGITIMACY, PROCESS EFFICIENCY\par
ID \tabto{0cm}HIGH-VELOCITY ENVIRONMENTS, INTERNATIONAL-BUSINESS, KNOWLEDGE TRANSFER, LARGE FIRMS, PERFORMANCE, PERSPECTIVE, PROCESS MODEL, PRODUCT DEVELOPMENT, TECHNOLOGICAL-INNOVATION, TOTAL QUALITY MANAGEMENT\par
AB \tabto{0cm}Drawing on two in-depth case studies, this paper develops a conceptual model of how \hl{absorpti}ve \hl{capacit}y routines and their underlying processes of evolution influence the efficiency of management innovation adaptation processes. The model highlights three important relations. First, although different configurations of \hl{absorpti}ve \hl{capacit}y routines can lead to the successful implementation of the same management innovation - namely the reconfiguration of firms' value chains through sourcing of business services from offshore countries - the sequence of developing routines, their adequacy, and the interdependencies fit between routines partly explain how rapidly and seamlessly a firm is able to implement a management innovation. Second, we identify managerial attention and organizational legitimacy as two critical and interrelated sources of variation of the efficiency in the process of adopting and adapting management innovations. Finally, attention direction by a top-level internal change agent is more effective than local problemistic search to foster managerial attention and organizational legitimacy to both the management innovation to be adopted, and the need to develop and put into practice an appropriate set of \hl{absorpti}ve \hl{capacit}y routines.\par
\clearpage

\vspace*{-2cm}
Nb \tabto{0cm}261/327 (article\_id: 588)\par
TI \tabto{0cm}Entrepreneurial orientation in turbulent environments: The moderating role of \hl{absorpti}ve \hl{capacit}y\par
AU \tabto{0cm}Engelen, Kube, Schmidt, Flatten\par
PY \tabto{0cm}2014, SO RESEARCH POLICY\par
DT \tabto{0cm}Article\par
PG \tabto{0cm}17, NR 117, TC 0\par
DE \tabto{0cm}\hl{ABSORPTI}VE \hl{CAPACIT}Y, ENTREPRENEURIAL ORIENTATION, SURVEY RESEARCH\par
ID \tabto{0cm}BUSINESS PERFORMANCE, COMPETITIVE ADVANTAGE, CORPORATE ENTREPRENEURSHIP, DYNAMIC CAPABILITIES, FIRM PERFORMANCE, KNOWLEDGE TRANSFER, MARKET ORIENTATION, RESEARCH-AND-DEVELOPMENT, STRATEGIC MANAGEMENT, VENTURE PERFORMANCE\par
AB \tabto{0cm}The literature on entrepreneurial orientation (EO) has confirmed the positive relationship between EO and firm performance and that relationship's dependence on several contingencies. The present study connects the resource-based view and its dynamic capability extension to introduce \hl{absorpti}ve \hl{capacit}y (ACAP) as a moderator of the relationship between EO and firm performance. This theoretically derived research model is empirically validated using survey data from 219 small and medium-sized enterprises in Germany. Our empirical findings are that ACAP strengthens the EO-performance relationship in turbulent markets. (C) 2014 Elsevier B.V. All rights reserved.\par
\clearpage

\vspace*{-2cm}
Nb \tabto{0cm}262/327 (article\_id: 589)\par
TI \tabto{0cm}Entrepreneurial Orientation in low- and medium-tech industries: The need for \hl{Absorpti}ve \hl{Capacit}y to increase performance\par
AU \tabto{0cm}Sciascia, D'Oria, Bruni, Larraneta\par
PY \tabto{0cm}2014, SO EUROPEAN MANAGEMENT JOURNAL\par
DT \tabto{0cm}Article\par
PG \tabto{0cm}9, NR 87, TC 0\par
DE \tabto{0cm}\hl{ABSORPTI}VE \hl{CAPACIT}Y, ENTREPRENEURIAL ORIENTATION, LMT INDUSTRIES, PERFORMANCE\par
ID \tabto{0cm}BUSINESS PERFORMANCE, COMBINATIVE CAPABILITIES, EMPIRICAL-ANALYSIS, FAMILY FIRMS, FIRM PERFORMANCE, INTERNATIONAL-JOINT-VENTURES, KNOWLEDGE TRANSFER, MEDIUM-TECHNOLOGY INDUSTRIES, MODERATING ROLE, ORGANIZATIONAL FORMS\par
AB \tabto{0cm}The implications of Entrepreneurial Orientation (EO) for firm performance in low- and medium-tech (LMT) industries are largely unexplored and seem to be limited. In this paper we seek to address this research gap studying how \hl{Absorpti}ve \hl{Capacit}y can act as a key factor determining the effectiveness of EO in such a context. Specifically, we adopt the knowledge-based view of the firm and explore the moderating effects of \hl{Absorpti}ve \hl{Capacit}y's Potential and Realized dimensions on the EO-performance relationship in LMT industries. Our regression results based on a lagged dataset of 103 medium-sized firms based in Italy confirm our hypotheses that, in LMT industries, EO has a positive effect on firm performance when coupled with high levels of both Potential and Realized \hl{Absorpti}ve \hl{Capacit}y. (C) 2014 Elsevier Ltd. All rights reserved.\par
\clearpage

\vspace*{-2cm}
Nb \tabto{0cm}263/327 (article\_id: 590)\par
TI \tabto{0cm}Knowing, Power and Materiality: A Critical Review and Reconceptualization of \hl{Absorpti}ve \hl{Capacit}y\par
AU \tabto{0cm}Marabelli, Newell\par
PY \tabto{0cm}2014, SO INTERNATIONAL JOURNAL OF MANAGEMENT REVIEWS\par
DT \tabto{0cm}Review\par
PG \tabto{0cm}21, NR 117, TC 1\par
DE \tabto{0cm}null\par
ID \tabto{0cm}BOUNDARY OBJECTS, DECISION-MAKING, INFORMATION-SYSTEMS, INNOVATION, KNOWLEDGE TRANSFER, NETWORK, ORGANIZATIONAL KNOWLEDGE, PERFORMANCE, PERSPECTIVE, TRANSLATION\par
AB \tabto{0cm}Most studies on \hl{absorpti}ve \hl{capacit}y (AC) are based on assumptions that are characteristic of viewing knowledge from an epistemology of possession (knowledge is possessed by individuals and is transferrable). However, the literature on managing knowledge (or better knowledge work) acknowledges also an epistemology of practice (knowledge is unpredictable and dynamic and constituted in and through practice). Moreover, the literature on AC is relatively silent on the relationship between knowledge and power. In this paper, the authors argue that the AC construct should be interpreted in light of the possession and the practice perspectives of knowledge and power. The analysis includes a systematic literature review of AC that supports the authors' claims and, based on this, they suggest an interpretation of the construct that takes into account knowledge-power relationships. This review and theorizing contribute to a richer and processual view of AC.\par
\clearpage

\vspace*{-2cm}
Nb \tabto{0cm}264/327 (article\_id: 591)\par
TI \tabto{0cm}Multinational Enterprises, \hl{Absorpti}ve \hl{Capacit}y and Export Spillovers: Evidence from Polish Firm-level Data\par
AU \tabto{0cm}Cieslik, Hagemejer\par
PY \tabto{0cm}2014, SO REVIEW OF DEVELOPMENT ECONOMICS\par
DT \tabto{0cm}Article\par
PG \tabto{0cm}18, NR 48, TC 0\par
DE \tabto{0cm}null\par
ID \tabto{0cm}BENEFIT, COMPETITION, DIRECT FOREIGN-INVESTMENT, DOMESTIC FIRMS, ECONOMIES, FDI, PANEL-DATA, PRODUCTIVITY, TECHNOLOGY, TRADE\par
AB \tabto{0cm}An important benefit attributed to the activity of multinational enterprises (MNEs) in developing and transition countries is its effect on international market access. Through a variety of channels the presence of MNEs is expected to reduce the costs faced by indigenous firms in breaking into international markets and in turn boost their export prospects. In this paper we use an extensive Polish firm-level dataset for the period 2000-2008 to verify whether MNEs have positively contributed to the export performance of indigenous firms. We track not only sectoral and geographical spillovers stemming from the activity of MNEs but also control for firm-specific characteristics that affect indigenous firms' decisions to export including their \hl{absorpti}ve \hl{capacit}y. Our empirical results support the existence of positive spillovers (related to MNE export activity) at the sectoral level but not at the regional level. Finally, we find that individual \hl{absorpti}ve \hl{capacit}y determines the size of export spillovers.\par
\clearpage

\vspace*{-2cm}
Nb \tabto{0cm}265/327 (article\_id: 592)\par
TI \tabto{0cm}Determinants of \hl{absorpti}ve \hl{capacit}y: contrasting manufacturing vs services enterprises\par
AU \tabto{0cm}Chang, Chen, Lin\par
PY \tabto{0cm}2014, SO R \& D MANAGEMENT\par
DT \tabto{0cm}Article\par
PG \tabto{0cm}18, NR 81, TC 0\par
DE \tabto{0cm}null\par
ID \tabto{0cm}COMPETITIVE ADVANTAGE, DYNAMIC CAPABILITIES, HIGH-TECH FIRMS, INNOVATION PERFORMANCE, PRODUCT DEVELOPMENT, RESEARCH-AND-DEVELOPMENT, RESOURCE-BASED VIEW, STRATEGIC FLEXIBILITY, TAIWAN, TECHNOLOGY FIRMS\par
AB \tabto{0cm}The purpose of this study is to explore the determinants and the consequent of \hl{absorpti}ve \hl{capacit}y from the resource-capability-performance framework in the Taiwanese manufacturing and service industries. Structural equation modeling is applied to verify the conceptual model. In both of the Taiwanese manufacturing industry and the Taiwanese service industry, this study verifies that resource commitment and resource flexibility are two antecedents of \hl{absorpti}ve \hl{capacit}y, and that new product development performance or service innovation performance is its consequent. Besides, the empirical results show that \hl{absorpti}ve \hl{capacit}y plays a partial mediator in the manufacturing industry but a full mediator in the service industry. If companies would like to raise their new product development performance or service innovation performance, they have to enhance their resource commitment, resource flexibility, and \hl{absorpti}ve \hl{capacit}y. Moreover, this study finds that the resource commitment of small and medium enterprises is significantly less than that of large enterprises in Taiwanese manufacturing industry. This study also finds that resource flexibility of established companies is significantly higher than that of younger companies in Taiwan.\par
\clearpage

\vspace*{-2cm}
Nb \tabto{0cm}266/327 (article\_id: 593)\par
TI \tabto{0cm}Start-up \hl{absorpti}ve \hl{capacit}y: Does the owner's human and social capital matter?\par
AU \tabto{0cm}Debrulle, Maes, Sels\par
PY \tabto{0cm}2014, SO INTERNATIONAL SMALL BUSINESS JOURNAL\par
DT \tabto{0cm}Review\par
PG \tabto{0cm}25, NR 105, TC 0\par
DE \tabto{0cm}BUSINESS OWNER, ENVIRONMENTAL TURBULENCE, HUMAN CAPITAL, ORGANISATIONAL \hl{ABSORPTI}VE \hl{CAPACIT}Y, SOCIAL CAPITAL, START-UP\par
ID \tabto{0cm}COMPETITIVE ADVANTAGE, CORPORATE ENTREPRENEURSHIP, ENTREPRENEURIAL ORIENTATION, FINANCIAL PERFORMANCE, KNOWLEDGE-ACQUISITION, MEDIUM-SIZED FIRMS, NASCENT ENTREPRENEURS, ORGANIZATIONAL ANTECEDENTS, TECHNOLOGY-BASED FIRMS, VENTURE PERFORMANCE\par
AB \tabto{0cm}This study investigates how a business owner's human and social capital affects start-up \hl{absorpti}ve \hl{capacit}y under different environmental conditions. From an analysis of a sample of 199 Flemish start-ups, the study observes that the owners' start-up experience and bridging social capital are positively and significantly related to the new venture's ability to acquire, assimilate and exploit external information. In addition, the findings reveal a positive but decreasing effect of owner-specific human capital as a function of environmental turbulence. Furthermore, the study finds that management experience significantly stimulates start-up \hl{absorpti}ve \hl{capacit}y within highly dynamic environments, whereas it hinders it within stable environments. Finally, implications of the study and opportunities for future research are provided.\par
\clearpage

\vspace*{-2cm}
Nb \tabto{0cm}267/327 (article\_id: 594)\par
TI \tabto{0cm}The Determinants of Green Radical and Incremental Innovation Performance: Green Shared Vision, Green \hl{Absorpti}ve \hl{Capacit}y, and Green Organizational Ambidexterity\par
AU \tabto{0cm}Chen, Chang, Lin\par
PY \tabto{0cm}2014, SO SUSTAINABILITY\par
DT \tabto{0cm}Article\par
PG \tabto{0cm}20, NR 65, TC 2\par
DE \tabto{0cm}GREEN \hl{ABSORPTI}VE \hl{CAPACIT}Y, GREEN EXPLOITATION LEARNING, GREEN EXPLORATION LEARNING, GREEN INCREMENTAL INNOVATION, GREEN INNOVATION, GREEN ORGANIZATIONAL AMBIDEXTERITY, GREEN RADICAL INNOVATION, GREEN SHARED VISION\par
ID \tabto{0cm}ANTECEDENTS, CAPABILITIES, COMPETITIVE ADVANTAGES, EQUATION MODELING SEM, EXPLOITATION, EXPLORATION, FIRM, MODERATING ROLE, PERCEIVED RISK, TECHNOLOGY\par
AB \tabto{0cm}This study proposes a new concept, green organisational ambidexterity, that integrates green exploration learning and green exploitation learning simultaneously. Besides, this study argues that the antecedents of green organisational ambidexterity are green shared vision and green \hl{absorpti}ve \hl{capacit}y and its consequents are green radical innovation performance and green incremental innovation performance. The results demonstrate that green exploration learning partially mediates the positive relationships between green radical innovation performance and its two antecedents-green shared vision and green \hl{absorpti}ve \hl{capacit}y. In addition, this study indicates that green exploitation learning partially mediates the positive relationships between green incremental innovation performance and its two antecedents-green shared vision and green \hl{absorpti}ve \hl{capacit}y. Hence, firms have to increase their green shared vision, green \hl{absorpti}ve \hl{capacit}y, and green organisational ambidexterity to raise their green radical innovation performance and green incremental innovation performance.\par
\clearpage

\vspace*{-2cm}
Nb \tabto{0cm}268/327 (article\_id: 595)\par
TI \tabto{0cm}\hl{ABSORPTI}VE \hl{CAPACIT}Y AS A PRECONDITION FOR BUSINESS PROCESS IMPROVEMENT\par
AU \tabto{0cm}Manfreda, Kovacic, Stemberger, Trkman\par
PY \tabto{0cm}2014, SO JOURNAL OF COMPUTER INFORMATION SYSTEMS\par
DT \tabto{0cm}Article\par
PG \tabto{0cm}9, NR 102, TC 1\par
DE \tabto{0cm}\hl{ABSORPTI}VE \hl{CAPACIT}Y, BUSINESS PROCESS MANAGEMENT, HEALTHCARE, HEALTHCARE INSURANCE COMPANY, MODELING\par
ID \tabto{0cm}FRAMEWORK, HEALTH-CARE COSTS, INFORMATION-SYSTEMS, INNOVATION, INTEGRATION, MANAGEMENT, ORGANIZATIONS, PERFORMANCE, PERSPECTIVE, SUPPLY CHAINS\par
AB \tabto{0cm}Improving organizational performance by redesigning business processes and supporting them with a proper information system is a daunting challenge. We analyze the possibilities of business process management in general and in the healthcare sector in particular. The role of business process modeling as a way to increase an organization's \hl{absorpti}ve \hl{capacit}y is analyzed. A longitudinal case study of a European public healthcare insurance company identifies the factors either increasing or hindering \hl{absorpti}ve \hl{capacit}y. The case presents that the dilemma between radical and incremental approach to improve business processes is somehow artificial since the radicalness of changes depends on the difference between the \hl{absorpti}ve \hl{capacit}y and the extent of the proposed changes. The paper shows that business process management projects should not merely focus on the development of methodologically correct models, but should be used as an opportunity to increase the \hl{absorpti}ve \hl{capacit}y of an organization.\par
\clearpage

\vspace*{-2cm}
Nb \tabto{0cm}269/327 (article\_id: 596)\par
TI \tabto{0cm}Innovation in tourism: Re-conceptualising and measuring the \hl{absorpti}ve \hl{capacit}y of the hotel sector\par
AU \tabto{0cm}Thomas, Wood\par
PY \tabto{0cm}2014, SO TOURISM MANAGEMENT\par
DT \tabto{0cm}Article\par
PG \tabto{0cm}10, NR 80, TC 3\par
DE \tabto{0cm}\hl{ABSORPTI}VE \hl{CAPACIT}Y, HOTELS, INNOVATION, INNOVATION POLICY, KNOWLEDGE MANAGEMENT\par
ID \tabto{0cm}CONFIGURATION, DYNAMIC CAPABILITIES, FIRMS, INDUSTRY, KNOWLEDGE TRANSFER, ORGANIZATIONAL ROUTINES, PERFORMANCE, PERSPECTIVE, RECONCEPTUALIZATION, STRATEGY\par
AB \tabto{0cm}Recent reviews of research on innovation in tourism have highlighted a number of weaknesses in the literature. Among these is the limited theorising and empirical investigation of innovative practices by tourism organisations. This paper responds to these concerns by examining one important dimension of innovation within commercial tourism organisations, namely their ability to acquire, assimilate and utilise external knowledge (\hl{absorpti}ve \hl{capacit}y) for competitive advantage. The topic is pertinent because there is evidence to suggest that tourism organisations are particularly dependent on external sources of knowledge when compared with businesses in other sectors. Following a discussion of the conceptual antecedents of \hl{absorpti}ve \hl{capacit}y and its dimensions, a validated instrument for its measurement is developed and used to measure the \hl{absorpti}ve \hl{capacit}y of the British hotel sector. The results suggest that current conceptions of \hl{absorpti}ve \hl{capacit}y have limitations when applied to tourism enterprises. \hl{Absorpti}ve \hl{capacit}y is re-conceptualised to overcome these deficiencies. The research and policy implications of the findings are discussed. (C) 2014 Elsevier Ltd. All rights reserved.\par
\clearpage

\vspace*{-2cm}
Nb \tabto{0cm}270/327 (article\_id: 597)\par
TI \tabto{0cm}Corporate entrepreneurship, customer-oriented selling, \hl{absorpti}ve \hl{capacit}y, and international sales performance in the international B2B setting: Conceptual framework and research propositions\par
AU \tabto{0cm}Javalgi, Hall, Cavusgil\par
PY \tabto{0cm}2014, SO INTERNATIONAL BUSINESS REVIEW\par
DT \tabto{0cm}Article\par
PG \tabto{0cm}10, NR 106, TC 0\par
DE \tabto{0cm}\hl{ABSORPTI}VE \hl{CAPACIT}Y, B2B MARKETING, CORPORATE ENTREPRENEURSHIP, CULTURAL DISTANCE, CUSTOMER-ORIENTED SELLING, INTERNATIONAL B2B SALES, KNOWLEDGE-BASED VIEW\par
ID \tabto{0cm}BEHAVIOR, BUSINESS, JAPANESE FIRMS, KNOWLEDGE, MANAGEMENT, MARKET ORIENTATION, MODEL, ORGANIZATIONAL CULTURE, SALESPERSON PERFORMANCE, SUSTAINED COMPETITIVE ADVANTAGE\par
AB \tabto{0cm}In the international business-to-business (B2B) setting, a firm's salespeople often have more direct, prolonged, and intimate contact with the customer and market environments than any other employees of the firm. In fact, for customers in many B2B markets, the salesperson is the face of the firm. The sales function can be characterized as an inherently entrepreneurial activity. Entrepreneurship is founded on knowing or seeing something others do not see, and the sales force has long been recognized as an important source of knowledge about a firm's customers and environment. However, there has been relatively little work linking entrepreneurship to international sales performance, especially in the B2B context.
This paper focuses on the intelligence-gathering role of salespeople to firms practicing corporate entrepreneurship in the international B2B setting. More specifically, drawing on the theories of corporate entrepreneurship and the knowledge-based view of the firm, the authors develop a conceptual model that proposes international sales performance for firms practicing corporate entrepreneurship will be enhanced when salespeople practice customer-oriented selling and the firm's \hl{absorpti}ve \hl{capacit}y is stronger. Recommended methodology for testing the proposed model is a single-informant survey of sales managers with firms in the domain of interest, using structural equation modeling with moderator tests. The paper concludes with implications and directions for future research. (C) 2014 Published by Elsevier Ltd.\par
\clearpage

\vspace*{-2cm}
Nb \tabto{0cm}271/327 (article\_id: 598)\par
TI \tabto{0cm}Developing relationship-specific memory and \hl{absorpti}ve \hl{capacit}y in interorganizational relationships\par
AU \tabto{0cm}Choi\par
PY \tabto{0cm}2014, SO INFORMATION TECHNOLOGY \& MANAGEMENT\par
DT \tabto{0cm}Article\par
PG \tabto{0cm}16, NR 52, TC 0\par
DE \tabto{0cm}\hl{ABSORPTI}VE \hl{CAPACIT}Y, INTERORGANIZATIONAL RELATIONSHIP, IT RESOURCES, IT-ENABLED CAPABILITY, KNOWLEDGE MANAGEMENT, RELATIONSHIP-SPECIFIC MEMORY\par
ID \tabto{0cm}CAPABILITIES, ELECTRONIC KNOWLEDGE REPOSITORIES, FIRM PERFORMANCE, FUTURE-RESEARCH, INFORMATION-TECHNOLOGY INFRASTRUCTURE, ORGANIZATIONAL MEMORY, PRODUCT DEVELOPMENT, SUPPLY CHAIN RELATIONSHIPS, SUSTAINED COMPETITIVE ADVANTAGE, SYSTEMS RESEARCH\par
AB \tabto{0cm}Drawing on the studies of relationship-specific memory and \hl{absorpti}ve \hl{capacit}y, this study examines whether physical and human IT resources deployed in interorganizational relationships influence the development of a firm's IT-enabled capabilities, namely relationship-specific memory and \hl{absorpti}ve \hl{capacit}y. In addition, the study explores whether these capabilities increase firm performance and also examines the relationship between relationship-specific memory and \hl{absorpti}ve \hl{capacit}y. To test the hypotheses, we conducted a partial least squares analysis using data collected from 115 firms. The results demonstrate that firms enhanced their relationship-specific memory and \hl{absorpti}ve \hl{capacit}y by leveraging their physical and human IT resources invested in interorganizational relationships and that these two capabilities increased their performance. Moreover, our results indicate that relationship-specific memory served as a knowledge base for the development of \hl{absorpti}ve \hl{capacit}y. The results offer empirical evidence on how firms could improve their performance by internally managing the relational knowledge obtained through their interorganizational relationships.\par
\clearpage

\vspace*{-2cm}
Nb \tabto{0cm}272/327 (article\_id: 599)\par
TI \tabto{0cm}FDI spillovers on firm survival in Italy: \hl{absorpti}ve \hl{capacit}y matters!\par
AU \tabto{0cm}Ferragina, Mazzotta\par
PY \tabto{0cm}2014, SO JOURNAL OF TECHNOLOGY TRANSFER\par
DT \tabto{0cm}Article\par
PG \tabto{0cm}39, NR 77, TC 1\par
DE \tabto{0cm}\hl{ABSORPTI}VE \hl{CAPACIT}Y, BACKWARD AND FORWARD LINKAGES, BUSINESS DYNAMICS, DURATION ANALYSIS, FIRM PERFORMANCE, INTERNATIONAL INVESTMENT, INTERNATIONAL LINKAGES TO DEVELOPMENT, ITALIAN PRODUCTION SYSTEM, MULTINATIONAL FIRMS, PRODUCTIVITY SPILLOVERS\par
ID \tabto{0cm}DOMESTIC FIRMS, FOREIGN DIRECT-INVESTMENT, LINKAGES, MANUFACTURING SECTOR, MULTINATIONAL COMPANIES, PLANT-SURVIVAL, PRODUCTIVITY SPILLOVERS, R-AND-D, STATE-OWNED ENTERPRISES, TECHNOLOGY-TRANSFER\par
AB \tabto{0cm}The aim of this paper is to explore the effects of spillovers driven by competition and forward and backward linkages between foreign firms and Italian firms. We adopt the firm dynamics framework, which allows us to test the impact of foreign firms' activity on the probability that local firms will exit. The empirical analysis relies on continuous survival models (Cox proportional hazard models) and uses a representative firm level database from the period of 2002-2010 with data concerning more than 4,000 Italian manufacturing firms. Our estimates regarding the whole sample show that horizontal and vertical linkages have no impact on firm survival. To further test this finding, we perform a more disaggregated analysis that allows for heterogeneity across firms and sectors. We obtain evidence that the effects of FDI spillovers on firm survival follow specific patterns at both the intra- and inter-industry levels based on differences in productivity between Italian firms and foreign firms and on the technological intensity of the industry. Foreign firms' activity reduces the exit probability of competitors and of downstream local customers (through forward linkages) with low productivity gap but has no impact on high productivity gap firms. Firms in high technology intensive sectors do not benefit from horizontal FDI while in low and medium technology sectors they do. Differences in \hl{absorpti}ve \hl{capacit}y may explain these results. However, we also find that vertical linkages with foreign firms in the upstream supplying industries spur firm duration in medium and high tech sectors.\par
\clearpage

\vspace*{-2cm}
Nb \tabto{0cm}273/327 (article\_id: 600)\par
TI \tabto{0cm}Systems thinking and \hl{absorpti}ve \hl{capacit}y in high-tech small and medium-sized enterprises from South Korea\par
AU \tabto{0cm}Kim, Akbar, Tzokas, Al-Dajani\par
PY \tabto{0cm}2014, SO INTERNATIONAL SMALL BUSINESS JOURNAL\par
DT \tabto{0cm}Review\par
PG \tabto{0cm}21, NR 130, TC 0\par
DE \tabto{0cm}\hl{ABSORPTI}VE \hl{CAPACIT}Y, HIGH-TECH SMES ORGANISATIONAL PERFORMANCE, SOUTH KOREA, SYSTEMS THINKING\par
ID \tabto{0cm}BUSINESS PERFORMANCE, COMPETITIVE ADVANTAGE, DYNAMIC CAPABILITY, ENTREPRENEURIAL ORIENTATION, INNOVATION PERFORMANCE, KNOWLEDGE CREATION, LEARNING CAPABILITY, MARKET ORIENTATION, ORGANIZATIONAL ANTECEDENTS, RESEARCH-AND-DEVELOPMENT\par
AB \tabto{0cm}This article develops and tests \hl{absorpti}ve \hl{capacit}y's relationship with one of its important forerunners - systems thinking - which, although postulated as an important element, has received little empirical attention in the \hl{absorpti}ve \hl{capacit}y literature. Our contribution lies in the introduction of unique pathways through which systems thinking influences \hl{absorpti}ve \hl{capacit}y and how it affects various interrelated dimensions of high-tech small and medium-sized enterprises' performance, by examining evidence from South Korea's semiconductor industry.\par
\clearpage

\vspace*{-2cm}
Nb \tabto{0cm}274/327 (article\_id: 601)\par
TI \tabto{0cm}Realising potential: The impact of business incubation on the \hl{absorpti}ve \hl{capacit}y of new technology-based firms\par
AU \tabto{0cm}Patton\par
PY \tabto{0cm}2014, SO INTERNATIONAL SMALL BUSINESS JOURNAL\par
DT \tabto{0cm}Article\par
PG \tabto{0cm}21, NR 72, TC 5\par
DE \tabto{0cm}\hl{ABSORPTI}VE \hl{CAPACIT}Y, BUSINESS SUPPORT, INCUBATION, MANAGERIAL KNOWLEDGE, NEW TECHNOLOGY FIRMS\par
ID \tabto{0cm}ENTREPRENEURSHIP, EVOLUTION, KNOWLEDGE TRANSFER, MANAGEMENT, OPEN INNOVATION, OPERATIONS, ORIENTATION, PERFORMANCE, PERSPECTIVE, RECONCEPTUALIZATION\par
AB \tabto{0cm}This article explores the potential of university technology business incubators to enhance the \hl{absorpti}ve \hl{capacit}y of new technology-based firms. The research pursues three critical themes: it employs the \hl{absorpti}ve \hl{capacit}y construct to analyse and evaluate the potential of incubation to strengthen the business model of new technology firms. It then explores the interaction between founders and incubator directors, mentors and business advisers to assess how this might enhance \hl{absorpti}ve \hl{capacit}y. Finally, it indicates how such interactions can facilitate the transition from potential to realised \hl{absorpti}ve \hl{capacit}y. The article interrogates the incubation process by using the \hl{absorpti}ve \hl{capacit}y framework to evaluate how it might strengthen the business model of new technology firms. The qualitative findings suggest that where founders, advisers, mentors and incubator directors engage collaboratively to create an iterative dialogue which informs the development of a viable business model, the process by which potential \hl{absorpti}ve \hl{capacit}y can be fully realised is substantially strengthened.\par
\clearpage

\vspace*{-2cm}
Nb \tabto{0cm}275/327 (article\_id: 602)\par
TI \tabto{0cm}\hl{Absorpti}ve \hl{capacit}y and autonomous R\&D climate roles in firm innovation\par
AU \tabto{0cm}Huang, Lin, Wu, Yu\par
PY \tabto{0cm}2015, SO JOURNAL OF BUSINESS RESEARCH\par
DT \tabto{0cm}Article\par
PG \tabto{0cm}8, NR 65, TC 2\par
DE \tabto{0cm}\hl{ABSORPTI}VE \hl{CAPACIT}Y, INNOVATION, R\&D AUTONOMY, R\&D CLIMATE, R\&D HUMAN CAPITAL\par
ID \tabto{0cm}CANADIAN BIOTECHNOLOGY, COMPETITIVE ADVANTAGE, ENVIRONMENT, INTERNAL CAPABILITIES, JOINT VENTURES, KNOWLEDGE, NETWORKS, ORGANIZATIONAL FORMS, PERFORMANCE, RECONCEPTUALIZATION\par
AB \tabto{0cm}\hl{Absorpti}ve \hl{capacit}y is frequently an outcome of a firm's cumulatively path-dependent R\&D investments. However, the query how \hl{absorpti}ve \hl{capacit}y transforms R\&D investment into firm innovation, in the context of autonomous R\&D climate remains unclear. Using 165 firms in the Taiwan's information and communication technology industry, the results indicate that \hl{absorpti}ve \hl{capacit}y partially mediates the relationship between R\&D investment and firm innovation. \hl{Absorpti}ve \hl{capacit}y accounts for 36\% effects of R\&D investment on firm innovation. The result also shows a negative moderating effect of R\&D autonomy on the relationship between \hl{absorpti}ve \hl{capacit}y and firm innovation. (C) 2014 Elsevier Inc. All rights reserved.\par
\clearpage

\vspace*{-2cm}
Nb \tabto{0cm}276/327 (article\_id: 603)\par
TI \tabto{0cm}Extending organizational antecedents of \hl{absorpti}ve \hl{capacit}y: Organizational characteristics that encourage experimentation\par
AU \tabto{0cm}Burcharth, Lettl, Ulhoi\par
PY \tabto{0cm}2015, SO TECHNOLOGICAL FORECASTING AND SOCIAL CHANGE\par
DT \tabto{0cm}Article\par
PG \tabto{0cm}16, NR 94, TC 2\par
DE \tabto{0cm}\hl{ABSORPTI}VE \hl{CAPACIT}Y, EXPERIMENTATION, INNOVATION, ORGANIZATIONAL ANTECEDENTS, SMES\par
ID \tabto{0cm}COMBINATIVE CAPABILITIES, EXTERNAL KNOWLEDGE, INNOVATION PERFORMANCE, KNOWLEDGE-BASED THEORY, LOCAL SEARCH, MANUFACTURING FIRMS, RADICAL INNOVATION, REGIONAL NETWORKS, RESEARCH-AND-DEVELOPMENT, STRATEGIC ALLIANCES\par
AB \tabto{0cm}\hl{Absorpti}ve \hl{capacit}y has generally been perceived as a 'passive' outcome of R\&D investments. Recently, however, a renewed debate on its 'proactive' dimensions has emerged. We tap into this development and complement the existing discussion on combinative capabilities with a perspective that focuses on organizational characteristics that encourage experimentation. Specifically, we argue that characteristics such as slack resources, tolerance for failure, willingness to cannibalize and external openness are important organizational antecedents for knowledge \hl{absorpti}on activities as they prevent inertia. Drawing on multi-informant survey data collected from SMEs in Denmark (n = 169), we find empirical support for the impact of these characteristics (except for tolerance for failure) on various aspects of \hl{absorpti}ve \hl{capacit}y (both potential and realized). Before concluding, we discuss the theoretical and managerial implications of our study. (C) 2014 Elsevier Inc. All rights reserved.\par
\clearpage

\vspace*{-2cm}
Nb \tabto{0cm}277/327 (article\_id: 604)\par
TI \tabto{0cm}Organizational learning, \hl{absorpti}ve \hl{capacit}y, imitation and innovation Empirical analyses of 115 firms across China\par
AU \tabto{0cm}Song\par
PY \tabto{0cm}2015, SO CHINESE MANAGEMENT STUDIES\par
DT \tabto{0cm}Article\par
PG \tabto{0cm}17, NR 69, TC 1\par
DE \tabto{0cm}\hl{ABSORPTI}VE \hl{CAPACIT}Y, CHINESE FIRMS, IMITATION, INNOVATION, ORGANIZATIONAL LEARNING\par
ID \tabto{0cm}ANTECEDENTS, CAPABILITY, DETERMINANTS, INDUSTRY, KNOWLEDGE MANAGEMENT, MARKET ORIENTATION, OPPORTUNITY, PERFORMANCE, PRODUCT INNOVATION, SPILLOVERS\par
AB \tabto{0cm}Purpose - The purpose of this paper is to examine the relationships among organizational learning, \hl{absorpti}ve \hl{capacit}y, imitation and innovation in the Chinese context.
Design/methodology/approach - Based on the organizational learning theory and innovation theory, the paper presents a framework linking organizational learning, \hl{absorpti}ve \hl{capacit}y, imitation and innovation. Using a key informant technique, a survey questionnaire was designed and sent to the middle or top management managers of 115 firms located in Peking, People's Republic (PR) of China. Structural equation modeling (SEM) with the maximum likelihood (ML) estimation procedures was applied to test the hypotheses developed in the research.
Findings - The empirical results show that both organizational learning and \hl{absorpti}ve \hl{capacit}y have positive impacts on innovation; imitation has a positive impact on \hl{absorpti}ve \hl{capacit}y; \hl{absorpti}ve \hl{capacit}y mediates the relationship between imitation and innovation.
Practical implications - This study has implications for firms aiming to enhance innovation by organizational learning, \hl{absorpti}ve \hl{capacit}y and imitation.
Originality/value - Despite the number of studies concerning organizational learning, \hl{absorpti}ve \hl{capacit}y, imitation and innovation, research that encompasses the interrelationships between the four concepts simultaneously remains scarce. The paper provides a framework linking organizational learning, imitation, \hl{absorpti}ve \hl{capacit}y and innovation, and it advances the argument that \hl{absorpti}ve \hl{capacit}y is an important factor in predicting the Chinese firms' transition from imitation to innovation.\par
\clearpage

\vspace*{-2cm}
Nb \tabto{0cm}278/327 (article\_id: 605)\par
TI \tabto{0cm}The construct of \hl{absorpti}ve \hl{capacit}y in knowledge management and intellectual capital research: content and text analyses\par
AU \tabto{0cm}Mariano, Walter\par
PY \tabto{0cm}2015, SO JOURNAL OF KNOWLEDGE MANAGEMENT\par
DT \tabto{0cm}Article\par
PG \tabto{0cm}29, NR 223, TC 2\par
DE \tabto{0cm}\hl{ABSORPTI}VE \hl{CAPACIT}Y, INTELLECTUAL CAPITAL, KNOWLEDGE MANAGEMENT, LITERATURE REVIEW\par
ID \tabto{0cm}ANTECEDENTS, DYNAMIC CAPABILITIES, FIRMS, INNOVATION PERFORMANCE, MANAGING KNOWLEDGE, NETWORK, ORGANIZATIONAL KNOWLEDGE, RECONCEPTUALIZATION, RESEARCH-AND-DEVELOPMENT, STRATEGY\par
AB \tabto{0cm}Purpose - The purpose of this paper is to provide a holistic picture of how and to what extent Cohen and Levinthal's (1990) seminal article on \hl{absorpti}ve \hl{capacit}y was used in knowledge management (KM) and intellectual capital (IC) research from 1990 to 2013.
Design/methodology/approach - In this paper, 186 articles extracted from eight KM and IC journals were reviewed by conducting both content and text analyses. To facilitate research comparison, content analysis followed the method used by Roberts et al. (2012) and thus was based on categories, conceptualizations, levels of analysis and, additionally, temporal evolution of \hl{absorpti}ve \hl{capacit}y from 1990 to 2013 was looked at. Text analysis was performed to identify major research themes developing the \hl{absorpti}ve \hl{capacit}y construct.
Findings - Finding showed that \hl{absorpti}ve \hl{capacit}y was largely underdeveloped in the KM and IC fields. KM, knowledge transfer and innovation were the top three research areas investigating \hl{absorpti}ve \hl{capacit}y in the KM and IC fields. Research limitations/implications - This study had limitations related to time frame, covering a period from April 1990 to November 2013, and accessibility of articles due to specific restrictions in journal subscriptions.
Originality/value - This paper is a first attempt to review \hl{absorpti}ve \hl{capacit}y in KM and IC research. It represented a primary reference for those interested to research \hl{absorpti}ve \hl{capacit}y in the KM and IC fields.\par
\clearpage

\vspace*{-2cm}
Nb \tabto{0cm}279/327 (article\_id: 606)\par
TI \tabto{0cm}Exploring \hl{absorpti}ve \hl{capacit}y in cross-sector social partnerships\par
AU \tabto{0cm}Pittz, Intindola\par
PY \tabto{0cm}2015, SO MANAGEMENT DECISION\par
DT \tabto{0cm}Article\par
PG \tabto{0cm}14, NR 68, TC 0\par
DE \tabto{0cm}DECISION MAKING, LEARNING ORGANIZATIONS, ORGANIZATIONAL DECISION MAKING, ORGANIZATIONAL LEARNING, PARTNERING, STRATEGY\par
ID \tabto{0cm}COLLABORATIONS, COORDINATION, CORPORATIONS, DESIGN, DYNAMIC CAPABILITIES, INNOVATION, PERFORMANCE, STRATEGIC ALLIANCES, TEAMS, TRUST MATTER\par
AB \tabto{0cm}Purpose - The purpose of this paper is to explore cross-sector partnerships (CSSPs) from a strategic perspective to consider collaborations that are long lasting and transcend initial objectives. The authors integrate the concept of \hl{absorpti}ve \hl{capacit}y (ACAP) with the CSSP literature and derive two new antecedents of ACAP, trust and goal interdependency, with relevance to this context. This work responds to a call from ACAP scholars to consider the construct in alternative settings and from collaboration scholars to employ strategy research that approaches CSSPs from a viewpoint beyond a mere transactional approach.
Design/methodology/approach - This manuscript presents a thorough analysis of the process literature regarding CSSPs and the construct of ACAP to consider the importance of knowledge sharing and participatory decision making in the success of collaboration efforts. The combination of these research streams results in a refined model of ACAP to be used in the CSSP context.
Findings - This manuscript provides conceptual and theoretical insights into how knowledge is acquired and exploited within CSSPs. A model for ACAP in CSSPs is proposed and suggests that two new antecedents of ACAP, trust and goal interdependence, be explored in this context through subsequent empirical research.
Research limitations/implications - This type of conceptual work can benefit greatly from subsequent empirical research to test the developed propositions. This model shows considerable promise for future testing, however, and has the potential to encourage additional research into knowledge sharing and long-term success of cross-sector collaborations.
Practical implications - This paper fulfills the need to apply a strategic lens to CSSPs and invites future research into the mutual organizational benefits derived from collaborations that cross economic sectors. It suggests that internal organizational mechanisms exist to be developed by managers that have the potential to enhance a firms ability to recognize the value of external knowledge, acquire it, and transform it for commercial and/or social objectives.
Social implications - As collaborations across economic sectors are proving vital for addressing complex social needs, this manuscript provides a new model to serve as a guidepost for successful partnerships.
Originality/value - This manuscript fulfills a need to integrate strategy scholarship with CSSPs that transcends the heretofore transactional perspective. Through an exploration of the literature, a new model for ACAP is proposed including two new antecedents, trust and goal interdependence, with application to the context of cross-sector collaborations.\par
\clearpage

\vspace*{-2cm}
Nb \tabto{0cm}280/327 (article\_id: 607)\par
TI \tabto{0cm}The effect of shipping knowledge and \hl{absorpti}ve \hl{capacit}y on organizational innovation and logistics value\par
AU \tabto{0cm}Lee, Song\par
PY \tabto{0cm}2015, SO INTERNATIONAL JOURNAL OF LOGISTICS MANAGEMENT\par
DT \tabto{0cm}Article\par
PG \tabto{0cm}20, NR 55, TC 0\par
DE \tabto{0cm}STRATEGIC MANAGEMENT, TRANSPORTATION MANAGEMENT\par
ID \tabto{0cm}COMBINATIVE CAPABILITIES, COMPETITIVE ADVANTAGE, DIVERSIFICATION, EMPIRICAL-EXAMINATION, FIRM, INTEGRATION, INTERNATIONAL JOINT VENTURES, MANAGEMENT, NETWORKS, PERFORMANCE\par
AB \tabto{0cm}Purpose - The purpose of this paper is to examine what types of shipping knowledge are crucial in order for shipping companies to survive in dynamic business environment, and to investigate how the shipping knowledge affects the company's performance (i.e. organizational innovation and logistics value). This paper also diagnoses the moderating effect of \hl{absorpti}ve \hl{capacit}y on the relationship between the shipping knowledge and its effectiveness.
Design/methodology/approach - Based on the literature, a theoretical framework and relevant hypotheses are established so as to show associated relationships between shipping knowledge, \hl{absorpti}ve \hl{capacit}y, and organizational innovation and logistics value. Data are collected for an empirical analysis and a moderated hierarchical regression analysis is conducted to test the hypotheses.
Findings - Results show that a high level of shipping knowledge has a positive influence on the organizational innovation and logistics value of shipping companies. The findings also indicate that, while the \hl{absorpti}ve \hl{capacit}y of shipping companies moderates the positive impact of shipping knowledge on the logistics value, it directly affects the improvement of organizational innovation.
Research limitations/implications - This research verifies that effective knowledge management of shipping companies plays a significant role in developing organizational innovation and improving logistics performance. The research findings provide shipping companies with a strategic insight into the identification of critical sources for competitive advantage and greater organizational performance from an organizational learning perspective.
Practical implications - This line of research is served as an indicator of a good strategic direction for the practitioners engaged in the maritime transport and logistics industry, in order for them to become better integrated entities in a global logistics system as well as maximize their competitive advantages.
Originality/value - This paper makes the first attempt in its kind at empirically examining the types of shipping knowledge and its overall effectiveness in terms of the improvement of organizational innovation and logistics value. The moderating role of \hl{absorpti}ve \hl{capacit}y on the impact of knowledge on organizational performance has also been initiated in the maritime logistics research.\par
\clearpage

\vspace*{-2cm}
Nb \tabto{0cm}281/327 (article\_id: 608)\par
TI \tabto{0cm}\hl{Absorpti}ve \hl{capacit}y and mass customization capability\par
AU \tabto{0cm}Zhang, Zhao, Lyles, Guo\par
PY \tabto{0cm}2015, SO INTERNATIONAL JOURNAL OF OPERATIONS \& PRODUCTION MANAGEMENT\par
DT \tabto{0cm}Article\par
PG \tabto{0cm}20, NR 49, TC 0\par
DE \tabto{0cm}\hl{ABSORPTI}VE \hl{CAPACIT}Y, KNOWLEDGE ACQUISITION, KNOWLEDGE APPLICATION, KNOWLEDGE ASSIMILATION, MASS CUSTOMIZATION CAPABILITY\par
ID \tabto{0cm}IMPACT, INNOVATION, MANUFACTURING PRACTICES, PERFORMANCE, PERSPECTIVE, PRODUCT MODULARITY, RECONCEPTUALIZATION, SUPPLY CHAIN INTEGRATION, TRUST, VIEW\par
AB \tabto{0cm}Purpose - The purpose of this paper is to investigate the effects of a manufacturer's \hl{absorpti}ve \hl{capacit}y (AC) on its mass customization capability (MCC).
Design/methodology/approach - The authors conceptualize AC within the supply chain context as four processes: knowledge acquisition from customers, knowledge acquisition from suppliers, knowledge assimilation, and knowledge application. The authors then propose and empirically test a model on the relationships among AC processes and MCC using structural equation modeling and data collected from 276 manufacturing firms in China.
Findings - The results show that AC significantly improves MCC. In particular, knowledge sourced from customers and suppliers enhances MCC in three ways: directly, indirectly through knowledge application, and indirectly through knowledge assimilation and application. The study also finds that knowledge acquisition significantly enhances knowledge assimilation and knowledge application, and that knowledge assimilation leads to knowledge application.
Originality/value - This study provides empirical evidence of the effects of AC processes on MCC. It also indicates the relationships among AC processes. Moreover, it reveals the mechanisms through which knowledge sourced from customers and suppliers contributes to MCC development, and demonstrates the importance of internal knowledge management practices in exploiting knowledge from supply chain partners. Furthermore, it provides guidelines for executives to decide how to manage supply chain knowledge and devote their efforts and resources in absorbing new knowledge for MCC development.\par
\clearpage

\vspace*{-2cm}
Nb \tabto{0cm}282/327 (article\_id: 609)\par
TI \tabto{0cm}The role of organizational and social capital in the firm's \hl{absorpti}ve \hl{capacit}y\par
AU \tabto{0cm}Aribi, Dupouet\par
PY \tabto{0cm}2015, SO JOURNAL OF KNOWLEDGE MANAGEMENT\par
DT \tabto{0cm}Article\par
PG \tabto{0cm}20, NR 48, TC 0\par
DE \tabto{0cm}\hl{ABSORPTI}VE \hl{CAPACIT}Y, ORGANIZATIONAL CAPITAL, SOCIAL CAPITAL\par
ID \tabto{0cm}ANTECEDENTS, CAPABILITIES, DYNAMIC THEORY, EVOLUTION, KNOWLEDGE CREATION, PERFORMANCE, PRODUCT DEVELOPMENT, RECONCEPTUALIZATION, TECHNOLOGICAL-INNOVATION\par
AB \tabto{0cm}Purpose - This paper aims to ask the question of the contingency of a firm's \hl{absorpti}ve \hl{capacit}y upon the type of expected outcome. Thus, this paper looks at different expected outputs in terms of more or less radical innovations and sees if there are consequences on the \hl{absorpti}ve process underpinning cognitive structures and processes, as embodied in its organizational and social capital.
Design/methodology/approach - To do so, a qualitative study was conducted. In total, 23 persons in three French industrial firms were interviewed about their firm's \hl{absorpti}ve \hl{capacit}y. One of these firms aims at "new-to-the-firm" innovations, while the other two aim at "new-to-the-world" innovations.
Findings - Results suggest that while "new-to-the-firm" innovations tend to favor the use of social capital, "new-to-the-world" innovations tend to rely more on organizational capital. These rather counterintuitive results are interpreted by the necessity to take into account other variables than knowledge distance in the \hl{absorpti}on of new knowledge. In particular, complexity and time-length would call for greater use of organizational capital, while speed and reactivity would instead require greater use of social capital.
Originality/value - This is to the best of the authors' knowledge that one of the first study evidencing the contingent nature of the \hl{absorpti}ve process. Further, results tend to show the form \hl{absorpti}ve \hl{capacit}y takes depends not only on cognitive aspects but also on the particular environment the firm evolves in.\par
\clearpage

\vspace*{-2cm}
Nb \tabto{0cm}283/327 (article\_id: 610)\par
TI \tabto{0cm}How \hl{Absorpti}ve \hl{Capacit}y is Formed in a Latecomer Economy: Different Roles of Foreign Patent and Know-how Licensing in Korea\par
AU \tabto{0cm}Chung, Lee\par
PY \tabto{0cm}2015, SO WORLD DEVELOPMENT\par
DT \tabto{0cm}Article\par
PG \tabto{0cm}17, NR 51, TC 0\par
DE \tabto{0cm}\hl{ABSORPTI}VE \hl{CAPACIT}Y, FOREIGN TECHNOLOGY, IN-HOUSE R\&D, KOREA, LICENSING, TACIT KNOWLEDGE\par
ID \tabto{0cm}ACQUISITION, DIRECT-INVESTMENT, FIRMS, GROWTH, INNOVATION, PERFORMANCE, POLICY, RESEARCH-AND-DEVELOPMENT, SYSTEMS, TECHNOLOGY\par
AB \tabto{0cm}Different from previous studies that tend to use in-house research and development (R\&D) as a proxy for \hl{absorpti}ve \hl{capacit}y but fail to reveal the origins of this R\&D ability, this paper attempts to determine the origin of \hl{absorpti}ve \hl{capacit}y (AC) after defining such concept as the capability of a firm to conduct in-house R\&D and to generate innovation outcomes. This paper distinguishes three forms of foreign technology acquisitions based on unique data from Korea, namely, know-how-only licensing, know-how-and-patent licensing, and patent-only licensing. An econometric analysis demonstrates that a firm tends to involve know-how licensing before it starts in-house R\&D, whereas patent licensing is not significantly related to conducting R\&D. Therefore, a substitution effect is found between foreign patent licensing and conducting in-house R\&D, which is in contrast to the inducing effect of know-how licensing for in-house R\&D. It is also found that conducting in-house R\&D, and know-how licensing by a firm, respectively, is significantly related to a generation of innovations or patent applications in next periods. This study shows that a learning process that involves foreign technology, especially tacit knowledge in the form of know-how, occurs before firms can conduct in-house R\&D and innovations. (C) 2014 Elsevier Ltd. All rights reserved.\par
\clearpage

\vspace*{-2cm}
Nb \tabto{0cm}284/327 (article\_id: 611)\par
TI \tabto{0cm}Technological convergence and the \hl{absorpti}ve \hl{capacit}y of standardisation\par
AU \tabto{0cm}Gauch, Blind\par
PY \tabto{0cm}2015, SO TECHNOLOGICAL FORECASTING AND SOCIAL CHANGE\par
DT \tabto{0cm}Article\par
PG \tabto{0cm}14, NR 41, TC 2\par
DE \tabto{0cm}\hl{ABSORPTI}VE \hl{CAPACIT}Y, PATENT CLASSIFICATION, STANDARDISATION, TECHNOLOGY CONVERGENCE\par
ID \tabto{0cm}CONSORTIA, INDUSTRY, INNOVATION, MODEL, SOCIOLOGY\par
AB \tabto{0cm}In this paper a method of identifying trends in technological convergence on the level of technical fields is proposed. Defining convergence as an inherently stable process of structuring inter-technological patterns over time, German patent data are used to project them onto the structure of the output of standards via a concordance list of International Patent Classifications (IPC) symbols and International Classification of Standards (ICS) classes. Using a set of criteria for a reliable measurement of technological convergence, a set of methods, such as explorative identification of agglomerations of technical fields, the analyses of the breadth of technical fields to differentiate between focused and diffused convergence trends and in-depth analysis using a revised version of the Cross-Impact Assessment method, are devised to measure the level and trend of technological convergence. The structures of convergence in technological development and standardisation are in general moderately positively correlated, but that there are significant differences on how these converging trends are covered in the stock of active standards at the level of technical fields. (C) 2014 Elsevier Inc. All rights reserved.\par
\clearpage

\vspace*{-2cm}
Nb \tabto{0cm}285/327 (article\_id: 612)\par
TI \tabto{0cm}Configuring \hl{absorpti}ve \hl{capacit}y as a key process for research intensive firms\par
AU \tabto{0cm}Patterson, Ambrosini\par
PY \tabto{0cm}2015, SO TECHNOVATION\par
DT \tabto{0cm}Article\par
PG \tabto{0cm}13, NR 70, TC 2\par
DE \tabto{0cm}\hl{ABSORPTI}VE \hl{CAPACIT}Y, BIOPHARMACEUTICAL FIRMS, PROCESS STUDIES\par
ID \tabto{0cm}BIOTECHNOLOGY, COMPETITIVE ADVANTAGE, DYNAMIC CAPABILITIES, MICROFOUNDATIONS, MODERATING ROLE, OPEN INNOVATION, PERSPECTIVE, RECONCEPTUALIZATION, ROUTINES, STRATEGIC ALLIANCES\par
AB \tabto{0cm}\hl{Absorpti}ve \hl{capacit}y is a dynamic capability which creates new firm resources by searching, acquiring, assimilating, transforming and exploiting external knowledge with internal resources and act as a process framework for innovation. Despite being one of the most frequently cited strategic management concepts, \hl{absorpti}ve \hl{capacit}y as a dynamic capability has limited empirical evidence with unverified assumptions. The concept is at risk of reification. With this study we contribute to the literature by providing empirical evidence for \hl{absorpti}ve \hl{capacit}y which challenge the assumptions of how the construct is configured. We follow the strategic factor of intellectual property rights (IPR) in European biopharmaceutical firms using a qualitative process study with temporal bracketing. By tracking IPR, we found evidence for \hl{absorpti}ve \hl{capacit}y in all firms we studied, but the process framework in use is different to Zahra and George's (2002. Acad. Manage. Rev. 27,185-203) and Todorova and Durisin's (2007. Acad. Manage. Rev. 32,774-786) theoretical models. Based on our evidence and literature review we develop some theoretical insights and propose a modified \hl{absorpti}ve \hl{capacit}y model. This new model puts a greater emphasis on assimilating knowledge from outside the firm and provides more clarity on how research intensive firms might use \hl{absorpti}ve \hl{capacit}y. (C) 2014 Elsevier Ltd. All rights reserved.\par
\clearpage

\vspace*{-2cm}
Nb \tabto{0cm}286/327 (article\_id: 613)\par
TI \tabto{0cm}Productivity spillovers from FDI and the role of domestic firm's \hl{absorpti}ve \hl{capacit}y in South Korean manufacturing industries\par
AU \tabto{0cm}Kim\par
PY \tabto{0cm}2015, SO EMPIRICAL ECONOMICS\par
DT \tabto{0cm}Article\par
PG \tabto{0cm}21, NR 37, TC 2\par
DE \tabto{0cm}\hl{ABSORPTI}VE \hl{CAPACIT}Y, FOREIGN DIRECT INVESTMENT (FDI), PRODUCTIVITY SPILLOVERS, SIMULTANEOUS QUANTILE REGRESSION\par
ID \tabto{0cm}EFFICIENCY, FOREIGN DIRECT-INVESTMENT, LINKAGES, PERFORMANCE, QUANTILE REGRESSION\par
AB \tabto{0cm}This paper investigates productivity spillovers from foreign direct investment (FDI) and the \hl{absorpti}ve \hl{capacit}y of domestic firms using the firm level data of South Korean manufacturing industries. This paper finds that the \hl{absorpti}ve \hl{capacit}y has as a role in mitigating the negative spillovers from FDI by capturing additional positive spillovers from FDI. Thus, firms without any \hl{absorpti}ve \hl{capacit}y suffer more from FDI than firms with \hl{absorpti}ve \hl{capacit}y. However, when the endogeneity problems in \hl{absorpti}ve \hl{capacit}y such as R\&D and export activity variables are considered, the role of \hl{absorpti}ve \hl{capacit}y becomes too insignificant to alleviate the negative spillovers from FDI. In addition, the results from simultaneous quantile regression indicate that the effect of R\&D and export activity to absorb the additional spillovers is heterogeneous depending on the conditional productivity distribution.\par
\clearpage

\vspace*{-2cm}
Nb \tabto{0cm}287/327 (article\_id: 614)\par
TI \tabto{0cm}Use of Social Media in Inbound Open Innovation: Building Capabilities for \hl{Absorpti}ve \hl{Capacit}y\par
AU \tabto{0cm}Ooms, Bell, Kok\par
PY \tabto{0cm}2015, SO CREATIVITY AND INNOVATION MANAGEMENT\par
DT \tabto{0cm}Article\par
PG \tabto{0cm}15, NR 58, TC 1\par
DE \tabto{0cm}null\par
ID \tabto{0cm}ADVANTAGE, FIRM, INFORMATION, KNOWLEDGE, NETWORKS, PERFORMANCE, PERSPECTIVE, RECONCEPTUALIZATION, RELATIONAL EMBEDDEDNESS, STRATEGY\par
AB \tabto{0cm}This study investigates the effects of the use of social media in inbound open innovation on capabilities for \hl{absorpti}ve \hl{capacit}y of companies. Seven explorative case studies were conducted in an R\&D and business context of two large global high-tech companies. The results suggest that if the necessary conditions are met, social media usage increases the transparent, moderational and multi-directional interactions that in turn influence four capabilities for \hl{absorpti}ve \hl{capacit}y: connectedness, socialization tactics, cross-functionality and receptivity, a hitherto overlooked capability. Hence, we observe that social media are boundary-spanning tools that can be used to build and increase companies' \hl{absorpti}ve \hl{capacit}y.\par
\clearpage

\vspace*{-2cm}
Nb \tabto{0cm}288/327 (article\_id: 615)\par
TI \tabto{0cm}Regional innovation system, \hl{absorpti}ve \hl{capacit}y and innovation performance: An empirical study\par
AU \tabto{0cm}Lau, Lo\par
PY \tabto{0cm}2015, SO TECHNOLOGICAL FORECASTING AND SOCIAL CHANGE\par
DT \tabto{0cm}Article\par
PG \tabto{0cm}16, NR 144, TC 4\par
DE \tabto{0cm}\hl{ABSORPTI}VE \hl{CAPACIT}Y, CHINA, EMPIRICAL STUDY, REGIONAL INNOVATION SYSTEM\par
ID \tabto{0cm}COMPETITIVE ADVANTAGE, DEVELOPMENT COOPERATION, EXTERNAL TECHNOLOGY, FIRMS, HONG-KONG, KNOWLEDGE TRANSFER, MEASUREMENT ERROR, MEDIATING ROLE, MODERATING ROLE, RESEARCH-AND-DEVELOPMENT\par
AB \tabto{0cm}Recent studies have separately investigated the importance of regional innovation systems (RISs) and \hl{absorpti}ve \hl{capacit}y (AC) to innovation performance. This study explores their interdependency on the premise that better utilization of RIS enhances innovation performance because the RIS enhances the \hl{absorpti}ve \hl{capacit}y of the firm. In this study, we cover three typical elements of RIS: regional innovation initiatives (RII), knowledge-intensive business services (KIBSs) and value chain information sources. This study explores the relationships of these elements on the acquisition, assimilation, transformation and exploitation learning processes of \hl{absorpti}ve \hl{capacit}y. Data were obtained through a mailed survey using a self-administered questionnaire. The results show that RII, KIBS and value chain information sources affect a firm's \hl{absorpti}ve \hl{capacit}y, leading to better innovation performance. Specifically, KIBS improves the acquisition process, value chain information sources improve the acquisition and assimilation processes, and RII improve the transformation process. This study contributes to the literature by exploring how a firm interacts with RIS by utilizing various RIS initiatives to enhance the firm's \hl{absorpti}ve \hl{capacit}y and innovation performance. It also increases our understanding of how AC learning processes relate to RIS and innovation performance. (C) 2014 Elsevier Inc. All rights reserved.\par
\clearpage

\vspace*{-2cm}
Nb \tabto{0cm}289/327 (article\_id: 616)\par
TI \tabto{0cm}Behavioral implications of \hl{absorpti}ve \hl{capacit}y: The role of technological effort and technological capability in leveraging alliance network technological resources\par
AU \tabto{0cm}Srivastava, Gnyawali, Hatfield\par
PY \tabto{0cm}2015, SO TECHNOLOGICAL FORECASTING AND SOCIAL CHANGE\par
DT \tabto{0cm}Article\par
PG \tabto{0cm}13, NR 51, TC 1\par
DE \tabto{0cm}\hl{ABSORPTI}VE \hl{CAPACIT}Y, ALLIANCE NETWORK, ALLIANCE PORTFOLIO, TECHNOLOGICAL CAPABILITY, TECHNOLOGICAL EFFORT\par
ID \tabto{0cm}COLLABORATION, COMPETITION, ESTABLISHED FIRMS, EXPLORATION, IMPACT, INNOVATION, KNOWLEDGE, LINKAGES, PERFORMANCE, STRATEGIC ALLIANCES\par
AB \tabto{0cm}This paper focuses on the moderating role of a firm's \hl{absorpti}ve \hl{capacit}y in realizing innovation benefits from the firm's alliance network technological resources. We conceptualize \hl{absorpti}ve \hl{capacit}y along two dimensions technological effort and technological capability and hypothesize that these two dimensions have opposing moderating effects which can result in theoretical and empirical misspecifications if ignored. We argue that firms with higher technological efforts have greater motivation to search for knowledge from their alliance partners, place more value on the external knowledge and mobilize such knowledge, and face lower internal resistance to assimilate and to use the knowledge. On the other hand, firms with stronger technological capability have lower motivation to search for knowledge from alliance partners, put lower value on the knowledge, make less intense efforts to mobilize it, and face greater internal resistance in assimilating and using the knowledge. By analyzing longitudinal data of 178 U.S.-based public semiconductor firms during 1988-2000 using negative binomial regression, we find that as firms increase their technological effort, the benefits from alliance network resources in terms of technological innovations come at a higher rate. In contrast, as technological capabilities of firms increase, the benefits from the alliance network resources in the form firm technological innovations come at a lower rate. We further discuss important theoretical and managerial implications of our findings. Published by Elsevier Inc.\par
\clearpage

\vspace*{-2cm}
Nb \tabto{0cm}290/327 (article\_id: 617)\par
TI \tabto{0cm}Mergers and Acquisitions Versus Greenfield Investment, \hl{Absorpti}ve \hl{Capacit}y, and Economic Growth: Evidence from 12 New Member States of the European Union\par
AU \tabto{0cm}Eren, Zhuang\par
PY \tabto{0cm}2015, SO EASTERN EUROPEAN ECONOMICS\par
DT \tabto{0cm}Article\par
PG \tabto{0cm}25, NR 52, TC 2\par
DE \tabto{0cm}CENTRAL AND EASTERN EUROPE, ECONOMIC GROWTH, EUROPEAN UNION NEW MEMBER STATES, FOREIGN DIRECT INVESTMENT, GREENFIELD INVESTMENT, MERGERS AND ACQUISITIONS\par
ID \tabto{0cm}COUNTRIES, EASTERN, EXPORT, FDI, FOREIGN DIRECT-INVESTMENT, IMPACT, SECTOR, TRANSITION ECONOMIES\par
AB \tabto{0cm}Using disaggregated FDI data on 12 new member states of the European Union from 1999 to 2010, this paper examines whether different types of FDI have differential effects on economic growth. Our results show that mergers and acquisitions (M\&As) and greenfield investment do not on their own have significant growth effects in these economies. In both cases, the availability of \hl{absorpti}ve \hl{capacit}y plays an important role in stimulating their growth effects. Moreover, a developed financial system complements the impact of M\& As on economic growth, while a minimum threshold level of human capital needs to be reached so that greenfield investment is beneficial for economic growth. Domestic investment is revealed to be a consistent contributor to economic growth.\par
\clearpage

\vspace*{-2cm}
Nb \tabto{0cm}291/327 (article\_id: 618)\par
TI \tabto{0cm}When does \hl{absorpti}ve \hl{capacit}y matter for international performance of firms? Evidence from China\par
AU \tabto{0cm}Wu, Voss\par
PY \tabto{0cm}2015, SO INTERNATIONAL BUSINESS REVIEW\par
DT \tabto{0cm}Article\par
PG \tabto{0cm}8, NR 77, TC 0\par
DE \tabto{0cm}\hl{ABSORPTI}VE \hl{CAPACIT}Y, CHINA, DURATION OF INTERNATIONALIZATION, EARLINESS OF INTERNATIONALIZATION, INTERNATIONAL PERFORMANCE\par
ID \tabto{0cm}BORN-GLOBAL FIRM, CAPABILITIES PERSPECTIVE, ENTREPRENEURIAL PROCLIVITY, ENTRY, GROWTH, KNOWLEDGE, MARKETING CAPABILITIES, MEDIATING ROLE, STRATEGIC CHANGE, VENTURES\par
AB \tabto{0cm}While learning plays an important role in firms' internationalization process, the impact \hl{absorpti}ve \hl{capacit}y has on the international performance when considering the timing of the internationalization is still unclear. Our research explores the role of \hl{absorpti}ve \hl{capacit}y in international performance of early internationalizing firms and international experienced firms. Combining established theories, we propose opposing effect of \hl{absorpti}ve \hl{capacit}y as the learning advantages of newness vanish over time and are replaced by organizational rigidities and inertia. Based on survey data from 162 Chinese firms, our empirical results indicate that the influence of \hl{absorpti}ve \hl{capacit}y on international performance becomes stronger when the firm enters foreign market in its earlier stage of life cycle. However, we find that as the learning advantages of newness diminish, so does the effectiveness of high levels of \hl{absorpti}ve \hl{capacit}y. \hl{Absorpti}ve \hl{capacit}y resources become captured by organizational and operational rigidities and contribute less to firm performance. (C) 2014 Elsevier Ltd. All rights reserved.\par
\clearpage

\vspace*{-2cm}
Nb \tabto{0cm}292/327 (article\_id: 619)\par
TI \tabto{0cm}THE SOCIAL UNDERPINNINGS OF \hl{ABSORPTI}VE \hl{CAPACIT}Y: THE MODERATING EFFECTS OF STRUCTURAL HOLES ON INNOVATION GENERATION BASED ON EXTERNAL KNOWLEDGE\par
AU \tabto{0cm}Tortoriello\par
PY \tabto{0cm}2015, SO STRATEGIC MANAGEMENT JOURNAL\par
DT \tabto{0cm}Article\par
PG \tabto{0cm}12, NR 38, TC 2\par
DE \tabto{0cm}\hl{ABSORPTI}VE \hl{CAPACIT}Y, INDIVIDUAL LEVEL, INNOVATIONS, KNOWLEDGE MANAGEMENT, SOCIAL NETWORKS\par
ID \tabto{0cm}ALLIANCES, COHESION, FIRMS, INDUSTRY, LOCAL SEARCH, PERFORMANCE, PERSPECTIVE, RANGE, TIES\par
AB \tabto{0cm}Building on \hl{absorpti}ve \hl{capacit}y and social network research, in this paper I investigate how individuals inside the organization use external knowledge to generate innovations. Through original sociometric data collected from 276 scientists, researchers, and engineers from the Research and Development division of a large multinational high-tech company, I show that the effects of external knowledge on individuals' innovativeness are contingent upon individuals' position in the internal social structure. In particular, results indicate that the positive effects of external knowledge on innovation generation become more positive when individuals sourcing external knowledge span structural holes in the internal knowledge-sharing network. Copyright (c) 2013 John Wiley \& Sons, Ltd.\par
\clearpage

\vspace*{-2cm}
Nb \tabto{0cm}293/327 (article\_id: 620)\par
TI \tabto{0cm}CROSS-BORDER INTELLECTUAL PROPERTY RIGHTS : CONTRACT ENFORCEMENT AND \hl{ABSORPTI}VE \hl{CAPACIT}Y\par
AU \tabto{0cm}Naghavi, Tsai\par
PY \tabto{0cm}2015, SO SCOTTISH JOURNAL OF POLITICAL ECONOMY\par
DT \tabto{0cm}Article\par
PG \tabto{0cm}16, NR 31, TC 2\par
DE \tabto{0cm}null\par
ID \tabto{0cm}DEVELOPING-COUNTRIES, FOREIGN DIRECT-INVESTMENT, IMITATION, INNOVATION, NORTH-SOUTH TRADE, PATENT PROTECTION, POLICY, TECHNOLOGY-TRANSFER, WELFARE\par
AB \tabto{0cm}This article studies the cross-border protection of intellectual property rights (IPR) as an outcome of a contract obtained through a Nash bargaining process between an innovative North and an imitative South. The level of disclosure required in such contract is higher, the more capable is the South in copying if bargaining breaks down. This raises questions about the suitability of universal IPR standards through a single contract. The threat of a penalty in case of non-compliance can, however, reduce the outside option of more advanced countries and make a stricter IPR regime enforceable by harmonizing their interests with relatively less developed nations.\par
\clearpage

\vspace*{-2cm}
Nb \tabto{0cm}294/327 (article\_id: 621)\par
TI \tabto{0cm}How Can Firms' Basic Research Turn Into Product Innovation? The Role of \hl{Absorpti}ve \hl{Capacit}y and Industry Appropriability\par
AU \tabto{0cm}Martinez-Senra, Quintas, Sartal, Vazquez\par
PY \tabto{0cm}2015, SO IEEE TRANSACTIONS ON ENGINEERING MANAGEMENT\par
DT \tabto{0cm}Article\par
PG \tabto{0cm}12, NR 101, TC 0\par
DE \tabto{0cm}\hl{ABSORPTI}VE \hl{CAPACIT}Y, APPROPRIABILITY, BASIC RESEARCH, PRODUCT INNOVATION\par
ID \tabto{0cm}CAPABILITIES, INTELLECTUAL PROPERTY, INTERNATIONAL GENERATION, KNOW-HOW, MODERATING ROLE, ORGANIZATION, PERFORMANCE, RESEARCH-AND-DEVELOPMENT, SCIENTIFIC-KNOWLEDGE, TECHNOLOGY\par
AB \tabto{0cm}We explain why companies seeking superior product innovation should invest in basic research. Our arguments highlight the role of \hl{absorpti}ve \hl{capacit}y and examine how industry appropriability influences these relations. Based on a rich dataset of 8 416 firms, we argue that basic research in firms increases their knowledge stock and flows, therefore improving their \hl{capacit}y to identify, assimilate, and exploit external knowledge, which allows them to enhance their product innovation performance. We also verify that strong appropriability regimes not only reduce the effect of basic research on \hl{absorpti}ve \hl{capacit}y, but also affect the relation between \hl{absorpti}ve \hl{capacit}y and product innovation in two ways. In businesses with a high \hl{absorpti}ve \hl{capacit}y, strong appropriability regimes exert a negative influence by reducing product innovation; however, businesses with a low \hl{absorpti}ve \hl{capacit}y see their level of product innovation increase. This evidence not only throws into question the attitude of many managers toward basic research; it also calls for open reflection on both the net effect of appropriability on innovative performance and the stages of the innovation process to which public resources should be allocated.\par
\clearpage

\vspace*{-2cm}
Nb \tabto{0cm}295/327 (article\_id: 622)\par
TI \tabto{0cm}Knowledge flows and the \hl{absorpti}ve \hl{capacit}y of regions\par
AU \tabto{0cm}Miguelez, Moreno\par
PY \tabto{0cm}2015, SO RESEARCH POLICY\par
DT \tabto{0cm}Article\par
PG \tabto{0cm}16, NR 95, TC 1\par
DE \tabto{0cm}\hl{ABSORPTI}VE \hl{CAPACIT}Y, INVENTOR MOBILITY, NETWORKS, PATENTS, REGIONAL INNOVATION\par
ID \tabto{0cm}DEVELOPMENT SPILLOVERS, DIFFUSION, EUROPE, INNOVATION, LABOR MOBILITY, PROXIMITY, RESEARCH COLLABORATION, RESEARCH-AND-DEVELOPMENT, SOCIAL NETWORKS, SPATIAL ECONOMETRICS\par
AB \tabto{0cm}This paper assesses the extent to which \hl{absorpti}ve \hl{capacit}y determines knowledge flows' impact on regional innovation. In particular, it looks at how regions with large \hl{absorpti}ve \hl{capacit}y make the most of external inflows of knowledge and information brought in by means of inventor mobility and networks, and fosters local innovation. The paper uses an unbalanced panel of 274 regions over 8 years to estimate a regional knowledge production function with fixed-effects. It finds evidence that inflows of inventors are critical for wealthier regions, while it has more nuanced effects for less developed areas. It also shows that regions' \hl{absorpti}ve \hl{capacit}y critically adds a premium to tap into remote knowledge pools conveyed by mobility and networks. (C) 2015 Elsevier B.V. All rights reserved.\par
\clearpage

\vspace*{-2cm}
Nb \tabto{0cm}296/327 (article\_id: 623)\par
TI \tabto{0cm}Depth and breadth of external knowledge search and performance: The mediating role of \hl{absorpti}ve \hl{capacit}y\par
AU \tabto{0cm}Ferreras-Mendez, Newell, Fernandez-Mesa, Alegre\par
PY \tabto{0cm}2015, SO INDUSTRIAL MARKETING MANAGEMENT\par
DT \tabto{0cm}Article\par
PG \tabto{0cm}12, NR 87, TC 1\par
DE \tabto{0cm}\hl{ABSORPTI}VE \hl{CAPACIT}Y, BREADTH, DEPTH, INNOVATION, PERFORMANCE\par
ID \tabto{0cm}BIOTECHNOLOGY, CAPABILITIES, FIRM PERFORMANCE, IMPACT, INNOVATION PERFORMANCE, MANUFACTURING FIRMS, NETWORKS, ORGANIZATIONAL ANTECEDENTS, PRODUCT INTRODUCTION, RESEARCH-AND-DEVELOPMENT\par
AB \tabto{0cm}Nowadays it is commonly accepted that exploiting external knowledge sources is important for firms' innovation and performance. However, it is still not clear how this effect takes place and what internal capabilities are involved in the process. We propose to open the black box between external knowledge search strategies, and innovation and performance by proposing \hl{absorpti}ve \hl{capacit}y (AC) as the mediating variable. A sample of 102 biotechnology firms from Spain is used to test the proposed theoretical model through structural equation modeling taking the partial least squares approach. Results suggest that AC acts as a full mediator in the relationship between the depth of external knowledge search and the innovation and business performance of the firm. Finally, some suggestions for managers and future lines of research are highlighted. (C) 2015 Elsevier Inc All rights reserved.\par
\clearpage

\vspace*{-2cm}
Nb \tabto{0cm}297/327 (article\_id: 624)\par
TI \tabto{0cm}Complex technological knowledge and value creation in science-to-industry technology transfer projects: The moderating effect of \hl{absorpti}ve \hl{capacit}y\par
AU \tabto{0cm}Winkelbach, Walter\par
PY \tabto{0cm}2015, SO INDUSTRIAL MARKETING MANAGEMENT\par
DT \tabto{0cm}Article\par
PG \tabto{0cm}11, NR 88, TC 0\par
DE \tabto{0cm}\hl{ABSORPTI}VE \hl{CAPACIT}Y, COMPLEX KNOWLEDGE, SCIENCE-TO-INDUSTRY TECHNOLOGY TRANSFER, VALUE CREATION\par
ID \tabto{0cm}COMPETITIVE ADVANTAGE, EMPIRICAL-TEST, ENTREPRENEURIAL ORIENTATION, FIRM PERFORMANCE, ORGANIZATIONAL RESEARCH, PATENT SCOPE, PRODUCT INNOVATION, RESEARCH-AND-DEVELOPMENT, SOCIAL DESIRABILITY BIAS, STRATEGIC ALLIANCES\par
AB \tabto{0cm}This study seeks to enhance the understanding of the interplay between complex knowledge, \hl{absorpti}ve \hl{capacit}y in terms of both \hl{absorpti}ve capabilities and prior knowledge, and value creation. Drawing on a database of 127 science-to-industry R\&D projects in technology-based markets, our study results show the inherent relevance of complexity and \hl{absorpti}ve capabilities for value creation. Contrary to expectations, prior knowledge has no significant effect on value creation per se. Instead, the impact of complex technological knowledge on value creation is enhanced at high levels of both prior knowledge and \hl{absorpti}ve capabilities. The findings suggest that following a well-worn path leads to competence traps, whereas knowledge-related learning capabilities enable a firm to deal with dynamic environments. The findings have implications for managerial decisions and theory regarding how value can be created from complex knowledge in technology transfer settings. (C) 2015 Elsevier Inc. All rights reserved.\par
\clearpage

\vspace*{-2cm}
Nb \tabto{0cm}298/327 (article\_id: 625)\par
TI \tabto{0cm}\hl{Absorpti}ve \hl{capacit}y and performance: The role of customer relationship and technological capabilities in high-tech SMEs\par
AU \tabto{0cm}Tzokas, Kim, Akbar, Al-Dajani\par
PY \tabto{0cm}2015, SO INDUSTRIAL MARKETING MANAGEMENT\par
DT \tabto{0cm}Article\par
PG \tabto{0cm}9, NR 123, TC 2\par
DE \tabto{0cm}\hl{ABSORPTI}VE \hl{CAPACIT}Y, CUSTOMER RELATIONSHIP CAPABILITY, HIGH TECH, SMES, TECHNOLOGICAL CAPABILITY\par
ID \tabto{0cm}COMPETITIVE ADVANTAGE, FIRM PERFORMANCE, INNOVATION PERFORMANCE, KNOWLEDGE CREATION CAPABILITY, MARKET ORIENTATION, ORGANIZATIONAL ANTECEDENTS, PRODUCT DEVELOPMENT, RELATIONSHIP MANAGEMENT, RESEARCH-AND-DEVELOPMENT, STRATEGIC FLEXIBILITY\par
AB \tabto{0cm}This study focuses on how the interplay between a firm's \hl{absorpti}ve \hl{capacit}y (ACAP), and its technological and customer relationship capability contributes to its overall performance. Using structural equation modeling in a sample of 158 firms (316 questionnaires, two respondents per firm) from South Korea's semiconductor industry, we find that a firm's ACAP leads to better performance in terms of new product development, market performance and profitability when used in combination with the firm's capability to engage state of the art technologies in its new product development program (NPD) (technological capability) as well as cultivate strong customer relationships to gain customer insight in NPD (customer relationship capability). By highlighting the interactive nature of \hl{absorpti}ve \hl{capacit}y's antecedents and how these relate to firms' performance, this study contributes to the understanding of the role of ACAP as a mechanism for translating external knowledge into tangible benefits in high-tech SMEs, thus leading to important theoretical and practical implications. (C) 2015 Elsevier Inc. All rights reserved.\par
\clearpage

\vspace*{-2cm}
Nb \tabto{0cm}299/327 (article\_id: 626)\par
TI \tabto{0cm}Knowledge management and innovation in knowledge-based and high-tech industrial markets: The role of openness and \hl{absorpti}ve \hl{capacit}y\par
AU \tabto{0cm}Martin-de Castro\par
PY \tabto{0cm}2015, SO INDUSTRIAL MARKETING MANAGEMENT\par
DT \tabto{0cm}Article\par
PG \tabto{0cm}4, NR 30, TC 1\par
DE \tabto{0cm}\hl{ABSORPTI}VE \hl{CAPACIT}Y, HIGH-TECH INDUSTRIES, INNOVATION, KNOWLEDGE MANAGEMENT, OPEN INNOVATION\par
ID \tabto{0cm}CAPABILITIES, DETERMINANTS, FIELD, MANUFACTURING FIRMS, NETWORKS, PERFORMANCE, PRODUCT DEVELOPMENT\par
AB \tabto{0cm}Knowledge, as resource, and technological innovation, as a dynamic capability, are key sources for firm's sustained competitive advantage and survival in knowledge-based and high-tech industries. Under this rationale has emerged a research stream where knowledge management, organizational learning, or intellectual capital, help to understand and constitute the key pieces of one of the most complex business phenomena; the 'firm's technological advantage'. This being so, it is also true that in knowledge-bated and high-tech industrial markets, competitive success comes directly from continuous technological innovations, where a single organization cannot successfully innovate in isolation; therefore, firms should rely on external relationships and networks in order to complement its knowledge domains, and then, develop better and faster innovations. In this sense, I would like to highlight the cross-fertilizing role of three constructs that are nurtured by different research traditions; 'collaborative/open innovation', from Strategy and Innovation Management research; '\hl{absorpti}ve \hl{capacit}y', from 'A Knowledge-Based View'; and 'market orientation', from Marketing research. (C) 2015 Elsevier Inc. All rights reserved.\par
\clearpage

\vspace*{-2cm}
Nb \tabto{0cm}300/327 (article\_id: 627)\par
TI \tabto{0cm}Micro-evidence on the determinants of innovation in the Netherlands: The relative importance of \hl{absorpti}ve \hl{capacit}y and agglomeration externalities\par
AU \tabto{0cm}Smit, Abreu, de Groot\par
PY \tabto{0cm}2015, SO PAPERS IN REGIONAL SCIENCE\par
DT \tabto{0cm}Article\par
PG \tabto{0cm}24, NR 89, TC 2\par
DE \tabto{0cm}\hl{ABSORPTI}VE \hl{CAPACIT}Y, AGGLOMERATION EXTERNALITIES, COMMUNITY INNOVATION SURVEY, FIRM BEHAVIOUR, INNOVATION, MICRODATA\par
ID \tabto{0cm}COMPETITIVE ADVANTAGE, ECONOMIC-GROWTH, FIRM, KNOWLEDGE, PERFORMANCE, PRODUCT INNOVATION, PROXIMITY, RESEARCH-AND-DEVELOPMENT, SECTORAL PATTERNS, TECHNOLOGICAL-CHANGE\par
AB \tabto{0cm}Although the benefits of clustering for innovation have received much attention in the theoretical as well as empirical literature, analyses at the regional level often disregard the characteristics of local firms. We tackle both at the same time: agglomeration externalities (Marshall, Porter, Jacobs) from census microdata, and firm data from the Community Innovation Survey. Importantly, we allow for sectoral heterogeneity of agglomeration forces. We find that the firm characteristics, including those that proxy for \hl{absorpti}ve \hl{capacit}y', have a much stronger relationship with the propensity to innovate than regular agglomeration externalities. The latter are only statistically significant for a few specific sectors, and even then only for some types of innovation. Sorting of innovation-prone firms into specific locations might therefore be much more important to explain spatial patterns of innovation than agglomeration externalities.\par
\clearpage

\vspace*{-2cm}
Nb \tabto{0cm}301/327 (article\_id: 628)\par
TI \tabto{0cm}Fostering \hl{absorpti}ve \hl{capacit}y through leadership: A cross-cultural analysis\par
AU \tabto{0cm}Flatten, Adams, Brettel\par
PY \tabto{0cm}2015, SO JOURNAL OF WORLD BUSINESS\par
DT \tabto{0cm}Article\par
PG \tabto{0cm}16, NR 141, TC 0\par
DE \tabto{0cm}\hl{ABSORPTI}VE \hl{CAPACIT}Y, LEADERSHIP, NATIONAL CULTURE\par
ID \tabto{0cm}CHARISMATIC LEADERSHIP, DYNAMIC CAPABILITIES, KNOWLEDGE TRANSFER, MANAGEMENT LEADERSHIP, MEASUREMENT ERROR, NATIONAL CULTURE, ORGANIZATIONAL PERFORMANCE, TRANSACTIONAL LEADERSHIP, TRANSFORMATIONAL LEADERSHIP, WORK-RELATED ATTITUDES\par
AB \tabto{0cm}Current business dynamics render a firm's \hl{absorpti}ve \hl{capacit}y, i.e., its ability to explore and exploit external knowledge, highly relevant for successfully competing in the global marketplace. Based on data from 608 firms in Austria, Brazil, Germany, India, Singapore, and the USA, we analyze how \hl{absorpti}ve \hl{capacit}y can be fostered through different leadership styles and whether national culture moderates this relationship. We reveal that transformational and transactional leadership have a positive impact on \hl{absorpti}ve \hl{capacit}y, which is moderated by power distance and uncertainty avoidance. Overall, the study's findings enable managers to foster knowledge \hl{absorpti}on and innovation generation in an international context. (C) 2014 Elsevier Inc. All rights reserved.\par
\clearpage

\vspace*{-2cm}
Nb \tabto{0cm}302/327 (article\_id: 629)\par
TI \tabto{0cm}A framework and model for \hl{absorpti}ve \hl{capacit}y in a dynamic multi-firm environment\par
AU \tabto{0cm}D'Souza, Kulkarni\par
PY \tabto{0cm}2015, SO INTERNATIONAL JOURNAL OF PRODUCTION ECONOMICS\par
DT \tabto{0cm}Article\par
PG \tabto{0cm}13, NR 81, TC 0\par
DE \tabto{0cm}\hl{ABSORPTI}VE \hl{CAPACIT}Y, ENVIRONMENT, MATHEMATICAL MODELING, MULTI-FIRM, SMALL FIRM\par
ID \tabto{0cm}CAPABILITIES, FORTUNE FAVORS, INNOVATION DIFFUSION, INTERNATIONAL JOINT VENTURES, KNOWLEDGE CREATION, ORGANIZATIONS, PERFORMANCE, PREPARED FIRM, RESEARCH-AND-DEVELOPMENT, TECHNOLOGY-TRANSFER\par
AB \tabto{0cm}Researchers have made significant strides toward understanding how \hl{absorpti}ve \hl{capacit}y influences firm performance. However, most of these developments have been theoretical in nature, and have been conducted in single-firm contexts. Our study answers prior calls for more empirical and mathematical treatments in multi-firm contexts that nurture a vibrant and balanced research stream. Our study adopts a mathematical modeling approach to investigate the influence of \hl{absorpti}ve \hl{capacit}y on the performance of a firm in a dynamic multi-firm context. We develop a theoretical framework that drives the modeling exercise to gain insights on two research questions: (1) Relative to the dominant player in the industry, what level of ACap should a firm be endowed with to increase its long-term value in a dynamic environment? (2) Is there a threshold value of ACap endowment that makes it more likely for a firm to challenge the dominant player in the industry? We provide an analytical result, and further, we conduct a numerical study with two firms and seven periods. Our results suggest that there are ACap hurdle rates that a firm must meet to survive and grow. In addition, our model suggests that smaller firms in an industry may do well to strive for unique combinations of ACap, prior knowledge, and initial firm value, to compete successfully against the dominant player in the industry. Our work serves to open new avenues for future research that addresses the influence of ACap endowments on firm performance in dynamic multi-firm environments. (C) 2015 Elsevier B.V. All rights reserved.\par
\clearpage

\vspace*{-2cm}
Nb \tabto{0cm}303/327 (article\_id: 630)\par
TI \tabto{0cm}How shared vision moderates the effects of \hl{absorpti}ve \hl{capacit}y and networking on clustered firms' innovation\par
AU \tabto{0cm}Exposito-Langa, Molina-Morales, Tomas-Miquel\par
PY \tabto{0cm}2015, SO SCANDINAVIAN JOURNAL OF MANAGEMENT\par
DT \tabto{0cm}Article\par
PG \tabto{0cm}10, NR 83, TC 1\par
DE \tabto{0cm}\hl{ABSORPTI}VE \hl{CAPACIT}Y, CLUSTER, NETWORKING, SHARED VISION\par
ID \tabto{0cm}COMPETITIVE CAPABILITIES, CREATION, DETERMINANTS, INDUSTRIAL DISTRICTS, KNOWLEDGE ACQUISITION, PRODUCT DEVELOPMENT, RECONCEPTUALIZATION, START-UPS, TIES, WINE CLUSTER\par
AB \tabto{0cm}This paper will contribute to the line of research that seeks to identify the determinants of firms' innovation performance. Focusing on the territorial dimension, we investigated the role played by shared vision in the effects of internal resources (\hl{absorpti}ve \hl{capacit}y) and external resources (network positioning) on the innovation of firms. To address the research questions, the empirical study drew on a sample of firms belonging to the Valencian textile cluster in Spain. Our findings suggest that networking and firm resources affect performance independently. Furthermore, internal and relational resources are positively active thanks to shared vision. More generally, we aim to contribute to the discussion on the degree to which firms should be involved in the cluster network in order to gain competitive advantages. (C) 2015 Elsevier Ltd. All rights reserved.\par
\clearpage

\vspace*{-2cm}
Nb \tabto{0cm}304/327 (article\_id: 631)\par
TI \tabto{0cm}Does open innovation apply to China? Exploring the contingent role of external knowledge sources and internal \hl{absorpti}ve \hl{capacit}y in Chinese large firms and SMEs\par
AU \tabto{0cm}Huang, Rice, Martin\par
PY \tabto{0cm}2015, SO JOURNAL OF MANAGEMENT \& ORGANIZATION\par
DT \tabto{0cm}Article\par
PG \tabto{0cm}20, NR 98, TC 1\par
DE \tabto{0cm}\hl{ABSORPTI}VE \hl{CAPACIT}Y, CHINA, KNOWLEDGE SOURCES, LARGE FIRMS, OPEN INNOVATION, SMES\par
ID \tabto{0cm}CAPABILITIES, COLLABORATION, INDUSTRY, LARGE-SIZE, MANAGEMENT, MANUFACTURING FIRMS, PERFORMANCE, RESEARCH-AND-DEVELOPMENT, STRATEGIC ALLIANCES, TECHNOLOGY\par
AB \tabto{0cm}While 'open innovation' is often considered to be an organisational strategy with universal application, its generalisability and applicability to organisations operating within emerging economies has yet to be fully explored. This study provides empirical evidence of its importance within a substantial sample of Chinese large firms and small and medium enterprises. Using Tobit regression analysis, our findings indicate that external knowledge sources from inter-firm networking are more important in creating the benefits of open innovation for Chinese small and medium enterprises than their larger peers. Linkages to university and research institutes generally have few direct effects on the innovation performance of both large and small firms in China. However, the role of universities and research institutes is shown to be important among our large firm sample when combined with evident internal \hl{absorpti}ve \hl{capacit}y. This interaction is generally limited to our large firm sample, and is not as evident among small firms.
Our study indicates that the barriers to the adoption of open innovation by Chinese firms might be largely related to the comparatively weak domestic research expertise and limited organisational \hl{absorpti}ve capabilities, with this most particularly evident for small and medium enterprises.
These findings suggest that, based on this evidence, there is no need for emerging economies like China to mimic the emergence path from closed to open innovation followed by developed countries. Chinese firms will be more likely to garner the benefits available from openness when they develop the capabilities required to identify, assimilate and commercialise knowledge and technologies obtained from external sources.\par
\clearpage

\vspace*{-2cm}
Nb \tabto{0cm}305/327 (article\_id: 632)\par
TI \tabto{0cm}INFORMATION TECHNOLOGY USE AS A LEARNING MECHANISM: THE IMPACT OF IT USE ON KNOWLEDGE TRANSFER EFFECTIVENESS, \hl{ABSORPTI}VE \hl{CAPACIT}Y, AND FRANCHISEE PERFORMANCE\par
AU \tabto{0cm}Iyengar, Sweeney, Montealegre\par
PY \tabto{0cm}2015, SO MIS QUARTERLY\par
DT \tabto{0cm}Article\par
PG \tabto{0cm}32, NR 143, TC 0\par
DE \tabto{0cm}\hl{ABSORPTI}VE \hl{CAPACIT}Y, FRANCHISING, IT USE, IT VALUE, KNOWLEDGE TRANSFER EFFECTIVENESS, ORGANIZATIONAL LEARNING\par
ID \tabto{0cm}BUSINESS STRATEGY, COMBINATIVE CAPABILITIES, COMPETITIVE ADVANTAGE, DYNAMIC CAPABILITIES, FIRM PERFORMANCE, ORGANIZATIONAL CAPABILITIES, PRODUCT DEVELOPMENT, RESEARCH AGENDA, STRATEGIC ALLIANCES, TURBULENT ENVIRONMENTS\par
AB \tabto{0cm}This study aims to contribute to the literature through the theoretical development and empirical investigation of the role of information technology use in organizational learning. We develop a theoretical framework that unpacks organizational learning into mechanisms and outcomes. The outcomes of organizational learning are distinguished at two levels: first-order and second-order. Based on the framework, we propose a research model set in the franchising context. We conceptualize franchisee use of IT provided by the franchisor as an important learning mechanism that impacts knowledge transfer effectiveness (first-order outcome) and \hl{absorpti}ve \hl{capacit}y (second-order outcome). Further, the influence of IT use on financial performance is mediated through \hl{absorpti}ve \hl{capacit}y. The model was tested on a sample of 783 independently owned real-estate franchisees using a comprehensive dataset comprised of primary and secondary data. The results indicate that IT use is an important learning mechanism for franchisees by impacting knowledge transfer effectiveness and \hl{absorpti}ve \hl{capacit}y. In turn, \hl{absorpti}ve \hl{capacit}y mediates the relationship between IT use and financial performance. The empirical support for the research model serves to affirm the underlying learning mechanisms-outcomes framework. The results are stable across the choice of statistical method and the operationalization of financial performance. Theoretical contributions, implications for practice, and limitations of the study are discussed.\par
\clearpage

\vspace*{-2cm}
Nb \tabto{0cm}306/327 (article\_id: 633)\par
TI \tabto{0cm}Examining \hl{Absorpti}ve \hl{Capacit}y in Supply Chains: Linking Responsive Strategy and Firm Performance\par
AU \tabto{0cm}Dobrzykowski, Leuschner, Hong, Roh\par
PY \tabto{0cm}2015, SO JOURNAL OF SUPPLY CHAIN MANAGEMENT\par
DT \tabto{0cm}Article\par
PG \tabto{0cm}26, NR 137, TC 1\par
DE \tabto{0cm}\hl{ABSORPTI}VE \hl{CAPACIT}Y, INFORMATION PROCESSING THEORY, KNOWLEDGE ACQUISITION, NEW PRODUCT DEVELOPMENT, ORGANIZATIONAL LEARNING, STRUCTURAL EQUATION MODELING\par
ID \tabto{0cm}BIG DATA, COMPETITIVE ADVANTAGE, CONTINGENCY APPROACH, DESIGN, INFORMATION-PROCESSING PERSPECTIVE, INNOVATION, INTEGRATION, MANUFACTURING STRATEGIES, PRODUCT DEVELOPMENT, RESEARCH-AND-DEVELOPMENT\par
AB \tabto{0cm}Information management is a core supply chain activity that is increasing in importance as firms strive to become more responsive to growing customer demand for innovative products. However, effective processing of information from customers and suppliers remains a struggle for most firms. \hl{Absorpti}ve \hl{capacit}y provides a useful view of information processing activities, but the current understanding of how firms use it to improve performance and why some firms seem to develop it while others do not remains unclear. This study is grounded in information processing theory, and examines the role of \hl{absorpti}ve \hl{capacit}y in linking a firm's responsive strategy and performance. We test a structural equation model on data from 711 manufacturing firms, and validate our results on a second sample of 677 firms. Our study makes three major contributions by providing evidence that: (1) \hl{absorpti}ve \hl{capacit}y is motivated by a firm's responsive strategy; (2) it fully mediates the relationship between responsive strategy and firm performance, indicating that \hl{absorpti}ve \hl{capacit}y is a necessary competence for firms that aim to deliver innovative products to customers; and (3) the relationship between responsive strategy and \hl{absorpti}ve \hl{capacit}y is U-shaped, indicating that when firms attempt to blend efficient and responsive strategies, their ability to develop \hl{absorpti}ve \hl{capacit}y is diminished.\par
\clearpage

\vspace*{-2cm}
Nb \tabto{0cm}307/327 (article\_id: 634)\par
TI \tabto{0cm}Developing \hl{Absorpti}ve \hl{Capacit}y through e-Business: The Case of International SMEs\par
AU \tabto{0cm}Raymond, Bergeron, Croteau, St-Pierre\par
PY \tabto{0cm}2015, SO JOURNAL OF SMALL BUSINESS MANAGEMENT\par
DT \tabto{0cm}Article\par
PG \tabto{0cm}20, NR 113, TC 0\par
DE \tabto{0cm}null\par
ID \tabto{0cm}ADVANCED MANUFACTURING TECHNOLOGY, COMPETITIVE ADVANTAGE, DYNAMIC CAPABILITIES, FIRM PERFORMANCE, FUTURE-RESEARCH, INFORMATION-SYSTEMS RESEARCH, MEDIUM-SIZED ENTERPRISES, OPERATIONAL CAPABILITIES, ORGANIZATIONAL CAPABILITIES, RESOURCE-BASED VIEW\par
AB \tabto{0cm}This research uses the \hl{absorpti}ve \hl{capacit}y concept as a theoretical lens to study the effect of e-business upon the internationalization performance of small and medium-sized enterprises (SMEs), addressing the following research issue: To what extent are manufacturing SMEs successful in developing their potential and realized \hl{absorpti}ve \hl{capacit}y in response to the environmental uncertainty brought about by their internationalization? Results of a survey study of 588 manufacturing SMEs indicate that e-business capabilities have a significant impact on internationalization performance to the extent that these capabilities are developed as a response to increased environmental uncertainty. Moreover, these capabilities are realized through the development of networking, advanced manufacturing, and marketing capabilities that also respond to environmental uncertainty.\par
\clearpage

\vspace*{-2cm}
Nb \tabto{0cm}308/327 (article\_id: 635)\par
TI \tabto{0cm}\hl{Absorpti}ve \hl{capacit}y, organizational antecedents, and environmental dynamism\par
AU \tabto{0cm}Roberts\par
PY \tabto{0cm}2015, SO JOURNAL OF BUSINESS RESEARCH\par
DT \tabto{0cm}Article\par
PG \tabto{0cm}8, NR 65, TC 1\par
DE \tabto{0cm}\hl{ABSORPTI}VE \hl{CAPACIT}Y, DATA INTEGRATION, ENVIRONMENTAL DYNAMISM, ORGANIZATIONAL CAPABILITY, ORGANIZATIONAL LEARNING\par
ID \tabto{0cm}BEHAVIORAL-RESEARCH, COMBINATIVE CAPABILITIES, INFORMATION-SYSTEMS, INNOVATION, KNOWLEDGE MANAGEMENT, MULTIBUSINESS FIRMS, PERFORMANCE, RECONCEPTUALIZATION, TECHNOLOGY, WEAK TIES\par
AB \tabto{0cm}I examine organizational antecedents to \hl{absorpti}ve \hl{capacit}y. Specifically, I identify and empirically test how the interaction between data integration and connectedness affects \hl{absorpti}ve \hl{capacit}y. I also integrate the role of the environment in my research model by determining whether the link between these organizational capabilities and \hl{absorpti}ve \hl{capacit}y varies across different levels of environmental dynamism.! test my model with data collected from 178 high-tech firms. The results show that data integration and connectedness jointly influence \hl{absorpti}ve \hl{capacit}y; however, connectedness does not contribute to \hl{absorpti}ve \hl{capacit}y above and beyond data integration for firms competing in dynamic environments. This study has implications for research and practice on \hl{absorpti}ve \hl{capacit}y and organizational learning. (C) 2015 Elsevier Inc. All rights reserved.\par
\clearpage

\vspace*{-2cm}
Nb \tabto{0cm}309/327 (article\_id: 636)\par
TI \tabto{0cm}Short- and Long-Term Performance Feedback and \hl{Absorpti}ve \hl{Capacit}y\par
AU \tabto{0cm}Ben-Oz, Greve\par
PY \tabto{0cm}2015, SO JOURNAL OF MANAGEMENT\par
DT \tabto{0cm}Article\par
PG \tabto{0cm}27, NR 82, TC 2\par
DE \tabto{0cm}\hl{ABSORPTI}VE \hl{CAPACIT}Y, KNOWLEDGE MANAGEMENT, ORGANIZATIONAL LEARNING, PERFORMANCE FEEDBACK\par
ID \tabto{0cm}BEHAVIORAL-THEORY, COMBINATIVE CAPABILITIES, COMPETITIVE ADVANTAGE, EXPLOITATION, EXPLORATION, FIRM PERFORMANCE, INNOVATION, KNOWLEDGE, ORGANIZATIONAL ANTECEDENTS, SEARCH\par
AB \tabto{0cm}Research on organizational learning from performance feedback has produced findings on how organizational change is influenced by performance relative to aspiration levels, but has focused on short-term goal variables. In this article, we examine how short- and long-term goals are related to short- and long-term actions, respectively. We do so by predicting changes in \hl{absorpti}ve \hl{capacit}y from performance relative to aspiration levels, and by testing whether long-term goals mainly affect potential \hl{absorpti}ve \hl{capacit}y, which has long-term effects, while short-term goals mainly affect the realized \hl{absorpti}ve \hl{capacit}y, which has short-term effects. Using data from surveys of 252 decision makers representing 129 Israeli early-stage high-tech organizations, our analysis yields supportive empirical findings. The findings imply that performance relative to aspiration levels has effects on long-term strategic actions as well as short-term ones, and thus argue against strict myopia.\par
\clearpage

\vspace*{-2cm}
Nb \tabto{0cm}310/327 (article\_id: 637)\par
TI \tabto{0cm}Entrepreneurial orientation-as-experimentation and firm performance: The enabling role of \hl{absorpti}ve \hl{capacit}y\par
AU \tabto{0cm}Patel, Kohtamaki, Parida, Wincent\par
PY \tabto{0cm}2015, SO STRATEGIC MANAGEMENT JOURNAL\par
DT \tabto{0cm}Article\par
PG \tabto{0cm}11, NR 61, TC 0\par
DE \tabto{0cm}\hl{ABSORPTI}VE \hl{CAPACIT}Y, ENTREPRENEURIAL ORIENTATION, EXPERIMENTATION, GROWTH, VARIATION\par
ID \tabto{0cm}ANTECEDENTS, CAPABILITIES, GROWTH, INDUSTRY, INNOVATION PERFORMANCE, KNOWLEDGE TRANSFER, MAXIMUM-LIKELIHOOD-ESTIMATION, PATENT CITATIONS, PERSPECTIVE, RISK\par
AB \tabto{0cm}According to the perspective of entrepreneurial orientation-as-experimentation, entrepreneurial orientation (EO) increases variability in innovation outcomes. Although increased variability in the innovation portfolio could increase performance, it could also lead to a decline in performance. We propose that \hl{absorpti}ve \hl{capacit}y plays a role in both increasing and managing variations in innovation outcomes. Potential \hl{absorpti}ve \hl{capacit}y enhances the effects of EO on variability in innovation outcomes, whereas realized \hl{absorpti}ve \hl{capacit}y helps transform and exploit variability in innovation outcomes to enhance firm performance. Copyright (c) 2014 John Wiley \& Sons, Ltd.\par
\clearpage

\vspace*{-2cm}
Nb \tabto{0cm}311/327 (article\_id: 638)\par
TI \tabto{0cm}Soaking It Up: \hl{Absorpti}ve \hl{Capacit}y in Interorganizational New Product Development Teams\par
AU \tabto{0cm}Backmann, Hoegl, Cordery\par
PY \tabto{0cm}2015, SO JOURNAL OF PRODUCT INNOVATION MANAGEMENT\par
DT \tabto{0cm}Review\par
PG \tabto{0cm}17, NR 103, TC 0\par
DE \tabto{0cm}null\par
ID \tabto{0cm}COMPETITIVE ADVANTAGE, DEVELOPMENT SUCCESS, DEVELOPMENT-PROJECTS, DIVERSITY, EMPIRICAL-TEST, INTERNATIONAL JOINT VENTURES, KNOWLEDGE TRANSFER, PERFORMANCE, SUPPLIER INTEGRATION, WORK GROUP COHESION\par
AB \tabto{0cm}Prior research has acknowledged the importance of an organization's \hl{absorpti}ve \hl{capacit}ythe ability to acquire new knowledge and information, assimilate, transform, and exploit itfor innovation purposes. Because innovations are usually developed by project teams, this suggests that \hl{absorpti}ve \hl{capacit}y, as a construct, may also be usefully applied at the team level. Consequently, this study developed a measure for team-level \hl{absorpti}ve \hl{capacit}y, investigated the potential influencing factors, and examined its relationship to team effectiveness in terms of product innovativeness in an interorganizational context. Specifically, building on the theory of homophily and information and decision-making theories, three factors (social-category similarity, work-style similarity, and knowledge complementarity between the recipient and the partner organization teams) were identified as likely antecedents of team \hl{absorpti}ve \hl{capacit}y. The hypotheses were tested on data from 98 interorganizational new product development teams and included responses from team members, team leaders, and team-external managers. With regard to the antecedents of team \hl{absorpti}ve \hl{capacit}y in interorganizational settings, the results showed a significant positive association with partners' work-style similarity and an inverted U-shaped relationship with partners' knowledge complementarity. Social-category similarity was not significantly associated with team \hl{absorpti}ve \hl{capacit}y. We also examined whether team \hl{absorpti}ve \hl{capacit}y was related to interorganizational team effectiveness and found a significant positive relationship between team \hl{absorpti}ve \hl{capacit}y and product innovativeness. The study demonstrates that \hl{absorpti}ve is indeed related to team effectiveness outcomes in an interorganizational context, which underlines the importance of team-level \hl{absorpti}ve \hl{capacit}y for product innovation management and suggests paying more attention to the lower levels of \hl{absorpti}ve \hl{capacit}y.\par
\clearpage

\vspace*{-2cm}
Nb \tabto{0cm}312/327 (article\_id: 639)\par
TI \tabto{0cm}Research and development project management best practices and \hl{absorpti}ve \hl{capacit}y: Empirical evidence from Spanish firms\par
AU \tabto{0cm}Vicente-Oliva, Martinez-Sanchez, Berges-Muro\par
PY \tabto{0cm}2015, SO INTERNATIONAL JOURNAL OF PROJECT MANAGEMENT\par
DT \tabto{0cm}Article\par
PG \tabto{0cm}13, NR 129, TC 0\par
DE \tabto{0cm}\hl{ABSORPTI}VE \hl{CAPACIT}Y, KNOWLEDGE MANAGEMENT, MANAGERIAL PRACTICES, R\&D PROJECT MANAGEMENT\par
ID \tabto{0cm}COMPETITIVE ADVANTAGE, DECISION-MAKING, DYNAMIC CAPABILITIES, INNOVATION PERFORMANCE, KNOWLEDGE TRANSFER, MODERATING ROLE, PRODUCT DEVELOPMENT, RESOURCE-BASED VIEW, STRATEGIC MANAGEMENT, TECHNOLOGY-TRANSFER\par
AB \tabto{0cm}Evidence from a study carried out in a sample of Spanish firms indicates that research and development (R\&D) project management practices are positively related to \hl{absorpti}ve \hl{capacit}y of knowledge (AC), although the influence of these practices differs for each AC dimension. Managers realize that learning from past experiences in R\&D projects develops the \hl{capacit}y to gain access to relevant external knowledge. However, the positive relationship between management practices and \hl{absorpti}ve \hl{capacit}y is only significant for transforming and exploiting external knowledge in R\&D projects. The article discusses the managerial implications of improving \hl{absorpti}ve \hl{capacit}y within the management of R\&D projects and the firm, for every AC dimension. (C) 2015 Elsevier Ltd. APM and IPMA. All rights reserved.\par
\clearpage

\vspace*{-2cm}
Nb \tabto{0cm}313/327 (article\_id: 640)\par
TI \tabto{0cm}Enhancing Green \hl{Absorpti}ve \hl{Capacit}y, Green Dynamic \hl{Capacit}ies and Green Service Innovation to Improve Firm Performance: An Analysis of Structural Equation Modeling (SEM)\par
AU \tabto{0cm}Chen, Lin, Chang\par
PY \tabto{0cm}2015, SO SUSTAINABILITY\par
DT \tabto{0cm}Article\par
PG \tabto{0cm}19, NR 64, TC 0\par
DE \tabto{0cm}FIRM PERFORMANCE, GREEN \hl{ABSORPTI}VE \hl{CAPACIT}Y, GREEN DYNAMIC \hl{CAPACIT}IES' GREEN SERVICE INNOVATION\par
ID \tabto{0cm}ALLIANCES, CAPABILITIES, DETERMINANTS, ENTREPRENEURSHIP, MANAGEMENT, MISSING LINK, PERSPECTIVE, PRODUCT DEVELOPMENT, SUPPLY CHAIN, SYSTEM\par
AB \tabto{0cm}This study discusses the influences of green \hl{absorpti}ve \hl{capacit}y, green dynamic \hl{capacit}ies, and green service innovation on firm performance. In order to fill the research gap, this study proposes the concept of green service innovation. The results are as follows: First, this study finds that green \hl{absorpti}ve \hl{capacit}y has positive effects on green dynamic \hl{capacit}ies, green service innovation, and firm performance. Second, this study points out that green dynamic \hl{capacit}ies have positive effects on green service innovation and firm performance. Third, this study observes that green dynamic \hl{capacit}ies and green service innovation intercede the positive connection between green \hl{absorpti}ve \hl{capacit}y and firm performance.\par
\clearpage

\vspace*{-2cm}
Nb \tabto{0cm}314/327 (article\_id: 641)\par
TI \tabto{0cm}Association between Innovative Entrepreneurial Orientation, \hl{Absorpti}ve \hl{Capacit}y, and Farm Business Performance\par
AU \tabto{0cm}Gellynck, Cardenas, Pieniak, Verbeke\par
PY \tabto{0cm}2015, SO AGRIBUSINESS\par
DT \tabto{0cm}Article\par
PG \tabto{0cm}16, NR 78, TC 0\par
DE \tabto{0cm}null\par
ID \tabto{0cm}AGRICULTURAL KNOWLEDGE INFRASTRUCTURE, CONSTRUCT, EMPIRICAL-EXAMINATION, FIRMS, INTEGRATIVE MODEL, LINKING, MARKET ORIENTATION, ORGANIZATIONAL TRUST, RESOURCES, VENTURES\par
AB \tabto{0cm}A growing interest emerges with regard to how entrepreneurship and innovation can contribute to the success and well-being of the farmers. The main objective of the study is to develop and test a model, which emphasizes the relationship of trust and innovative entrepreneurial orientation on \hl{absorpti}ve \hl{capacit}y in an agricultural context. This article examines also the effect of \hl{absorpti}ve \hl{capacit}y on innovation as well as the connection of both innovative entrepreneurial orientation and innovation outcome on farm business performance. Cross-sectional data were collected from a sample of 199 banana farmers in 2008-2009 in Ecuador. A structural equation modeling approach has been applied to validate the hypothesized model. The findings show that all relationships established in the model are significantly and positively associated, except for the linkage between innovation outcome and farm business performance. [EconLit citation: L260; O310; Q160]. (C) 2014 Wiley Periodicals, Inc.\par
\clearpage

\vspace*{-2cm}
Nb \tabto{0cm}315/327 (article\_id: 642)\par
TI \tabto{0cm}\hl{Absorpti}ve \hl{capacit}y and business model innovation as rapid development strategies for regional growth\par
AU \tabto{0cm}Moutinho\par
PY \tabto{0cm}2016, SO INVESTIGACION ECONOMICA\par
DT \tabto{0cm}Article\par
PG \tabto{0cm}46, NR 124, TC 0\par
DE \tabto{0cm}\hl{ABSORPTI}VE \hl{CAPACIT}Y, INNOVATION PROCESS MANAGEMENT, R\&D MANAGEMENT, REGIONAL INNOVATION SYSTEMS\par
ID \tabto{0cm}BLUE OCEAN STRATEGY, CLUSTERS, CREATIVE DESTRUCTION, ECONOMIC-GROWTH, INCREASING RETURNS, KNOWLEDGE-SPILLOVER, NEURAL-NETWORKS, PERFORMANCE, RESEARCH-AND-DEVELOPMENT, SYSTEMS\par
AB \tabto{0cm}Innovation remains a complex phenomenon and the task of managing it at the Regional Innovation Systems (RIS) architecture level is discussed herein, namely involving joint initiatives, close to organizational realities and their competitive advantages, up and beyond the uncertainty that surrounds Governmental R\&D Investment (GRI) effectiveness, either due to misuse or ineffective, application of resources. Artificial Neural Networks (ANN) modelling was applied to the study of RIS structure, aiming to identify the 'hidden' mediating variables that may influence the overall effect of GRI on economic and employment growth. In general, \hl{Absorpti}ve \hl{Capacit}y, is the most rapid and balanced development strategy for regions characterised by environments, which are adverse to change and innovation, and characterized by low industrialization and income levels.\par
\clearpage

\vspace*{-2cm}
Nb \tabto{0cm}316/327 (article\_id: 643)\par
TI \tabto{0cm}Catching up of emerging economies: the role of capital goods imports, FDI inflows, domestic investment and \hl{absorpti}ve \hl{capacit}y\par
AU \tabto{0cm}Glas, Hubler, Nunnenkamp\par
PY \tabto{0cm}2016, SO APPLIED ECONOMICS LETTERS\par
DT \tabto{0cm}Article\par
PG \tabto{0cm}4, NR 5, TC 0\par
DE \tabto{0cm}\hl{ABSORPTI}VE \hl{CAPACIT}Y, BRICS, F14, F21, FOREIGN DIRECT INVESTMENT, IMPORTS, O47, TOTAL FACTOR PRODUCTIVITY\par
ID \tabto{0cm}FOREIGN DIRECT-INVESTMENT\par
AB \tabto{0cm}We show that the impact of capital goods imports and FDI inflows on economic convergence depends on the local \hl{capacit}y of emerging economies to absorb superior technologies.\par
\clearpage

\vspace*{-2cm}
Nb \tabto{0cm}317/327 (article\_id: 644)\par
TI \tabto{0cm}The emergence of \hl{absorpti}ve \hl{capacit}y through micro-macro level interactions\par
AU \tabto{0cm}Martinkenaite, Breunig\par
PY \tabto{0cm}2016, SO JOURNAL OF BUSINESS RESEARCH\par
DT \tabto{0cm}Article\par
PG \tabto{0cm}9, NR 78, TC 0\par
DE \tabto{0cm}\hl{ABSORPTI}VE \hl{CAPACIT}Y, EMERGENCE, LEARNING PROCESS, MICRO-MACRO LEVEL INTERACTIONS\par
ID \tabto{0cm}COMBINATIVE CAPABILITIES, DYNAMIC THEORY, FIRM, INNOVATION, KNOWLEDGE TRANSFER, NETWORK POSITION, ORGANIZATIONAL PRACTICES, PERFORMANCE, PERSPECTIVE, STRATEGIC ALLIANCES\par
AB \tabto{0cm}A firm's \hl{absorpti}ve \hl{capacit}y involves two dimensions: horizontal and vertical. The horizontal dimension refers to a dynamic interplay between internal and external environments of the firm, which is extensively covered in the \hl{absorpti}ve \hl{capacit}y research. However, the literature ignores vertical dimension involving individual-organization interactions. Scant knowledge is available about the mechanisms through which \hl{absorpti}ve \hl{capacit}y emerges as an organizational learning capability. This study reviews the seminal works of Cohen and Levinthal and finds that the stickiness of knowledge, the multiple antecedents of \hl{absorpti}ve \hl{capacit}y and their interattions are explicitly addressed therein, but are insufficiently problematized in subsequent research. Drawing on the knowledge-based view of the firm and the micro-foundations lens of organizational capabilities, the present study re-conceptualizes \hl{absorpti}ve \hl{capacit}y as a set of three sequentially linked learning processes where individual and organizational antecedents interact, and explains how value recognition, assimilation and application capabilities emerge as organizational (macro) level phenomena. (C) 2015 Elsevier Inc. All rights reserved.\par
\clearpage

\vspace*{-2cm}
Nb \tabto{0cm}318/327 (article\_id: 645)\par
TI \tabto{0cm}Importing, Productivity and \hl{Absorpti}ve \hl{Capacit}y in Sub-Saharan African Manufacturing and Services Firms\par
AU \tabto{0cm}Foster-McGregor, Isaksson, Kaulich\par
PY \tabto{0cm}2016, SO OPEN ECONOMIES REVIEW\par
DT \tabto{0cm}Article\par
PG \tabto{0cm}31, NR 74, TC 0\par
DE \tabto{0cm}\hl{ABSORPTI}VE \hl{CAPACIT}Y, HUMAN CAPITAL, IMPORTING, PRODUCTIVITY, SUB-SAHARAN AFRICA\par
ID \tabto{0cm}1ST EVIDENCE, CAPITAL GOODS, DEVELOPMENT SPILLOVERS, EXPORTS, INDUSTRY, INTERNATIONAL-TRADE, LEVEL EVIDENCE, PERFORMANCE, PLANT-LEVEL, REGRESSION\par
AB \tabto{0cm}This paper examines the relationship between importing and firm-level productivity in Sub-Saharan Africa. Using a recent firm-level survey for 19 Sub-Saharan African countries, the paper shows that there is a positive and significant relationship between importing and productivity for both manufacturing and services firms. Using a series of robustness tests, the paper finds that the importer-productivity relationship is robust in the case of manufacturing firms, but the results for services appear sensitive to the presence of extreme values. Finally, the paper shows that manufacturing firms with the highest levels of human capital show the strongest relationship between importing and productivity, a result consistent with a role for \hl{absorpti}ve \hl{capacit}y in maximising the benefits of importing.\par
\clearpage

\vspace*{-2cm}
Nb \tabto{0cm}319/327 (article\_id: 646)\par
TI \tabto{0cm}Green Process Innovation and Financial Performance in Emerging Economies: Moderating Effects of \hl{Absorpti}ve \hl{Capacit}y and Green Subsidies\par
AU \tabto{0cm}Xie, Huo, Qi, Zhu\par
PY \tabto{0cm}2016, SO IEEE TRANSACTIONS ON ENGINEERING MANAGEMENT\par
DT \tabto{0cm}Article\par
PG \tabto{0cm}12, NR 72, TC 0\par
DE \tabto{0cm}\hl{ABSORPTI}VE \hl{CAPACIT}Y, EMERGING ECONOMIES, FINANCIAL PERFORMANCE, GREEN PROCESS INNOVATION, GREEN SUBSIDIES\par
ID \tabto{0cm}ECO-INNOVATION, EMPIRICAL-EVIDENCE, END-OF-PIPE, ENVIRONMENTAL PERFORMANCE, FIRM PERFORMANCE, MANAGEMENT, PANEL-DATA, PERSPECTIVE, RESEARCH-AND-DEVELOPMENT, RESOURCE-BASED VIEW\par
AB \tabto{0cm}Being "green" is socially, desirable, yet whether it pays to be "green" is unclear. This question has become more important to manufacturing industries as environmental concerns are escalating, particularly in emerging economies. We examine the effects of green process innovation on the financial performance of manufacturing industries with a focus on the moderating effects of government subsidies versus industries' own \hl{absorpti}ve capability. Using sulfur dioxide emissions as an environmental index, we establish a dynamic model using ten years panel data from 28 industries in China where environmental concerns have been severe and government subsidies have been commonly believed to be instrumental. The results show that clean technologies and end-of-pipe technologies are positively related to financial performance at the industry level, thus it pays to be "green." Furthermore, strong \hl{absorpti}ve \hl{capacit}y tends to enhance this relationship, but surprisingly green subsidies turn out to weaken this relationship. We also find the effects are different between clean technologies and end-of-pile technologies, which leads to finer grained insights into these two types of technologies and their adoption. We conclude that manufacturing industries can benefit more from green process innovation by leveraging their internal \hl{absorpti}ve capability, as opposed to replying on external government subsidies as commonly believed. Our findings provide finer grained insights on the benefits of green process innovation, and shed light on the study of green process innovation in other emerging economies.\par
\clearpage

\vspace*{-2cm}
Nb \tabto{0cm}320/327 (article\_id: 647)\par
TI \tabto{0cm}Direct and mediated ties to universities: "Scientific" \hl{absorpti}ve \hl{capacit}y and innovation performance of pharmaceutical firms\par
AU \tabto{0cm}Belderbos, Gilsing, Suzuki\par
PY \tabto{0cm}2016, SO STRATEGIC ORGANIZATION\par
DT \tabto{0cm}Article\par
PG \tabto{0cm}21, NR 96, TC 0\par
DE \tabto{0cm}ALLIANCE PORTFOLIOS, INDUSTRY-SCIENCE LINKAGES, R\&D COLLABORATION, SCIENTIFIC \hl{ABSORPTI}VE \hl{CAPACIT}Y\par
ID \tabto{0cm}BIOTECHNOLOGY FIRMS, INDUSTRY, INTERORGANIZATIONAL COLLABORATION, KNOWLEDGE, NETWORK DYNAMICS, RESEARCH-AND-DEVELOPMENT, SCIENCE, SOCIAL-STRUCTURE, STRATEGIC ORGANIZATION, TECHNOLOGY\par
AB \tabto{0cm}Extant literature on firm-university collaboration has emphasized two different strategies that firms in science-based industries adopt to source scientific knowledge and expertise. On one hand, firms engage in direct research collaborations with universities. On the other hand, firms establish indirect, mediated, ties to universities by engaging in research collaborations with dedicated biotech firms that are themselves strongly linked to universitieswith the dedicated biotech firm taking the role of broker. We argue that the relative benefits of direct and mediated ties depend on the extent to which firms have organized their research and development to facilitate the \hl{absorpti}on, assimilation, transformation, and exploitation of scientific knowledge, which we coin scientific \hl{absorpti}ve \hl{capacit}y. Drawing on patent and publication data in a panel of 33 vertically integrated pharmaceutical firms, we find that direct university collaboration is more beneficial for firms with relatively high scientific \hl{absorpti}ve \hl{capacit}y, while only mediated ties are associated with greater innovative performance for firms with relatively low scientific \hl{absorpti}ve \hl{capacit}y. The latter association is reduced if the mediated ties are with top universities. Our findings are suggestive of the importance of a fit between the nature of a firm's research and development organization and its strategy to access scientific knowledge.\par
\clearpage

\vspace*{-2cm}
Nb \tabto{0cm}321/327 (article\_id: 648)\par
TI \tabto{0cm}Balancing \hl{absorpti}ve \hl{capacit}y and inbound open innovation for sustained innovative performance: An attention-based view\par
AU \tabto{0cm}Kim, Foss\par
PY \tabto{0cm}2016, SO EUROPEAN MANAGEMENT JOURNAL\par
DT \tabto{0cm}Article\par
PG \tabto{0cm}11, NR 61, TC 0\par
DE \tabto{0cm}\hl{ABSORPTI}VE \hl{CAPACIT}Y, ATTENTION-BASED VIEW, BOUNDARY-SPANNING, SUSTAINED INNOVATION\par
ID \tabto{0cm}ACQUISITION, ANTECEDENTS, COMPETITIVE ADVANTAGE, EXPLOITATION, EXPLORATION, FIRM, KNOWLEDGE, NETWORK, RECONCEPTUALIZATION, RESEARCH-AND-DEVELOPMENT\par
AB \tabto{0cm}How can a firm develop new ideas and turn them into profitable innovations on a sustained basis? We address this fundamental issue in a novel way by developing an integrative framework of \hl{absorpti}ve \hl{capacit}y (AC) and inbound open innovation that is rooted in the attention-based view of the firm. We specifically address why a balance between open and closed innovation is important from the perspective of \hl{absorpti}ve \hl{capacit}y, and show how it may be brought about. Pursuing either open or closed inbound innovation alone may result in an imbalance between potential AC and realized AC as well as inward-looking AC and outward-looking AC, which will hinder innovative performance. We argue that practicing open and closed inbound innovation repeatedly and alternately by switching organizational attentions, and thus developing the associated AC, can facilitate balancing \hl{absorpti}ve \hl{capacit}y and lead to innovative performance. (C) 2015 Elsevier Ltd. All rights reserved.\par
\clearpage

\vspace*{-2cm}
Nb \tabto{0cm}322/327 (article\_id: 649)\par
TI \tabto{0cm}\hl{Absorpti}ve \hl{capacit}y: a non-linear process\par
AU \tabto{0cm}Aribi, Dupouet\par
PY \tabto{0cm}2016, SO KNOWLEDGE MANAGEMENT RESEARCH \& PRACTICE\par
DT \tabto{0cm}Article\par
PG \tabto{0cm}12, NR 44, TC 0\par
DE \tabto{0cm}\hl{ABSORPTI}VE \hl{CAPACIT}Y, FEEDBACK LOOPS IN THE \hl{ABSORPTI}VE PROCESS, NON-LINEARITY OF \hl{ABSORPTI}VE \hl{CAPACIT}Y\par
ID \tabto{0cm}CREATION, DYNAMIC CAPABILITIES, FIRM, ORGANIZATIONAL KNOWLEDGE, PERFORMANCE, RECONCEPTUALIZATION, TECHNOLOGICAL-INNOVATION\par
AB \tabto{0cm}Since the Cohen and Levinthal article on \hl{absorpti}ve \hl{capacit}y was published, 'the ability to recognize the value of new information, to assimilate it, and apply it to commercial ends' (p. 128) is seen as an essential competence for a firm's long-term performance. However, the way \hl{absorpti}ve \hl{capacit}y is actually implemented in firms remains relatively poorly known. The few existing works present \hl{absorpti}ve \hl{capacit}y as an essentially linear process, and the way the different phases of this process are actually carried out remains understudied. In order to enhance our understanding of the way firms absorb external knowledge, we gathered data from 23 interviews of managers from three different industrial firms. Our results suggest that, far from being linear, the process displays several feedback loops, both within and between each phase of \hl{absorpti}on. In this study, we enrich previous \hl{absorpti}ve \hl{capacit}y models.\par
\clearpage

\vspace*{-2cm}
Nb \tabto{0cm}323/327 (article\_id: 650)\par
TI \tabto{0cm}Knowledge transfer in university quadruple helix ecosystems: an \hl{absorpti}ve \hl{capacit}y perspective\par
AU \tabto{0cm}Miller, McAdam, Moffett, Alexander, Puthusserry\par
PY \tabto{0cm}2016, SO R \& D MANAGEMENT\par
DT \tabto{0cm}Article\par
PG \tabto{0cm}17, NR 76, TC 0\par
DE \tabto{0cm}null\par
ID \tabto{0cm}CONSTRUCT, ENTREPRENEURSHIP, FUTURE, INDUSTRY, MANAGEMENT, OPEN INNOVATION, STRATEGIES, SYSTEMS, TECHNOLOGY-TRANSFER, UK\par
AB \tabto{0cm}Increased understanding of knowledge transfer (KT) from universities to the wider regional knowledge ecosystem offers opportunities for increased regional innovation and commercialisation. The aim of this article is to improve the understanding of the KT phenomena in an open innovation context where multiple diverse quadruple helix stakeholders are interacting. An \hl{absorpti}ve \hl{capacit}y-based conceptual framework is proposed, using a priori constructs which portrays the multidimensional process of KT between universities and its constituent stakeholders in pursuit of open innovation and commercialisation. Given the lack of overarching theory in the field, an exploratory, inductive theory building methodology was adopted using semi-structured interviews, document analysis and longitudinal observation data over a three-year period. The findings identify five factors, namely human centric factors, organisational factors, knowledge characteristics, power relationships and network characteristics, which mediate both the ability of stakeholders to engage in KT and the effectiveness of knowledge acquisition, assimilation, transformation and exploitation. This research has implications for policy makers and practitioners by identifying the need to implement interventions to overcome the barriers to KT effectiveness between regional quadruple helix stakeholders within an open innovation ecosystem.\par
\clearpage

\vspace*{-2cm}
Nb \tabto{0cm}324/327 (article\_id: 651)\par
TI \tabto{0cm}Emergence of innovation networks from R\&D cooperation with endogenous \hl{absorpti}ve \hl{capacit}y\par
AU \tabto{0cm}Savin, Egbetokun\par
PY \tabto{0cm}2016, SO JOURNAL OF ECONOMIC DYNAMICS \& CONTROL\par
DT \tabto{0cm}Article\par
PG \tabto{0cm}22, NR 43, TC 0\par
DE \tabto{0cm}\hl{ABSORPTI}VE \hl{CAPACIT}Y, AGENT-BASED MODELING, COGNITIVE DISTANCE, INNOVATION, KNOWLEDGE SPILLOVERS, NETWORKS\par
ID \tabto{0cm}COLLABORATION, EVOLUTION, FIRM, INDUSTRIES, KNOWLEDGE, MODEL, RELATIONAL EMBEDDEDNESS, SMALL WORLDS\par
AB \tabto{0cm}This paper extends the existing literature on strategic R\&D alliances by presenting a model of innovation networks with endogenous \hl{absorpti}ve \hl{capacit}y. The networks emerge as a result of dynamic cooperation between firms occupying different locations in the knowledge space. Partner selection is driven by \hl{absorpti}ve \hl{capacit}y which is itself influenced by cognitive distance and R\&D investment allocation. Under different knowledge regimes, we examine the structure of networks that emerge and how firms perform within such networks. We find networks that exhibit small world properties which are generally robust to changes in the knowledge regime. Certain network strategies such as occupying brokerage positions or maximising accessibility to potential partners pay off, especially in 'young' industries with limited involuntary but abundant voluntary spillovers. This particular result is driven by endogenous \hl{absorpti}ve \hl{capacit}y. (C) 2016 Elsevier B.V. All rights reserved.\par
\clearpage

\vspace*{-2cm}
Nb \tabto{0cm}325/327 (article\_id: 652)\par
TI \tabto{0cm}\hl{Absorpti}ve \hl{capacit}y and benefits from FDI: Evidence from Chinese manufactured exports\par
AU \tabto{0cm}Tang, Zhang\par
PY \tabto{0cm}2016, SO INTERNATIONAL REVIEW OF ECONOMICS \& FINANCE\par
DT \tabto{0cm}Article\par
PG \tabto{0cm}7, NR 34, TC 1\par
DE \tabto{0cm}\hl{ABSORPTI}VE CAPABILITY (AC), FOREIGN DIRECT INVESTMENT (FDI), MANUFACTURED EXPORTS (MX)\par
ID \tabto{0cm}COUNTRY, FOREIGN DIRECT-INVESTMENT, GROWTH, LINKAGES, MULTINATIONALS, PRODUCTIVITY, SPILLOVERS, TRADE\par
AB \tabto{0cm}China's global export rank rose from the 32nd in 1978 to the 1st in 2009, and in the same period China had been a top recipient of foreign direct investment (FDI) in the world. Does large FDI inflow automatically lead to the export boom? Or is it a must for China to have certain \hl{absorpti}ve \hl{capacit}y (AC) given FDI? This work investigates how manufacturing exports (MX) are affected by the AC-FDI interaction. MX performance is assessed by three indicators: export \hl{capacit}y, export intensity, and export quality. AC is defined as a host-country's ability to capture potential benefits from FDI, and such ability is proxied by govemment FDI policy, human capital, R\&D, and infrastructure. Estimates are conducted with the data on 21 manufacturing sectors for 31 regions over 8 years (2005-2012). We find that (a) AC is necessary condition for China to benefit from FDI in MX, and contributions of FDI alone to MX are limited; (b) China's strong AC largely comes from well-designed FDI policy and high quality infrastructure, both of which complement with FDI in strengthening export \hl{capacit}y, intensity and quality; and (c) human capital and R\&D seems to be more helpful for China to capture spillovers from FDI to export quality. (C) 2015 Elsevier Inc. All rights reserved.\par
\clearpage

\vspace*{-2cm}
Nb \tabto{0cm}326/327 (article\_id: 653)\par
TI \tabto{0cm}Overcoming the false dichotomy between internal R\&D and external knowledge acquisition: \hl{Absorpti}ve \hl{capacit}y dynamics over time\par
AU \tabto{0cm}Denicolai, Ramirez, Tidd\par
PY \tabto{0cm}2016, SO TECHNOLOGICAL FORECASTING AND SOCIAL CHANGE\par
DT \tabto{0cm}Article\par
PG \tabto{0cm}9, NR 62, TC 0\par
DE \tabto{0cm}\hl{ABSORPTI}VE \hl{CAPACIT}Y, KNOWLEDGE ASSETS, LISTED COMPANIES, TECHNOLOGY SOURCING\par
ID \tabto{0cm}ANTECEDENTS, DETERMINANTS, EXPLOITATION, FIRM PERFORMANCE, HIGH-TECH, INDUSTRY, MANAGEMENT, OPEN INNOVATION, RESOURCES, TECHNOLOGICAL CAPABILITY\par
AB \tabto{0cm}An important challenge in open innovation is the capability to absorb and exploit external inbound knowledge, and how internal R\&D may facilitate or hinder this. Conventionally, internal R\&D expenditure is used as a proxy for \hl{absorpti}ve \hl{capacit}y, but in the context of open innovation, this can be problematic. Internal R\&D may also constrain present and future \hl{absorpti}on, and restrict exploitation for a number of reasons, e.g. degree of development, structural, geographical or relevance to existing business units and markets. Conversely, external sources of innovation can be difficult to identify, evaluate and absorb, but may be more codified, as by definition they are available in the market, and more fully-developed to demonstrate commercial potential. Using panel data of 325 firms over five years, we find that contrary to the prescriptions of transaction cost analysis, externally-sourced knowledge takes less time to absorb and exploit than internally-generated knowledge, but that internal knowledge creates higher returns over the longer term. Significantly, the relationship between internal and external knowledge and performance changes over time, while the ideal strategic balance needs to consider decisions taken at different times. (C) 2015 Elsevier Inc. All rights reserved.\par
\clearpage

\vspace*{-2cm}
Nb \tabto{0cm}327/327 (article\_id: 654)\par
TI \tabto{0cm}Using fuzzy-set qualitative comparative analysis to develop an \hl{absorpti}ve \hl{capacit}y-based view of training\par
AU \tabto{0cm}Hernandez-Perlines, Moreno-Garcia, Yanez-Araque\par
PY \tabto{0cm}2016, SO JOURNAL OF BUSINESS RESEARCH\par
DT \tabto{0cm}Article\par
PG \tabto{0cm}6, NR 39, TC 0\par
DE \tabto{0cm}\hl{ABSORPTI}VE \hl{CAPACIT}Y, CASE STUDY METHOD, FSQCA, ORGANIZATIONAL PERFORMANCE, PLS-SEM, TRAINING\par
ID \tabto{0cm}BEHAVIOR, CAPABILITIES, INNOVATION, MANAGEMENT, MODELS, PERFORMANCE, RECONCEPTUALIZATION\par
AB \tabto{0cm}Numerous studies examine the importance of training on organizational performance, albeit without resolving how the transformation process occurs. This study bridges this research gap. The study presents a model that shows that \hl{absorpti}ve \hl{capacit}y, particularly exploitation capability, mediates the relationship between training and organizational performance. Three analysis methods converge with the findings: case studies (6 cases), PLS-SEM (112 cases), and fsQCA (25 cases). Using these three data analysis methods in a single piece of research represents a considerable methodological contribution and enables the confirmation of the model's validity and robustness. Another of the contributions is that the use of fsQCA overcomes other methods' deficiencies. (c) 2015 Elsevier Inc. All rights reserved.\par
\clearpage

\end{document}
